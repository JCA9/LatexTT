\chapter{Planteamiento del problema}

\section{Antecedentes}
Esta idea surgió a partir de nuestra propia experiencia, ya que, al cumplir los 21 años, nos dimos cuenta de lo fácil que es perder el control sobre los gastos cotidianos. 
Ambos, al igual que muchos de nuestros amigos y personas cercanas, solíamos ignorar la importancia de llevar un registro detallado de nuestros ingresos y egresos. 
A medida que nos adentramos en la vida adulta, nos dimos cuenta de que la gestión de los gastos y el ahorro era algo que muchas veces se pasaba por alto, 
lo que provocaba una falta de conciencia sobre cuánto dinero realmente se estaba gastando. 

Comenzamos a investigar el impacto que tenía la falta de seguimiento financiero, y cómo las tecnologías actuales podrían ayudarnos a simplificar el proceso. 
Observamos que, a pesar de contar con una gran cantidad de herramientas digitales disponibles, la mayoría de estas eran demasiado complejas o no ofrecían un enfoque claro y 
accesible para personas sin experiencia en finanzas. Entonces recordamos nuestras propias experiencias como estudiantes, donde las dificultades para ahorrar o manejar el dinero 
se volvían recurrentes. 

Nos dimos cuenta de que había una desconexión entre las herramientas existentes y las necesidades reales de las personas, sobre todo de quienes están comenzando a tener 
responsabilidades financieras y que aún no tienen los conocimientos necesarios para optimizar sus finanzas. En nuestras conversaciones con amigos, colegas y familiares, 
descubrimos que la falta de un sistema práctico de seguimiento de gastos era un problema común, y que muchas personas sentían que no podían controlar sus finanzas debido a 
la falta de visibilidad y de herramientas efectivas. Esta experiencia compartida fue el motor que nos impulsó a crear algo que pudiera ayudar a las personas a tomar el control 
de sus finanzas desde una edad temprana. 

\section{Planteamiento del problema}
En México, la gestión inadecuada de las finanzas personales es una preocupación creciente. 
Según la Encuesta Nacional sobre Salud Financiera (ENSAFI) 2023, solo el 53.2\% de la población realiza algún tipo de registro o control de sus gastos, y únicamente el 32.8\% únicamente el 32.8\% cumple con dicho registro. \cite{inegi2024}Este comportamiento refleja una falta de conciencia y disciplina financiera que puede llevar a 
decisiones económicas desfavorables. 

Además, un estudio del Instituto Nacional de Estadística y Geografía (INEGI) revela que el 45.9\% de la población menciona que casi nunca o nunca le sobra dinero al final 
del mes, lo que indica una limitada capacidad de ahorro y planificación financiera. \cite{inegi2024} Esta situación se ve agravada por la creciente preocupación por el futuro financiero; 
un informe de Statista muestra que un alto porcentaje de personas en diversos países están preocupadas por su situación financiera futura. \cite{statista2025} 

A pesar de la disponibilidad de herramientas digitales, el uso de aplicaciones de finanzas personales ha sido limitado. Según un informe de Adjust, aunque las instalaciones 
de aplicaciones de finanzas aumentaron un 50\% en comparación con 2022, la retención de usuarios sigue siendo un desafío, con una disminución constante en la retención después 
del día 7 de uso. \cite{adjust2023}  Esto sugiere que, aunque existe interés inicial, los usuarios enfrentan barreras para mantener el uso continuo de estas herramientas. 

Estos datos evidencian una necesidad urgente de soluciones que fomenten hábitos financieros saludables, mejoren la educación financiera y faciliten el seguimiento de los gastos
personales. La falta de registros precisos y la escasa utilización de herramientas digitales eficaces contribuyen a una gestión financiera deficiente, afectando la estabilidad 
económica de los individuos. 

\section{Propuesta de solución}
Para abordar las problemáticas identificadas en la sección anterior, proponemos utilizar un conjunto de tecnologías digitales y algoritmos de análisis de datos que faciliten 
la gestión financiera personal de manera intuitiva y accesible (De las cuales hablaremos en el Capítulo 2, Marco Teórico.). Las herramientas seleccionadas se centran en automatizar la captura y categorización de gastos, ofrecer análisis 
visuales y generar alertas y recomendaciones basadas en los hábitos del usuario. 

En primer lugar, la digitalización de los tickets de compra mediante Reconocimiento Óptico de Caracteres (OCR, por sus siglas en inglés) permitirá capturar de manera automática 
la información de gastos, evitando la dependencia de registros manuales, los cuales según la ENSAFI 2023 solo realizan el 53.2\% de los mexicanos. \cite{inegi2024} Esto contribuiría a que 
los usuarios tengan un registro más preciso de sus gastos, reduciendo la omisión de información crítica. Por otra parte, el OCR al no ser 100\% efectivo, los usuarios podrán 
ingresar o corregir los datos del ticket de manera manual si esta falla. 

Además, la aplicación integrará algoritmos de categorización automática que clasificarán los gastos en diferentes rubros (alimentación, transporte, entretenimiento, etc.). 

Para mejorar la comprensión de los gastos, se emplearán tableros interactivos y gráficos dinámicos, basados en bibliotecas de visualización como ECharts, D3.js u otras 
alternativas similares, que permitirán al usuario identificar patrones de gasto y tendencias a lo largo del tiempo. Estudios de la OCDE (2023) destacan que la visualización 
de datos financieros mejora la toma de decisiones y la comprensión de los hábitos de consumo, aumentando la probabilidad de ahorro hasta en un 20\%. \cite{patel2022} 

De igual manera, para mejorar la comprensión de los gastos y permitir un seguimiento financiero más estructurado, se implementará un sistema de generación de reportes 
automáticos, tanto mensuales como anuales, que consolide la información capturada por la aplicación móvil y la plataforma web. Estos reportes incluirán tablas de resumen, 
gráficos interactivos y análisis comparativos de categorías de gasto, ingresos y ahorro. Además, los reportes pueden ser exportados a formatos comunes como PDF o Excel, 
facilitando la revisión histórica, el seguimiento de objetivos y la planificación de presupuestos futuros. 

Finalmente, se implementará un sistema de notificaciones y alertas inteligentes que informen al usuario sobre sus límites de gasto o posibles desviaciones de su presupuesto. 

En conjunto, estas tecnologías permiten atacar todas las problemáticas detectadas: la falta de registro de gastos, la escasa conciencia sobre hábitos financieros, la baja 
retención en el uso de aplicaciones y la dificultad de análisis financiero. La integración de OCR, visualización interactiva y alertas inteligentes crea un ecosistema digital 
que facilita el seguimiento y mejora la educación financiera del usuario. 

 
\section{Objetivos}
Teniendo en cuenta las problemáticas y la propuesta de solución planteadas, definimos los siguientes objetivos para el desarrollo del proyecto:

\subsection{Objetivo general}
Desarrollar una aplicación y plataforma web interactiva para la gestión de finanzas personales que registre automáticamente los gastos de tickets físicos y digitales, 
categorizándolos para facilitar el análisis y seguimiento del presupuesto del usuario. 

\subsection{Objetivos específicos}
\begin{itemize}
    \item Desarrollar el módulo de escaneo de tickets físicos mediante la aplicación móvil.
    \item Desarrollar el módulo de escaneo de tickets digitales mediante la plataforma web.
    \item Implementar el módulo de categorización automática de gastos.
    \item Implementar el módulo de alertas y notificaciones financieras.
    \item Desarrollar el módulo de generación de informes y reportes de gastos mensuales, trimestrales y anuales.
    \item Desarrollar el módulo de análisis de gastos con gráficas interactivas y visualización de tendencias.
    \item Desarrollar el módulo de ingreso y corrección manual de datos de gastos.
    \item Implementar el módulo de creación de planes financieros y seguimiento de deudas.
\end{itemize}

\section{Justificación}
La gestión adecuada de las finanzas personales es fundamental para garantizar la estabilidad económica y el bienestar financiero a largo plazo. Sin embargo, como se ha dicho en 
la sección de planteamiento del problema, una gran parte de la población en México enfrenta dificultades para llevar un control efectivo de sus gastos e ingresos, lo que puede 
llevar a problemas financieros significativos. \cite{inegi2024}

El desarrollo de una aplicación y plataforma web que facilite la gestión de las finanzas personales mediante la automatización del registro y análisis de gastos es una solución 
innovadora y necesaria. 
Al utilizar tecnologías como el reconocimiento óptico de caracteres (OCR) para capturar automáticamente los datos de los tickets de compra, se reduce la carga administrativa 
para el usuario, permitiéndole centrarse en la toma de decisiones financieras informadas. Además, la categorización automática de gastos y la generación de reportes visuales 
proporcionan una comprensión clara de los hábitos de consumo, lo que es esencial para identificar áreas de mejora y oportunidades de ahorro. Ademas, la implementación de alertas
y notificaciones inteligentes también contribuye a mantener a los usuarios informados sobre su situación financiera, ayudándoles a evitar gastos excesivos y a adherirse a sus 
presupuestos. Por lo que esta solución se orienta a usuarios jóvenes y adultos que buscan mejorar su educación financiera y adoptar hábitos de gasto más saludables.

La viabilidad del proyecto se sustenta en la disponibilidad de smartphones o una computadora, ya que según la INEGI, el 81.4\% de la población cuenta con acceso a dispositivos móviles y
el 43.8\% con acceso a computadoras. \cite{Encuesta Uso de Tecnologias INEGI}

El impacto esperado se medirá a través de la precisión de extracción de datos, el tiempo ahorrado frente a la captura manual y la tasa de retención de usuarios.
Con esto, este sistema busca demostrar que la tecnología puede ser una herramienta poderosa para mejorar la gestión financiera personal.

Este proyecto no solo aborda una necesidad crítica en la gestión financiera personal, sino que también empodera a los usuarios para tomar el control de sus finanzas, 
promoviendo hábitos financieros saludables y sostenibles a largo plazo. La combinación de tecnologías avanzadas y un enfoque centrado en el usuario hace que esta solución 
sea relevante y valiosa en el contexto actual, donde la digitalización y la automatización son clave para mejorar la eficiencia y la efectividad en diversas áreas de 
la vida cotidiana.
