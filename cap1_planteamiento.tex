\chapter{Planteamiento del problema}

\section{Antecedentes}
Esta idea surgió a partir de nuestra propia experiencia, ya que, al tener 21 años, nos dimos cuenta de lo fácil que es perder el control sobre los gastos cotidianos. 
Ambos, al igual que muchos de nuestros amigos y personas cercanas, solíamos ignorar la importancia de llevar un registro detallado de nuestros ingresos y egresos. 
A medida que nos adentramos en la vida adulta, nos dimos cuenta de que la gestión de los gastos y el ahorro era algo que muchas veces se pasaba por alto, 
lo que provocaba una falta de conciencia sobre cuánto dinero realmente se estaba gastando. 

Comenzamos a investigar el impacto que tenía la falta de seguimiento financiero, y cómo las tecnologías actuales podrían ayudarnos a simplificar el proceso. 
Observamos que, a pesar de contar con una gran cantidad de herramientas digitales disponibles, la mayoría de estas eran demasiado complejas o no ofrecían un enfoque claro y 
accesible para personas sin experiencia en finanzas. Entonces recordamos nuestras propias experiencias como estudiantes, donde las dificultades para ahorrar o manejar el dinero 
se volvían recurrentes. 

Nos dimos cuenta de que había una desconexión entre las herramientas existentes y las necesidades reales de las personas, sobre todo de quienes están comenzando a tener 
responsabilidades financieras y que aún no tienen los conocimientos necesarios para optimizar sus finanzas. En nuestras conversaciones con amigos, colegas y familiares, 
descubrimos que la falta de un sistema práctico de seguimiento de gastos era un problema común, y que muchas personas sentían que no podían controlar sus finanzas debido a 
la falta de visibilidad y de herramientas efectivas. Esta experiencia compartida fue el motor que nos impulsó a crear algo que pudiera ayudar a las personas a tomar el control 
de sus finanzas desde una edad temprana. 

\section{Planteamiento del problema}
En México, la gestión inadecuada de las finanzas personales es una preocupación creciente. 
Según la Encuesta Nacional sobre Salud Financiera (ENSAFI) 2023, solo el 53.2\% de la población realiza algún tipo de registro o control de sus gastos, y de este porcentaje, 
únicamente el 32.8\% cumple con dicho registro. [1] Este comportamiento refleja una falta de conciencia y disciplina financiera que puede llevar a 
decisiones económicas desfavorables. 

Además, un estudio del Instituto Nacional de Estadística y Geografía (INEGI) revela que el 45.9\% de la población menciona que casi nunca o nunca le sobra dinero al final 
del mes, lo que indica una limitada capacidad de ahorro y planificación financiera [1]. Esta situación se ve agravada por la creciente preocupación por el futuro financiero; 
un informe de Statista muestra que un alto porcentaje de personas en diversos países están preocupadas por su situación financiera futura [2]. 

A pesar de la disponibilidad de herramientas digitales, el uso de aplicaciones de finanzas personales ha sido limitado. Según un informe de Adjust, aunque las instalaciones 
de aplicaciones de finanzas aumentaron un 50\% en comparación con 2022, la retención de usuarios sigue siendo un desafío, con una disminución constante en la retención después 
del día 7 de uso [3].  Esto sugiere que, aunque existe interés inicial, los usuarios enfrentan barreras para mantener el uso continuo de estas herramientas. 

Estos datos evidencian una necesidad urgente de soluciones que fomenten hábitos financieros saludables, mejoren la educación financiera y faciliten el seguimiento de los gastos
personales. La falta de registros precisos y la escasa utilización de herramientas digitales eficaces contribuyen a una gestión financiera deficiente, afectando la estabilidad 
económica de los individuos. 

\section{Propuesta de solución}
Para abordar las problemáticas identificadas en la sección anterior, se propone utilizar un conjunto de tecnologías digitales y algoritmos de análisis de datos que faciliten 
la gestión financiera personal de manera intuitiva y accesible. Las herramientas seleccionadas se centran en automatizar la captura y categorización de gastos, ofrecer análisis 
visuales y generar alertas y recomendaciones basadas en los hábitos del usuario. 

En primer lugar, la digitalización de los tickets de compra mediante Reconocimiento Óptico de Caracteres (OCR, por sus siglas en inglés) permitirá capturar de manera automática 
la información de gastos, evitando la dependencia de registros manuales, los cuales según la ENSAFI 2023 solo realizan el 53.2\% de los mexicanos [1]. Esto contribuiría a que 
los usuarios tengan un registro más preciso de sus gastos, reduciendo la omisión de información crítica. Por otra parte, el OCR al no ser 100\% efectivo, los usuarios podrán 
ingresar o corregir los datos del ticket de manera manual si esta falla. 

Además, la aplicación integrará algoritmos de categorización automática que clasificarán los gastos en diferentes rubros (alimentación, transporte, entretenimiento, etc.). 

Para mejorar la comprensión de los gastos, se emplearán tableros interactivos y gráficos dinámicos, basados en bibliotecas de visualización como ECharts, D3.js u otras 
alternativas similares, que permitirán al usuario identificar patrones de gasto y tendencias a lo largo del tiempo. Estudios de la OCDE (2023) destacan que la visualización 
de datos financieros mejora la toma de decisiones y la comprensión de los hábitos de consumo, aumentando la probabilidad de ahorro hasta en un 20\% [4]. 

De igual manera, para mejorar la comprensión de los gastos y permitir un seguimiento financiero más estructurado, se implementará un sistema de generación de reportes 
automáticos, tanto mensuales como anuales, que consolide la información capturada por la aplicación móvil y la plataforma web. Estos reportes incluirán tablas de resumen, 
gráficos interactivos y análisis comparativos de categorías de gasto, ingresos y ahorro. Además, los reportes pueden ser exportados a formatos comunes como PDF o Excel, 
facilitando la revisión histórica, el seguimiento de objetivos y la planificación de presupuestos futuros. 

Finalmente, se implementará un sistema de notificaciones y alertas inteligentes que informen al usuario sobre sus límites de gasto o posibles desviaciones de su presupuesto. 

En conjunto, estas tecnologías permiten atacar todas las problemáticas detectadas: la falta de registro de gastos, la escasa conciencia sobre hábitos financieros, la baja 
retención en el uso de aplicaciones y la dificultad de análisis financiero. La integración de OCR, visualización interactiva y alertas inteligentes crea un ecosistema digital 
que facilita el seguimiento y mejora la educación financiera del usuario. 

 

\section{Objetivo general}
Desarrollar una aplicación y plataforma web interactiva para la gestión de finanzas personales que registre automáticamente los gastos de tickets físicos y digitales, 
categorizándolos para facilitar el análisis y seguimiento del presupuesto del usuario. 

\section{Objetivos específicos}
\begin{itemize}
    \item Desarrollar el módulo de escaneo de tickets físicos mediante la aplicación móvil.
    \item Desarrollar el módulo de escaneo de tickets digitales mediante la plataforma web.
    \item Implementar el módulo de categorización automática de gastos.
    \item Implementar el módulo de alertas y notificaciones financieras.
    \item Desarrollar el módulo de generación de informes y reportes de gastos mensuales, trimestrales y anuales.
    \item Desarrollar el módulo de análisis de gastos con gráficas interactivas y visualización de tendencias.
    \item Desarrollar el módulo de ingreso y corrección manual de datos de gastos.
    \item Implementar el módulo de creación de planes financieros y seguimiento de deudas.
\end{itemize}

\section{Justificación}
La correcta gestión de las finanzas personales es un desafío creciente, especialmente en países como México, donde una gran parte de la población enfrenta dificultades para 
llevar un control adecuado de sus gastos e ingresos. Según la Encuesta Nacional sobre Salud Financiera (ENSAFI) 2023, solo el 53.2\% de los mexicanos realiza algún tipo de 
control de sus gastos, lo que refleja una carencia de conciencia sobre la importancia de la planificación financiera. Esta situación se ve reflejada en la incapacidad de muchas 
personas para ahorrar, pagar deudas o planificar para el futuro, lo que compromete su estabilidad económica.

Este proyecto propone una solución digital que combina tecnologías como el Reconocimiento Óptico de Caracteres (OCR), análisis de datos, categorización automática de gastos 
y visualización interactiva para facilitar la toma de decisiones financieras de los usuarios. Al automatizar el registro de gastos a partir de tickets físicos y digitales, 
eliminamos parcialmente la barrera del registro manual y promovemos un control preciso y rápido de los gastos. Esto no solo reduce el tiempo de seguimiento de finanzas, sino 
que también mejora la precisión y la visibilidad de las categorías de gasto, lo cual es crucial para la salud financiera a largo plazo.

La adopción de tecnologías como el OCR y la clasificación automática de datos no solo mejora la eficiencia, sino que también ofrece a los usuarios una experiencia 
personalizada, mediante la cual pueden establecer presupuestos, recibir notificaciones de alerta y generar reportes automáticos sobre su comportamiento financiero. La 
plataforma web, por otro lado, proporciona una visión más detallada de los gastos mensuales, anuales y comparaciones entre categorías, lo que facilita la toma de decisiones 
informadas.

El enfoque técnico del proyecto se justifica por la necesidad de herramientas accesibles, eficientes y fáciles de usar que permitan a los usuarios mejorar sus hábitos 
financieros sin la necesidad de una formación financiera previa. Esta solución facilitará la educación financiera de los usuarios, proporcionándoles las herramientas necesarias 
para tener un control más riguroso de su dinero y contribuir a una cultura de finanzas personales saludables.