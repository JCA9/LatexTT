\section{Diseño del Sistema}

\subsection{Arquitectura del Sistema}
Según Tan Dang en Orient Software \cite{Arquitectura}, la arquitectura de un sistema representa el plano o estructura general del mismo, describiendo sus componentes principales, 
las relaciones que existen entre ellos y la manera en que interactúan para cumplir los objetivos del software.
Esta arquitectura sirve como guía para el desarrollo, mantenimiento y evolución del sistema, garantizando que todos los módulos funcionen de manera integrada y coherente.

\begin{itemize}
    \item \textbf{Modelo de arquitectura adoptado}
    El sistema que proponemos se basa en una arquitectura Cliente–Servidor bajo un modelo en capas, con el objetivo de separar las responsabilidades de cada componente y garantizar 
    la mantenibilidad, escalabilidad y modularidad del software.
    En este modelo, los clientes representados por la aplicación móvil y la plataforma web envían peticiones al servidor, el cual procesa la información y responde con los 
    resultados solicitados.
    El servidor central gestiona la lógica de negocio, la conexión con la base de datos y los procesos automatizados, como la extracción de texto mediante OCR y la 
    categorización de gastos.

    \begin{figure}[H]
    \centering
    \includegraphics[width=0.5\linewidth]{./images/EjemploArquitecturaCapas.png}
    \caption{Ejemplo de arquitectura en capas}
    \end{figure}

    \item \textbf{Arquitectura en capas}
    Se emplea una arquitectura multicapa (n-tier) que separa responsabilidades en presentación, aplicación/control, lógica de negocio y datos. La separación clara facilita 
    la mantenibilidad, permite evolucionar capas de forma independiente y reduce el acoplamiento, tal como describen los patrones clásicos de layered architecture y las 
    guías de arquitectura de Microsoft Azure \cite{ArquitecturaCapas}. Cada capa cumple una función específica dentro del flujo general de procesamiento, 
    desde la captura de información hasta la generación de reportes.

    \item \textbf{Capa de presentación}
    La capa de presentación (app móvil y web) se encarga de la interacción con el usuario y de enviar/recibir solicitudes al backend mediante HTTP/HTTPS. 
    En una arquitectura n-tier, esta capa debe permanecer delgada, delegando la lógica de negocio a capas inferiores; es el enfoque recomendado para aplicaciones 
    empresariales y está alineado con las guías de N-tier de Azure y con la organización en capas para aplicaciones empresariales descrita por Fowler. \cite{ArquitecturaPresenctacion}

    \item \textbf{Capa de aplicación o control}
    La capa de aplicación expone una API (REST) que coordina flujos entre la interfaz y los módulos de negocio (procesamiento de tickets, categorización, reportes), 
    maneja sesiones/autenticación y orquesta transacciones. El estilo REST promueve interfaces uniformes, desacoplamiento cliente-servidor y escalabilidad del sistema, 
    según la tesis de Roy T. Fielding; en seguridad de endpoints y sesiones, se siguen lineamientos de OWASP ASVS. \cite{ArquitecturaAplicacion}

    \item \textbf{Capa de lógica de negocio}
    Aquí residen los casos de uso y las reglas del dominio: validación de datos extraídos del ticket, normalización, categorización basada en reglas (regex/diccionarios) y 
    generación de reportes. Los patrones de enterprise application recomiendan encapsular la lógica de dominio en esta capa para aislarla de presentación y datos, facilitando 
    pruebas, evolución y reuso. \cite{ArquitecturaNegocio}

    \item \textbf{Capa de datos}
    La capa de datos se encarga del almacenamiento, recuperación y actualización de la información del sistema.
    Utiliza una base de datos MySQL, donde se gestionan las tablas de usuarios, tickets, categorías, presupuestos, reportes y notificaciones.
\end{itemize}

    \begin{figure}[H]
    \centering
    \includegraphics[width=0.7\linewidth]{./images/ArquitecturaCapasFinara.JPG}
    \caption{Arquitectura en capas del sistema propuesto}
    \end{figure}

\subsection{Diseño de la Base de Datos}
La base de datos del sistema la diseñaremos bajo un enfoque relacional, utilizando MySQL como gestor por su estabilidad, escalabilidad y compatibilidad con sistemas cliente-servidor.
El diseño busca garantizar la integridad, consistencia y disponibilidad de la información financiera procesada por el sistema, permitiendo almacenar de forma 
estructurada los datos provenientes de la aplicación móvil y la plataforma web.

\begin{itemize}
    \item \textbf{Modelo entidad-relación}
    
El modelo entidad-relación constituye la base del diseño lógico de la base de datos, al permitir identificar las entidades del dominio del problema y las relaciones existentes entre ellas.
De acuerdo con Connolly y Begg \cite{ModeloER}, un modelo E-R “proporciona una representación conceptual de alto nivel que describe los datos de una organización y las asociaciones entre ellos”.

    \begin{figure}[H]
    \centering
    \includegraphics[width=1\linewidth]{./images/Modelo Entidad - Relacion.png}
    \caption{Modelo entidad-relación del sistema propuesto en módulos}
    \end{figure}

En la Figura anterior presentamos el modelo entidad-relación del sistema de finanzas personales, diseñado bajo un enfoque relacional utilizando MySQL como gestor de base de datos.
El modelo se compone de cinco módulos funcionales:

Usuarios y control: define la entidad central Usuarios, responsable de almacenar los datos personales, credenciales y fecha de registro.

Tickets y productos: almacena los comprobantes procesados por el módulo OCR y los artículos asociados a cada ticket.

Finanzas y movimientos: gestiona la información de ingresos y egresos, enlazados a categorías personalizadas por cada usuario.

Deudas y planes financieros: permite registrar compromisos económicos y metas de ahorro.

Alertas: almacena notificaciones automáticas generadas por el sistema.

Este modelo asegura integridad referencial mediante claves primarias (PK) y foráneas (FK), reflejando el flujo lógico del sistema:
el usuario registra un ticket, el sistema procesa los datos mediante OCR, clasifica la información y la vincula con sus categorías, deudas o alertas correspondientes.

    \item \textbf{Tablas y relaciones}
    
El modelo físico se implementa mediante un conjunto de tablas interrelacionadas que representan las entidades identificadas en el modelo E-R.
Cada tabla posee una clave primaria que garantiza la unicidad de los registros y claves foráneas que preservan la coherencia entre los módulos del sistema.

Las relaciones más relevantes son:

Usuarios-Tickets: relación 1:0..N, donde un usuario puede registrar varios tickets.

Tickets-ProductosTickets: relación 1:0..N, cada ticket puede contener múltiples productos asociados.

Usuarios-Movimientos: relación 1:0..N, un usuario genera múltiples registros financieros.

Categorías-Movimientos: relación 1:N, cada movimiento pertenece a una categoría definida por el usuario.

Usuarios-Deudas-Usuarios: relación 1:0..N, que permite asignar deudas a usuarios específicos.

Categorías-Presupuestos: relación 1:1, cada categoria puede tener un presupuesto.

Según Elmasri y Navathe \cite{TablasYRelaciones}, un buen diseño relacional debe “mantener una estructura normalizada, clara y eficiente, minimizando redundancias y asegurando la 
integridad de los datos”.
En el presente sistema se implementaron índices en los campos más consultados (id-usuario, id-categoria, fecha-creacion) para optimizar el rendimiento en operaciones de 
lectura y filtrado, especialmente en la generación de reportes y estadísticas financieras.

    \item \textbf{Normalización}
    
El diseño de la base de datos se normalizó hasta la Tercera Forma Normal (3FN), garantizando que cada tabla contenga datos atómicos, dependientes únicamente de la clave 
primaria, y que no existan dependencias transitivas.
De acuerdo con Silberschatz et al. \cite{Normalización}, la normalización “reduce la redundancia y evita anomalías en las operaciones de inserción, actualización y eliminación de datos”.

En el sistema desarrollado:

Se eliminaron dependencias parciales mediante la creación de tablas intermedias (Deudas-Usuarios y Productos-Tickets).

Cada atributo depende exclusivamente de su clave primaria, cumpliendo con los requisitos de la 3FN.

No existen campos multivaluados ni datos repetidos entre entidades.

Este proceso asegura una estructura coherente, favorece la integridad referencial y mejora la eficiencia en consultas complejas, especialmente en operaciones de categorización y generación de reportes financieros.
\end{itemize}

\subsection{Modelado UML}
\begin{itemize}
    \item \textbf{Diagramas de casos de uso}

    Aquí ilustraremos los casos de uso identificados anteriormente en la fase de requerimientos, mostrando las interacciones entre los actores y el sistema.

    \begin{figure}[H]
    \centering
    \includegraphics[width=1\linewidth]{./images/CasosUsoApp.JPG}
    \caption{Diagrama UML de Casos de Uso - App Móvil}
    \end{figure}

    \begin{figure}[H]
    \centering
    \includegraphics[width=1\linewidth]{./images/CasosUsoWeb.JPG}
    \caption{Diagrama UML de Casos de Uso - Plataforma Web}
    \end{figure}

    \item \textbf{Diagramas de secuencia}
    
    Un diagrama de secuencia muestra cómo los objetos interactúan en un escenario particular de un caso de uso,
    detallando el orden de los mensajes intercambiados entre ellos a lo largo del tiempo. \cite{DIagramaSecuencia}
    
    \begin{figure}[H]
    \centering
    \includegraphics[width=1\linewidth]{./images/DiagramaSecuenciaInicioSesion.png}
    \caption{Diagrama UML de Secuencia - Inicio de Sesión}
    \end{figure}

    \begin{figure}[H]
    \centering
    \includegraphics[width=1\linewidth]{./images/DiagramaSecuenciaOCR.png}
    \caption{Diagrama UML de Secuencia - Procesamiento OCR}
    \end{figure}

    \begin{figure}[H]
    \centering
    \includegraphics[width=1\linewidth]{./images/DiagramaSecuenciaCategorizacion.png}
    \caption{Diagrama UML de Secuencia - Procesamiento de Categorización}
    \end{figure}

    \begin{figure}[H]
    \centering
    \includegraphics[width=1\linewidth]{./images/DiagramaSecuenciaReportes.png}
    \caption{Diagrama UML de Secuencia - Generación de Reportes}
    \end{figure}

    \item \textbf{Análisis del flujo del Sistema}

    En esta sección, se presenta un análisis detallado del flujo del sistema, identificando los diferentes procesos y cómo interactúan entre sí. 
    Utilizamos un diagrama para ilustrar estos procesos de manera clara y concisa.

    \begin{figure}[H]
    \centering
    \includegraphics[width=1\linewidth, height=0.3\textheight]{./images/Flujo del Sistema.jpg}
    \caption{Diagrama del sistema propuesto}
    \end{figure}

    \item \textbf{Análisis del flujo del OCR}
    
    El siguiente diagrama ilustra un prototipo del flujo del proceso de Reconocimiento Óptico de Caracteres (OCR) dentro del sistema.
    
    \begin{landscape}
    \begin{figure}[H]
    \centering
    \includegraphics[width=\linewidth, height=\textheight, keepaspectratio]{./images/FlujoOCR.png}
    \caption{Diagrama del flujo del OCR}
    \end{figure}
    \end{landscape}

\end{itemize}

\subsection{Diseño de la Interfaz de Usuario}
\begin{itemize}
    \item \textbf{Wireframes App Móvil}

Para el diseño de la interfaz de usuario de la aplicación móvil, creamos wireframes que representan las diferentes pantallas y funcionalidades de la aplicación. 
Estos wireframes sirven como guía visual para el desarrollo de la interfaz, asegurando una experiencia de usuario intuitiva y coherente.

\begin{figure}[H]
    \centering
    \includegraphics[width=0.5\linewidth]{./images/wireframesApp/LogIn.png}
    \caption{Wireframe de la pantalla de inicio de sesión - App Móvil}
\end{figure}

\begin{figure}[H]
    \centering
    \includegraphics[width=0.6\linewidth]{./images/wireframesApp/Carrusel.png}
    \caption{Wireframe de la pantalla de carrusel - App Móvil}
\end{figure}

\begin{figure}[H]
    \centering
    \includegraphics[width=1\linewidth]{./images/wireframesApp/Principal.png}
    \caption{Wireframe de la pantalla de principal - App Móvil}
\end{figure}

\begin{figure}[H]
    \centering
    \includegraphics[width=1\linewidth]{./images/wireframesApp/Transacciones.png}
    \caption{Wireframe de la pantalla de transacciones - App Móvil}
\end{figure}

\begin{figure}[H]
    \centering
    \includegraphics[width=1\linewidth]{./images/wireframesApp/Presupuestos.png}
    \caption{Wireframe de la pantalla de presupuestos - App Móvil}
\end{figure}

\begin{figure}[H]
    \centering
    \includegraphics[width=1\linewidth]{./images/wireframesApp/Deudas.png}
    \caption{Wireframe de la pantalla de deudas - App Móvil}
\end{figure}

\begin{figure}[H]
    \centering
    \includegraphics[width=1\linewidth]{./images/wireframesApp/Notificaciones.png}
    \caption{Wireframe de la pantalla de Notificaciones - App Móvil}
\end{figure}

\begin{figure}[H]
    \centering
    \includegraphics[width=0.6\linewidth]{./images/wireframesApp/Escanear Tickets.png}
    \caption{Wireframe de la pantalla de Escaneo de Tickets - App Móvil}
\end{figure}

    \item \textbf{Wireframes Panel Web}
Para el diseño de la interfaz de usuario del panel web, creamos wireframes que representan las diferentes pantallas y funcionalidades de la plataforma web (Dashboard).
Estos wireframes sirven como guía visual para el desarrollo de la interfaz, asegurando una experiencia de usuario intuitiva y coherente.

\begin{figure}[H]
    \centering
    \includegraphics[width=1\linewidth]{./images/wireframesWeb/LandingPage.png}
    \caption{Wireframe de la Landing Page - Panel Web}
\end{figure}

\begin{figure}[H]
    \centering
    \includegraphics[width=1\linewidth]{./images/wireframesWeb/IniciarSesion.png}
    \caption{Wireframe de la pantalla de Iniciar Sesión - Panel Web}
\end{figure}

\begin{figure}[H]
    \centering
    \includegraphics[width=1\linewidth]{./images/wireframesWeb/DashInicio.png}
    \caption{Wireframe de la pantalla de Dashboard Inicial - Panel Web}
\end{figure}

\begin{figure}[H]
    \centering
    \includegraphics[width=1\linewidth]{./images/wireframesWeb/DashTransacciones.png}
    \caption{Wireframe de la pantalla de Transacciones - Panel Web}
\end{figure}

\begin{figure}[H]
    \centering
    \includegraphics[width=1\linewidth]{./images/wireframesWeb/DashPresupuestos.png}
    \caption{Wireframe de la pantalla de Presupuestos - Panel Web}
\end{figure}

\begin{figure}[H]
    \centering
    \includegraphics[width=1\linewidth]{./images/wireframesWeb/DashDeudas.png}
    \caption{Wireframe de la pantalla de Deudas - Panel Web}
\end{figure}

\begin{figure}[H]
    \centering
    \includegraphics[width=1\linewidth]{./images/wireframesWeb/DashRecomendaciones.png}
    \caption{Wireframe de la pantalla de Recomendaciones por IA - Panel Web}
\end{figure}

\end{itemize}