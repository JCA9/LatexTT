
\subsection{Reglas de negocio}
Las reglas de negocio son directrices específicas que definen o restringen ciertos aspectos del funcionamiento del sistema, asegurando que las operaciones se realicen 
de acuerdo con las políticas y objetivos establecidos. Estas reglas son fundamentales para mantener la integridad y coherencia del sistema. A continuación, presentamos 
las principales reglas de negocio identificadas para nuestro sistema. \cite{ReglasNegocio}

\begin{table}[H]
\centering
\renewcommand{\arraystretch}{1.2}
\begin{tabularx}{\textwidth}{@{} l l X @{}}
\toprule
\textbf{Código} & \textbf{Nombre} & \textbf{Descripción} \\
\midrule
RN01 & Registro único de usuario &
Cada usuario debe tener una cuenta única asociada a un correo electrónico válido y no podrá registrar duplicados. \\

RN02 & Validación de credenciales &
Las credenciales ingresadas deben ser verificadas en la base de datos antes de permitir el acceso al sistema. \\

RN03 & Límite de tamaño de ticket &
Las imágenes de los tickets no deben superar los 10 MB para evitar saturación del servidor. \\

RN04 & Formato de imagen permitido &
Solo se aceptan imágenes en formato JPG, PNG o PDF para la carga de tickets. \\

RN05 & Procesamiento OCR controlado &
Cada ticket se procesa una sola vez para evitar duplicación de datos en la base de datos. \\

RN06 & Categorización automática &
Los gastos se clasifican automáticamente según palabras clave o coincidencias con categorías predefinidas. \\

RN07 & Corrección manual de datos &
El usuario podrá editar los datos extraídos, pero los cambios quedarán registrados en la bitácora. \\

RN08 & Generación de reportes &
El sistema solo generará reportes cuando existan datos válidos en el rango de fechas seleccionado. \\

RN09 & Notificaciones de presupuesto &
Las alertas se enviarán únicamente cuando el gasto acumulado supere el 80\% del presupuesto configurado. \\

RN10 & Respaldo y seguridad &
Toda la información importante del usuario debe respaldarse automáticamente cada semana en la nube. \\

RN11 & Eliminación de datos &
Los usuarios podrán eliminar sus datos personales y gastos registrados, lo cual será irreversible. \\

RN12 & Acceso por roles &
Solo los usuarios autenticados podrán acceder a sus datos y reportes personales. \\

RN13 & Actualización de precios o montos &
Los valores de los gastos no podrán modificarse después de generar un reporte cerrado. \\

\bottomrule
\end{tabularx}
\caption{Reglas de Negocio (RN01–RN13)}
\end{table}

\newpage
\begin{table}[H]
\centering
\renewcommand{\arraystretch}{1.2}
\begin{tabularx}{\textwidth}{@{} l l X @{}}
\toprule
\textbf{Código} & \textbf{Nombre} & \textbf{Descripción} \\
\midrule

RN14 & Periodos de análisis &
Los reportes financieros se generan por día, semana, mes o año, según la elección del usuario. \\

RN15 & Tolerancia de OCR &
Si el reconocimiento de texto tiene menos del 60\% de precisión, se solicitará revisión manual al usuario. \\

\bottomrule
\end{tabularx}
\caption{Reglas de Negocio (RN14 - RN01–RN15)}
\end{table}