
\subsection{Reglas de negocio}
Las reglas de negocio son directrices específicas que definen o restringen ciertos aspectos del funcionamiento del sistema de administración de finanzas personales, asegurando que las operaciones se realicen de acuerdo con las políticas financieras y objetivos establecidos. Estas reglas son fundamentales para mantener la integridad, coherencia y funcionalidad del sistema orientado a jóvenes adultos. A continuación, presentamos las principales reglas de negocio identificadas para nuestro sistema. \cite{ReglasNegocio}

\begin{table}[H]
\centering
\renewcommand{\arraystretch}{1.2}
\begin{tabularx}{\textwidth}{@{} l l X @{}}
\toprule
\textbf{Código} & \textbf{Nombre} & \textbf{Descripción} \\
\midrule

\multicolumn{3}{c}{\textbf{GESTIÓN DE USUARIOS Y SEGURIDAD}} \\
\midrule

RN01 & Registro único de usuario &
Cada usuario debe tener una cuenta única asociada a un correo electrónico válido. No se permiten cuentas duplicadas con el mismo correo electrónico. \\

RN02 & Autenticación múltiple &
Los usuarios pueden acceder a la aplicación móvil y la plataforma web con las mismas credenciales. \\

RN03 & Tiempo de sesión &
Las sesiones de usuario expiran automáticamente después de 5 minutos de inactividad por motivos de seguridad. \\

RN04 & Validación de credenciales &
Todas las credenciales deben ser verificadas contra la base de datos antes de permitir el acceso al sistema. \\

\midrule
\multicolumn{3}{c}{\textbf{PROCESAMIENTO OCR Y TICKETS}} \\
\midrule

RN05 & Límite de tamaño de imagen &
Las imágenes de tickets no deben superar los 10 MB para optimizar el rendimiento del servidor y la experiencia del usuario. \\

RN06 & Formatos de imagen permitidos &
Solo se aceptan imágenes en formato JPG, PNG o PDF para la carga y procesamiento de tickets. \\

RN07 & Idioma de procesamiento OCR &
El sistema de OCR únicamente procesa tickets en idioma español. Tickets en otros idiomas requieren registro manual. \\

RN08 & Antigüedad recomendada de tickets &
Se recomienda fotografiar tickets al momento de la compra. Tickets con más de 30 días de antigüedad pueden tener menor calidad de escaneo. \\

RN09 & Procesamiento único por ticket &
Cada imagen de ticket se procesa una sola vez para evitar duplicación de datos. Si se requiere reprocesamiento, debe eliminarse el registro anterior. \\

RN10 & Validación manual obligatoria &
Cuando el OCR presenta discrepancias o baja confianza en la extracción, se solicita validación manual obligatoria del usuario. \\

\bottomrule
\end{tabularx}
\caption{Reglas de Negocio del Sistema (RN01–RN10)}
\end{table}

\begin{table}[H]
\centering
\renewcommand{\arraystretch}{1.2}
\begin{tabularx}{\textwidth}{@{} l l X @{}}
\toprule
\textbf{Código} & \textbf{Nombre} & \textbf{Descripción} \\
\midrule

\multicolumn{3}{c}{\textbf{CATEGORIZACIÓN Y PRESUPUESTOS}} \\
\midrule

RN11 & Límite de categorías personalizadas &
Cada usuario puede crear máximo 10 categorías personalizadas, siguiendo el modelo de presupuestación por sobres de Dave Ramsey. \\

RN12 & Categorías del sistema no eliminables &
Las categorías básicas del sistema (Alimentos/Despensa, Transporte, Ahorro) no pueden ser eliminadas por el usuario. \\

RN13 & Asignación única de categoría &
Cada gasto debe asignarse únicamente a una categoría. No se permite la división de gastos entre múltiples categorías. \\

RN14 & Categorización automática &
Los gastos se clasifican automáticamente según palabras clave y patrones de coincidencia con categorías predefinidas. \\

RN15 & Presupuestos mensuales &
Los presupuestos se configuran y controlan por períodos mensuales. \\

RN16 & Reasignación automática &
Al modificar o eliminar una categoría, todos los gastos asociados se reasignan automáticamente según las reglas de categorización definidas. \\

\midrule
\multicolumn{3}{c}{\textbf{GESTIÓN DE INGRESOS Y EGRESOS}} \\
\midrule

RN17 & Registro sin monto mínimo &
No existe monto mínimo para registrar ingresos o egresos. Se acepta cualquier valor mayor a cero. \\

RN18 & Ingresos recurrentes &
El sistema permite configurar ingresos recurrentes mensuales que se registran automáticamente (ej: salario, mesada). \\

\bottomrule
\end{tabularx}
\caption{Reglas de Negocio del Sistema (RN11–RN18)}
\end{table}

\begin{table}[H]
\centering
\renewcommand{\arraystretch}{1.2}
\begin{tabularx}{\textwidth}{@{} l l X @{}}
\toprule
\textbf{Código} & \textbf{Nombre} & \textbf{Descripción} \\
\midrule

\multicolumn{3}{c}{\textbf{GESTIÓN DE DEUDAS}} \\
\midrule

RN19 & Límite máximo de deudas &
Cada usuario puede registrar máximo 20 deudas activas simultáneamente para mantener un control financiero efectivo. \\

RN20 & Tipos de deuda soportados &
El sistema maneja diferentes tipos de deuda: tarjetas de crédito, préstamos personales, deudas familiares, y otras deudas sin clasificar. \\

RN21 & Alertas de vencimiento &
Se generan alertas automáticas para recordar pagos de deudas según la configuración del usuario. \\

\midrule
\multicolumn{3}{c}{\textbf{NOTIFICACIONES Y ALERTAS}} \\
\midrule

RN22 & Configuración personalizable &
Los usuarios pueden habilitar o deshabilitar tipos específicos de alertas por módulo (presupuestos, deudas, reportes). \\

RN23 & Alertas de presupuesto &
Se envían alertas cuando el gasto acumulado supera el 80\% del presupuesto configurado por categoría. \\

\bottomrule
\end{tabularx}
\caption{Reglas de Negocio del Sistema (RN19–RN23)}
\end{table}

\begin{table}[H]
\centering
\renewcommand{\arraystretch}{1.2}
\begin{tabularx}{\textwidth}{@{} l l X @{}}
\toprule
\textbf{Código} & \textbf{Nombre} & \textbf{Descripción} \\
\midrule

\multicolumn{3}{c}{\textbf{REPORTES Y ANÁLISIS}} \\
\midrule

RN24 & Períodos de reporte &
Los reportes se generan por día, semana, mes o año desde la plataforma web. \\

RN25 & Rango de fechas limitado &
Los reportes se generan únicamente para fechas donde existan transacciones registradas en el sistema. \\

RN26 & Formatos de exportación &
Los reportes pueden exportarse en formato PDF para visualización o Excel para análisis adicional. \\

\midrule
\multicolumn{3}{c}{\textbf{INTEGRIDAD Y SEGURIDAD DE DATOS}} \\
\midrule

RN27 & Eliminación irreversible &
La eliminación de datos personales, transacciones o categorías es irreversible y requiere confirmación explícita del usuario. \\

RN28 & Integridad referencial &
Al eliminar una categoría, todos los gastos asociados deben reasignarse a una categoría válida antes de completar la operación. \\

RN29 & Validación de montos &
Todos los montos ingresados deben ser valores numéricos positivos mayores a cero con máximo dos decimales. \\


\bottomrule
\end{tabularx}
\caption{Reglas de Negocio del Sistema (RN24–RN29)}
\end{table}