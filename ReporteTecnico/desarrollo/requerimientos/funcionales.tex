\section{Requerimientos}
En esta sección detallaremos los requerimientos que debe cumplir el sistema.

\subsection{Requerimientos Funcionales - Casos de Uso}
Los requerimientos funcionales describen las funciones específicas que el sistema debe realizar para satisfacer las necesidades del usuario. 
A continuación, presentamos los principales requerimientos funcionales identificados para nuestro sistema y posteriormente en la sección de diseño
mostraremos los diagramas UML. \cite{RequisitosFuncionales}

\begin{itemize}
    \item \textbf{Requerimientos/Casos de Uso Compartidos (App Móvil y Plataforma Web)} 
        
    \begin{table}[H]
    \centering
    \renewcommand{\arraystretch}{1.2}
    \begin{tabularx}{\textwidth}{@{} l l X @{}}
    \toprule
    \textbf{Código} & \textbf{Nombre} & \textbf{Descripción} \\
    \midrule
    RF01 & Registro de usuario &
    El sistema debe permitir el registro de nuevos usuarios mediante correo electrónico y contraseña, validando los datos ingresados antes de crear la cuenta. \\

    RF02 & Inicio de sesión &
    El sistema debe permitir que los usuarios registrados inicien sesión de forma segura tanto en la aplicación móvil como en la plataforma web. \\

    RF03 & Recuperación de contraseña &
    El sistema debe permitir al usuario recuperar su contraseña mediante un enlace de restablecimiento enviado a su correo electrónico registrado. \\

    \bottomrule
    \end{tabularx}
    \caption{Requisitos Funcionales Compartidos (RF01–RF03)}
    \end{table}

    \begin{table}[H]
    \centering
    \renewcommand{\arraystretch}{1.2}
    \begin{tabularx}{\textwidth}{@{} l l X @{}}
    \toprule
    \textbf{Código} & \textbf{Nombre} & \textbf{Descripción} \\
    \midrule
    RF04 & Edición de perfil &
    El usuario podrá modificar su información personal, como nombre, correo electrónico o fotografía de perfil, desde la aplicación o la web. \\


    RF05 & Eliminación de cuenta &
    El sistema debe permitir al usuario eliminar permanentemente su cuenta y los datos asociados, solicitando confirmación previa antes de proceder. \\

    RF06 & Registro de ingresos &
    El usuario podrá registrar manualmente sus ingresos indicando el monto, la fecha y una descripción opcional, tanto desde la aplicación móvil como desde la plataforma web. \\

    RF07 & Registro de egresos &
    El usuario podrá registrar egresos de forma manual o automática, especificando monto, fecha, categoría y descripción. \\

    RF08 & Registro de egresos manualmente &
    El sistema permitirá al usuario ingresar egresos de manera manual cuando no disponga de un ticket digital o imagen para procesar automáticamente. \\

    RF09 & Registro de ticket manualmente &
    El usuario podrá registrar un ticket introduciendo manualmente los datos del comprobante, como fecha, monto y establecimiento. \\

    RF10 & Configuración de presupuesto &
    El sistema debe permitir al usuario definir presupuestos por categoría o periodo, estableciendo límites de gasto que serán monitoreados automáticamente. \\

    RF11 & Visualización de historial de gastos &
    El usuario podrá consultar el historial de gastos registrados en ambos entornos, filtrando por fecha, categoría o tipo de transacción. \\

    RF12 & Visualización de gráficas o estadísticas &
    El sistema debe generar gráficas y estadísticas interactivas que representen el comportamiento financiero del usuario de manera clara y comprensible. \\

    \bottomrule
    \end{tabularx}
    \caption{Requisitos Funcionales Compartidos (RF04–RF12)}
    \end{table}

    \begin{table}[H]
    \centering
    \renewcommand{\arraystretch}{1.2}
    \begin{tabularx}{\textwidth}{@{} l l X @{}}
    \toprule
    \textbf{Código} & \textbf{Nombre} & \textbf{Descripción} \\
    \midrule
    RF13 & Generación de reportes en PDF &
    La plataforma web y la aplicación móvil deberán permitir la generación de reportes financieros en formato PDF, con opción de descarga o visualización en pantalla. \\

    RF14 & Configuración de metas de ahorro &
    El usuario podrá establecer metas de ahorro personalizadas, definir plazos y montos objetivo, y consultar su progreso mediante indicadores visuales. \\
    \bottomrule
    \end{tabularx}
    \caption{Requisitos Funcionales Compartidos (RF03–RF14)}
    \end{table}

    \item \textbf{Requerimientos/Casos de Uso Exclusivos de la App Móvil} 
    
    \begin{table}[H]
    \centering
    \renewcommand{\arraystretch}{1.2}
    \begin{tabularx}{\textwidth}{@{} l l X @{}}
    \toprule
    \textbf{Código} & \textbf{Nombre} & \textbf{Descripción} \\
    \midrule
    RF15 & Cambio de tema de la aplicación &
    El sistema debe permitir al usuario cambiar el tema visual de la aplicación (modo claro u oscuro) para mejorar la experiencia de uso. \\

    RF16 & Captura de foto del ticket &
    La aplicación móvil debe permitir al usuario tomar una fotografía del ticket de compra utilizando la cámara del dispositivo para su posterior procesamiento. \\

    RF17 & Captura múltiple de tickets &
    El sistema debe permitir tomar varias fotografías del mismo ticket en caso de que la información no sea visible en una sola imagen. \\

    RF18 & Confirmación de datos escaneados &
    El sistema debe mostrar los datos extraídos por el OCR para que el usuario los revise y confirme antes de su almacenamiento definitivo. \\

    RF19 & Etiquetado personalizado de tickets &
    El usuario podrá agregar etiquetas personalizadas a los tickets para clasificar los gastos según sus propios criterios o proyectos personales. \\

    RF20 & Corrección de datos escaneados &
    El sistema debe permitir al usuario editar manualmente los datos que fueron mal reconocidos durante el proceso de OCR. \\

    \bottomrule
    \end{tabularx}
    \caption{Requisitos Funcionales App Móvil (RF15–RF22)}
    \end{table}

    \begin{table}[H]
    \centering
    \renewcommand{\arraystretch}{1.2}
    \begin{tabularx}{\textwidth}{@{} l l X @{}}
    \toprule
    \textbf{Código} & \textbf{Nombre} & \textbf{Descripción} \\
    \midrule
    
    RF21 & Visualización de historial de notificaciones &
    El usuario podrá consultar un historial de notificaciones dentro de la aplicación móvil, incluyendo alertas de presupuesto, deudas y movimientos recientes. \\

    RF22 & Visualización de estadísticas por categoría &
    El sistema debe permitir al usuario visualizar las estadísticas de gasto organizadas por categoría y periodo, mostrando porcentajes o montos totales. \\

    RF23 & Modificación de categorías de gasto &
    El usuario podrá crear, editar o eliminar categorías de gasto personalizadas para una organización más detallada de sus finanzas. \\

    RF24 & Establecimiento de alertas de deudas &
    El sistema debe permitir configurar alertas automáticas para recordar al usuario fechas de vencimiento o pago de sus deudas registradas. \\

    RF25 & Configuración de deudas &
    El usuario podrá registrar nuevas deudas, especificando monto, acreedor, fecha de vencimiento y frecuencia de pago. \\

    RF26 & Registro de deudas &
    El sistema debe almacenar la información de las deudas registradas por el usuario y permitir su actualización o eliminación posterior. \\

    RF27 & Planificación de pago de deudas &
    El sistema debe generar un plan de pago de deudas que muestre fechas, montos parciales y estado de cada obligación financiera. \\

    RF28 & Planificación de presupuestos personales &
    El sistema debe permitir al usuario definir presupuestos personalizados por categoría o tipo de gasto para un mejor control financiero. \\

    \bottomrule
    \end{tabularx}
    \caption{Requisitos Funcionales App Móvil (RF21–RF28)}
    \end{table}

    \begin{table}[H]
    \centering
    \renewcommand{\arraystretch}{1.2}
    \begin{tabularx}{\textwidth}{@{} l l X @{}}
    \toprule
    \textbf{Código} & \textbf{Nombre} & \textbf{Descripción} \\
    \midrule
    RF29 & Consulta de tickets &
    El usuario podrá consultar los tickets previamente registrados, filtrando por fecha, categoría o monto. \\

    RF30 & Visualización de gastos &
    El sistema debe mostrar al usuario un resumen de sus gastos organizados por día, semana y mes, junto con las categorías más frecuentes. \\

    RF31 & Eliminación de tickets &
    El sistema debe permitir al usuario eliminar tickets registrados de forma manual o automática, previa confirmación para evitar pérdidas accidentales de información. \\

    \bottomrule
    \end{tabularx}
    \caption{Requisitos Funcionales App Móvil (RF29–RF31)}
    \end{table}

    \item \textbf{Requerimientos/Casos de Uso Exclusivos de la Plataforma Web} 
    
    \begin{table}[H]
    \centering
    \renewcommand{\arraystretch}{1.2}
    \begin{tabularx}{\textwidth}{@{} l l X @{}}
    \toprule
    \textbf{Código} & \textbf{Nombre} & \textbf{Descripción} \\
    \midrule
RF32 & Consulta de consejos personalizados &
La plataforma web debe mostrar consejos financieros personalizados basados en los hábitos de consumo del usuario, apoyándose en el análisis de sus gastos registrados. \\

RF33 & Búsqueda y filtrado de tickets &
El sistema debe permitir al usuario buscar y filtrar tickets según criterios como fecha, categoría, monto o palabras clave dentro de las descripciones. \\

RF34 & Generación de reportes en Excel &
El sistema debe permitir generar y descargar reportes financieros en formato Excel, con hojas separadas por categorías o periodos. \\

RF35 & Compartir reportes por correo electrónico &
El usuario podrá compartir los reportes generados directamente por correo electrónico desde la plataforma web, sin necesidad de descargarlos localmente. \\

RF36 & Comparación de gastos entre meses &
El sistema debe generar comparativas gráficas y tabulares de los gastos registrados en distintos meses, mostrando incrementos o reducciones porcentuales. \\

    \bottomrule
    \end{tabularx}
    \caption{Requisitos Funcionales Plataforma Web (RF32–RF36)}
    \end{table}

        \begin{table}[H]
    \centering
    \renewcommand{\arraystretch}{1.2}
    \begin{tabularx}{\textwidth}{@{} l l X @{}}
    \toprule
    \textbf{Código} & \textbf{Nombre} & \textbf{Descripción} \\
    \midrule
RF37 & Consulta de balance vs egresos &
El sistema debe mostrar un balance general que compare los ingresos frente a los egresos, calculando el resultado neto del periodo seleccionado. \\

RF38 & Visualización de estadísticas avanzadas &
La plataforma debe ofrecer un panel con estadísticas detalladas del comportamiento financiero del usuario, incluyendo tendencias, promedios y porcentajes de gasto por categoría. \\

RF39 & Carga de ticket digital &
El sistema debe permitir la carga de tickets en formato digital (PDF o imagen) desde el explorador web, procesándolos automáticamente con el módulo OCR. \\

RF40 & Configuración de metas de ahorro desde la web &
El usuario podrá definir y modificar metas de ahorro directamente desde la plataforma web, sincronizándose con los datos de la aplicación móvil. \\
    \bottomrule
    \end{tabularx}
    \caption{Requisitos Funcionales Plataforma Web (RF37–RF40)}
    \end{table}


\end{itemize}

