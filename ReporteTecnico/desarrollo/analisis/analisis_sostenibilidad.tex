\section{Análisis de sostenibilidad}
El proyecto se desarrolló con un enfoque académico y de investigación aplicada, orientado a la creación de un prototipo funcional de un sistema de gestión de finanzas personales.
Dado su carácter experimental, la sostenibilidad del sistema se centra en su mantenimiento técnico y académico, más que en su operación económica a largo plazo.

El sistema está diseñado bajo una arquitectura modular y escalable, que permite incorporar nuevas funcionalidades sin modificar la estructura base.
Además, se emplearon tecnologías gratuitas y de código abierto, tales como:
\begin{itemize}
	\item Next.js y React — plataforma web.
	\item Flutter — aplicación móvil.
	\item MySQL — sistema gestor de base de datos.
	\item Python y Tesseract (OCR) — procesamiento de los tickets.
\end{itemize}

Estas herramientas garantizan que el sistema pueda mantenerse y evolucionar sin requerir licencias ni infraestructura de alto costo.
Estas herramientas garantizan que el sistema pueda mantenerse y evolucionar sin requerir licencias ni infraestructura de alto costo.
El despliegue del prototipo se realiza en entornos locales, simulando la arquitectura cliente-servidor con servidores de desarrollo y una base de datos local.

Desde la perspectiva \textbf{operativa}, la modularidad y documentación del código permiten que futuros desarrolladores puedan continuar su evolución sin depender de componentes cerrados.  
La adopción de estándares abiertos facilita la interoperabilidad con otros sistemas y la integración de APIs externas para futuras versiones.

Finalmente, en el ámbito \textbf{social y académico}, el proyecto contribuye al fomento de la educación tecnológica y la cultura de la gestión financiera responsable, ofreciendo una herramienta potencialmente accesible para estudiantes, profesionales y público general.  
Esto genera un impacto positivo al promover el uso consciente de la tecnología para la toma de decisiones financieras informadas.