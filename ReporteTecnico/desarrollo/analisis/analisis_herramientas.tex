\section{Análisis de herramientas a utilizar}
Hemos realizado un análisis de las herramientas que se investigaron para poder dar una solución a la problemática que tenemos,
por lo que a continuación se presentan los resultados de dichas tecnologías.

\subsection{Análisis de los sistemas Operativos}

\begin{itemize}
    \item Windows
    
    Windows es un sistema operativo desarrollado por Microsoft. Es uno de los sistemas operativos más utilizados en el mundo, especialmente en computadoras personales y 
    de oficina. Ofrece una interfaz gráfica de usuario (GUI) que facilita la interacción con el sistema y una amplia compatibilidad con software y hardware. \cite{WindowsLenovo}
    La mejor parte de Windows es la amplia gama de opciones de software y hardware disponibles. Puedes elegir entre gama baja, media o alta. No solo eso, también tiene muchas 
    opciones para dispositivos periféricos. Además, Windows es conocido por su compatibilidad con una amplia variedad de aplicaciones, lo que lo convierte en una opción 
    popular para usuarios domésticos y profesionales. \cite{WindowsVSMac}

    \item Linux
    
    Linux es un sistema operativo de código abierto basado en Unix. Es conocido por su estabilidad, seguridad y flexibilidad. Linux es ampliamente utilizado en servidores,
    supercomputadoras y dispositivos embebidos. Ofrece una amplia variedad de distribuciones (como Ubuntu, Fedora y Debian) que se adaptan a diferentes necesidades y 
    preferencias de los usuarios. \cite{LinuxUbuntu}
    Una de las principales ventajas de Linux es su naturaleza de código abierto, lo que permite a los usuarios personalizar y modificar el sistema según sus necesidades. 
    Además, Linux es conocido por su seguridad robusta y su capacidad para manejar tareas complejas de manera eficiente. \cite{LinuxUbuntu}
    

    \item macOS
    
    macOS es el sistema operativo desarrollado por Apple para sus computadoras. Se destaca por su diseño elegante y su enfoque en la experiencia del usuario. \cite{MacOS}
    Al igual que Windows, macOS ofrece una amplia gama de aplicaciones y es conocido por su estabilidad y seguridad. Sin embargo, su compatibilidad con hardware es 
    más limitada, ya que está diseñado para funcionar exclusivamente en dispositivos Apple. Una de las desventajas de macOS, ademas de su costo, es que la mayoria de
    aplicaciones y herramientas de desarrollo son de pago, lo que puede ser una barrera para algunos desarrolladores. \cite{WindowsVSMac}

\end{itemize}

Para el desarrollo del proyecto elegimos Windows como sistema operativo principal, ya que es el que utilizamos en nuestros equipos de trabajo por lo cual ya 
estamos acostumbrados a este sistema y nos ofrece buena compatibilidad con las herramientas de desarrollo. Además, nos facilita la instalación de Visual Studio Code y 
las pruebas locales del proyecto sin complicaciones adicionales.

\subsection{Análisis de los Editores/IDEs}

\begin{itemize}
    \item Visual Studio Code
    
    Visual Studio Code (VS Code) es un editor de código fuente desarrollado por Microsoft. Es conocido por su ligereza, velocidad y amplia gama de extensiones que
    permiten personalizar el entorno de desarrollo según las necesidades del programador. VS Code es compatible con múltiples lenguajes de programación y ofrece 
    características como resaltado de sintaxis, autocompletado, depuración integrada y control de versiones. \cite{VSCode} La ventaja principal de VS Code es su 
    flexibilidad y facilidad de uso, lo que lo convierte en una opción popular entre desarrolladores de todos los niveles.

    \item Android Studio

    Android Studio es el entorno de desarrollo integrado (IDE) oficial para el desarrollo de aplicaciones Android. Basado en IntelliJ IDEA, ofrece herramientas potentes
    para la creación, depuración y optimización de aplicaciones Android. \cite{AndroidStudio} Incluye un emulador de dispositivos, 
    un editor de interfaz gráfica y soporte para el lenguaje Kotlin, entre otras características. Una desventaja de Android Studio es que puede ser pesado y 
    consumir muchos recursos del sistema, lo que puede afectar el rendimiento en computadoras con hardware limitado.

    \item JetBrains

    JetBrains es una empresa conocida por sus herramientas de desarrollo, incluyendo IDEs como IntelliJ IDEA, PyCharm y WebStorm. 
    Estas herramientas son altamente valoradas por su inteligencia de código, refactorización y soporte para múltiples lenguajes de programación. \cite{JetBrains}
    Una desventaja de JetBrains es que muchas de sus herramientas son de pago, lo que puede ser una barrera para algunos desarrolladores.

\end{itemize}

Como entorno de desarrollo se utilizara Visual Studio Code, por ser ligero, rápido y tambien cuenta con una gran cantidad de extensiones que nos facilitaran la 
programación en los diferentes lenguajes que utilizaremos. También nos permite integrar control de versiones, depuración y el trabajo colaborativo sin necesidad de 
herramientas adicionales, ya no tenemos que descargar GitHub o algún otro software.

\subsection{Análisis de los Lenguajes de programación}

En cuanto a los lenguajes de programación, se trabajara con tres principales:

\begin{itemize}
    \item TypeScript
    
    TypeScript es un superconjunto de JavaScript que añade tipado estático y características avanzadas al lenguaje. El código TypeScript se convierte a JavaScript, que se 
    ejecuta en cualquier lugar donde se ejecute JavaScript: En un navegador, en Node.js, Deno, Bun y en tus aplicaciones. \cite{TypeScript}

    Este lenguaje se usará para el desarrollo del frontend con Next.js por su compatibilidad con React, su tipado estático y su integración natural con VS Code.
    
    \item Dart

    Dart es un lenguaje optimizado para el cliente para desarrollar aplicaciones rápidas en cualquier plataforma. Su objetivo es ofrecer el lenguaje de programación más 
    productivo para desarrollo multiplataforma, combinado con un plataforma de tiempo de ejecución flexible para marcos de aplicaciones. \cite{Dart} El lenguaje Dart es 
    seguro en cuanto a tipos; utiliza verificación de tipos estáticos para garantizarlo el valor de una variable siempre coincide con el tipo estático de la variable.

    Dart, se utilizará para el desarrollo móvil con Flutter, debido a su capacidad de compilar de forma nativa para Android.
    
    \item Python
    
    Python es un lenguaje de programación ampliamente utilizado en las aplicaciones web, el desarrollo de software, la ciencia de datos y el machine learning (ML). 
    Los desarrolladores utilizan Python porque es eficiente y fácil de aprender, además de que se puede ejecutar en muchas plataformas diferentes. \cite{Python}
    Los beneficios de usar Python incluyen su sintaxis clara y concisa, una gran comunidad de soporte y una amplia variedad de bibliotecas y frameworks que facilitan el 
    desarrollo. \cite{Python}

    Python, se usará para los procesos de análisis y extracción de datos, ya que cuenta con librerías muy completas para el manejo de archivos, OCR y automatización.

\end{itemize}

\subsection{Análisis de las Bases de datos}
\begin{itemize}
    \item MySQL

    MySQL es un sistema de gestión de bases de datos relacional (RDBMS) de código abierto. Es conocido por su rapidez y fiabilidad, y es una de las bases de datos más 
    populares del mundo. \cite{MySQL} MySQL es ampliamente utilizado en aplicaciones web y es compatible con una variedad de lenguajes de programación y plataformas.
    Una desventaja de MySQL es que, aunque es de código abierto, algunas características avanzadas solo están disponibles en la versión comercial. Por otra parte, 
    su arquitectura puede no ser tan flexible como la de algunas bases de datos NoSQL para ciertos tipos de aplicaciones.
    
    \item PostgreSQL
    
    PostgreSQL es un sistema de gestión de bases de datos relacional (RDBMS) de código abierto y avanzado. Es conocido por su robustez, escalabilidad y soporte para 
    características avanzadas como transacciones complejas, procedimientos almacenados y tipos de datos personalizados. \cite{PostgreSQL} PostgreSQL es ampliamente 
    utilizado en aplicaciones empresariales y es compatible con una variedad de lenguajes de programación y plataformas. Una desventaja de PostgreSQL es que puede ser más 
    complejo de configurar y administrar en comparación con otras bases de datos, lo que puede requerir un mayor conocimiento técnico por parte del administrador de la 
    base de datos.
    
    \item MongoDB
    MongoDB es una base de datos NoSQL orientada a documentos que almacena datos en formato BSON (una extensión binaria de JSON). 
    Es conocida por su flexibilidad, escalabilidad y facilidad de uso. \cite{MongoDB} MongoDB es ampliamente utilizado en aplicaciones web y móviles, especialmente 
    aquellas que requieren un manejo eficiente de grandes volúmenes de datos no estructurados.
    Una desventaja de MongoDB es que, al ser una base de datos NoSQL, puede no ser la mejor opción para aplicaciones que requieren transacciones complejas o integridad 
    referencial estricta,
    ya que carece de algunas de las características tradicionales de las bases de datos relacionales.
\end{itemize}

La base de datos seleccionada será MySQL, porque es fácil de administrar, tiene buen rendimiento para transacciones y es ampliamente soportada por la mayoría de 
servicios en la nube. Además, nos decidimos por esta ya que es de uso gratuito y contamos con experiencia trabajando en esta base.

\subsection{Análisis de Software para Extracción de Texto}

\begin{itemize}
    \item Tesseract OCR
    
    Tesseract OCR es un motor de reconocimiento óptico de caracteres (OCR) de código abierto desarrollado inicialmente por Hewlett-Packard y ahora mantenido por Google.
    Es conocido por su alta precisión en el reconocimiento de texto impreso y manuscrito en una variedad de idiomas. \cite{Tesseract} Tesseract es compatible con múltiples 
    plataformas y se puede integrar fácilmente en aplicaciones mediante su API. Una desventaja de Tesseract es que puede requerir una configuración y entrenamiento 
    adicionales para optimizar su rendimiento en casos específicos, como documentos con formatos complejos o baja calidad de imagen. \cite{Tesseract}

    \item Paddle OCR
    
    Paddle OCR es una herramienta de reconocimiento óptico de caracteres (OCR) desarrollada por Baidu. Es parte del ecosistema Paddle, una plataforma de aprendizaje
    profundo de código abierto. Paddle OCR es conocido por su alta precisión y velocidad en el reconocimiento de texto en imágenes y documentos. \cite{PaddleOCR}
    Ofrece soporte para múltiples idiomas y es capaz de manejar una variedad de formatos de texto, incluyendo texto impreso y manuscrito. Una desventaja de Paddle OCR es 
    que puede requerir una configuración y entrenamiento adicionales para optimizar su rendimiento en casos específicos. \cite{PaddleOCR}

\end{itemize}

Para la extracción de texto se optó por Tesseract OCR, por ser una herramienta gratuita, de código abierto y con buena precisión en documentos impresos o escaneados. 
Esto permitió realizar el reconocimiento de texto de los tickets sin depender de servicios externos.

\subsection{Análisis de Frameworks frontend web}

\begin{itemize}
    \item Next.js

    Next.js es un framework de React que permite la renderización del lado del servidor y la generación de sitios estáticos. Es conocido por su rendimiento y facilidad de uso, 
    lo que lo convierte en una opción popular para aplicaciones web modernas. Next.js ofrece características como enrutamiento automático, soporte para TypeScript y 
    una optimización de aplicaciones a otro nivel gracias a sus características automáticas integradas. Estas optimizaciones no sólo simplifican el trabajo del desarrollador, 
    sino que también garantizan que tu aplicación sea rápida, eficiente y lista para la puesta en producción desde el primer momento.\cite{NextJSOfficial}

    \item Remix
    
    Remix es un framework de React que se centra en la experiencia del desarrollador y el rendimiento de la aplicación. Ofrece una arquitectura basada en rutas y
    proporciona herramientas para manejar datos y estados de manera eficiente. Remix es conocido por su enfoque en la optimización del rendimiento y la facilidad de uso. 
    Una desventaja de Remix en contra de Next.js es que es un framework más nuevo y puede tener una comunidad y ecosistema de plugins más pequeños. \cite{Remix}

    \item SvelteKit

    SvelteKit es un framework para construir aplicaciones web utilizando Svelte. Ofrece una experiencia de desarrollo optimizada y un rendimiento excepcional, 
    gracias a su enfoque en la compilación en tiempo de construcción. SvelteKit permite la generación de sitios estáticos y la renderización del lado del servidor, 
    lo que lo convierte en una opción versátil para desarrolladores. \cite{SvelteKit}


\end{itemize}

En el frontend se eligió Next.js, por su integración directa con React, su rendimiento en renderizado del lado del servidor y su flexibilidad para generar 
páginas estáticas o dinámicas. Además de la experiencia previa que contamos con este framework.

\subsection{Análisis de Frameworks móviles}
\begin{itemize}
    \item Flutter
    
    Flutter es un framework de código abierto desarrollado por Google para construir aplicaciones nativas multiplataforma desde una única base de código. Utiliza el lenguaje
    de programación Dart y ofrece un rendimiento cercano al nativo, así como una amplia gama de widgets personalizables. Flutter es conocido por su rapidez en el desarrollo y 
    su capacidad para crear interfaces de usuario atractivas y fluidas. \cite{WhyFlutter}

    \item React Native

    React Native es un framework de código abierto desarrollado por Facebook que permite construir aplicaciones móviles utilizando JavaScript y React. Ofrece una experiencia 
    de desarrollo similar a la de React, lo que facilita la creación de interfaces de usuario nativas. React Native es conocido por su capacidad para compartir código entre 
    plataformas y su amplia comunidad de desarrolladores. La desventaja de React Native en comparación con Flutter es que puede tener un rendimiento ligeramente inferior en 
    ciertas situaciones, ya que utiliza un puente para comunicarse con los componentes nativos. \cite{WhyReactNative}

    \item Kotlin

    Kotlin es un lenguaje de programación moderno que se ejecuta en la máquina virtual de Java (JVM) y es totalmente interoperable con Java. Es conocido por su concisión, 
    seguridad y facilidad de uso, lo que lo convierte en una opción popular para el desarrollo de aplicaciones Android. Kotlin también ofrece características avanzadas como 
    la programación funcional y la extensión de funciones. Kotlin es mucho más difícil de aprender en comparación con otros lenguajes como Dart o React Native, 
    lo que puede ser un obstáculo para algunos desarrolladores. \cite{WhyFlutter}

\end{itemize}

Para la aplicación móvil se utilizará Flutter, ya que permite desarrollar para Android e iOS con una sola base de código y ofrece un rendimiento cercano al nativo.

\subsection{Análisis de Backend}

\begin{itemize}
    \item Node.js con Express
    
    Node.js es un entorno de ejecución de JavaScript del lado del servidor que permite construir aplicaciones escalables y de alto rendimiento. Express es un framework
    minimalista y flexible para Node.js que facilita la creación de aplicaciones web y APIs. Juntos, Node.js y Express ofrecen una solución potente para el desarrollo backend,
    con una amplia gama de módulos y una gran comunidad de desarrolladores. \cite{NodeExpress}

    \item Django
    
    Django es un framework web de alto nivel para Python que sigue el patrón de diseño Modelo-Vista-Controlador (MVC). Es conocido por su rapidez en el desarrollo,
    su seguridad y su enfoque en la reutilización de código. Django ofrece una amplia gama de características integradas, como un sistema de autenticación, un panel de 
    administración y soporte para bases de datos. \cite{Django} Una desventaja de Django es que puede ser más pesado y complejo en comparación con otros frameworks,
    lo que puede requerir un mayor conocimiento técnico por parte del desarrollador.

    \item Spring Boot
    
    Spring Boot es un framework para construir aplicaciones Java basadas en el ecosistema Spring. Ofrece una configuración automática y una amplia gama de herramientas
    para facilitar el desarrollo de aplicaciones web y microservicios. Spring Boot es conocido por su escalabilidad, seguridad y soporte para una variedad de bases de datos 
    y servicios. \cite{SpringBoot}. Debido a las tecnologías utilizadas en el frontend y la aplicación móvil, Spring Boot no es tan compatible como las otras opciones 
    mencionadas.
    

\end{itemize}

Finalmente, en el backend se implementará en Node.js con Express, por su alta velocidad en el manejo de peticiones concurrentes, su compatibilidad con TypeScript y 
la facilidad de crear APIs que se integran directamente con el frontend web y móvil.
