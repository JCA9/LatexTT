\renewcommand{\bibname}{Referencias} % <-- cambia "Bibliografía" a "Referencias"
\addcontentsline{toc}{chapter}{Referencias} % <-- Agrega la sección al índice

\begin{thebibliography}{99}

\bibitem{inegi2024} %1
\textsc{INEGI.} (2024). Encuesta Nacional sobre Salud Financiera (ENSAFI) 2023.
Recuperado de: \url{https://www.inegi.org.mx/contenidos/saladeprensa/boletines/2024/ENSAFI/ENSAFI.pdf}

\bibitem{statista2025} %2 
\textsc{Statista.} (2025). La situación financiera personal en el futuro, ¿fuente de preocupación?.
Recuperado de: \url{https://es.statista.com/grafico/27448/porcentaje-de-encuestados-que-estan-preocupados-por-su-futuro-financiero/}

\bibitem{adjust2023}    %3
\textsc{Adjust.} (2023). El uso de las aplicaciones de finanzas continúa creciendo en 2023.
Recuperado de: \url{https://www.adjust.com/es/blog/finance-app-usage/}

\bibitem{OECD_INFE2023} %4
\textsc{OECD.} (2023). OECD/INFE 2023 International Survey of Adult Financial Literacy. OECD Publishing. 
Recuperado de: \url{https://doi.org/10.1787/56003a32-en}

\bibitem{oecd2023} %6
\textsc{OECD.} (2023). Financial literacy and decision-making: Data visualization impact.
\url{https://www.oecd.org/finance/financial-education/}

\bibitem{banxico2023}  %7
\textsc{Banco de México.} (2023). Educación financiera y adherencia al presupuesto.
\url{https://www.banxico.org.mx/educacion-financiera/}

\bibitem{EncINEGI}  %8
\textsc{INEGI, ENDUTIH} (2023). Comunicado de prensa 372/24, 13-jun-2024. Encuesta Nacional sobre Disponibilidad y Uso de Tecnologías de la Información en los Hogares (ENDUTIH) 2023.
\url{https://www.inegi.org.mx/contenidos/saladeprensa/boletines/2024/ENDUTIH/ENDUTIH_23.pdf}

\bibitem{Athento} %9
\textsc{M. Aguirre and M. Aguirre} (2025). “Procesamiento de documentos: etapas y tecnologías esenciales,” Athento - Smart Digital Content Platform, Aug. 13.
\url{https://www.athento.com/es/procesamiento-de-documentos-etapas-y-tecnologias-esenciales/}

\bibitem{T. Portal} %10
\textsc{Portal TIC} (2025). “Digitalización de documentos: ¿qué es y en qué consiste el proceso?”
\url{https://www.ticportal.es/temas/sistema-gestion-documental/digitalizacion-de-documentos}

\bibitem{AWS} %11
\textsc{Servicios web de Amazon, Inc.} (n.d.). ¿Qué es el OCR? - Explicación del reconocimiento óptico de caracteres - AWS
\url{https://aws.amazon.com/es/what-is/ocr/}

\bibitem{IBM} %12
\textsc{C. Stryker and J. Holdsworth} (2025). “Procesamiento De Lenguaje Natural
\url{https://www.ibm.com/mx-es/think/topics/natural-language-processing}

\bibitem{AWS NLP} %13
\textsc{Amazon Web Services, Inc.} (n.d.). ¿Qué es el NLP? - Explicación del procesamiento de lenguaje natural - AWS
\url{https://aws.amazon.com/es/what-is/nlp/}

\bibitem{Latenode} %14
\textsc{Latenode} (2025). El mejor software de escaneo y OCR para dispositivos móviles en 2024
\url{https://latenode.com/es/blog/the-best-mobile-scanning-and-ocr-software-in-2024}

\bibitem{IBM OCR} %15
\textsc{IBM} (n.d.).  “Reconocimiento óptico de caracteres,” ¿Qué es el reconocimiento óptico de caracteres (OCR)? 
\url{https://www.ibm.com/mx-es/think/topics/optical-character-recognition}

\bibitem{Paseur} %16
\textsc{N. Gunnoo} (2024). Reconocimiento óptico de caracteres (OCR): guía completa
\url{https://parseur.com/es/blog/que-es-el-reconocimiento-optico-de-caracteres}

\bibitem{Paseur Extracción} %17
\textsc{N. Gunnoo} (2025). Extrae texto de un PDF en 2025
\url{https://parseur.com/es/caso-de-uso/extraer-texto-de-pdf}

\bibitem{Statistics Canada} %18
\textsc{Government of Canada, Statistics Canada} (2022). Document Intelligence: The art of PDF information extraction
\url{https://www.statcan.gc.ca/en/data-science/network/pdf-extraction}

\bibitem{Timbox} %19
\textsc{K. Madrid} (2023). Timbrar CFDI 4.0
\url{https://www.timbox.com.mx/timbrar-cfdi-4-0/}

\bibitem{BBVA Finanzas} %20
\textsc{BBVA MEXICO and BBVA} (2024). “¿Qué son las finanzas personales y su importancia?,” ¿Qué son las finanzas personales y su importancia?, Nov. 25, 2024.
\url{https://www.bbva.mx/educacion-financiera/banca-digital/cuenta-digital-que-son-finanzas-personales.html}

\bibitem{Coral Prous} %21
\textsc{HeadTeam Marketing} (2024). 33 conceptos clave de finanzas personales,
\url{https://coralprous.es/educacion-financiera/conceptos-de-finanzas-personales/}

\bibitem{CFPB_Budgeting} %22
\textsc{Consumer Financial Protection Bureau.} (2021). Budgeting: How to create a budget and stick with it. Recuperado de \url{https://www.consumerfinance.gov/about-us/blog/budgeting-how-to-create-a-budget-and-stick-with-it/}

\bibitem{Britannica_503020} %23
\textsc{Encyclopaedia Britannica.} (2024). What is the 50/30/20 rule?. Recuperado de \url{https://www.britannica.com/money/what-is-the-50-30-20-rule}

\bibitem{Investopedia_ZBB} %24
\textsc{Investopedia.} (2023). Zero-Based Budgeting (ZBB). Recuperado de \url{https://www.investopedia.com/terms/z/zbb.asp}

\bibitem{CFPB_AutomaticSaving} %25
\textsc{Consumer Financial Protection Bureau.} (2020). Set a goal, make a plan, and save automatically. Recuperado de \url{https://www.consumerfinance.gov/about-us/blog/set-a-goal-make-a-plan-and-save-automatically/}

\bibitem{CFPB_SpendingTracker} %26
\textsc{Consumer Financial Protection Bureau.} (2019). Track your spending with this easy tool. Recuperado de \url{https://www.consumerfinance.gov/about-us/blog/track-your-spending-with-this-easy-tool/}

\bibitem{CFPB_MoneyRules} %27
\textsc{Consumer Financial Protection Bureau.} (2022). Creating your own financial rules to live by. Recuperado de \url{https://www.consumerfinance.gov/about-us/blog/creating-your-own-financial-rules-to-live-by/}

\bibitem{La Vanguardia} %28
\textsc{N. Bourass} (2025). "Dave Ramsey, experto en finanzas, comparte su mejor método de ahorro: Si no le dices a tu dinero dónde ir, ¡te vas a preguntar dónde fue!"
\url{https://www.lavanguardia.com/dinero/20250305/10447218/dave-ramsey-experto-finanzas-comparte-mejor-metodo-ahorro-dices-dinero-donde-ir-preguntar-donde-fue-gvm.html}

\bibitem{IBM Banca} %30
\textsc{IBM} (2024). “Transformación digital de la banca,” ¿Qué es la transformación digital de la banca y los servicios financieros?
\url{https://www.ibm.com/mx-es/think/topics/digital-transformation-banking}

% MARCO TEORICO - TECNOLOGIAS DE DESARROLLO

\bibitem{FowlerArchStyles}
\textsc{Fowler, M.} (2023). Software Architecture Guide. Recuperado de \url{https://martinfowler.com/architecture/}

\bibitem{ClienteServidor}
\textsc{Oscar  Blancarte} (n.d.). Arquitectura Cliente-Servidor.
\url{https://reactiveprogramming.io/blog/es/estilos-arquitectonicos/cliente-servidor}

\bibitem{NextJSOfficial}
\textsc{Vercel.} (2025). Next.js Documentation: Introduction. Recuperado de \url{https://nextjs.org/docs}

\bibitem{WhyFlutter}
\textsc{Ilia Lotarev} (2025). Flutter vs Kotlin: ¿Cuál elegir para tu proyecto?
\url{https://adapty.io/blog/flutter-vs-kotlin/}

\bibitem{WhyMySQL}
\textsc{Oracle.} (2024). MySQL: The world's most popular open source database. Recuperado de \url{https://www.mysql.com/why-mysql/}

\bibitem{WhyRest}
\textsc{Codecademy} (n.d.). What is REST API (RESTful API)? Explained
\url{https://www.codecademy.com/article/what-is-rest-api}

% MARCO TEORICO - CATEGORIZACION
\bibitem{ProcessMakerExpense}
\textsc{ProcessMaker Inc.} (2025). Walkthrough of ProcessMaker's free expense report tool. Recuperado de \url{https://www.processmaker.com/es/blog/walkthrough-of-processmakers-free-expense-report-tool/}

\bibitem{CategorizacionGastos}
\textsc{Navan} (2023). “¿Qué es la Automatización de Procesos Digitales (DPA)? | Procesador,” Creador de procesos
\url{https://www.processmaker.com/es/blog/what-is-digital-process-automation-dpa/}

\bibitem{WhyRegex}
\textsc{Lenovo} (n.d.). ¿Qué es Regex y cómo usarlo? | Lenovo México.
\url{https://www.lenovo.com/mx/es/glosario/expresion-regular-regex/}

% MARCO TEORICO - METODOLOGIAS DE DESARROLLO
\bibitem{Metodologias}
\textsc{R. S. Pressman and B. R. Maxim} (2020). Software Engineering: A Practitioner’s Approach, 9th ed. New York, NY, USA: McGraw-Hill.
\url{}

\bibitem{ManifestoAgil} 
\textsc{K. Beck et al.} (2001). Manifesto for Agile Software Development
\url{https://agilemanifesto.org}

\bibitem{WhyXP}
\textsc{I. R. Cano} (2018). “¿Qué ventajas aporta Extreme Programming? - Viewnext.com
\url{https://www.viewnext.com/ventajas-extreme-programming/}

\bibitem{WhyKanban}
\textsc{J. Martins} (2025). ¿Qué es la metodología Kanban y cómo funciona? • Asana
\url{https://asana.com/es/resources/what-is-kanban}

% ANÁLISIS - REQUISITOS

\bibitem{RequisitosFuncionales}
\textsc{A. Jain and A. Jain} (2025). “¿Qué son los requisitos funcionales? Ejemplos y plantillas,” Visure Solutions.
\url{https://visuresolutions.com/es/alm-guide/functional-requirements/}

\bibitem{RequisitosNoFuncionales}
\textsc{A. Jain and A. Jain} (2025). “¿Qué son los requisitos no funcionales? Tipos, ejemplos y enfoques,” Visure Solutions.
\url{https://visuresolutions.com/es/alm-guide/non-functional-requirements/}

\bibitem{ReglasNegocio}
\textsc{Sydle} (2024). “¿Qué son las reglas de negocio y qué importancia tienen? Conoce algunos ejemplos,” Blog SYDLE.
\url{https://www.sydle.com/es/blog/reglas-de-negocio-5f6333be1e43744c69d995e0}

% ANALISIS HERRAMIENTAS
\bibitem{WindowsLenovo}
\textsc{Lenovo México.} (n.d.). ¿Qué es Microsoft Windows? 
\url{https://www.lenovo.com/mx/es/glosario/que-es-microsoft-windows/}

\bibitem{WindowsVSMac}
\textsc{EDteam} (2023). Windows vs. Mac: ¿Cuál es mejor para programadores?,
\url{https://ed.team/blog/windows-vs-mac-cual-es-mejor-para-programadores}

\bibitem{LinuxUbuntu}
\textsc{RedHat} (n.d.). ¿Qué es Linux? - Definición y ventajas de Linux
\url{https://www.redhat.com/es/topics/linux/what-is-linux}

\bibitem{MacOS}
\textsc{G. Caina} (2023). “MacOS: ¿Qué es y cómo funciona?,” Mac Center Colombia
\url{https://mac-center.com/blogs/mac-center/macos-que-es-y-como-funciona}

\bibitem{VSCode}
\textsc{Fernán García de Zúñiga} (2025). ¿Qué es Visual Studio Code y cuáles son sus ventajas?
\url{https://www.arsys.es/blog/que-es-visual-studio-code-y-cuales-son-sus-ventajas}

\bibitem{AndroidStudio}
\textsc{Google Developers} (n.d.). ¿Qué es Android Studio? - Android Developers
\url{https://developer.android.com/studio/intro?hl=es-419}

\bibitem{JetBrains}
\textsc{Bestnet Softwares S.L.U} (2024). JetBrains - Bestnet Softwares Herramientas de Desarrollo de Software
\url{https://www.bestnetsoft.com/jetbrains/}

\bibitem{TypeScript}
\textsc{TypeScript} (n.d.). JavaScript with syntax for types.
\url{https://www.typescriptlang.org/}

\bibitem{Dart}
\textsc{Dart} (n.d.). Dart overview.
\url{https://dart.dev/overview}

\bibitem{Python}
\textsc{Amazon Web Services, Inc} (n.d.). ¿Qué es Python? - Explicación del lenguaje Python - AWS
\url{https://aws.amazon.com/es/what-is/python/}

\bibitem{MySQL}
\textsc{Jeffrey Erickson - Oracle} (2024). MySQL: qué es y cómo se usa
\url{https://www.oracle.com/latam/mysql/what-is-mysql/}

\bibitem{PostgreSQL}
\textsc{PostgreSQL Global Development Group} (n.d.). ¿Qué es PostgreSQL?
\url{https://www.postgresql.org/about/}

\bibitem{MongoDB}
\textsc{MongoDB, Inc.} (n.d.). ¿Qué es MongoDB? - Definición y características de MongoDB
\url{https://www.mongodb.com/es/what-is-mongodb}

\bibitem{Tesseract}
\textsc{N. Bispo and N. Bispo} (2025). Tesseract OCR Guide: Exploring Capabilities and Performance
\url{https://unstract.com/blog/guide-to-optical-character-recognition-with-tesseract-ocr/}

\bibitem{PaddleOCR}
\textsc{PaddlePaddle} (n.d.). PaddleOCR: An Easy-to-Use OCR Tool Based on PaddlePaddle
\url{https://github.com/PaddlePaddle/PaddleOCR}

\bibitem{AzureDocumentI}
\textsc{Microsoft Corporation} (n.d.). ¿Qué es la extracción de información de documentos? - Azure Applied AI Services
\url{https://azure.microsoft.com/es-es/products/ai-services/ai-document-intelligence}

\bibitem{Amazon Web Services}
\textsc{Amazon Web Services, Inc.} (n.d.). what-is-aws
\url{https://aws.amazon.com/es/what-is-aws/}

\bibitem{MicrosoftAzure}
\textsc{Microsoft Corporation} (n.d.). ¿Qué es Microsoft Azure? - Definición y servicios de Azure
\url{https://azure.microsoft.com/es-es/overview/what-is-azure/}

\bibitem{GoogleCloud}
\textsc{Google Cloud} (n.d.). ¿Qué es Google Cloud? - Definición y servicios de Google Cloud
\url{https://cloud.google.com/what-is-cloud-computing}

\bibitem{Remix}
\textsc{Remix Software Inc.} (n.d.). Remix - The full stack web framework that scales from static sites to dynamic, data-driven apps.
\url{https://remix.run/}

\bibitem{SvelteKit}
\textsc{SvelteKit} (n.d.). SvelteKit - The framework for building web applications with Svelte.
\url{https://kit.svelte.dev/}

\bibitem{WhyReactNative}
\textsc{I. Sastre} (2023). Comparativa React Native Vs Flutter, ¿Cuál es mejor? - Batura Mobile
\url{https://baturamobile.com/blog/comparativa-react-native-vs-flutter/}

\bibitem{NodeExpress}
\textsc{Mozilla Org} (n.d.). Introducción a Express/Node - Aprende desarrollo web | MDN
\url{https://developer.mozilla.org/es/docs/Learn_web_development/Extensions/Server-side/Express_Nodejs/Introduction}

\bibitem{Django}
\textsc{Django Software Foundation} (n.d.). ¿Qué es Django? - Definición y características de Django
\url{https://www.djangoproject.com/}

\bibitem{SpringBoot}
\textsc{Pivotal Software, Inc.} (n.d.). ¿Qué es Spring Boot? - Definición y características de Spring Boot
\url{https://spring.io/projects/spring-boot}

% ANALISIS DE Riesgos
\bibitem{AnalisisRiesgos}
\textsc{T. Asana} (2025). Cómo realizar un análisis de riesgos y ejemplos
\url{https://asana.com/es/resources/project-risks}

% ARQUITECTURA DEL SISTEMA
\bibitem{Arquitectura}
\textsc{Tan Dang} (2024). What Is System Architecture? A Simple Explanation, Orient Software Blog
\url{https://www.orientsoftware.com/blog/system-architecture/}

\bibitem{ArquitecturaCapas}
\textsc{Frank Buschmann, Kevlin Henney} (2008). Pattern-Oriented Software Architecture
\url{https://www.researchgate.net/publication/228722744_Pattern-oriented_software_architecture}

\bibitem{ArquitecturaPresenctacion}
\textsc{Microsoft Corporation} (n.d.). N-tier architecture - Azure Architecture Center
\url{https://learn.microsoft.com/en-us/azure/architecture/guide/architecture-styles/n-tier}

\bibitem{ArquitecturaNegocio}
\textsc{M. Fowler} (2003). Patterns of Enterprise Application Architecture

% ARQUITECTURA BD
\bibitem{ModeloER}
\textsc{T. Connolly and C. Begg} (2015). Database Systems: A Practical Approach to Design, Implementation, and Management, 6th ed. Pearson.

\bibitem{TablasYRelaciones}
\textsc{R. Elmasri and S. B. Navathe} (2016). Fundamentals of Database Systems, 7th ed. Pearson.

\bibitem{Normalización}
\textsc{A. Silberschatz, H. F. Korth, and S. Sudarshan} (2020). Database System Concepts, 7th ed. McGraw-Hill.

% MODELADOS UML
\bibitem{DIagramaSecuencia}
\textsc{Lucidchart} (2025). Diagrama de secuencia UML
\url{ https://www.lucidchart.com/pages/es/diagrama-de-secuencia}

% guidelines
\bibitem{flutter2024}
\textsc{Flutter.} (2024). Flutter code style guide. Flutter documentation. Google.
Recuperado de: \url{https://dart.dev/guides/language/effective-dart/style}

\bibitem{react2024}
\textsc{React.} (2024). Accessibility – React documentation. Meta Platforms, Inc.
Recuperado de: \url{https://legacy.reactjs.org/docs/accessibility.html}

\bibitem{infinum2024}
\textsc{Infinum.} (2024). React guidelines and best practices. Infinum.
Recuperado de: \url{https://infinum.com/handbook/frontend/react/react-guidelines-and-best-practices}

\end{thebibliography}