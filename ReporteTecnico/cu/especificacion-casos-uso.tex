\section{Especificación de Casos de Uso}

\subsection{CU-01: Iniciar Sesión}

\noindent\textbf{Actor:} Administrador Financiero

\noindent\textbf{Descripción:} El administrador financiero accede al sistema utilizando sus credenciales de autenticación (correo/contraseña o cuenta de Google).
\noindent\textbf{Precondiciones:}
\begin{itemize}
    \item El sistema debe estar disponible
    \item El usuario debe tener una cuenta registrada en el sistema
    \item Para autenticación con Google, el usuario debe tener una cuenta de Google válida
\end{itemize}

\noindent\textbf{Postcondiciones:}
\begin{itemize}
    \item El usuario queda autenticado en el sistema
    \item Se establece una sesión activa
    \item El usuario es redirigido al dashboard principal
\end{itemize}
\noindent\textbf{Flujo Principal:}
\begin{enumerate}
    \item El usuario accede a la pantalla de inicio de sesión
    \item El sistema presenta las opciones de autenticación:
    \begin{itemize}
        \item Correo electrónico y contraseña
        \item Iniciar sesión con Google
    \end{itemize}
    \item El usuario selecciona el método de autenticación deseado
    \item Si selecciona correo/contraseña:
    \begin{enumerate}
        \item Ingresa su correo electrónico
        \item Ingresa su contraseña
        \item Presiona el botón "Iniciar Sesión"
    \end{enumerate}
    \item Si selecciona Google:
    \begin{enumerate}
        \item Presiona el botón "Iniciar con Google"
        \item Es redirigido a la autenticación de Google
        \item Autoriza el acceso a la aplicación
    \end{enumerate}
    \item El sistema valida las credenciales
    \item El sistema establece la sesión del usuario
    \item El usuario es redirigido al dashboard principal
\end{enumerate}
\noindent\textbf{Flujos Alternativos:}

\textbf{FA-01: Credenciales inválidas}
\begin{enumerate}
    \item En el paso 6, si las credenciales son incorrectas
    \item El sistema muestra un mensaje de error
    \item El usuario regresa al paso 3
\end{enumerate}

\textbf{FA-02: Error de autenticación con Google}
\begin{enumerate}
    \item En el paso 5c, si la autenticación con Google falla
    \item El sistema muestra un mensaje de error
    \item El usuario regresa al paso 2
\end{enumerate}

\subsection{CU-02: Registrar Ingreso}

\noindent\textbf{Actor:} Administrador Financiero

\noindent\textbf{Descripción:} El administrador financiero registra un ingreso de dinero en el sistema, especificando el monto y agregando notas descriptivas.
\noindent\textbf{Precondiciones:}
\begin{itemize}
    \item El usuario debe estar autenticado en el sistema
    \item Debe existir un ingreso real de dinero para registrar
\end{itemize}

\noindent\textbf{Postcondiciones:}
\begin{itemize}
    \item El ingreso queda registrado en el sistema
    \item Se actualiza el balance general del usuario
    \item El registro es visible en el historial de transacciones
\end{itemize}
\noindent\textbf{Flujo Principal:}
\begin{enumerate}
    \item El usuario accede a la opción "Registrar Ingreso" desde el dashboard
    \item El sistema presenta el formulario de registro de ingreso
    \item El usuario ingresa la información requerida:
    \begin{itemize}
        \item Monto del ingreso
        \item Fecha del ingreso (por defecto la fecha actual)
        \item Fuente del ingreso (opcional)
        \item Notas descriptivas (opcional)
    \end{itemize}
    \item El usuario confirma la información ingresada
    \item El sistema valida los datos ingresados
    \item El sistema registra el ingreso en la base de datos
    \item El sistema actualiza el balance del usuario
    \item El sistema muestra un mensaje de confirmación
    \item El usuario regresa al dashboard actualizado
\end{enumerate}
\noindent\textbf{Flujos Alternativos:}

\textbf{FA-01: Datos inválidos}
\begin{enumerate}
    \item En el paso 5, si los datos no son válidos (ej: monto negativo)
    \item El sistema muestra mensajes de error específicos
    \item El usuario corrige los datos y regresa al paso 4
\end{enumerate}

\subsection{CU-03: Registrar Egreso}

\noindent\textbf{Actor:} Administrador Financiero

\noindent\textbf{Descripción:} El administrador financiero inicia el proceso de registro de un egreso, seleccionando entre las opciones disponibles: registro manual o captura de ticket.

\noindent\textbf{Precondiciones:}
\begin{itemize}
    \item El usuario debe estar autenticado en el sistema
    \item Debe existir un gasto real para registrar
\end{itemize}

\noindent\textbf{Postcondiciones:}
\begin{itemize}
    \item Se inicia el flujo de registro correspondiente (manual o por foto)
    \item El usuario es dirigido a la pantalla apropiada
\end{itemize}

\noindent\textbf{Flujo Principal:}
\begin{enumerate}
    \item El usuario presiona el botón "Registrar Egreso" desde el dashboard
    \item El sistema presenta las opciones de registro:
    \begin{itemize}
        \item Registro Manual
        \item Tomar Foto de Ticket
    \end{itemize}
    \item El usuario selecciona la opción deseada
    \item Si selecciona "Registro Manual": se ejecuta CU-04
    \item Si selecciona "Tomar Foto de Ticket": se ejecuta CU-05
\end{enumerate}

\noindent\textbf{Flujos Alternativos:}

\textbf{FA-01: Cancelar operación}
\begin{enumerate}
    \item En cualquier momento del proceso
    \item El usuario puede cancelar y regresar al dashboard
\end{enumerate}

\subsection{CU-04: Registrar Egreso Manualmente}

\noindent\textbf{Actor:} Administrador Financiero

\noindent\textbf{Descripción:} El administrador financiero registra manualmente un egreso ingresando todos los datos del gasto, incluyendo artículos, costos y lugar de compra.

\noindent\textbf{Precondiciones:}
\begin{itemize}
    \item El usuario debe estar autenticado en el sistema
    \item Se debe haber seleccionado "Registro Manual" en CU-03
\end{itemize}

\noindent\textbf{Postcondiciones:}
\begin{itemize}
    \item El egreso queda registrado en el sistema
    \item Se actualiza el balance general del usuario
    \item El gasto queda categorizado
    \item El registro es visible en el historial de transacciones
\end{itemize}

\noindent\textbf{Flujo Principal:}
\begin{enumerate}
    \item El sistema presenta el formulario de registro manual
    \item El usuario ingresa la información del gasto:
    \begin{itemize}
        \item Descripción/artículos comprados
        \item Monto total del gasto
        \item Lugar de compra (establecimiento)
        \item Fecha del gasto (por defecto la fecha actual)
        \item Notas adicionales (opcional)
    \end{itemize}
    \item El usuario confirma los datos ingresados
    \item El sistema procesa automáticamente la categorización del gasto
    \item El sistema presenta la categoría sugerida al usuario
    \item El usuario confirma si la categorización es correcta:
    \begin{itemize}
        \item Si es correcta: procede al paso 8
        \item Si no es correcta: selecciona la categoría apropiada
    \end{itemize}
    \item El sistema registra el egreso con la categoría confirmada
    \item El sistema actualiza el balance del usuario
    \item El sistema muestra un mensaje de confirmación
    \item El usuario regresa al dashboard actualizado
\end{enumerate}

\noindent\textbf{Flujos Alternativos:}

\textbf{FA-01: Datos inválidos}
\begin{enumerate}
    \item En el paso 3, si los datos no son válidos
    \item El sistema muestra mensajes de error específicos
    \item El usuario corrige los datos y regresa al paso 3
\end{enumerate}

\textbf{FA-02: Error en categorización automática}
\begin{enumerate}
    \item En el paso 4, si no se puede categorizar automáticamente
    \item El sistema solicita al usuario seleccionar una categoría manualmente
    \item El usuario selecciona la categoría apropiada
    \item Continúa en el paso 7
\end{enumerate}

\subsection{CU-05: Tomar Foto de Ticket}

\noindent\textbf{Actor:} Administrador Financiero

\noindent\textbf{Descripción:} El administrador financiero registra un egreso tomando una o varias fotografías del ticket de compra, confirmando los datos escaneados por OCR.

\noindent\textbf{Precondiciones:}
\begin{itemize}
    \item El usuario debe estar autenticado en el sistema
    \item Se debe haber seleccionado "Tomar Foto de Ticket" en CU-03
    \item El dispositivo debe tener cámara disponible
    \item Debe existir un ticket físico para fotografiar
\end{itemize}

\noindent\textbf{Postcondiciones:}
\begin{itemize}
    \item El egreso queda registrado en el sistema
    \item Se almacenan las fotografías del ticket
    \item Se actualiza el balance general del usuario
    \item El gasto queda categorizado
    \item Se asigna la etiqueta opcional si fue proporcionada
\end{itemize}

\noindent\textbf{Flujo Principal:}
\begin{enumerate}
    \item El sistema activa la cámara del dispositivo
    \item El usuario toma una o varias fotografías del ticket
    \item El sistema procesa las imágenes con tecnología OCR
    \item El sistema extrae automáticamente:
    \begin{itemize}
        \item Artículos comprados
        \item Precios individuales y total
        \item Establecimiento
        \item Fecha de compra
    \end{itemize}
    \item El sistema presenta los datos escaneados para confirmación
    \item El usuario revisa y confirma/corrige los datos extraídos
    \item El usuario puede agregar opcionalmente una etiqueta descriptiva (ej: "compra mensual noviembre")
    \item El sistema categoriza automáticamente el gasto
    \item El usuario confirma la categorización sugerida
    \item El sistema registra el egreso con todos los datos confirmados
    \item El sistema almacena las fotografías asociadas al registro
    \item El sistema actualiza el balance del usuario
    \item El sistema muestra un mensaje de confirmación
    \item El usuario regresa al dashboard actualizado
\end{enumerate}

\noindent\textbf{Flujos Alternativos:}

\textbf{FA-01: OCR no funciona correctamente}
\begin{enumerate}
    \item En el paso 3, si el OCR no puede procesar las imágenes
    \item El sistema informa que no se pudieron extraer los datos
    \item Se ejecuta CU-06: Registrar Ticket Manualmente
\end{enumerate}

\textbf{FA-02: Datos extraídos incorrectos}
\begin{enumerate}
    \item En el paso 6, si los datos extraídos son incorrectos
    \item El usuario corrige manualmente los datos erróneos
    \item Continúa en el paso 7
\end{enumerate}

\textbf{FA-03: Calidad de imagen insuficiente}
\begin{enumerate}
    \item En el paso 3, si la calidad de la imagen es insuficiente
    \item El sistema solicita tomar nuevas fotografías
    \item El usuario regresa al paso 2
\end{enumerate}

\subsection{CU-06: Registrar Ticket Manualmente}

\noindent\textbf{Actor:} Administrador Financiero

\noindent\textbf{Descripción:} El administrador financiero registra manualmente los datos de un ticket cuando el OCR no funciona correctamente, digitando los artículos y costos mientras mantiene las fotografías del ticket.

\noindent\textbf{Precondiciones:}
\begin{itemize}
    \item El usuario debe estar autenticado en el sistema
    \item Se deben haber tomado fotografías del ticket
    \item El OCR debe haber fallado en la extracción de datos
\end{itemize}

\noindent\textbf{Postcondiciones:}
\begin{itemize}
    \item El egreso queda registrado en el sistema con datos ingresados manualmente
    \item Se mantienen las fotografías del ticket asociadas
    \item Se actualiza el balance general del usuario
    \item El gasto queda categorizado
    \item Se asigna la etiqueta opcional si fue proporcionada
\end{itemize}

\noindent\textbf{Flujo Principal:}
\begin{enumerate}
    \item El sistema presenta un formulario de entrada manual con las fotografías del ticket visibles
    \item El usuario digita manualmente los datos del ticket:
    \begin{itemize}
        \item Lista de artículos comprados
        \item Precio de cada artículo
        \item Monto total
        \item Establecimiento/lugar de compra
        \item Fecha de compra
    \end{itemize}
    \item El usuario puede agregar opcionalmente una etiqueta descriptiva
    \item El usuario confirma los datos ingresados
    \item El sistema intenta categorizar automáticamente el gasto basado en los artículos
    \item Si la categorización automática es exitosa:
    \begin{enumerate}
        \item El sistema presenta la categoría sugerida
        \item El usuario confirma o corrige la categorización
    \end{enumerate}
    \item Si la categorización automática falla:
    \begin{enumerate}
        \item El sistema solicita al usuario seleccionar una categoría
        \item El usuario selecciona la categoría apropiada
    \end{enumerate}
    \item El sistema registra el egreso con todos los datos confirmados
    \item El sistema mantiene las fotografías asociadas al registro
    \item El sistema actualiza el balance del usuario
    \item El sistema muestra un mensaje de confirmación
    \item El usuario regresa al dashboard actualizado
\end{enumerate}

\noindent\textbf{Flujos Alternativos:}

\textbf{FA-01: Datos inválidos}
\begin{enumerate}
    \item En el paso 4, si los datos ingresados no son válidos
    \item El sistema muestra mensajes de error específicos
    \item El usuario corrige los datos y regresa al paso 4
\end{enumerate}

\textbf{FA-02: Totales no coinciden}
\begin{enumerate}
    \item En el paso 4, si la suma de artículos no coincide con el total
    \item El sistema alerta sobre la discrepancia
    \item El usuario revisa y corrige los montos
    \item Regresa al paso 4
\end{enumerate}

\newpage

\subsection{CU-07: Ver Historial de Notificaciones}

\noindent\textbf{Actor:} Administrador Financiero

\noindent\textbf{Descripción:} El administrador financiero consulta el historial completo de notificaciones recibidas del sistema, con capacidad de filtrado por tipo de alerta para revisar alertas de presupuestos, recordatorios de deudas y otras notificaciones importantes.

\noindent\textbf{Precondiciones:}
\begin{itemize}
    \item El usuario debe estar autenticado en el sistema
    \item El usuario debe haber configurado y permitido las notificaciones en su dispositivo
    \item Debe existir al menos una notificación previa en el sistema
    \item El usuario debe tener presupuestos configurados y alertas activas
\end{itemize}

\noindent\textbf{Postcondiciones:}
\begin{itemize}
    \item Se muestra el historial de notificaciones según los filtros aplicados
    \item Las notificaciones no leídas quedan marcadas como leídas
    \item Se actualiza el contador de notificaciones pendientes
\end{itemize}

\noindent\textbf{Flujo Principal:}
\begin{enumerate}
    \item El usuario accede a la opción "Historial de Notificaciones" desde el menú principal
    \item El sistema presenta la lista completa de notificaciones ordenadas por fecha (más recientes primero)
    \item El sistema muestra para cada notificación:
    \begin{itemize}
        \item Tipo de notificación (Alerta de Presupuesto, Recordatorio de Deuda, etc.)
        \item Fecha y hora de generación
        \item Mensaje descriptivo
        \item Estado (leída/no leída)
    \end{itemize}
    \item El usuario puede aplicar filtros por tipo de notificación:
    \begin{itemize}
        \item Alertas de Presupuesto
        \item Recordatorios de Deudas
        \item Notificaciones del Sistema
        \item Todas las notificaciones
    \end{itemize}
    \item El sistema actualiza la lista según el filtro seleccionado
    \item El usuario puede seleccionar una notificación específica para ver detalles completos
    \item El sistema marca automáticamente las notificaciones como leídas al visualizarlas
    \item El usuario puede eliminar notificaciones individuales o en lote
\end{enumerate}

\noindent\textbf{Flujos Alternativos:}

\textbf{FA-01: Sin notificaciones disponibles}
\begin{enumerate}
    \item En el paso 2, si no existen notificaciones en el sistema
    \item El sistema muestra mensaje: "No hay notificaciones disponibles"
    \item Se sugiere al usuario configurar alertas y presupuestos
\end{enumerate}

\textbf{FA-02: Filtro sin resultados}
\begin{enumerate}
    \item En el paso 5, si el filtro aplicado no devuelve resultados
    \item El sistema muestra mensaje: "No hay notificaciones del tipo seleccionado"
    \item El usuario puede cambiar el filtro o ver todas las notificaciones
\end{enumerate}

\subsection{CU-08: Visualizar Estadísticas por Categoría}

\noindent\textbf{Actor:} Administrador Financiero

\noindent\textbf{Descripción:} El administrador financiero visualiza estadísticas detalladas de sus gastos organizadas por categorías, incluyendo gráficos comparativos de presupuesto vs gasto real, análisis de productos por categoría y seguimiento mensual de patrones de consumo.

\noindent\textbf{Precondiciones:}
\begin{itemize}
    \item El usuario debe estar autenticado en el sistema
    \item Debe tener al menos una categoría de gastos configurada
    \item Debe existir historial de transacciones registradas
    \item Debe tener presupuestos asignados por categorías
\end{itemize}

\noindent\textbf{Postcondiciones:}
\begin{itemize}
    \item Se visualizan las estadísticas actualizadas por categoría
    \item Se muestran gráficos comparativos de presupuesto vs gasto real
    \item El usuario obtiene insights sobre sus patrones de gasto
\end{itemize}

\noindent\textbf{Flujo Principal:}
\begin{enumerate}
    \item El usuario accede a la opción "Estadísticas por Categoría" desde el dashboard
    \item El sistema presenta la vista general con gráfico de pastel mostrando:
    \begin{itemize}
        \item Distribución del presupuesto total por categorías
        \item Porcentaje gastado vs presupuesto por cada categoría
        \item Indicadores visuales de categorías en alerta (cerca del límite)
    \end{itemize}
    \item El usuario puede seleccionar una categoría específica para ver detalles
    \item El sistema muestra información detallada de la categoría seleccionada:
    \begin{itemize}
        \item Presupuesto asignado vs cantidad gastada
        \item Lista de productos/gastos registrados en la categoría
        \item Tendencia de gastos (comparación con mes anterior)
        \item Porcentaje de presupuesto utilizado
    \end{itemize}
    \item El sistema permite cambiar el período de análisis (mes actual, mes anterior, trimestre)
    \item El usuario puede exportar las estadísticas en formato gráfico
    \item El sistema actualiza automáticamente las estadísticas con cada nueva transacción
\end{enumerate}

\noindent\textbf{Flujos Alternativos:}

\textbf{FA-01: Sin datos suficientes}
\begin{enumerate}
    \item En el paso 2, si no hay suficientes datos para generar estadísticas
    \item El sistema muestra mensaje informativo sobre datos insuficientes
    \item Se sugiere registrar más transacciones para obtener análisis completos
\end{enumerate}

\textbf{FA-02: Categoría sin movimientos}
\begin{enumerate}
    \item En el paso 4, si la categoría seleccionada no tiene transacciones
    \item El sistema muestra mensaje: "No hay gastos registrados en esta categoría"
    \item Se muestra solo el presupuesto asignado y se sugiere comenzar a registrar gastos
\end{enumerate}

\subsection{CU-09: Modificar Categoría de Gastos}

\noindent\textbf{Actor:} Administrador Financiero

\noindent\textbf{Descripción:} El administrador financiero modifica las configuraciones de sus categorías de gastos, incluyendo cambios de presupuesto, reasignación de productos, modificación de nombres y eliminación de productos mal categorizados.

\noindent\textbf{Precondiciones:}
\begin{itemize}
    \item El usuario debe estar autenticado en el sistema
    \item Debe existir al menos una categoría de gastos creada
    \item El usuario debe tener permisos para modificar categorías
\end{itemize}

\noindent\textbf{Postcondiciones:}
\begin{itemize}
    \item La categoría queda actualizada con las nuevas configuraciones
    \item Los gastos existentes se reasignan automáticamente si se cambió la categoría
    \item Se actualizan las estadísticas y gráficos relacionados
    \item Se registra el cambio en el log de auditoría
\end{itemize}

\noindent\textbf{Flujo Principal:}
\begin{enumerate}
    \item El usuario accede a "Gestión de Categorías" desde el menú de configuración
    \item El sistema muestra la lista de categorías existentes con sus respectivos presupuestos
    \item El usuario selecciona la categoría que desea modificar
    \item El sistema presenta las opciones de modificación disponibles:
    \begin{itemize}
        \item Cambiar nombre de la categoría
        \item Ajustar presupuesto asignado
        \item Ver y reasignar productos categorizados
        \item Eliminar productos de la categoría
    \end{itemize}
    \item El usuario realiza las modificaciones deseadas:
    \begin{enumerate}
        \item Si cambia el nombre: ingresa el nuevo nombre y confirma
        \item Si ajusta presupuesto: especifica el nuevo monto límite
        \item Si reasigna productos: selecciona productos y elige nueva categoría destino
        \item Si elimina productos: marca los productos a eliminar y confirma
    \end{enumerate}
    \item El sistema valida que los cambios no generen conflictos
    \item El usuario confirma todos los cambios realizados
    \item El sistema procesa las modificaciones y reasigna automáticamente los gastos afectados
    \item El sistema actualiza todas las estadísticas y gráficos relacionados
    \item Se muestra mensaje de confirmación con resumen de cambios aplicados
\end{enumerate}

\noindent\textbf{Flujos Alternativos:}

\textbf{FA-01: Nombre duplicado}
\begin{enumerate}
    \item En el paso 6, si el nuevo nombre ya existe en otra categoría
    \item El sistema muestra error: "Ya existe una categoría con este nombre"
    \item El usuario debe ingresar un nombre diferente
    \item Regresa al paso 5
\end{enumerate}

\textbf{FA-02: Presupuesto inválido}
\begin{enumerate}
    \item En el paso 6, si el presupuesto ingresado es menor o igual a cero
    \item El sistema muestra error de validación
    \item El usuario debe corregir el monto
    \item Continúa en el paso 6
\end{enumerate}

\textbf{FA-03: Error en reasignación}
\begin{enumerate}
    \item En el paso 8, si ocurre un error al reasignar productos
    \item El sistema muestra mensaje de error específico
    \item Se revierten los cambios parciales realizados
    \item El usuario puede reintentar la operación
\end{enumerate}

\subsection{CU-10: Planificar Pago de Deudas}

\noindent\textbf{Actor:} Administrador Financiero

\noindent\textbf{Descripción:} El administrador financiero visualiza y organiza sus deudas pendientes, recibe sugerencias del sistema para estrategias de pago optimizadas y planifica el orden y cronograma de pagos para minimizar intereses y cumplir con los compromisos financieros.

\noindent\textbf{Precondiciones:}
\begin{itemize}
    \item El usuario debe estar autenticado en el sistema
    \item Debe tener al menos una deuda registrada en el sistema
    \item Las deudas deben tener información completa (monto, tasa de interés, fecha de vencimiento)
\end{itemize}

\noindent\textbf{Postcondiciones:}
\begin{itemize}
    \item Se genera un plan de pago optimizado para las deudas
    \item Se establecen prioridades de pago basadas en las estrategias seleccionadas
    \item Se programan alertas de recordatorio para los pagos planificados
    \item Se actualiza el dashboard con el cronograma de pagos
\end{itemize}

\noindent\textbf{Flujo Principal:}
\begin{enumerate}
    \item El usuario accede a la opción "Planificar Pago de Deudas" desde el menú principal
    \item El sistema muestra la lista completa de deudas pendientes con:
    \begin{itemize}
        \item Nombre del acreedor
        \item Monto adeudado
        \item Tasa de interés (si aplica)
        \item Fecha de vencimiento
        \item Pago mínimo requerido
    \end{itemize}
    \item El sistema presenta estrategias de pago sugeridas:
    \begin{itemize}
        \item Avalancha de deudas (priorizar por mayor tasa de interés)
        \item Bola de nieve (priorizar por menor monto)
        \item Por fecha de vencimiento (priorizar próximos vencimientos)
        \item Personalizada (orden definido por el usuario)
    \end{itemize}
    \item El usuario selecciona la estrategia de pago preferida
    \item El usuario ingresa el monto total disponible mensualmente para pagos de deudas
    \item El sistema calcula y presenta el plan de pagos optimizado mostrando:
    \begin{itemize}
        \item Orden de prioridad de pagos
        \item Monto sugerido para cada deuda
        \item Tiempo estimado para liquidar cada deuda
        \item Total de intereses que se pagarán
    \end{itemize}
    \item El usuario revisa y puede ajustar manualmente los montos asignados
    \item El usuario confirma el plan de pagos
    \item El sistema programa alertas de recordatorio para cada fecha de pago
    \item Se genera un calendario de pagos visible en el dashboard
\end{enumerate}

\noindent\textbf{Flujos Alternativos:}

\textbf{FA-01: Monto insuficiente para pagos mínimos}
\begin{enumerate}
    \item En el paso 6, si el monto disponible no cubre los pagos mínimos requeridos
    \item El sistema muestra alerta: "El monto disponible no cubre los pagos mínimos requeridos"
    \item Se sugiere revisar las deudas o incrementar el monto disponible
    \item El usuario debe ajustar el monto o revisar las deudas registradas
\end{enumerate}

\textbf{FA-02: Sin deudas registradas}
\begin{enumerate}
    \item En el paso 2, si no existen deudas registradas en el sistema
    \item El sistema muestra mensaje: "No tienes deudas registradas"
    \item Se ofrece la opción de registrar nuevas deudas
    \item Se redirige al caso de uso CU-12: Registrar Deudas
\end{enumerate}

\subsection{CU-11: Establecer Alertas de Pago de Deudas}

\noindent\textbf{Actor:} Administrador Financiero

\noindent\textbf{Descripción:} El administrador financiero configura alertas y recordatorios automáticos para los pagos de sus deudas, estableciendo diferentes tipos de notificaciones para evitar pagos tardíos y cargos por intereses adicionales.

\noindent\textbf{Precondiciones:}
\begin{itemize}
    \item El usuario debe estar autenticado en el sistema
    \item Debe tener al menos una deuda registrada
    \item Las notificaciones deben estar habilitadas en el dispositivo
    \item El usuario debe tener permisos para configurar alertas
\end{itemize}

\noindent\textbf{Postcondiciones:}
\begin{itemize}
    \item Las alertas quedan configuradas y activas para las deudas seleccionadas
    \item Se programan las notificaciones según la frecuencia establecida
    \item Se actualiza el sistema de notificaciones con las nuevas alertas
    \item Las alertas aparecen en el calendario del usuario
\end{itemize}

\noindent\textbf{Flujo Principal:}
\begin{enumerate}
    \item El usuario accede a "Configurar Alertas de Deudas" desde el menú de deudas
    \item El sistema muestra la lista de deudas disponibles para configurar alertas
    \item El usuario selecciona las deudas para las cuales desea establecer alertas
    \item Para cada deuda seleccionada, el sistema presenta opciones de configuración:
    \begin{itemize}
        \item Tipo de recordatorio (fecha de vencimiento, recordatorio previo)
        \item Días de anticipación (3, 7, 15 días antes del vencimiento)
        \item Frecuencia de recordatorio (una vez, diario hasta el pago, semanal)
        \item Método de notificación (push, email, in-app, todos)
    \end{itemize}
    \item El usuario configura cada parámetro según sus preferencias
    \item El sistema muestra una vista previa del calendario de alertas programadas
    \item El usuario confirma la configuración de alertas
    \item El sistema activa las alertas y las programa en el calendario
    \item Se muestra mensaje de confirmación con resumen de alertas configuradas
    \item El sistema comienza a enviar notificaciones según la programación establecida
\end{enumerate}

\noindent\textbf{Flujos Alternativos:}

\textbf{FA-01: Notificaciones deshabilitadas}
\begin{enumerate}
    \item En cualquier paso, si las notificaciones están deshabilitadas en el dispositivo
    \item El sistema muestra alerta: "Las notificaciones están deshabilitadas"
    \item Se proporcionan instrucciones para habilitar notificaciones
    \item El usuario debe habilitar notificaciones para continuar
\end{enumerate}

\textbf{FA-02: Deuda sin fecha de vencimiento}
\begin{enumerate}
    \item En el paso 4, si una deuda no tiene fecha de vencimiento definida
    \item El sistema solicita establecer una fecha de vencimiento
    \item Se redirige al caso de uso CU-12: Configurar Deudas
    \item Una vez establecida la fecha, regresa al paso 4
\end{enumerate}

\subsection{CU-12: Configurar Deudas}

\noindent\textbf{Actor:} Administrador Financiero

\noindent\textbf{Descripción:} El administrador financiero modifica y actualiza la información de deudas existentes, incluyendo ajustes de montos, plazos, tasas de interés y programación de pagos periódicos para mantener la información actualizada y precisa.

\noindent\textbf{Precondiciones:}
\begin{itemize}
    \item El usuario debe estar autenticado en el sistema
    \item Debe existir al menos una deuda registrada previamente
    \item El usuario debe tener permisos para modificar información de deudas
\end{itemize}

\noindent\textbf{Postcondiciones:}
\begin{itemize}
    \item La información de la deuda queda actualizada en el sistema
    \item Se actualizan automáticamente los planes de pago relacionados
    \item Se ajustan las alertas programadas según los nuevos parámetros
    \item Se registra el cambio en el historial de modificaciones
\end{itemize}

\noindent\textbf{Flujo Principal:}
\begin{enumerate}
    \item El usuario accede a "Gestión de Deudas" desde el menú principal
    \item El sistema muestra la lista de deudas registradas con información básica
    \item El usuario selecciona la deuda que desea modificar
    \item El sistema presenta el formulario con la información actual de la deuda:
    \begin{itemize}
        \item Nombre del acreedor
        \item Monto total adeudado
        \item Tasa de interés (si aplica)
        \item Fecha de vencimiento
        \item Pago mínimo mensual
        \item Notas adicionales
    \end{itemize}
    \item El usuario modifica los campos necesarios:
    \begin{itemize}
        \item Actualizar monto adeudado
        \item Cambiar fecha de vencimiento
        \item Ajustar tasa de interés
        \item Modificar pago mínimo
        \item Programar pagos periódicos automáticos
    \end{itemize}
    \item El sistema valida que los nuevos datos sean consistentes y válidos
    \item El usuario confirma los cambios realizados
    \item El sistema actualiza la información de la deuda
    \item Se recalculan automáticamente los planes de pago que incluyan esta deuda
    \item Se ajustan las alertas programadas según los nuevos parámetros
    \item Se registra la modificación en el historial con fecha y cambios realizados
    \item Se muestra mensaje de confirmación con resumen de cambios aplicados
\end{enumerate}

\noindent\textbf{Flujos Alternativos:}

\textbf{FA-01: Datos inválidos}
\begin{enumerate}
    \item En el paso 6, si los datos ingresados no son válidos
    \item El sistema muestra mensajes de error específicos para cada campo
    \item El usuario debe corregir los datos señalados
    \item Regresa al paso 5 para realizar las correcciones
\end{enumerate}

\textbf{FA-02: Conflicto con plan de pagos activo}
\begin{enumerate}
    \item En el paso 9, si los cambios afectan un plan de pagos activo
    \item El sistema muestra alerta sobre el impacto en el plan existente
    \item Se ofrece la opción de actualizar automáticamente el plan o mantenerlo sin cambios
    \item El usuario decide cómo proceder con el plan de pagos
\end{enumerate}

\subsection{CU-13: Registrar Deudas}

\noindent\textbf{Actor:} Administrador Financiero

\noindent\textbf{Descripción:} El administrador financiero registra nuevas deudas en el sistema, capturando toda la información necesaria para su seguimiento y gestión, incluyendo detalles del acreedor, montos, plazos y condiciones de pago.

\noindent\textbf{Precondiciones:}
\begin{itemize}
    \item El usuario debe estar autenticado en el sistema
    \item El usuario debe tener una nueva deuda real para registrar
    \item El sistema debe tener espacio disponible para nuevas deudas
\end{itemize}

\noindent\textbf{Postcondiciones:}
\begin{itemize}
    \item La nueva deuda queda registrada y activa en el sistema
    \item Se actualiza el dashboard con la información de la nueva deuda
    \item La deuda está disponible para planificación de pagos y alertas
    \item Se registra la fecha de creación en el historial
\end{itemize}

\noindent\textbf{Flujo Principal:}
\begin{enumerate}
    \item El usuario accede a la opción "Registrar Nueva Deuda" desde el menú de deudas
    \item El sistema presenta el formulario de registro de deuda
    \item El usuario ingresa la información obligatoria:
    \begin{itemize}
        \item Nombre del acreedor (banco, tarjeta de crédito, prestamista, etc.)
        \item Monto total adeudado
        \item Fecha de la deuda (cuándo se adquirió)
        \item Fecha de vencimiento o plazo para pagar
    \end{itemize}
    \item El usuario puede agregar información opcional:
    \begin{itemize}
        \item Tasa de interés anual
        \item Pago mínimo mensual requerido
        \item Tipo de deuda (tarjeta de crédito, préstamo personal, hipoteca, etc.)
        \item Notas adicionales o condiciones especiales
    \end{itemize}
    \item El usuario confirma la información ingresada
    \item El sistema valida que todos los datos obligatorios estén completos y sean válidos
    \item El sistema registra la nueva deuda en la base de datos
    \item Se actualiza automáticamente el dashboard con la nueva deuda visible
    \item El sistema ofrece opciones de configuración inmediata:
    \begin{itemize}
        \item Establecer alertas de pago (CU-11)
        \item Incluir en plan de pagos (CU-10)
        \item Configurar pagos periódicos
    \end{itemize}
    \item Se muestra mensaje de confirmación con resumen de la deuda registrada
\end{enumerate}

\noindent\textbf{Flujos Alternativos:}

\textbf{FA-01: Información incompleta}
\begin{enumerate}
    \item En el paso 6, si faltan campos obligatorios
    \item El sistema resalta los campos faltantes con mensajes de error
    \item El usuario debe completar la información requerida
    \item Regresa al paso 3 para completar datos
\end{enumerate}

\textbf{FA-02: Deuda duplicada}
\begin{enumerate}
    \item En el paso 7, si ya existe una deuda similar registrada
    \item El sistema muestra alerta: "Parece que ya tienes una deuda similar registrada"
    \item Se muestra la deuda existente para comparación
    \item El usuario puede confirmar que es una deuda diferente o cancelar el registro
\end{enumerate}

\textbf{FA-03: Monto inválido}
\begin{enumerate}
    \item En el paso 6, si el monto ingresado es menor o igual a cero
    \item El sistema muestra error de validación
    \item El usuario debe ingresar un monto válido mayor a cero
    \item Continúa en el paso 5
\end{enumerate}