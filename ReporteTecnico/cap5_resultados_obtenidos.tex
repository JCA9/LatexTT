\chapter{Resultados Obtenidos}

En este Trabajo Terminal 1 concluimos con éxito el análisis y la planificación del proyecto, en donde definimos los requerimientos del sistema tanto para la 
aplicación móvil como para la plataforma web. Durante esta fase, establecimos las prioridades del proyecto y definimos los requisitos específicos para cada plataforma.

En cuanto al levantamiento y análisis de requerimientos, realizamos la identificación de los casos de uso principales, 
además de la elaboración de diagramas UML que sirvieron para documentar los flujos de trabajo y las interacciones dentro del sistema. 
Completamos también la redacción de la documentación que describe los requerimientos funcionales y no funcionales del sistema.

En la fase de planeación técnica y diseño preliminar, trabajamos en la creación de los esquemas iniciales de las funcionalidades del sistema. 
Asimismo, definimos el estilo visual inicial del sistema, lo que nos permitió establecer una identidad visual coherente y atractiva para los usuarios finales.

En cuanto al prototipado y diseño de la experiencia de usuario, creamos prototipos navegables de la interfaz móvil mediante Balsamiq, lo que nos permitió tener una 
representación visual interactiva del sistema desde las primeras etapas de desarrollo. Este diseño también nos permitió y permitirá detectar posibles mejoras en la navegación y la 
disposición de los elementos.

Configuramos el entorno de desarrollo y comenzamos con la implementación del FrontEnd de la página web, donde logramos avances en la construcción de la estructura básica 
y los componentes visuales que permiten una interacción fluida y eficiente.

Finalmente, desarrollamos la versión inicial del módulo principal del proyecto, el escaneo de tickets. 
En esta etapa, se realizaron pruebas preliminares que han permitido categorizar productos a partir de una biblioteca definida para cada categoría, 
sentando las bases para las funcionalidades avanzadas que se implementarán en el próximo trabajo terminal.