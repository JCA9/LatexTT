\renewcommand{\bibname}{Referencias} % <-- cambia "Bibliografía" a "Referencias"
\addcontentsline{toc}{chapter}{Referencias} % <-- Agrega la sección al índice

\begin{thebibliography}{99}

\bibitem{inegi2024} %1
\textsc{INEGI.} (2024). Encuesta Nacional sobre Salud Financiera (ENSAFI) 2023.
Recuperado de: \url{https://www.inegi.org.mx/contenidos/saladeprensa/boletines/2024/ENSAFI/ENSAFI.pdf}

\bibitem{statista2025} %2 
\textsc{Statista.} (2025). La situación financiera personal en el futuro, ¿fuente de preocupación?.
Recuperado de: \url{https://es.statista.com/grafico/27448/porcentaje-de-encuestados-que-estan-preocupados-por-su-futuro-financiero/}

\bibitem{adjust2023}    %3
\textsc{Adjust.} (2023). El uso de las aplicaciones de finanzas continúa creciendo en 2023.
Recuperado de: \url{https://www.adjust.com/es/blog/finance-app-usage/}

\bibitem{patel2022} %4
\textsc{Patel, V., et al.} (2022). Optical Character Recognition for document automation: Accuracy and applications.
ScienceDirect. \url{https://www.sciencedirect.com/science/article/pii/S2352711022000329}

\bibitem{statista2024} %5 estas todavia no las cito
\textsc{Statista.} (2024). Usage of finance apps and retention statistics.
\url{https://www.statista.com/statistics/finance-app-usage/}

\bibitem{oecd2023} %6
\textsc{OECD.} (2023). Financial literacy and decision-making: Data visualization impact.
\url{https://www.oecd.org/finance/financial-education/}

\bibitem{banxico2023}  %7
\textsc{Banco de México.} (2023). Educación financiera y adherencia al presupuesto.
\url{https://www.banxico.org.mx/educacion-financiera/}

\bibitem{Encuesta Uso de Tecnologias INEGI}  %8
\textsc{INEGI, ENDUTIH} (2023). Comunicado de prensa 372/24, 13-jun-2024. Encuesta Nacional sobre Disponibilidad y Uso de Tecnologías de la Información en los Hogares (ENDUTIH) 2023.
\url{https://www.inegi.org.mx/contenidos/saladeprensa/boletines/2024/ENDUTIH/ENDUTIH_23.pdf}

\bibitem{Athento} %9
\textsc{M. Aguirre and M. Aguirre} (2025). “Procesamiento de documentos: etapas y tecnologías esenciales,” Athento - Smart Digital Content Platform, Aug. 13.
\url{https://www.athento.com/es/procesamiento-de-documentos-etapas-y-tecnologias-esenciales/}

\bibitem{T. Portal} %10
\textsc{Portal TIC} (2025). “Digitalización de documentos: ¿qué es y en qué consiste el proceso?”
\url{https://www.ticportal.es/temas/sistema-gestion-documental/digitalizacion-de-documentos}

\bibitem{AWS} %11
\textsc{Servicios web de Amazon, Inc.} (n.d.). ¿Qué es el OCR? - Explicación del reconocimiento óptico de caracteres - AWS
\url{https://aws.amazon.com/es/what-is/ocr/}

\bibitem{IBM} %12
\textsc{C. Stryker and J. Holdsworth} (2025). “Procesamiento De Lenguaje Natural
\url{https://www.ibm.com/mx-es/think/topics/natural-language-processing}

\bibitem{AWS NLP} %13
\textsc{Amazon Web Services, Inc.} (n.d.). ¿Qué es el NLP? - Explicación del procesamiento de lenguaje natural - AWS
\url{https://aws.amazon.com/es/what-is/nlp/}

\bibitem{Latenode} %14
\textsc{Latenode} (2025). El mejor software de escaneo y OCR para dispositivos móviles en 2024
\url{https://latenode.com/es/blog/the-best-mobile-scanning-and-ocr-software-in-2024}

\bibitem{IBM OCR} %15
\textsc{IBM} (n.d.).  “Reconocimiento óptico de caracteres,” ¿Qué es el reconocimiento óptico de caracteres (OCR)? 
\url{https://www.ibm.com/mx-es/think/topics/optical-character-recognition}

\bibitem{Paseur} %16
\textsc{N. Gunnoo} (2024). Reconocimiento óptico de caracteres (OCR): guía completa
\url{https://parseur.com/es/blog/que-es-el-reconocimiento-optico-de-caracteres}

\bibitem{Paseur Extracción} %17
\textsc{N. Gunnoo} (2025). Extrae texto de un PDF en 2025
\url{https://parseur.com/es/caso-de-uso/extraer-texto-de-pdf}

\bibitem{Statistics Canada} %18
\textsc{Government of Canada, Statistics Canada} (2022). Document Intelligence: The art of PDF information extraction
\url{https://www.statcan.gc.ca/en/data-science/network/pdf-extraction}

\bibitem{Timbox} %19
\textsc{K. Madrid} (2023). Timbrar CFDI 4.0
\url{https://www.timbox.com.mx/timbrar-cfdi-4-0/}

\bibitem{BBVA Finanzas} %20
\textsc{BBVA MEXICO and BBVA} (2024). “¿Qué son las finanzas personales y su importancia?,” ¿Qué son las finanzas personales y su importancia?, Nov. 25, 2024.
\url{https://www.bbva.mx/educacion-financiera/banca-digital/cuenta-digital-que-son-finanzas-personales.html}

\bibitem{Coral Prous} %21
\textsc{HeadTeam Marketing} (2024). 33 conceptos clave de finanzas personales,
\url{https://coralprous.es/educacion-financiera/conceptos-de-finanzas-personales/}

\bibitem{La Vanguardia} %22
\textsc{N. Bourass} (2025). “Dave Ramsey, experto en finanzas, comparte su mejor método de ahorro: Si no le dices a tu dinero dónde ir, ¡te vas a preguntar dónde fue!”
\url{https://www.lavanguardia.com/dinero/20250305/10447218/dave-ramsey-experto-finanzas-comparte-mejor-metodo-ahorro-dices-dinero-donde-ir-preguntar-donde-fue-gvm.html}

\bibitem{FasterCapital} %23
\textsc{FasterCapital} (n.d.). Metodos de control presupuestario  metodos eficaces de control presupuestario en la gestion de costes
\url{https://fastercapital.com/es/contenido/Metodos-de-control-presupuestario--metodos-eficaces-de-control-presupuestario-en-la-gestion-de-costes.html}

\bibitem{IBM Banca} %24
\textsc{IBM} (2024). “Transformación digital de la banca,” ¿Qué es la transformación digital de la banca y los servicios financieros?
\url{https://www.ibm.com/mx-es/think/topics/digital-transformation-banking}

% MARCO TEORICO - TECNOLOGIAS DE DESARROLLO

\bibitem {C4Model} 
\textsc{S. Brown} (2023). The C4 Model for Visualising Software Architecture, 2nd ed. London, UK: Leanpub.
\url{https://c4model.com/}

\bibitem{Architectural Styles} 
\textsc{R. T. Fielding} (2000). Architectural Styles and the Design of Network-based Software Architectures
\url{https://ics.uci.edu/~fielding/pubs/dissertation/top.htm}

\bibitem{ClienteServidor}
\textsc{Oscar  Blancarte} (n.d.). Arquitectura Cliente-Servidor.
\url{https://reactiveprogramming.io/blog/es/estilos-arquitectonicos/cliente-servidor}

\bibitem{WhyNextJS}
\textsc{A. Barragán} (2025). “Por qué Next.js es el framework perfecto para aplicaciones React,” OpenWebinars.net
\url{https://www.scribbr.com/citation/generator/folders/2g0gvq0LC3VK7b2A3veRRC/lists/wAVEo7xYXGbmK1pEkbZIQ/}

\bibitem{WhyFlutter}
\textsc{Ilia Lotarev} (2025). Flutter vs Kotlin: ¿Cuál elegir para tu proyecto?
\url{https://adapty.io/blog/flutter-vs-kotlin/}

\bibitem{WhyMySQL}
\textsc{J. Erickson} (2024). MySQL: Understanding what it is and how its used
\url{https://www.scribbr.com/citation/generator/folders/2g0gvq0LC3VK7b2A3veRRC/lists/wAVEo7xYXGbmK1pEkbZIQ/}

\bibitem{WhyRest}
\textsc{Codecademy} (n.d.). What is REST API (RESTful API)? Explained
\url{https://www.codecademy.com/article/what-is-rest-api}

% MARCO TEORICO - CATEGORIZACION
\bibitem{ProcessMaker}
\textsc{ProcessMaker Inc.} (2024). ¿Qué es la categorización de gastos y por qué es importante?
\url{https://www.processmaker.com/es/resources/what-is-expense-categorization}

\bibitem{CategorizacionGastos}
\textsc{Navan} (2023). “¿Qué es la Automatización de Procesos Digitales (DPA)? | Procesador,” Creador de procesos
\url{https://www.processmaker.com/es/blog/what-is-digital-process-automation-dpa/}

\bibitem{WhyRegex}
\textsc{Lenovo} (n.d.). ¿Qué es Regex y cómo usarlo? | Lenovo México.
\url{https://www.lenovo.com/mx/es/glosario/expresion-regular-regex/}

% MARCO TEORICO - METODOLOGIAS DE DESARROLLO
\bibitem{Metodologias}
\textsc{R. S. Pressman and B. R. Maxim} (2020). Software Engineering: A Practitioner’s Approach, 9th ed. New York, NY, USA: McGraw-Hill.
\url{}

\bibitem{ManifestoAgil} 
\textsc{K. Beck et al.} (2001). Manifesto for Agile Software Development
\url{https://agilemanifesto.org}

\bibitem{WhyXP}
\textsc{I. R. Cano} (2018). “¿Qué ventajas aporta Extreme Programming? - Viewnext.com
\url{https://www.viewnext.com/ventajas-extreme-programming/}

\bibitem{WhyKanban}
\textsc{J. Martins} (2025). ¿Qué es la metodología Kanban y cómo funciona? • Asana
\url{https://asana.com/es/resources/what-is-kanban}

\end{thebibliography}