\renewcommand{\bibname}{Referencias} % <-- cambia "Bibliografía" a "Referencias"
\addcontentsline{toc}{chapter}{Referencias} % <-- Agrega la sección al índice

\begin{thebibliography}{99}

\bibitem{inegi2024} %1
\textsc{INEGI.} (2024). Encuesta Nacional sobre Salud Financiera (ENSAFI) 2023.
Recuperado de: \url{https://www.inegi.org.mx/contenidos/saladeprensa/boletines/2024/ENSAFI/ENSAFI.pdf}

\bibitem{statista2025} %2 
\textsc{Statista.} (2025). La situación financiera personal en el futuro, ¿fuente de preocupación?.
Recuperado de: \url{https://es.statista.com/grafico/27448/porcentaje-de-encuestados-que-estan-preocupados-por-su-futuro-financiero/}

\bibitem{adjust2023}    %3
\textsc{Adjust.} (2023). El uso de las aplicaciones de finanzas continúa creciendo en 2023.
Recuperado de: \url{https://www.adjust.com/es/blog/finance-app-usage/}

\bibitem{patel2022} %4
\textsc{Patel, V., et al.} (2022). Optical Character Recognition for document automation: Accuracy and applications.
ScienceDirect. \url{https://www.sciencedirect.com/science/article/pii/S2352711022000329}

\bibitem{statista2024} %5 estas todavia no las cito
\textsc{Statista.} (2024). Usage of finance apps and retention statistics.
\url{https://www.statista.com/statistics/finance-app-usage/}

\bibitem{oecd2023} %6
\textsc{OECD.} (2023). Financial literacy and decision-making: Data visualization impact.
\url{https://www.oecd.org/finance/financial-education/}

\bibitem{banxico2023}  %7
\textsc{Banco de México.} (2023). Educación financiera y adherencia al presupuesto.
\url{https://www.banxico.org.mx/educacion-financiera/}

\bibitem{inegiSus}  %7
\textsc{cambiame chucho} (2025). chucho te extraño
\url{soy tu ex}

\end{thebibliography}