\section*{Glosario}

\begin{description}
  \item[API (Interfaz de Programación de Aplicaciones):] Conjunto de reglas y protocolos que permite la comunicación e intercambio de datos entre diferentes componentes de software.

  \item[Aplicación móvil:] Programa diseñado para ejecutarse en dispositivos móviles, en este caso utilizado para capturar y registrar los tickets de compra.

  \item[Arquitectura cliente-servidor:] Modelo de diseño en el que un servidor central gestiona los datos y servicios solicitados por múltiples clientes, garantizando la coherencia y la disponibilidad de la información.

  \item[Base de datos relacional:] Estructura de almacenamiento de información que organiza los datos en tablas vinculadas mediante claves primarias y foráneas, permitiendo consultas y operaciones eficientes.

  \item[Categorización de gastos:] Proceso mediante el cual se clasifican las transacciones financieras en grupos o categorías (por ejemplo, alimentación, transporte, salud) con el fin de facilitar su análisis.

  \item[Flutter:] Framework de código abierto desarrollado por Google que permite crear aplicaciones móviles multiplataforma a partir de una única base de código fuente.

  \item[Frontend:] Capa de una aplicación que interactúa directamente con el usuario, encargada de la presentación y la experiencia visual.

  \item[JSON (JavaScript Object Notation):] Formato de texto ligero utilizado para el intercambio de datos entre el frontend y el backend.

  \item[Metodología de desarrollo:] Conjunto de principios, técnicas y prácticas utilizadas para planificar, diseñar, construir y mantener sistemas de software.

  \item[MySQL:] Sistema gestor de bases de datos relacional de código abierto ampliamente utilizado en aplicaciones web y sistemas distribuidos.

  \item[Next.js:] Framework basado en React que permite el desarrollo de aplicaciones web modernas con soporte para renderizado del lado del servidor (SSR) y generación estática (SSG).

  \item[OCR (Reconocimiento Óptico de Caracteres):] Tecnología que permite convertir imágenes de texto impreso en texto digital, utilizada para extraer información de los tickets de compra.

  \item[Prototipo:] Versión inicial o modelo funcional de un sistema que permite evaluar su diseño, características y viabilidad técnica antes de su implementación definitiva.

  \item[React:] Biblioteca de JavaScript empleada para la construcción de interfaces de usuario interactivas y componentes reutilizables.

  \item[Reporte financiero:] Documento generado por el sistema que presenta un resumen de los ingresos, egresos y tendencias de gasto del usuario en un periodo determinado.

  \item[SQL (Structured Query Language):] Lenguaje estándar utilizado para la gestión y manipulación de datos en sistemas de bases de datos relacionales.

  \item[Ticket digitalizado:] Imagen o documento electrónico que representa un comprobante de compra capturado y procesado mediante el módulo OCR.

  \item[UML (Lenguaje Unificado de Modelado):] Conjunto de notaciones gráficas que permiten describir, visualizar y documentar los componentes y relaciones de un sistema de software.

\end{description}