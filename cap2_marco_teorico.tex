\chapter{Marco Teórico}


\section{Procesamiento de Documentos Digitales}
El procesamiento de documentos es un conjunto de procedimientos y tecnologías diseñados para poder transformar y gestionar documentos, contenidos e información 
tanto de archivos físicos como digitales. Este procesamiento puede abarcar desde la captura y digitalización de documentos y tiene como objetivo optimizar la eficiencia y 
accesibilidad de la información \cite{Athento}.

\subsection{Digitalización de documentos}
Es el proceso de transformar documentos en papel a un formato digital. No es lo mismo el escanear que digitalizar, ya que el escaneo implica obtener una foto del documento 
y la digitalización lo convierte en algo editable, reconociendo el texto que contiene \cite{T. Portal}. 

\subsection{OCR (Reconocimiento Óptico de Caracteres)}
El OCR nos permite una digitalización y clasificación simultanea de la documentación e información, confiriendo metadatos a los documentos y permitiendo una búsqueda 
instantánea de la información deseada posteriormente \cite{T. Portal}. Dicho de otra forma, es el proceso por el cual se convierte una imagen de texto en un formato de texto que pueden 
leer las maquinas.


\subsubsection{Casos de Uso del OCR}

\textbf{Búsqueda inteligente de archivos de documentos} \\
El OCR permite la creacion de archivos digitales con capacidad de búsqueda mediante la extracción de texto de documentos PDF y basados en imágenes. 
Una vez que se reconoce el texto, se puede indexar y utilizar en sistemas de búsqueda basados en IA. Por ejemplo, al buscar un nombre de cliente específico, 
se devolverían todas las órdenes de pago, las facturas y los formularios que se enviaron originalmente como documentación \cite{AWS}.

\textbf{Procesamiento de lenguaje natural} \\
El OCR reconoce y extrae texto por palabra, línea o celda de tabla, lo que ofrece un mayor control sobre cómo se prepara el contenido para las tareas posteriores de 
procesamiento de lenguaje natural, como la clasificación de documentos, el resumen, el análisis de opiniones, el modelado de temas, el reconocimiento de entidades, etc. \cite{AWS}.

\textbf{Estandarización de datos} \\
El OCR ayuda a normalizar datos de documentos no estructurados.

\textbf{Automatización del procesamiento de formularios} \\
Puede identificar campos y extraer información estructurada de varios tipos de formularios, lo que permite a las empresas integrar estos datos directamente en las 
bases de datos sin necesidad de ingresarlos manualmente. \cite{AWS}.

\begin{figure}[H]
    \centering
    \includegraphics[width=1\linewidth]{./images/OCR_Estructura.JPG}
    \caption{Funcionamiento del OCR}
    \label{fig:placeholder}
\end{figure}

\subsubsection{Proceso del OCR}

El software OCR emplea un escáner para leer los documentos y así convertirlos en datos binarios. Analiza la imagen escaneada y clasifica las áreas claras como el fondo 
y las áreas oscuras como el texto. \cite{IBM OCR}

\begin{itemize}
\item \textbf{Procesamiento Previo} \\
El OCR primero limpia la imagen y elimina errores para prepararla para la lectura. 

\item \textbf{Reconocimiento de Texto} \\
Los dos tipos principales de algoritmos de OCR o procesos de software que utiliza el software de OCR para el reconocimiento de texto se denominan coincidencia de patrones 
y extracción de características. 

\item \textbf{Coincidencia de patrones} \\
La coincidencia de patrones aísla una imagen de carácter, llamada glifo, y la compara con un glifo almacenado de manera similar. El reconocimiento de patrones solo funciona 
si el glifo almacenado tiene una fuente y una escala similares a las del glifo de entrada. Este método funciona bien con imágenes escaneadas de documentos que se han escrito 
en una fuente conocida. 

\item \textbf{Extracción de características} \\
La extracción de características divide o descompone los glifos en características como líneas, circuitos cerrados, dirección de línea e intersecciones de línea. Luego, 
utiliza estas características para encontrar la mejor coincidencia o el vecino más cercano entre los glifos almacenados.

\item \textbf{Procesamiento Posterior} \\
Después del análisis, el sistema convierte los datos de texto extraídos en documentos de texto que pueden leer las máquinas \cite{AWS}.
\end{itemize}

% \begin{figure}[H]
%     \centering
%     \includegraphics[width=1\linewidth]{ProcessOCR.jpg}
%     \caption{Proceso general de un programa de OCR}
%     \label{fig:placeholder}
% \end{figure}

\subsubsection{Tipos de OCR}
Actualmente hay 4 tipos de programas de OCR que son cada vez más sofisticados y avanzados.\cite{IBM OCR}

\textbf{OCR simple:} Es un análisis de coincidencias de patrones, yendo carácter por carácter, comparando las incidencias entre los caracteres recién escaneados y los 
glifos almacenados previamente. Esta limitado a la gran cantidad de posibles combinaciones de fuentes, idiomas y caracteres, volviendo complicado el escaneo de algunos 
tipos de documentos. 

\textbf{Reconocimiento óptico de marcas (OMR):} Este análisis sirve para identificar logotipos, firmas, símbolos, marcas de agua, así como marcas dentro de un documento. 
Estos pueden identificarse haciendo coincidir imágenes almacenadas, así como en el OCR simple. 

\textbf{Reconocimiento inteligente de caracteres (ICR):} Para este tipo de programas se vuelve indispensable el uso de Machine Learning o aprendizaje profundo, estos 
programas aprender a leer, así como un humano: mediante practica y entrenamiento continuo. Usa una red neuronal que revisa el texto repetidamente en busca de atributos 
distintivos: ubicación de curvas, intersecciones, líneas y bucles. 

\textbf{Reconocimiento inteligente de palabras:} Basado en el mismo principio del ICR con la diferencia de que este usa la Inteligencia Artificial entrenada para reconocer 
las palabras de una imagen, volviéndola más rápida. 

% \begin{figure}[H]
%     \centering
%     \includegraphics[width=1\linewidth]{ExOCR.png}
%     \caption{Ejemeplo de reconocimiento de OCR}
%     \label{fig:placeholder}
% \end{figure}

\subsection{Extracción estructurada de datos (PDF y XML)}
Para la extracción de datos de un documento PDF se recomienda seguir un flujo que detecte dos casos comunes:  

\begin{enumerate}
    \item El PDF es nativo con texto seleccionable, entonces el flujo será la extracción de dicho texto manteniendo la estructura con técnicas de extracción de contexto y 
    segmentación.

    \item El PDF es un conjunto de imágenes o escaneos, aquí el flujo debe ser el aplicar un OCR como lo mencionado anteriormente en la sección 3.1. \cite{Paseur}\cite{Paseur Extracción}
\end{enumerate}
Por ejemplo el repositorio SEDAR, Statistics Canada implementó un pipeline que primero identifica las páginas relevantes por densidad de tablas y características de layout, 
y luego aplica un algoritmo de extracción de tablas (SLICE) para transformar las tablas detectadas en datos tabulares, lo que ilustra la importancia de combinar detección de 
páginas relevantes, extracción de tablas y modelos de clasificación para obtener datasets de calidad a gran escala.\cite{Statistics Canada}

Este tipo de flujos nos ayudan a la automatización de diferentes aplicaciones como extraer datos de factura, análisis de texto, exportar datos, convertidores de 
PDF a texto. \cite{Paseur}\cite{Paseur Extracción}

Si es el caso también se puede hacer una extracción estructurada cuando tenemos comprobantes en formato XML, así como el estándar del SAT, los CFDI. La forma fiable de 
obtener la información proveniente de estos documentos es parseando el XML. \cite{Timbox}

Los CFDI mantienen un estándar que permite la extracción exacta de información como fechas, RFC, conceptos, importes e impuestos. 


\section{Finanzas Personales y Gestión de Gastos}
Modelo de sobres de dinero de Dave Ramsey
\subsection{Conceptos básicos de finanzas personales}
Las finanzas personales se conforman de ciertas piezas que sabiendo acomodar le permiten al individuo tener una toma de decisiones informadas y construir un futuro 
financiero solido. Estas piezas son conformadas por los \textbf{ingresos, gastos, ahorros, inversiones y deudas}. Es de gran importancia desarrollar hábitos financieros 
desde joven, para en un futuro tener buenas bases para enfrentarte a desafíos económicos que puedan surgir.

Dentro de los conceptos que se deben tener claros para la salud financiera tenemos:
\begin{itemize}
    \item \textbf{Ingresos activos/pasivos:} Son todas las formas en las que generas ingresos, ya sea ingresos por nomina, renta de un inmobiliario, las ganancias de alguna 
    inversión o si usas tu coche como auto de transporte particular.
    \item \textbf{Ahorro:} Es el dinero destinado para ser almacenado con un propósito especifico. No es dinero que sobro y solo queda guardar, si no que en cuanto se reciba 
    un ingreso, el ahorro deberá corresponder al 10\% de los ingresos, idealmente.
    \item \textbf{Deudas:} Incluye tarjetas de créditos, hipoteca, créditos de vivienda o un préstamo familiar. Es importante ordenar tus deudas de la mas pequeña a la mas 
    grande, hacer un plan de pagos y empezar a liquidar las deudas pequeñas primero.
    Si no llegaras a tener deudas, pero quisieras tomar un préstamo, deberías revisar tu capacidad de endeudamiento.
    \item \textbf{Balance:} Es un registro donde debes identificar tus gastos e ingresos, si tus gastos superan tus ingresos, las deudas te consumirán en algún punto por 
    lo que deberías cambiar de estrategia financiera. Por otra parte si tus gastos igualan tus ingresos, no estas generando ahorros lo que no te permitirá planear un futuro 
    pleno.
\end{itemize}
Hay otros temas relevantes que tocar hablando de conceptos y buenos hábitos, como el no caer en estafas que prometen generar inversiones irreales en tiempos récord, 
investiga y usa aplicaciones confiables para meter tu dinero. Los consultores también son de tomar en cuenta puesto que analizan tu salud financiera, cuales son tus 
objetivos y que productos financieros posees. Un consultor traza un plan personalizado según tus posibilidades y necesidades para cumplir los objetivos que te llevaran 
a lograr tus metas, el consultor también deberá ayudarte a resolver tus dudas y acompañarte año con año para llegar a tus objetivos.\cite{BBVA Finanzas}\cite{Coral Prous}


    \subsubsection{Metas financieras}
    El banco BBVA recomienda establecer metas financieras siguiendo los siguientes consejos: \cite{BBVA Finanzas}
    \begin{itemize}
        \item Ser especifico: plantear metas y objetivos a lograr, como el conseguir un auto, una casa o viajar.
        \item Establece plazos: fijar fechas limites de cuando quieres cumplir tus metas.
        \item Crear un plan: divide tus metas en pequeños objetivos logrables y asigna presupuestos a cada uno.
    \end{itemize}
    
\subsection{Modelos y metodologías de gestión financiera}
    \subsubsection{Modelo de sobres de Dave Ramsey}
    Dave Ramsey un escritor y experto en asesoramiento financiero, elaboro un sistema de ahorro basado en la presupuestación de sobres para el control financiero y el ahorro, 
    basándose en su propia experiencia de quiebra en la juventud. \cite{La Vanguardia}
    
    \textbf{Categorías:} El sistema define categorías según tus gastos mensuales, se recomienda realizar un listado de los gastos personales y a partir de ellos realizar la 
    categorización. Algunos gastos fijos comunes son el alquiler, el transporte, alimento, el internet y el ocio. La idea de Ramsey para este método no unicamente el ahorra 
    intensivamente, sino que el que aplique este método goce de un consumo responsable en su ocio, manteniendo un balance entre ingresos y gastos. 
    
    \textbf{Presupuesto:} El sistema de Ramsey esta pensado para usarse con efectivo, donde a un sobre le escribes la categoría e introduces el presupuesto en efectivo. El 
    presupuesto se puede definir analizando facturas y gastos de meses anteriores, calculando un estimado de presupuesto por categoría.
    La presupuestación ayuda a que se tome consciencia de los gastos mensuales para tener un mejor control financiero. Si por ejemplo vas al supermercado sin tener un 
    presupuesto definido, es probable que termines gastando mas de lo que realmente era necesario.

    \textbf{Rellenar Sobres:} Ramsey explica que los sobres deben pensarse de manera mensual, por lo que, si tus ingresos son semanales, el sistema debe ajustarse para 
    cubrir los gastos mensuales. Si los gastos son quincenales, debes introducir a los sobres la mitad correspondiente al presupueste mensual. Con la digitalización hay 
    gastos domiciliados para los que usar el método de los sobres es muy eficiente.

    \textbf{Disciplina:} Como cualquier método de ahorro requiere de constancia y enfoque, este método permite al que lo use conseguir pequeños logros que motivan a largo 
    plazo al usuario a lograr una mayor libertad económica.
    
\subsubsection{Presupuestación y control de gastos}
La presupuestación y el control de gastos son un proceso sistemático para planificar, ejecutar y evaluar el uso del dinero. Su finalidad es alinear los recursos con 
objetivos financieros (ahorro, inversión, reducción de deudas) y corregir desviaciones a tiempo, manteniendo estabilidad y disciplina. \cite{FasterCapital}

\textbf{Estructura del método:} 
Se inicia con el registro histórico de ingresos y gastos, se clasifican por categorías (vivienda, transporte, alimentación, salud, educación, ocio, ahorro, inversión 
y deudas), y se asignan límites de gasto en función de los ingresos netos y prioridades financieras.

\textbf{Técnicas de presupuestación:} 
Incluye enfoques como presupuesto incremental (ajustar el del periodo anterior), presupuesto base cero (cada gasto se justifica desde cero), 50/30/20 (necesidades/deseos
/ahorro-deuda), y presupuestos rodantes (se actualizan mensualmente o trimestralmente). La elección depende de la estabilidad de ingresos y de la complejidad de los gastos.

\textbf{Ciclo de control:} 
El control se realiza con un ciclo continuo: (1) planificar el presupuesto por categoría; (2) registrar gastos reales; (3) analizar variaciones (\emph{desviación} = gasto
 real – presupuesto); (4) aplicar medidas correctivas (recortes, reubicaciones, negociación de tarifas); (5) revisar y ajustar metas.

\textbf{Seguimiento práctico:} 
El registro puede ser manual (hojas de cálculo), con aplicaciones financieras o mediante políticas de sólo efectivo para categorías variables (por ejemplo, alimentación
/ocio). La trazabilidad mejora usando etiquetas por categoría, fecha y método de pago; se recomienda conciliación semanal y cierre mensual.

\textbf{Indicadores clave (KPIs):} 
Relación gasto/ingreso, tasa de ahorro, ratio de endeudamiento (cuotas/ingreso), gasto esencial vs. discrecional, cumplimiento por categoría (% dentro del presupuesto), y variación acumulada. Estos indicadores permiten evaluar si el presupuesto es realista y sostenible.

\textbf{Medidas de corrección:} 
Priorizar gastos esenciales, reducir gastos discrecionales, renegociar servicios (telecomunicaciones, seguros), optimizar compras (listas, ofertas, sustituciones), y 
reestructurar deudas (pago acelerado de saldos pequeños o de mayor interés, según estrategia).

\textbf{Buenas prácticas:} 
Definir objetivos claros (fondo de emergencia, inversión, amortización de deuda), automatizar el ahorro al inicio de cada ingreso, establecer límites por categoría 
con márgenes razonables, revisar mensualmente y ajustar por cambios (inflación, ingresos variables), y mantener evidencia (facturas/recibos) para auditoría personal.

\textbf{Riesgos comunes:} 
Subestimar gastos variables, no contemplar imprevistos, usar supuestos optimistas, falta de disciplina en el registro, y no actualizar el presupuesto ante cambios. 
El antídoto es la revisión periódica, el fondo de emergencia y la adaptación flexible.

\textbf{Resultado esperado:} 
Mejor control del flujo de efectivo, reducción de desviaciones, mayor tasa de ahorro, menor estrés financiero y toma de decisiones informada para metas de corto y 
largo plazo.
    
\subsection{Digitalización financiera y aplicaciones modernas}
La digitalización en la banca se trata de integrar tecnologías y estrategias para optimizar operaciones practicas y mejorar las experiencias personalizadas digitalmente. 
La evolución del mercado impulsa a los bancos tradicionales a seguir las tendencias de emplear aplicaciones móviles y sitios web para realizar transacciones. La nueva 
demanda demanda de Inteligencia Artificial, Internet de las Cosas y el blockchain son razones por las que los bancos implementan estrategias para la transformación digital. \cite{IBM Banca}
En la actualidad, los clientes buscan experiencias personalizadas, transparentes y seguras para realizar la gestión de sus cuentas, todo al alcance de tu mano en tiempo 
real y con una atención eficiente y rápida.
Algunos de los factores clave para la transformacional digital son:
\begin{itemize}
    \item \textbf{Recorrido del cliente:} tener siempre en cuenta la experiencia del cliente, centrándonos en el uso de datos y nuevas tecnologías para adaptar sus servicios bancarios de forma personalizada.
    \item \textbf{Infraestructura modernizada:} El uso de nuevas tecnologías como la Inteligencia Artificial o la automatización, sirven para agilizar operaciones internas o impulsar la eficiencia de procesos para ofrecerles a los bancos y servicios financieros una ventaja competitiva.
    \item \textbf{Análisis de datos:} usando herramientas de análisis de datos avanzados, los bancos pueden tomar decisiones mas informadas y fundamentadas para una mejor gestión de riesgos e innovación.
    \item \textbf{Medidas de seguridad:} Un punto necesario durante la innovación y la adoptar de la digitalización deben a su vez implementar medidas de ciberseguridad tan robustas como sus procesos para proteger de la mejor manera los datos confidenciales de los clientes, así como detección de fraudes y la garantización del cumplimiento de normativas.
    \item \textbf{Digitalización:} Esta dentro del sector financiero la necesidad de alinearse junto con otros sectores que adoptan el avance digital, por ello se vuelve crucial la iniciativa de transformación digital así como el asociarse con nuevas compañías de tecnología financiera o infraestructuras bancarias abiertas.

\end{itemize}

Las nuevas aplicaciones bancarias aplican diferentes tecnologías para la digitalización eficiente de sus procesos, tales como: 
\textbf{Interfaces de programación de aplicaciones (API):} Las APIs permiten que dos o mas aplicaciones integren servicios y transferencia de datos en lugar de hacer desarrollos desde cero, permitiendo crear nuevos productos y servicios para los clientes y mejorar la eficiencia operacional.
\textbf{Computación en la nube:} El entorno en la nube permite mejores operaciones y una infraestructura flexible, ágil y escalable. Es el acceso bajo demando de los recursos informáticos que se han estandarizado en los últimos tiempos.
\textbf{IA y Machine Learning (ML):} Se están adoptando estas tecnologías para el análisis de big data, automatización de procesos y la mejora de experiencia del usuario a través de asistentes en linea o chatbots, diseñados para solucionar problemas básicos del cliente. El Machine Learning en la digitalización bancaria nos facilitan el seguimiento de los cambios de comportamiento del usuario y detecta las posibles actividades fraudulentas de manera oportuna. 
\textbf{Internet de las cosas (IoT):} Esto hace referencias a toda la red de dispositivos electrónicos inteligentes que el usuario utiliza cotidianamente, que permiten a los usuarios realizar pagos instantáneos sin contacto ni interactuar con su banca móvil. Esta tecnología nos aporta una gestión de riesgos y avances en el proceso de autorización muy avanzado.

La digitalización de el sistema bancario conlleva una tarea compleja pero que trae nuevas oportunidades para las compañías de brindar mayos satisfacción al cliente con soluciones eficientes, rápidas e interactivas. Los beneficios de la digitalización nos traen una oportunidad de inversión mas centrada en los clientes, permitiéndoles tener objetivos a corto plazo, todo en una sola plataforma digital.\\
Cumplir con procesos tediosos se vuelve cosa del pasado con la digitalización puesto que actividades de los empleados se ven remplazadas por automatizaciones eficientes que permiten al empleado utilizar mas de su tiempo en actividades importantes. El uso de sistemas basados en la nube proporciona actualizaciones oportunas y el cumplimiento de la seguridad requerida para los datos.

A pesar del miedo de la administración y protección de datos de los clientes, surgen sofisticados servicios de desarrollo de software disponibles para la protección de la información sensible de los clientes y evitar vulnerabilidades a sus centros de información.

La digitalización nos permite ademas de toda la seguridad y automatización, el poder acceder a nuevos clientes constantemente, en cuestión de clic un nuevo usuario puede registrar sus datos y comenzar su experiencia dentro de las aplicaciones bancarias, que ademas de tener un crecimiento diario, puesto que con la digitalización, las empresas bancarias pueden tener una retroalimentación diaria de sus servicios, pudiendo mejorar e innovar según las necesidades detectadas en el usuario, brindando así, una experiencia única y eficiente a cada persona. \cite{IBM Banca}


\section{Tecnologías de Desarrollo y Arquitectura del Sistema}
La solucion que se propuso se fundamenta en una arquitectura cliente - servidor con una separación clara entre la captura y visualizacion (aplicación movil y web),
la logica de negocio (servidor backend) y la persistencia (base de datos). Este enfoque promueve la modularidad, el despliegue independiente y la escalabilidad
horizontal cuando los componentes requieren mayor capacidad. \cite{C4Model} \cite{Architectural Styles}

\subsection{Arquitectura cliente-servidor y modularidad}
\subsection{Tecnologías web y móviles}
\subsection{Bases de datos y comunicación entre sistemas}


\section{Automatización de Procesos y Categorización de Gastos}
Para tener un buen sistema de gestion de finanzas personales, es necesario implementar técnicas de automatización para poder reducir la intervencion manual, disminuir errores
y poder optimizar tiempos en el procesamiento de los datos. \cite{ProcessMaker}
Por otra parte, la categorizacion de gastos es fundamental para que el usuario pueda tener un control detallado de sus finanzas personales, pudiendo identificar patrones de 
consumo y áreas de mejora en su gestión financiera. \cite{CategorizacionGastos}

\subsection{Reglas y patrones de categorización}
\subsection{Procesamiento automatizado de transacciones}
\subsection{Generación de reportes y visualizaciones automáticas}

\section{Metodologías de Desarrollo}
Las metodologías nos proporcionan un marco estructurado que nos permite organizar el trabajo, poder gestionar los requisitos que quiere el cliente,
poder reducir los riesgos y nos ayuda a garantizar la calidad del producto final. \cite{Metodologias}
Para desarrollar una aplicacion de finanzas personales, tenemos que adoptar enfoques adaptativos que nos permitan mantener la flexibilidad ante cambios
en los requerimientos. \cite{ManifestoAgil}



\subsection{Metodologías ágiles}
\subsection{Ciclo de vida del desarrollo del software}
\subsection{Pruebas y validación del sistema}