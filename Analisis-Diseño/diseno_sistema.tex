\chapter{Diseño del Sistema}

En este capitulo se detalla el diseño arquitectónico y los componentes principales del sistema. 
Se describen las decisiones tomadas para la implementación de las funcionalidades clave.

\section{Arquitectura General}

El sistema sigue una arquitectura de tres capas:
\begin{itemize}
    \item \textbf{Capa de Presentación:} Interfaz de usuario móvil
    \item \textbf{Capa de Lógica de Negocio:} Procesamiento de reglas financieras
    \item \textbf{Capa de Datos:} Almacenamiento y persistencia de información
\end{itemize}

\section{Componentes Principales}

\subsection{Módulo de Autenticación}
\textbf{Función:} Gestionar el acceso seguro al sistema
\textbf{Tecnologías:} Firebase Authentication, OAuth 2.0

\subsection{Módulo OCR}
\textbf{Función:} Procesar imágenes de tickets y extraer información
\textbf{Tecnologías:} Google ML Kit, Tesseract

\subsection{Módulo de Categorización}
\textbf{Función:} Clasificar automáticamente gastos e ingresos
\textbf{Tecnologías:} Algoritmos de machine learning, análisis de patrones

\subsection{Módulo de Presupuestación}
\textbf{Función:} Implementar metodología Dave Ramsey
\textbf{Características:} Máximo 10 categorías, alertas automáticas

\subsection{Módulo de Reportes}
\textbf{Función:} Generar estadísticas y visualizaciones
\textbf{Tecnologías:} Charts.js, generación de PDFs

\section{Diagrama de Contenedores}

\begin{figure}[H]
    \centering
    \includegraphics[width=1\linewidth]{./images/Diagrama-Contenedores.JPG}
    \caption{Diagrama de Contenedores del Sistema}
\end{figure}

\section{Base de Datos}

\subsection{Modelo Entidad-Relación}
\textbf{Entidades principales:}
\begin{itemize}
    \item Usuario
    \item Transaccion
    \item Categoria
    \item Presupuesto
    \item Deuda
    \item Notificacion
\end{itemize}

\subsection{Almacenamiento}
\textbf{Base de datos principal:} Firebase Firestore
\textbf{Almacenamiento de imágenes:} Firebase Storage
\textbf{Datos locales:} SQLite para funcionalidad offline

\section{Interfaces de Usuario}

\subsection{Principios de Diseño}
\begin{itemize}
    \item Simplicidad y claridad
    \item Navegación intuitiva
    \item Feedback visual inmediato
    \item Accesibilidad para todos los usuarios
\end{itemize}

\subsection{Pantallas Principales}
\begin{itemize}
    \item Dashboard principal
    \item Captura de tickets (cámara)
    \item Gestión de categorías
    \item Reportes y estadísticas
    \item Configuración de deudas
    \item Historial de transacciones
\end{itemize}

\section{Seguridad}

\subsection{Encriptación}
\begin{itemize}
    \item Datos en tránsito: TLS 1.3
    \item Datos en reposo: AES-256
    \item Comunicación con APIs: OAuth 2.0 + JWT
\end{itemize}

\subsection{Privacidad}
\begin{itemize}
    \item Datos personales nunca compartidos
    \item Procesamiento OCR local cuando sea posible
    \item Eliminación automática de imágenes procesadas
\end{itemize}

\section{Integración con Servicios Externos}

\subsection{API de alguna Inteligencia Artificial}
Para realizar las recomendaciones personalizadas e intentar mejorar la categorización de gastos, proponemos integrar una API de inteligencia artificial que analizará los 
patrones de gasto del usuario y ofrecerá sugerencias basadas en sus hábitos financieros. 