\chapter{Especificación de Casos de Uso}

\section{Introducción}

Los casos de uso describen las interacciones entre los usuarios (actores) y el sistema de gestión financiera personal. Cada caso de uso detalla un flujo de trabajo específico que el sistema debe soportar para cumplir con los objetivos del usuario.

El sistema está diseñado para un único tipo de actor principal: el \textbf{Administrador Financiero}, quien es el usuario propietario de la aplicación y responsable de gestionar sus finanzas personales.

\section{Mapeo de Requerimientos Funcionales a Casos de Uso}

La siguiente tabla muestra la relación entre los 40 requerimientos funcionales identificados y los casos de uso especificados:

\begin{table}[H]
\centering
\begin{tabular}{|l|l|l|}
\hline
\textbf{RF} & \textbf{Nombre} & \textbf{Caso de Uso} \\
\hline
\multicolumn{3}{|c|}{\textbf{Casos de Uso Compartidos (App Móvil y Web)}} \\
\hline
RF01 & Registro de usuario & CU-01: Registrar Usuario \\
RF02 & Inicio de sesión & CU-02: Iniciar Sesión \\
RF03 & Recuperación de contraseña & CU-03: Recuperar Contraseña \\
RF04 & Edición de perfil & CU-21: Editar Perfil \\
RF05 & Eliminación de cuenta & CU-22: Eliminar Cuenta \\
RF06 & Registro de ingresos & CU-04: Registrar Ingreso \\
RF07-RF08 & Registro de egresos & CU-05: Registrar Egreso \\
RF08 & Registro de egresos manualmente & CU-06: Registrar Egreso Manualmente \\
RF09 & Registro de ticket manualmente & CU-08: Registrar Ticket Manualmente \\
RF10 & Configuración de presupuesto & CU-16: Configurar Presupuestos \\
RF11 & Visualización de historial & CU-18: Consultar Historial de Transacciones \\
RF12 & Visualización de gráficas & CU-23: Visualizar Gráficas \\
RF13 & Generación de reportes PDF & CU-24: Generar Reportes PDF \\
RF14 & Configuración de metas de ahorro & CU-17: Configurar Metas de Ahorro \\
\hline
\multicolumn{3}{|c|}{\textbf{Casos de Uso Exclusivos App Móvil}} \\
\hline
RF15 & Cambio de tema & CU-25: Cambiar Tema Aplicación \\
RF16-RF20 & Captura de foto del ticket & CU-07: Tomar Foto de Ticket \\
RF19 & Etiquetado personalizado & CU-26: Etiquetar Tickets \\
RF20 & Corrección de datos OCR & CU-27: Corregir Datos Escaneados \\
RF21 & Historial de notificaciones & CU-09: Ver Historial de Notificaciones \\
RF22 & Estadísticas por categoría & CU-10: Visualizar Estadísticas por Categoría \\
RF23 & Modificación de categorías & CU-11: Modificar Categoría de Gastos \\
RF24 & Alertas de deudas & CU-13: Establecer Alertas de Pago de Deudas \\
RF25 & Configuración de deudas & CU-14: Configurar Deudas \\
RF26 & Registro de deudas & CU-15: Registrar Deudas \\
RF27 & Planificación de pago de deudas & CU-12: Planificar Pago de Deudas \\
RF28 & Presupuestos personales & CU-28: Planificar Presupuestos Personales \\
RF29 & Consulta de tickets & CU-19: Consultar Tickets \\
RF30 & Visualización de gastos & CU-29: Visualizar Gastos \\
RF31 & Eliminación de tickets & CU-20: Eliminar Tickets \\
\hline
\multicolumn{3}{|c|}{\textbf{Casos de Uso Exclusivos Plataforma Web}} \\
\hline
RF32 & Consejos personalizados & CU-30: Consultar Consejos Personalizados \\
RF33 & Búsqueda y filtrado & CU-31: Buscar y Filtrar Tickets \\
RF34 & Reportes en Excel & CU-32: Generar Reportes Excel \\
RF35 & Compartir reportes por correo & CU-33: Compartir Reportes por Correo \\
RF36 & Comparación entre meses & CU-34: Comparar Gastos entre Meses \\
RF37 & Balance vs egresos & CU-35: Consultar Balance vs Egresos \\
RF38 & Estadísticas avanzadas & CU-36: Visualizar Estadísticas Avanzadas \\
RF39 & Carga de ticket digital & CU-37: Cargar Ticket Digital \\
RF40 & Metas de ahorro desde web & CU-38: Configurar Metas desde Web \\
\hline
\end{tabular}
\caption{Mapeo Completo de Requerimientos Funcionales a Casos de Uso}
\end{table}

\section{CU-01: Registrar Usuario}

\noindent\textbf{Actor:} Administrador Financiero

\noindent\textbf{Descripción:} El usuario se registra por primera vez en el sistema proporcionando sus datos personales y creando credenciales de acceso.

\noindent\textbf{Precondiciones:}
\begin{itemize}
    \item El sistema debe estar disponible
    \item El usuario no debe tener una cuenta existente
    \item El usuario debe tener acceso a un correo electrónico válido
\end{itemize}

\noindent\textbf{Postcondiciones:}
\begin{itemize}
    \item Se crea una nueva cuenta de usuario en el sistema
    \item Se envía un correo de confirmación al usuario
    \item El usuario puede iniciar sesión con sus credenciales
\end{itemize}

\noindent\textbf{Flujo Principal:}
\begin{enumerate}
    \item El usuario accede a la pantalla de registro
    \item El sistema presenta el formulario de registro
    \item El usuario ingresa la información requerida:
    \begin{itemize}
        \item Nombre completo
        \item Correo electrónico
        \item Contraseña
        \item Confirmación de contraseña
    \end{itemize}
    \item El usuario acepta los términos y condiciones
    \item El usuario presiona el botón ``Registrarse''
    \item El sistema valida los datos ingresados
    \item El sistema verifica que el correo no esté registrado
    \item El sistema crea la cuenta del usuario
    \item El sistema envía correo de confirmación
    \item El sistema muestra mensaje de registro exitoso
    \item El usuario es redirigido a la pantalla de inicio de sesión
\end{enumerate}

\noindent\textbf{Flujos Alternativos:}

\textbf{FA-01: Correo ya registrado}
\begin{enumerate}
    \item En el paso 7, si el correo ya está registrado
    \item El sistema muestra mensaje de error
    \item El usuario debe usar un correo diferente
    \item Regresa al paso 3
\end{enumerate}

\textbf{FA-02: Contraseñas no coinciden}
\begin{enumerate}
    \item En el paso 6, si las contraseñas no coinciden
    \item El sistema muestra mensaje de error
    \item El usuario debe corregir las contraseñas
    \item Regresa al paso 3
\end{enumerate}

\textbf{FA-03: Datos inválidos}
\begin{enumerate}
    \item En el paso 6, si los datos no cumplen los requisitos
    \item El sistema muestra mensajes de error específicos
    \item El usuario corrige los datos
    \item Regresa al paso 3
\end{enumerate}

\section{CU-02: Iniciar Sesión}

\noindent\textbf{Actor:} Administrador Financiero

\noindent\textbf{Descripción:} El administrador financiero accede al sistema utilizando sus credenciales de autenticación (correo/contraseña o cuenta de Google).

\noindent\textbf{Precondiciones:}
\begin{itemize}
    \item El sistema debe estar disponible
    \item El usuario debe tener una cuenta registrada en el sistema
    \item Para autenticación con Google, el usuario debe tener una cuenta de Google válida
\end{itemize}

\noindent\textbf{Postcondiciones:}
\begin{itemize}
    \item El usuario queda autenticado en el sistema
    \item Se establece una sesión activa
    \item El usuario es redirigido al dashboard principal
\end{itemize}

\noindent\textbf{Flujo Principal:}
\begin{enumerate}
    \item El usuario accede a la pantalla de inicio de sesión
    \item El sistema presenta las opciones de autenticación:
    \begin{itemize}
        \item Correo electrónico y contraseña
        \item Iniciar sesión con Google
    \end{itemize}
    \item El usuario selecciona el método de autenticación deseado
    \item Si selecciona correo/contraseña:
    \begin{enumerate}
        \item Ingresa su correo electrónico
        \item Ingresa su contraseña
        \item Presiona el botón ``Iniciar Sesión''
    \end{enumerate}
    \item Si selecciona Google:
    \begin{enumerate}
        \item Presiona el botón ``Iniciar con Google''
        \item Es redirigido a la página de autenticación de Google
        \item Autoriza el acceso a la aplicación
    \end{enumerate}
    \item El sistema valida las credenciales
    \item El sistema establece la sesión del usuario
    \item El usuario es redirigido al dashboard principal
\end{enumerate}

\noindent\textbf{Flujos Alternativos:}

\textbf{FA-01: Credenciales inválidas}
\begin{enumerate}
    \item En el paso 6, si las credenciales son incorrectas
    \item El sistema muestra un mensaje de error
    \item El usuario regresa al paso 3
\end{enumerate}

\textbf{FA-02: Error de autenticación con Google}
\begin{enumerate}
    \item En el paso 5, si la autenticación con Google falla
    \item El sistema muestra un mensaje de error
    \item El usuario regresa al paso 2
\end{enumerate}

\section{CU-03: Recuperar Contraseña}

\noindent\textbf{Actor:} Administrador Financiero

\noindent\textbf{Descripción:} El usuario recupera su contraseña cuando la ha olvidado, mediante un enlace de restablecimiento enviado a su correo electrónico.

\noindent\textbf{Precondiciones:}
\begin{itemize}
    \item El sistema debe estar disponible
    \item El usuario debe tener una cuenta registrada
    \item El usuario debe tener acceso a su correo electrónico registrado
\end{itemize}

\noindent\textbf{Postcondiciones:}
\begin{itemize}
    \item Se genera un enlace temporal de restablecimiento
    \item Se envía el enlace al correo del usuario
    \item El usuario puede crear una nueva contraseña
\end{itemize}

\noindent\textbf{Flujo Principal:}
\begin{enumerate}
    \item El usuario accede a la pantalla de inicio de sesión
    \item El usuario presiona ``¿Olvidaste tu contraseña?''
    \item El sistema presenta el formulario de recuperación
    \item El usuario ingresa su correo electrónico registrado
    \item El usuario presiona ``Enviar enlace de recuperación''
    \item El sistema valida que el correo esté registrado
    \item El sistema genera un token temporal de recuperación
    \item El sistema envía correo con enlace de restablecimiento
    \item El sistema muestra mensaje de confirmación
    \item El usuario accede al enlace desde su correo
    \item El sistema presenta formulario de nueva contraseña
    \item El usuario ingresa su nueva contraseña
    \item El usuario confirma la nueva contraseña
    \item El sistema actualiza la contraseña
    \item El sistema muestra confirmación de cambio exitoso
\end{enumerate}

\noindent\textbf{Flujos Alternativos:}

\textbf{FA-01: Correo no registrado}
\begin{enumerate}
    \item En el paso 6, si el correo no está registrado
    \item El sistema muestra mensaje de error
    \item El usuario debe verificar el correo o registrarse
    \item Regresa al paso 4
\end{enumerate}

\textbf{FA-02: Enlace expirado}
\begin{enumerate}
    \item En el paso 10, si el enlace ha expirado
    \item El sistema muestra mensaje de enlace expirado
    \item El usuario debe solicitar un nuevo enlace
    \item Regresa al paso 1
\end{enumerate}

\section{CU-04: Registrar Ingreso}

\noindent\textbf{Actor:} Administrador Financiero

\noindent\textbf{Descripción:} El administrador financiero registra un ingreso de dinero en el sistema, especificando el monto y agregando notas descriptivas.

\noindent\textbf{Precondiciones:}
\begin{itemize}
    \item El usuario debe estar autenticado en el sistema
    \item Debe existir un ingreso real de dinero para registrar
\end{itemize}

\noindent\textbf{Postcondiciones:}
\begin{itemize}
    \item El ingreso queda registrado en el sistema
    \item Se actualiza el balance general del usuario
    \item El registro es visible en el historial de transacciones
\end{itemize}

\noindent\textbf{Flujo Principal:}
\begin{enumerate}
    \item El usuario accede a la opción ``Registrar Ingreso'' desde el dashboard
    \item El sistema presenta el formulario de registro de ingreso
    \item El usuario ingresa la información requerida:
    \begin{itemize}
        \item Monto del ingreso
        \item Fecha del ingreso
        \item Fuente del ingreso (opcional)
        \item Notas descriptivas (opcional)
    \end{itemize}
    \item El usuario confirma la información ingresada
    \item El sistema valida los datos ingresados
    \item El sistema registra el ingreso en la base de datos
    \item El sistema actualiza el balance del usuario
    \item El sistema muestra mensaje de confirmación
    \item El usuario regresa al dashboard actualizado
\end{enumerate}

\noindent\textbf{Flujos Alternativos:}

\textbf{FA-01: Datos inválidos}
\begin{enumerate}
    \item En el paso 5, si los datos no son válidos (monto negativo, fechas invalidas)
    \item El sistema muestra mensaje de error específico
    \item El sistema marca los campos con errores
    \item El usuario corrige los datos y regresa al paso 4
\end{enumerate}

\section{CU-05: Registrar Egreso}

\noindent\textbf{Actor:} Administrador Financiero

\noindent\textbf{Descripción:} El administrador financiero inicia el proceso de registro de un egreso, seleccionando entre las opciones disponibles: registro manual o captura de ticket.

\noindent\textbf{Precondiciones:}
\begin{itemize}
    \item El usuario debe estar autenticado en el sistema
    \item Debe existir un gasto real para registrar
\end{itemize}

\noindent\textbf{Postcondiciones:}
\begin{itemize}
    \item Se inicia el flujo de registro correspondiente (manual o por foto)
    \item El usuario es dirigido a la pantalla de confirmación de datos
\end{itemize}

\noindent\textbf{Flujo Principal:}
\begin{enumerate}
    \item El usuario presiona el botón ``Registrar Egreso'' desde el dashboard
    \item El sistema presenta las opciones de registro:
    \begin{itemize}
        \item Registro Manual
        \item Tomar Foto de Ticket
    \end{itemize}
    \item El usuario selecciona el método deseado
    \item Si selecciona ``Registro Manual'': se ejecuta CU-06
    \item Si selecciona ``Tomar Foto de Ticket'': se ejecuta CU-07
\end{enumerate}

\noindent\textbf{Flujos Alternativos:}

\textbf{FA-01: Cancelar operación}
\begin{enumerate}
    \item En cualquier momento del proceso
    \item El usuario puede cancelar y regresar al dashboard
\end{enumerate}

\section{CU-06: Registrar Egreso Manualmente}

\noindent\textbf{Actor:} Administrador Financiero

\noindent\textbf{Descripción:} El administrador financiero registra manualmente un egreso ingresando todos los datos del gasto, incluyendo artículos, costos y lugar de compra.

\noindent\textbf{Precondiciones:}
\begin{itemize}
    \item El usuario debe estar autenticado en el sistema
    \item Se debe haber seleccionado ``Registro Manual'' en CU-03
\end{itemize}

\noindent\textbf{Postcondiciones:}
\begin{itemize}
    \item El egreso queda registrado en el sistema
    \item Se actualiza el balance general del usuario
    \item Se actualizan las estadísticas por categoría
    \item El registro es visible en el historial de transacciones
\end{itemize}

\noindent\textbf{Flujo Principal:}
\begin{enumerate}
    \item El sistema presenta el formulario de registro manual
    \item El usuario ingresa la información básica:
    \begin{itemize}
        \item Lugar de compra
        \item Fecha y hora de la transacción
        \item Monto total
    \end{itemize}
    \item El usuario agrega los artículos comprados:
    \begin{itemize}
        \item Nombre del artículo
        \item Cantidad
        \item Precio unitario
        \item Precio total
    \end{itemize}
    \item El sistema realiza categorización automática de artículos
    \item El usuario revisa y confirma las categorías asignadas
    \item El usuario puede agregar una etiqueta personalizada (opcional)
    \item El usuario confirma toda la información
    \item El sistema valida los datos y calcula totales
    \item El sistema registra el egreso en la base de datos
    \item El sistema actualiza balances y estadísticas
    \item El sistema muestra mensaje de confirmación
    \item El usuario regresa al dashboard actualizado
\end{enumerate}

\noindent\textbf{Flujos Alternativos:}

\textbf{FA-01: Datos inválidos}
\begin{enumerate}
    \item En el paso 3, si los datos no son válidos
    \item El sistema muestra mensaje de error específico
    \item El sistema marca los campos con errores
    \item El usuario corrige los datos y regresa al paso 3
\end{enumerate}

\textbf{FA-02: Error en categorización automática}
\begin{enumerate}
    \item En el paso 4, si no se puede categorizar automáticamente
    \item El sistema solicita al usuario categorizar manualmente
    \item El usuario selecciona la categoría apropiada
    \item Continúa en el paso 7
\end{enumerate}

\section{CU-07: Tomar Foto de Ticket}

\noindent\textbf{Actor:} Administrador Financiero

\noindent\textbf{Descripción:} El administrador financiero registra un egreso tomando una o varias fotografías del ticket de compra, confirmando los datos escaneados por OCR.

\noindent\textbf{Precondiciones:}
\begin{itemize}
    \item El usuario debe estar autenticado en el sistema
    \item Se debe haber seleccionado ``Tomar Foto de Ticket'' en CU-03
    \item La cámara del dispositivo debe estar disponible
    \item Debe existir un ticket físico para fotografiar
\end{itemize}

\noindent\textbf{Postcondiciones:}
\begin{itemize}
    \item El egreso queda registrado en el sistema
    \item Se almacenan las fotografías del ticket
    \item Se actualiza el balance general del usuario
    \item Se actualizan las estadísticas por categoría
    \item El registro es visible en el historial de transacciones
    \item Se asigna la etiqueta opcional si fue proporcionada
\end{itemize}

\noindent\textbf{Flujo Principal:}
\begin{enumerate}
    \item El sistema activa la cámara del dispositivo
    \item El usuario toma una o varias fotografías del ticket
    \item El sistema procesa las imágenes con OCR
    \item El sistema extrae la información del ticket:
    \begin{itemize}
        \item Lugar de compra
        \item Fecha y hora
        \item Lista de artículos con precios
        \item Monto total
    \end{itemize}
    \item El sistema presenta los datos extraídos para confirmación
    \item El usuario revisa y edita los datos si es necesario
    \item El sistema realiza categorización automática de artículos
    \item El usuario revisa y confirma las categorías asignadas
    \item El usuario puede agregar una etiqueta personalizada (opcional)
    \item El usuario confirma toda la información
    \item El sistema valida los datos y calcula totales
    \item El sistema registra el egreso en la base de datos
    \item El sistema almacena las fotografías del ticket
    \item El sistema actualiza balances y estadísticas
    \item El sistema muestra mensaje de confirmación
    \item El usuario regresa al dashboard actualizado
\end{enumerate}

\noindent\textbf{Flujos Alternativos:}

\textbf{FA-01: OCR no funciona correctamente}
\begin{enumerate}
    \item En el paso 3, si el OCR no puede procesar las imágenes
    \item El sistema muestra mensaje informativo
    \item Se ejecuta CU-08: Registrar Ticket Manualmente
\end{enumerate}

\textbf{FA-02: Datos extraídos incorrectos}
\begin{enumerate}
    \item En el paso 6, si los datos extraídos son incorrectos
    \item El usuario corrige manualmente los datos
    \item Continúa en el paso 7
\end{enumerate}

\textbf{FA-03: Calidad de imagen insuficiente}
\begin{enumerate}
    \item En el paso 3, si la calidad de la imagen es insuficiente
    \item El sistema sugiere tomar una nueva fotografía
    \item El usuario regresa al paso 2
\end{enumerate}

\section{CU-08: Registrar Ticket Manualmente}

\noindent\textbf{Actor:} Administrador Financiero

\noindent\textbf{Descripción:} El administrador financiero registra manualmente los datos de un ticket cuando el OCR no funciona correctamente, digitando los artículos y costos mientras mantiene las fotografías del ticket.

\noindent\textbf{Precondiciones:}
\begin{itemize}
    \item El usuario debe estar autenticado en el sistema
    \item Deben existir fotografías del ticket previamente capturadas
    \item El OCR debe haber fallado en la extracción de datos
\end{itemize}

\noindent\textbf{Postcondiciones:}
\begin{itemize}
    \item El egreso queda registrado en el sistema con datos ingresados manualmente
    \item Se mantienen las fotografías del ticket asociadas
    \item Se actualiza el balance general del usuario
    \item Se actualizan las estadísticas por categoría
    \item El registro es visible en el historial de transacciones
    \item Se asigna la etiqueta opcional si fue proporcionada
\end{itemize}

\noindent\textbf{Flujo Principal:}
\begin{enumerate}
    \item El sistema presenta un formulario de entrada manual con las fotografías del ticket visibles
    \item El usuario consulta las fotografías para extraer la información
    \item El usuario ingresa la información del ticket:
    \begin{itemize}
        \item Lugar de compra
        \item Fecha y hora de la transacción
        \item Lista de artículos comprados
        \item Precios individuales y cantidades
        \item Monto total
    \end{itemize}
    \item El usuario confirma que la suma de artículos coincide con el total
    \item El sistema realiza categorización automática de artículos
    \item El usuario revisa y confirma las categorías asignadas
    \item El usuario puede agregar una etiqueta personalizada (opcional)
    \item El usuario confirma toda la información
    \item El sistema valida los datos y calcula totales
    \item El sistema registra el egreso en la base de datos
    \item El sistema mantiene la asociación con las fotografías del ticket
    \item El sistema actualiza balances y estadísticas
    \item El sistema muestra mensaje de confirmación
    \item El usuario regresa al dashboard actualizado
\end{enumerate}

\noindent\textbf{Flujos Alternativos:}

\textbf{FA-01: Datos inválidos}
\begin{enumerate}
    \item En el paso 4, si los datos ingresados no son válidos
    \item El sistema muestra mensaje de error específico
    \item El sistema marca los campos con errores
    \item El usuario corrige los datos y regresa al paso 4
\end{enumerate}

\textbf{FA-02: Totales no coinciden}
\begin{enumerate}
    \item En el paso 4, si la suma de artículos no coincide con el total
    \item El sistema muestra mensaje de advertencia
    \item El usuario debe revisar y corregir los importes
    \item Regresa al paso 4
\end{enumerate}

\section{CU-09: Ver Historial de Notificaciones}

\noindent\textbf{Actor:} Administrador Financiero

\noindent\textbf{Descripción:} El administrador financiero consulta el historial completo de notificaciones recibidas del sistema, con capacidad de filtrado por tipo de alerta para revisar alertas de presupuestos, recordatorios de deudas y otras notificaciones importantes.

\noindent\textbf{Precondiciones:}
\begin{itemize}
    \item El usuario debe estar autenticado en el sistema
    \item Deben existir notificaciones previas en el sistema
    \item El usuario debe tener presupuestos configurados y alertas activas
\end{itemize}

\noindent\textbf{Postcondiciones:}
\begin{itemize}
    \item Se muestra el historial de notificaciones según los filtros aplicados
    \item Las notificaciones consultadas se marcan como leídas
    \item Se actualiza el contador de notificaciones pendientes
\end{itemize}

\noindent\textbf{Flujo Principal:}
\begin{enumerate}
    \item El usuario accede a la opción ``Historial de Notificaciones'' desde el menú principal
    \item El sistema muestra la lista completa de notificaciones ordenadas por fecha
    \item El usuario puede aplicar filtros:
    \begin{itemize}
        \item Por tipo de notificación (presupuesto, deuda, recordatorio)
        \item Por rango de fechas
        \item Por estado (leída/no leída)
    \end{itemize}
    \item El sistema actualiza la lista según los filtros aplicados
    \item El usuario puede seleccionar notificaciones individuales para ver detalles
    \item El sistema muestra información detallada de la notificación seleccionada
    \item El usuario puede marcar notificaciones como leídas/no leídas
    \item El usuario puede eliminar notificaciones individuales o en lote
\end{enumerate}

\noindent\textbf{Flujos Alternativos:}

\textbf{FA-01: Sin notificaciones disponibles}
\begin{enumerate}
    \item En el paso 2, si no existen notificaciones en el sistema
    \item El sistema muestra mensaje informativo
    \item Se sugiere al usuario configurar alertas y presupuestos
\end{enumerate}

\textbf{FA-02: Filtro sin resultados}
\begin{enumerate}
    \item En el paso 5, si el filtro aplicado no devuelve resultados
    \item El sistema muestra mensaje de ``sin resultados''
    \item El usuario puede cambiar el filtro o ver todas las notificaciones
\end{enumerate}

\section{CU-10: Visualizar Estadísticas por Categoría}

\noindent\textbf{Actor:} Administrador Financiero

\noindent\textbf{Descripción:} El administrador financiero visualiza estadísticas detalladas de sus gastos organizadas por categorías, incluyendo gráficos comparativos de presupuesto vs gasto real, análisis de productos por categoría y seguimiento mensual de patrones de consumo.

\noindent\textbf{Precondiciones:}
\begin{itemize}
    \item El usuario debe estar autenticado en el sistema
    \item Deben existir transacciones registradas en el sistema
    \item Debe tener presupuestos asignados por categorías
\end{itemize}

\noindent\textbf{Postcondiciones:}
\begin{itemize}
    \item Se visualizan las estadísticas actualizadas por categoría
    \item Se muestran comparativos con períodos anteriores
    \item El usuario obtiene insights sobre sus patrones de gasto
\end{itemize}

\noindent\textbf{Flujo Principal:}
\begin{enumerate}
    \item El usuario accede a la opción ``Estadísticas por Categoría'' desde el dashboard
    \item El sistema presenta un resumen general con todas las categorías
    \item El usuario puede seleccionar una categoría específica para análisis detallado
    \item El sistema muestra estadísticas detalladas de la categoría:
    \begin{itemize}
        \item Gráfico de presupuesto vs gasto real
        \item Lista de productos más comprados
        \item Tendencia de gastos por mes
        \item Comparativo con periodos anteriores
    \end{itemize}
    \item El usuario puede cambiar el período de análisis (mensual, trimestral, anual)
    \item El sistema actualiza las estadísticas según el período seleccionado
    \item El usuario puede exportar los reportes estadísticos
    \item El sistema actualiza automáticamente las estadísticas con cada nueva transacción
\end{enumerate}

\noindent\textbf{Flujos Alternativos:}

\textbf{FA-01: Sin datos suficientes}
\begin{enumerate}
    \item En el paso 2, si no hay suficientes datos para generar estadísticas
    \item El sistema muestra mensaje informativo
    \item Se sugiere registrar más transacciones para obtener análisis completos
\end{enumerate}

\textbf{FA-02: Categoría sin movimientos}
\begin{enumerate}
    \item En el paso 4, si la categoría seleccionada no tiene transacciones
    \item El sistema muestra mensaje específico
    \item Se muestra solo el presupuesto asignado y se sugiere comenzar a registrar gastos
\end{enumerate}

\section{CU-11: Modificar Categoría de Gastos}

\noindent\textbf{Actor:} Administrador Financiero

\noindent\textbf{Descripción:} El administrador financiero modifica las configuraciones de sus categorías de gastos, incluyendo cambios de presupuesto, reasignación de productos, modificación de nombres y eliminación de productos mal categorizados.

\noindent\textbf{Precondiciones:}
\begin{itemize}
    \item El usuario debe estar autenticado en el sistema
    \item Deben existir categorías previamente configuradas
    \item El usuario debe tener permisos para modificar categorías
\end{itemize}

\noindent\textbf{Postcondiciones:}
\begin{itemize}
    \item La categoría queda actualizada con las nuevas configuraciones
    \item Se actualizan las estadísticas y presupuestos relacionados
    \item Los productos reasignados cambian de categoría
    \item Se registra el cambio en el log de auditoría
\end{itemize}

\noindent\textbf{Flujo Principal:}
\begin{enumerate}
    \item El usuario accede a ``Gestión de Categorías'' desde el menú de configuración
    \item El sistema muestra la lista de categorías existentes
    \item El usuario selecciona la categoría a modificar
    \item El sistema presenta el formulario de edición con los datos actuales
    \item El usuario puede modificar:
    \begin{itemize}
        \item Nombre de la categoría
        \item Presupuesto asignado
        \item Color de identificación
        \item Productos asignados a la categoría
    \end{itemize}
    \item El sistema valida los cambios realizados
    \item Si existe reasignación de productos:
    \begin{enumerate}
        \item El sistema muestra lista de productos a reasignar
        \item El usuario confirma las reasignaciones
        \item El sistema actualiza las transacciones históricas afectadas
    \end{enumerate}
    \item El usuario confirma todos los cambios
    \item El sistema actualiza la categoría y sus relaciones
    \item El sistema recalcula estadísticas y presupuestos
    \item Se muestra mensaje de confirmación con resumen de cambios aplicados
\end{enumerate}

\noindent\textbf{Flujos Alternativos:}

\textbf{FA-01: Nombre duplicado}
\begin{enumerate}
    \item En el paso 6, si el nuevo nombre ya existe en otra categoría
    \item El sistema muestra mensaje de error
    \item El usuario debe ingresar un nombre diferente
    \item Regresa al paso 5
\end{enumerate}

\textbf{FA-02: Presupuesto inválido}
\begin{enumerate}
    \item En el paso 6, si el presupuesto ingresado es menor o igual a cero
    \item El sistema muestra mensaje de error
    \item El usuario debe corregir el valor del presupuesto
    \item Continúa en el paso 6
\end{enumerate}

\textbf{FA-03: Error en reasignación}
\begin{enumerate}
    \item En el paso 8, si ocurre un error al reasignar productos
    \item El sistema muestra mensaje de error detallado
    \item Se revierten los cambios parciales realizados
    \item El usuario puede reintentar la operación
\end{enumerate}

\section{CU-12: Planificar Pago de Deudas}

\noindent\textbf{Actor:} Administrador Financiero

\noindent\textbf{Descripción:} El administrador financiero visualiza y organiza sus deudas pendientes, recibe sugerencias del sistema para estrategias de pago optimizadas y planifica el orden y cronograma de pagos para minimizar intereses y cumplir con los compromisos financieros.

\noindent\textbf{Precondiciones:}
\begin{itemize}
    \item El usuario debe estar autenticado en el sistema
    \item Deben existir deudas registradas en el sistema
    \item Las deudas deben tener información completa (monto, tasa de interés, fecha de vencimiento)
\end{itemize}

\noindent\textbf{Postcondiciones:}
\begin{itemize}
    \item Se genera un plan de pago optimizado para las deudas
    \item Se establecen recordatorios automáticos para los pagos
    \item Se calcula el cronograma de liberación de deudas
    \item Se actualiza el dashboard con el cronograma de pagos
\end{itemize}

\noindent\textbf{Flujo Principal:}
\begin{enumerate}
    \item El usuario accede a la opción ``Planificar Pago de Deudas'' desde el menú principal
    \item El sistema muestra la lista de todas las deudas pendientes
    \item El usuario ingresa el monto mensual disponible para pago de deudas
    \item El sistema calcula y presenta el plan de pagos optimizado:
    \begin{itemize}
        \item Orden de pago de deudas
        \item Montos mensuales por deuda
        \item Tiempo estimado para liquidar cada deuda
        \item Total de intereses a pagar
        \item Fecha estimada de liberación total
    \end{itemize}
    \item El usuario revisa y ajusta el plan si es necesario
    \item El usuario confirma y activa el plan de pagos
    \item El sistema establece recordatorios automáticos
    \item El sistema registra el plan en el cronograma personal
    \item Se genera un calendario de pagos visible en el dashboard
\end{enumerate}

\noindent\textbf{Flujos Alternativos:}

\textbf{FA-01: Monto insuficiente para pagos mínimos}
\begin{enumerate}
    \item En el paso 6, si el monto disponible no cubre los pagos mínimos requeridos
    \item El sistema muestra alerta con ña falta calculada
    \item Se sugiere revisar el presupuesto o consolidar deudas
    \item El usuario debe ajustar el monto o revisar las deudas registradas
\end{enumerate}

\textbf{FA-02: Sin deudas registradas}
\begin{enumerate}
    \item En el paso 2, si no existen deudas registradas en el sistema
    \item El sistema muestra mensaje informativo
    \item Se ofrece la opción de registrar deudas
    \item Se redirige al caso de uso CU-15: Registrar Deudas
\end{enumerate}

\section{CU-13: Establecer Alertas de Pago de Deudas}

\noindent\textbf{Actor:} Administrador Financiero

\noindent\textbf{Descripción:} El administrador financiero configura alertas y recordatorios automáticos para los pagos de sus deudas, estableciendo diferentes tipos de notificaciones para evitar pagos tardíos y cargos por intereses adicionales.

\noindent\textbf{Precondiciones:}
\begin{itemize}
    \item El usuario debe estar autenticado en el sistema
    \item Deben existir deudas registradas con fechas de vencimiento
    \item Las notificaciones deben estar habilitadas en el dispositivo
\end{itemize}

\noindent\textbf{Postcondiciones:}
\begin{itemize}
    \item Las alertas quedan configuradas y activas para las deudas seleccionadas
    \item Se programan las notificaciones según los intervalos establecidos
    \item Se registran las preferencias de alerta del usuario
\end{itemize}

\noindent\textbf{Flujo Principal:}
\begin{enumerate}
    \item El usuario accede a ``Configurar Alertas de Deudas'' desde el menú de deudas
    \item El sistema muestra la lista de deudas disponibles para configurar alertas
    \item El usuario selecciona las deudas para las cuales desea alertas
    \item Para cada deuda seleccionada, el usuario configura:
    \begin{itemize}
        \item Días de anticipación para la alerta
        \item Tipo de notificación
        \item Recordatorios recurrentes
    \end{itemize}
    \item El sistema valida que las fechas de vencimiento estén definidas
    \item El usuario confirma la configuración de alertas
    \item El sistema programa las notificaciones automáticas
    \item Se muestra confirmación con resumen de alertas configuradas
    \item El sistema comienza a enviar notificaciones según la programación establecida
\end{enumerate}

\noindent\textbf{Flujos Alternativos:}

\textbf{FA-01: Notificaciones deshabilitadas}
\begin{enumerate}
    \item En cualquier paso, si las notificaciones están deshabilitadas en el dispositivo
    \item El sistema detecta la restricción y muestra alerta
    \item El usuario debe habilitar notificaciones para continuar
\end{enumerate}

\textbf{FA-02: Deuda sin fecha de vencimiento}
\begin{enumerate}
    \item En el paso 4, si una deuda no tiene fecha de vencimiento definida
    \item El sistema solicita establecer la fecha de vencimiento
    \item El usuario ingresa la fecha requerida
    \item Una vez establecida la fecha, regresa al paso 4
\end{enumerate}

\section{CU-14: Configurar Deudas}

\noindent\textbf{Actor:} Administrador Financiero

\noindent\textbf{Descripción:} El administrador financiero modifica y actualiza la información de deudas existentes, incluyendo ajustes de montos, plazos y tasas de interés periódicos para mantener la información actualizada y precisa.

\noindent\textbf{Precondiciones:}
\begin{itemize}
    \item El usuario debe estar autenticado en el sistema
    \item Deben existir deudas previamente registradas
\end{itemize}

\noindent\textbf{Postcondiciones:}
\begin{itemize}
    \item La información de la deuda queda actualizada en el sistema
    \item Se recalculan los planes de pago asociados si existen
    \item Se actualizan las alertas programadas relacionadas
    \item Se registra el cambio en el historial de modificaciones
\end{itemize}

\noindent\textbf{Flujo Principal:}
\begin{enumerate}
    \item El usuario accede a ``Gestión de Deudas'' desde el menú principal
    \item El sistema muestra la lista de deudas registradas
    \item El usuario selecciona la deuda a modificar
    \item El sistema presenta el formulario de edición con los datos actuales
    \item El usuario puede modificar:
    \begin{itemize}
        \item Nombre del acreedor
        \item Monto actual de la deuda
        \item Tasa de interés
        \item Fecha de vencimiento
        \item Pago mínimo mensual
        \item Notas adicionales
    \end{itemize}
    \item El sistema valida los datos ingresados
    \item Si los cambios afectan planes de pago existentes:
    \begin{enumerate}
        \item El sistema muestra advertencia sobre impacto en planes
        \item El usuario confirma que desea continuar
        \item El sistema recalcula automáticamente los planes afectados
    \end{enumerate}
    \item El usuario confirma todos los cambios
    \item El sistema actualiza la información de la deuda
    \item El sistema actualiza alertas y recordatorios asociados
    \item Se registra la modificación en el historial
    \item Se muestra mensaje de confirmación con resumen de cambios aplicados
\end{enumerate}

\noindent\textbf{Flujos Alternativos:}

\textbf{FA-01: Datos inválidos}
\begin{enumerate}
    \item En el paso 6, si los datos ingresados no son válidos
    \item El sistema muestra mensajes de error específicos
    \item Se marcan los campos con errores
    \item Regresa al paso 5 para realizar las correcciones
\end{enumerate}

\textbf{FA-02: Conflicto con plan de pagos activo}
\begin{enumerate}
    \item En el paso 9, si los cambios afectan un plan de pagos activo
    \item El sistema muestra opciones: mantener plan actual o recalcular
    \item El usuario decide cómo proceder con el plan de pagos
\end{enumerate}

\section{CU-15: Registrar Deudas}

\noindent\textbf{Actor:} Administrador Financiero

\noindent\textbf{Descripción:} El administrador financiero registra nuevas deudas en el sistema, capturando toda la información necesaria para su seguimiento y gestión, incluyendo detalles del acreedor, montos, plazos y condiciones de pago.

\noindent\textbf{Precondiciones:}
\begin{itemize}
    \item El usuario debe estar autenticado en el sistema
    \item Debe existir una deuda real para registrar
    \item El sistema debe tener espacio disponible para nuevas deudas
\end{itemize}

\noindent\textbf{Postcondiciones:}
\begin{itemize}
    \item La nueva deuda queda registrada y activa en el sistema
    \item Se incluye en los cálculos de planificación financiera
    \item Está disponible para configuración de alertas y planes de pago
    \item Se registra la fecha de creación en el historial
\end{itemize}

\noindent\textbf{Flujo Principal:}
\begin{enumerate}
    \item El usuario accede a la opción ``Registrar Nueva Deuda'' desde el menú de deudas
    \item El sistema presenta el formulario de registro de deuda
    \item El usuario ingresa la información requerida:
    \begin{itemize}
        \item Nombre del acreedor (banco, empresa, persona)
        \item Tipo de deuda (tarjeta de crédito, préstamo personal, hipoteca, etc.)
        \item Monto total de la deuda
        \item Tasa de interés anual
        \item Fecha de vencimiento o plazo
        \item Pago mínimo mensual requerido
        \item Fecha del primer pago
        \item Notas adicionales (opcional)
    \end{itemize}
    \item El usuario revisa toda la información ingresada
    \item El sistema valida los datos:
    \begin{itemize}
        \item Monto debe ser mayor a cero
        \item Fechas deben ser válidas y coherentes
        \item Tasa de interés debe ser un valor válido
    \end{itemize}
    \item El sistema verifica si existe una deuda similar ya registrada
    \item El usuario confirma el registro de la nueva deuda
    \item El sistema registra la deuda en la base de datos
    \item El sistema calcula proyecciones iniciales de pagos
    \item El sistema ofrece configurar alertas inmediatamente
    \item Se actualiza el dashboard con la nueva información
    \item Se muestra mensaje de confirmación con resumen de la deuda registrada
\end{enumerate}

\noindent\textbf{Flujos Alternativos:}

\textbf{FA-01: Información incompleta}
\begin{enumerate}
    \item En el paso 6, si faltan campos obligatorios
    \item El sistema marca los campos faltantes
    \item Se muestra mensaje indicando los datos requeridos
    \item Regresa al paso 3 para completar datos
\end{enumerate}

\textbf{FA-02: Deuda duplicada}
\begin{enumerate}
    \item En el paso 7, si ya existe una deuda similar registrada
    \item El sistema muestra la deuda existente para comparación
    \item El usuario puede confirmar que es una deuda diferente o cancelar el registro
\end{enumerate}

\textbf{FA-03: Monto inválido}
\begin{enumerate}
    \item En el paso 6, si el monto ingresado es menor o equal a cero
    \item El sistema muestra mensaje de error
    \item El usuario debe corregir el monto
    \item Continúa en el paso 5
\end{enumerate}

\section{CU-16: Configurar Presupuestos}

\noindent\textbf{Actor:} Administrador Financiero

\noindent\textbf{Descripción:} El administrador financiero configura presupuestos personalizados por categoría o período para controlar sus gastos según la metodología Dave Ramsey (máximo 10 categorías).

\noindent\textbf{Precondiciones:}
\begin{itemize}
    \item El usuario debe estar autenticado en el sistema
    \item Deben existir categorías de gastos creadas
    \item El usuario debe tener ingresos registrados para establecer presupuestos
\end{itemize}

\noindent\textbf{Postcondiciones:}
\begin{itemize}
    \item Los presupuestos quedan configurados y activos
    \item Se establecen límites de gasto por categoría
    \item Se activan alertas automáticas de límites
    \item El sistema comienza a monitorear los gastos contra el presupuesto
\end{itemize}

\noindent\textbf{Flujo Principal:}
\begin{enumerate}
    \item El usuario accede a ``Configuración de Presupuestos'' desde el menú principal
    \item El sistema muestra las categorías disponibles (máximo 10)
    \item Para cada categoría, el usuario:
    \begin{itemize}
        \item Establece el monto límite
        \item Define el porcentaje de alerta
        \item Configura acciones automáticas al exceder límites
    \end{itemize}
    \item El sistema valida que la suma no exceda los ingresos disponibles
    \item El usuario revisa el resumen del presupuesto
    \item El usuario confirma la configuración
    \item El sistema guarda los presupuestos
    \item Se muestra confirmación con resumen de presupuestos activos
\end{enumerate}

\noindent\textbf{Flujos Alternativos:}

\textbf{FA-01: Presupuesto excede ingresos}
\begin{enumerate}
    \item En el paso 5, si la suma de presupuestos excede los ingresos
    \item El sistema muestra alerta de presupuesto inviable
    \item El sistema sugiere ajustes automáticos
    \item El usuario debe reducir los montos asignados
    \item Regresa al paso 4
\end{enumerate}

\section{CU-17: Configurar Metas de Ahorro}

\noindent\textbf{Actor:} Administrador Financiero

\noindent\textbf{Descripción:} El administrador financiero establece metas de ahorro personalizadas con plazos y montos específicos, monitoreando su progreso mediante indicadores visuales.

\noindent\textbf{Precondiciones:}
\begin{itemize}
    \item El usuario debe estar autenticado en el sistema
    \item Debe tener presupuestos configurados
    \item Debe existir capacidad de ahorro en su presupuesto
\end{itemize}

\noindent\textbf{Postcondiciones:}
\begin{itemize}
    \item La meta de ahorro queda registrada y activa
    \item Se establece un plan de ahorro automático
    \item Se configuran recordatorios periódicos
    \item Se actualiza el dashboard con indicadores de progreso
\end{itemize}

\noindent\textbf{Flujo Principal:}
\begin{enumerate}
    \item El usuario accede a ``Metas de Ahorro'' desde el menú principal
    \item El usuario presiona ``Nueva Meta de Ahorro''
    \item El usuario ingresa la información de la meta:
    \begin{itemize}
        \item Nombre descriptivo de la meta
        \item Monto total objetivo
        \item Fecha límite para alcanzar la meta
        \item Prioridad de la meta (alta, media, baja)
    \end{itemize}
    \item El sistema calcula el ahorro mensual requerido
    \item El sistema verifica disponibilidad en el presupuesto
    \item El usuario revisa el plan de ahorro sugerido
    \item El usuario puede ajustar el plan o la fecha límite
    \item El usuario confirma la meta de ahorro
    \item El sistema registra la meta y el plan
    \item El sistema configura recordatorios automáticos
    \item Se muestra confirmación con detalles del plan de ahorro
\end{enumerate}

\noindent\textbf{Flujos Alternativos:}

\textbf{FA-01: Ahorro requerido excede capacidad}
\begin{enumerate}
    \item En el paso 5, si el ahorro mensual requerido excede la capacidad
    \item El sistema muestra alerta de meta no viable
    \item El sistema sugiere ajustar la fecha límite o el monto
    \item El usuario debe modificar los parámetros
    \item Regresa al paso 3
\end{enumerate}

\textbf{FA-02: Conflicto con metas existentes}
\begin{enumerate}
    \item En el paso 5, si la nueva meta conflicta con metas existentes
    \item El sistema muestra las metas que entran en conflicto
    \item El usuario debe repriorizar las metas
    \item El sistema recalcula los planes de ahorro
    \item Continúa en el paso 6
\end{enumerate}

\section{CU-18: Consultar Historial de Transacciones}

\noindent\textbf{Actor:} Administrador Financiero

\noindent\textbf{Descripción:} El administrador financiero consulta el historial completo de sus transacciones (ingresos y egresos) con capacidades de filtrado por fecha, categoría y tipo.

\noindent\textbf{Precondiciones:}
\begin{itemize}
    \item El usuario debe estar autenticado en el sistema
    \item Deben existir transacciones registradas en el sistema
\end{itemize}

\noindent\textbf{Postcondiciones:}
\begin{itemize}
    \item Se muestra el historial filtrado según los criterios seleccionados
    \item El usuario puede exportar los resultados
    \item Se actualiza la información de última consulta
\end{itemize}

\noindent\textbf{Flujo Principal:}
\begin{enumerate}
    \item El usuario accede a ``Historial de Transacciones'' desde el menú principal
    \item El sistema muestra todas las transacciones ordenadas por fecha (más recientes primero)
    \item El usuario puede aplicar filtros:
    \begin{itemize}
        \item Rango de fechas (desde/hasta)
        \item Tipo de transacción (ingreso/egreso/ambos)
        \item Categoría específica
        \item Rango de montos (mínimo/máximo)
        \item Método de registro (manual/OCR)
    \end{itemize}
    \item El sistema actualiza la lista según los filtros aplicados
    \item El usuario puede seleccionar transacciones individuales para ver detalles
    \item El sistema muestra información detallada incluyendo:
    \begin{itemize}
        \item Tickets asociados (si aplica)
        \item Categorización aplicada
        \item Fecha y hora exacta
        \item Método de registro utilizado
    \end{itemize}
    \item El usuario puede exportar los resultados en PDF o Excel
    \item El usuario puede editar transacciones individuales (si tiene permisos)
\end{enumerate}

\noindent\textbf{Flujos Alternativos:}

\textbf{FA-01: Sin transacciones en el período}
\begin{enumerate}
    \item En el paso 4, si no existen transacciones para los filtros aplicados
    \item El sistema muestra mensaje de ``sin resultados''
    \item El usuario puede ajustar los filtros o crear nueva transacción
    \item Se sugiere expandir el rango de fechas
\end{enumerate}

\textbf{FA-02: Error en exportación}
\begin{enumerate}
    \item En el paso 7, si ocurre error al exportar
    \item El sistema muestra mensaje de error específico
    \item El usuario puede reintentar con diferente formato
    \item Se ofrece opción de reporte simplificado
\end{enumerate}

\section{CU-19: Consultar Tickets}

\noindent\textbf{Actor:} Administrador Financiero

\noindent\textbf{Descripción:} El administrador financiero consulta los tickets registrados previamente, pudiendo filtrar por fecha, categoría, monto y visualizar las imágenes asociadas.

\noindent\textbf{Precondiciones:}
\begin{itemize}
    \item El usuario debe estar autenticado en el sistema
    \item Deben existir tickets registrados en el sistema
\end{itemize}

\noindent\textbf{Postcondiciones:}
\begin{itemize}
    \item Se muestran los tickets según los filtros aplicados
    \item El usuario puede acceder a las imágenes originales
    \item Se actualiza la información de última consulta
\end{itemize}

\noindent\textbf{Flujo Principal:}
\begin{enumerate}
    \item El usuario accede a ``Consultar Tickets'' desde el menú principal
    \item El sistema muestra todos los tickets con vista previa
    \item El usuario puede aplicar filtros:
    \begin{itemize}
        \item Fecha de registro
        \item Establecimiento/lugar
        \item Categoría de productos
        \item Rango de montos
        \item Método de procesamiento (OCR exitoso/manual)
    \end{itemize}
    \item El sistema actualiza la lista según los filtros
    \item El usuario selecciona un ticket para ver detalles completos
    \item El sistema muestra:
    \begin{itemize}
        \item Imágenes originales del ticket
        \item Datos extraídos o ingresados manualmente
        \item Lista detallada de productos
        \item Categorización aplicada
        \item Fecha y método de registro
    \end{itemize}
    \item El usuario puede:
    \begin{itemize}
        \item Descargar las imágenes originales
        \item Editar la información del ticket
        \item Recategorizar productos
        \item Agregar notas adicionales
    \end{itemize}
\end{enumerate}

\noindent\textbf{Flujos Alternativos:}

\textbf{FA-01: Imagen no disponible}
\begin{enumerate}
    \item En el paso 6, si la imagen original no está disponible
    \item El sistema muestra mensaje informativo
    \item Se muestran solo los datos extraídos
    \item Se ofrece opción de subir nueva imagen
\end{enumerate}

\textbf{FA-02: Sin tickets registrados}
\begin{enumerate}
    \item En el paso 2, si no existen tickets registrados
    \item El sistema muestra mensaje informativo
    \item Se ofrece opción de registrar primer ticket
    \item Se redirige a CU-07: Tomar Foto de Ticket
\end{enumerate}

\section{CU-20: Eliminar Tickets}

\noindent\textbf{Actor:} Administrador Financiero

\noindent\textbf{Descripción:} El administrador financiero elimina tickets registrados, ya sea de forma individual o masiva, con confirmaciones previas para evitar pérdidas accidentales.

\noindent\textbf{Precondiciones:}
\begin{itemize}
    \item El usuario debe estar autenticado en el sistema
    \item Deben existir tickets registrados para eliminar
    \item El usuario debe tener permisos de eliminación
\end{itemize}

\noindent\textbf{Postcondiciones:}
\begin{itemize}
    \item Los tickets seleccionados se eliminan permanentemente
    \item Se actualizan las estadísticas y balances
    \item Se registra la acción en el log de auditoría
    \item Se liberan los recursos de almacenamiento
\end{itemize}

\noindent\textbf{Flujo Principal:}
\begin{enumerate}
    \item El usuario accede a la funcionalidad desde ``Consultar Tickets''
    \item El usuario selecciona los tickets a eliminar:
    \begin{itemize}
        \item Selección individual con checkbox
        \item Selección múltiple con filtros
        \item Selección de rango por fechas
    \end{itemize}
    \item El usuario presiona ``Eliminar Seleccionados''
    \item El sistema muestra resumen de tickets a eliminar:
    \begin{itemize}
        \item Cantidad de tickets
        \item Monto total afectado
        \item Impacto en estadísticas
        \item Advertencia de acción irreversible
    \end{itemize}
    \item El usuario confirma la eliminación ingresando su contraseña
    \item El sistema valida la autorización
    \item El sistema elimina los tickets y sus imágenes asociadas
    \item El sistema recalcula balances y estadísticas
    \item El sistema registra la acción en el log
    \item Se muestra confirmación de eliminación exitosa
\end{enumerate}

\noindent\textbf{Flujos Alternativos:}

\textbf{FA-01: Cancelar eliminación}
\begin{enumerate}
    \item En cualquier momento antes del paso 6
    \item El usuario puede cancelar la operación
    \item No se realizan cambios en el sistema
    \item El usuario regresa a la consulta de tickets
\end{enumerate}

\textbf{FA-02: Contraseña incorrecta}
\begin{enumerate}
    \item En el paso 6, si la contraseña es incorrecta
    \item El sistema muestra mensaje de error
    \item El usuario puede reintentar (máximo 3 veces)
    \item Después de 3 intentos fallidos, se cancela la operación
\end{enumerate}

\textbf{FA-03: Error en eliminación}
\begin{enumerate}
    \item En el paso 7, si ocurre error durante la eliminación
    \item El sistema muestra mensaje de error específico
    \item Se revierten los cambios parciales
    \item El usuario puede reintentar la operación
\end{enumerate}

\section{CU-21: Editar Perfil}

\noindent\textbf{Actor:} Administrador Financiero

\noindent\textbf{Descripción:} El administrador financiero modifica su información personal como nombre, correo electrónico o fotografía de perfil desde la aplicación o la web.

\noindent\textbf{Precondiciones:}
\begin{itemize}
    \item El usuario debe estar autenticado en el sistema
    \item Debe existir un perfil de usuario previamente registrado
\end{itemize}

\noindent\textbf{Postcondiciones:}
\begin{itemize}
    \item La información del perfil queda actualizada
    \item Los cambios se reflejan en toda la aplicación
    \item Se registra la modificación en el log de actividad
\end{itemize}

\noindent\textbf{Flujo Principal:}
\begin{enumerate}
    \item El usuario accede a ``Configuración de Perfil'' desde el menú
    \item El sistema presenta el formulario con datos actuales
    \item El usuario puede modificar:
    \begin{itemize}
        \item Nombre completo
        \item Correo electrónico
        \item Fotografía de perfil
        \item Preferencias de notificación
    \end{itemize}
    \item El usuario confirma los cambios
    \item El sistema valida los nuevos datos
    \item El sistema actualiza la información del perfil
    \item Se muestra confirmación de cambios exitosos
\end{enumerate}

\noindent\textbf{Flujos Alternativos:}

\textbf{FA-01: Correo ya en uso}
\begin{enumerate}
    \item En el paso 5, si el nuevo correo ya está registrado
    \item El sistema muestra mensaje de error
    \item El usuario debe usar un correo diferente
    \item Regresa al paso 3
\end{enumerate}

\section{CU-22: Eliminar Cuenta}

\noindent\textbf{Actor:} Administrador Financiero

\noindent\textbf{Descripción:} El administrador financiero elimina permanentemente su cuenta y todos los datos asociados, con confirmaciones previas para prevenir eliminaciones accidentales.

\noindent\textbf{Precondiciones:}
\begin{itemize}
    \item El usuario debe estar autenticado en el sistema
    \item No deben existir transacciones pendientes o planes de pago activos
\end{itemize}

\noindent\textbf{Postcondiciones:}
\begin{itemize}
    \item La cuenta del usuario es eliminada permanentemente
    \item Todos los datos asociados son eliminados
    \item Se cierra la sesión activa
    \item Se envía confirmación por correo electrónico
\end{itemize}

\noindent\textbf{Flujo Principal:}
\begin{enumerate}
    \item El usuario accede a ``Configuración de Cuenta''
    \item El usuario selecciona ``Eliminar Cuenta''
    \item El sistema muestra advertencia sobre permanencia de la acción
    \item El sistema solicita confirmación ingresando contraseña
    \item El usuario confirma ingresando su contraseña
    \item El sistema valida la contraseña
    \item El sistema elimina todos los datos del usuario
    \item El sistema cierra la sesión
    \item Se envía correo de confirmación de eliminación
\end{enumerate}

\noindent\textbf{Flujos Alternativos:}

\textbf{FA-01: Contraseña incorrecta}
\begin{enumerate}
    \item En el paso 6, si la contraseña es incorrecta
    \item El sistema muestra mensaje de error
    \item El usuario puede reintentar o cancelar
    \item Regresa al paso 5
\end{enumerate}

\section{CU-23: Visualizar Gráficas}

\noindent\textbf{Actor:} Administrador Financiero

\noindent\textbf{Descripción:} El administrador financiero visualiza gráficas y estadísticas interactivas que representan su comportamiento financiero de manera clara y comprensible.

\noindent\textbf{Precondiciones:}
\begin{itemize}
    \item El usuario debe estar autenticado en el sistema
    \item Deben existir transacciones registradas para generar gráficas
\end{itemize}

\noindent\textbf{Postcondiciones:}
\begin{itemize}
    \item Se muestran gráficas actualizadas del comportamiento financiero
    \item El usuario puede exportar o compartir las gráficas
    \item Se actualiza la información de última consulta
\end{itemize}

\noindent\textbf{Flujo Principal:}
\begin{enumerate}
    \item El usuario accede a ``Gráficas y Estadísticas''
    \item El sistema presenta diferentes tipos de gráficas:
    \begin{itemize}
        \item Gráfica circular de gastos por categoría
        \item Gráfica de barras de ingresos vs egresos
        \item Línea de tendencia temporal
        \item Gráfica de progreso de metas
    \end{itemize}
    \item El usuario puede personalizar las gráficas:
    \begin{itemize}
        \item Seleccionar período de tiempo
        \item Filtrar por categorías
        \item Cambiar tipo de visualización
    \end{itemize}
    \item El sistema actualiza las gráficas según selección
    \item El usuario puede exportar gráficas en PDF o imagen
\end{enumerate}

\section{CU-24: Generar Reportes PDF}

\noindent\textbf{Actor:} Administrador Financiero

\noindent\textbf{Descripción:} El administrador financiero genera reportes financieros en formato PDF para descarga o visualización, tanto desde la aplicación móvil como desde la plataforma web.

\noindent\textbf{Precondiciones:}
\begin{itemize}
    \item El usuario debe estar autenticado en el sistema
    \item Deben existir datos financieros para incluir en el reporte
\end{itemize}

\noindent\textbf{Postcondiciones:}
\begin{itemize}
    \item Se genera un reporte PDF completo
    \item El reporte está disponible para descarga
    \item Se registra la generación del reporte
\end{itemize}

\noindent\textbf{Flujo Principal:}
\begin{enumerate}
    \item El usuario accede a ``Generar Reportes''
    \item El usuario selecciona tipo de reporte:
    \begin{itemize}
        \item Reporte mensual completo
        \item Reporte por categorías
        \item Reporte de metas y ahorro
        \item Reporte de deudas y pagos
    \end{itemize}
    \item El usuario configura parámetros:
    \begin{itemize}
        \item Período de tiempo
        \item Categorías a incluir
        \item Nivel de detalle
    \end{itemize}
    \item El sistema genera el reporte PDF
    \item El sistema presenta opciones:
    \begin{itemize}
        \item Descargar archivo
        \item Visualizar en pantalla
        \item Enviar por correo
    \end{itemize}
\end{enumerate}

\section{CU-25: Cambiar Tema Aplicación}

\noindent\textbf{Actor:} Administrador Financiero

\noindent\textbf{Descripción:} El administrador financiero cambia el tema visual de la aplicación móvil entre modo claro y oscuro para mejorar la experiencia de uso.

\noindent\textbf{Precondiciones:}
\begin{itemize}
    \item El usuario debe estar autenticado en la aplicación móvil
    \item La aplicación debe soportar múltiples temas
\end{itemize}

\noindent\textbf{Postcondiciones:}
\begin{itemize}
    \item El tema visual de la aplicación cambia inmediatamente
    \item La preferencia se guarda para futuras sesiones
    \item Todos los elementos de la interfaz se actualizan
\end{itemize}

\noindent\textbf{Flujo Principal:}
\begin{enumerate}
    \item El usuario accede a ``Configuración'' en la app móvil
    \item El usuario selecciona ``Apariencia''
    \item El sistema muestra opciones de tema:
    \begin{itemize}
        \item Modo claro
        \item Modo oscuro
        \item Automático (según sistema)
    \end{itemize}
    \item El usuario selecciona el tema deseado
    \item El sistema aplica el cambio inmediatamente
    \item El sistema guarda la preferencia del usuario
\end{enumerate}

\section{CU-26: Etiquetar Tickets}

\noindent\textbf{Actor:} Administrador Financiero

\noindent\textbf{Descripción:} El administrador financiero agrega etiquetas personalizadas a los tickets para clasificar los gastos según sus propios criterios o proyectos personales.

\noindent\textbf{Precondiciones:}
\begin{itemize}
    \item El usuario debe estar autenticado en el sistema
    \item Debe existir un ticket registrado para etiquetar
\end{itemize}

\noindent\textbf{Postcondiciones:}
\begin{itemize}
    \item El ticket queda etiquetado con las etiquetas seleccionadas
    \item Las etiquetas están disponibles para filtros y búsquedas
    \item Se actualiza el sistema de categorización personalizada
\end{itemize}

\noindent\textbf{Flujo Principal:}
\begin{enumerate}
    \item El usuario selecciona un ticket desde la consulta de tickets
    \item El usuario presiona ``Agregar Etiquetas''
    \item El sistema presenta:
    \begin{itemize}
        \item Lista de etiquetas existentes
        \item Opción para crear nueva etiqueta
    \end{itemize}
    \item El usuario selecciona etiquetas existentes o crea nuevas
    \item El usuario puede asignar múltiples etiquetas al ticket
    \item El sistema guarda las etiquetas asociadas al ticket
    \item Se actualiza la visualización del ticket con las etiquetas
\end{enumerate}

\section{CU-27: Corregir Datos Escaneados}

\noindent\textbf{Actor:} Administrador Financiero

\noindent\textbf{Descripción:} El administrador financiero corrige manualmente los datos que fueron mal reconocidos durante el proceso de OCR.

\noindent\textbf{Precondiciones:}
\begin{itemize}
    \item El usuario debe estar autenticado en el sistema
    \item Debe existir un ticket procesado por OCR con datos incorrectos
\end{itemize}

\noindent\textbf{Postcondiciones:}
\begin{itemize}
    \item Los datos del ticket quedan corregidos
    \item Se mejora el sistema de reconocimiento OCR
    \item El ticket corregido se almacena correctamente
\end{itemize}

\noindent\textbf{Flujo Principal:}
\begin{enumerate}
    \item El usuario identifica datos incorrectos en un ticket OCR
    \item El usuario selecciona ``Corregir Datos''
    \item El sistema presenta formulario editable con:
    \begin{itemize}
        \item Imagen original del ticket
        \item Datos extraídos por OCR
        \item Campos editables para corrección
    \end{itemize}
    \item El usuario corrige los datos incorrectos
    \item El usuario confirma las correcciones
    \item El sistema actualiza los datos del ticket
    \item El sistema aprende de la corrección para mejorar futuro OCR
\end{enumerate}

\section{CU-28: Planificar Presupuestos Personales}

\noindent\textbf{Actor:} Administrador Financiero

\noindent\textbf{Descripción:} El administrador financiero define presupuestos personalizados por categoría o tipo de gasto específicamente en la aplicación móvil para un mejor control financiero.

\noindent\textbf{Precondiciones:}
\begin{itemize}
    \item El usuario debe estar autenticado en la aplicación móvil
    \item Deben existir categorías de gasto configuradas
    \item El usuario debe tener ingresos registrados
\end{itemize}

\noindent\textbf{Postcondiciones:}
\begin{itemize}
    \item Los presupuestos personalizados quedan configurados
    \item Se establecen alertas automáticas específicas
    \item El sistema móvil monitorea los gastos contra estos presupuestos
\end{itemize}

\noindent\textbf{Flujo Principal:}
\begin{enumerate}
    \item El usuario accede a ``Presupuestos Personales'' en la app móvil
    \item El usuario selecciona ``Crear Presupuesto Personalizado''
    \item El usuario define parámetros específicos:
    \begin{itemize}
        \item Categorías incluidas
        \item Monto total o por categoría
        \item Período (semanal, quincenal, mensual)
        \item Alertas personalizadas (50%, 75%, 90%)
    \end{itemize}
    \item El usuario configura notificaciones móviles
    \item El sistema valida y guarda el presupuesto personalizado
    \item Se activa el monitoreo automático en la app móvil
\end{enumerate}

\section{CU-29: Visualizar Gastos}

\noindent\textbf{Actor:} Administrador Financiero

\noindent\textbf{Descripción:} El administrador financiero visualiza un resumen de sus gastos organizados por día, semana y mes, junto con las categorías más frecuentes en la aplicación móvil.

\noindent\textbf{Precondiciones:}
\begin{itemize}
    \item El usuario debe estar autenticado en la aplicación móvil
    \item Deben existir gastos registrados en el sistema
\end{itemize}

\noindent\textbf{Postcondiciones:}
\begin{itemize}
    \item Se muestran los gastos organizados según el período seleccionado
    \item Se identifican las categorías más frecuentes
    \item Se actualiza la información de consulta
\end{itemize}

\noindent\textbf{Flujo Principal:}
\begin{enumerate}
    \item El usuario accede a ``Visualizar Gastos'' en la app móvil
    \item El sistema presenta opciones de visualización:
    \begin{itemize}
        \item Vista diaria con gastos del día
        \item Vista semanal con resumen semanal
        \item Vista mensual con totales mensuales
    \end{itemize}
    \item El usuario selecciona el período deseado
    \item El sistema muestra:
    \begin{itemize}
        \item Gastos organizados por fecha
        \item Categorías más frecuentes
        \item Gráficas de tendencias
        \item Comparativa con períodos anteriores
    \end{itemize}
    \item El usuario puede filtrar por categorías específicas
\end{enumerate}

\section{CU-30: Consultar Consejos Personalizados}

\noindent\textbf{Actor:} Administrador Financiero

\noindent\textbf{Descripción:} El administrador financiero consulta consejos financieros personalizados basados en sus hábitos de consumo desde la plataforma web.

\noindent\textbf{Precondiciones:}
\begin{itemize}
    \item El usuario debe estar autenticado en la plataforma web
    \item Deben existir suficientes datos de gastos para análisis
    \item El sistema debe tener patrones de comportamiento identificados
\end{itemize}

\noindent\textbf{Postcondiciones:}
\begin{itemize}
    \item Se muestran consejos personalizados actualizados
    \item El usuario recibe recomendaciones específicas
    \item Se registra la consulta para mejorar futuros consejos
\end{itemize}

\noindent\textbf{Flujo Principal:}
\begin{enumerate}
    \item El usuario accede a ``Consejos Personalizados'' en la web
    \item El sistema analiza los patrones de gasto del usuario
    \item El sistema genera consejos basados en:
    \begin{itemize}
        \item Categorías de mayor gasto
        \item Tendencias de consumo
        \item Comparación con metas establecidas
        \item Oportunidades de ahorro identificadas
    \end{itemize}
    \item El sistema presenta consejos categorizados:
    \begin{itemize}
        \item Consejos de reducción de gastos
        \item Estrategias de ahorro
        \item Optimización de presupuestos
        \item Planificación financiera
    \end{itemize}
    \item El usuario puede marcar consejos como útiles
\end{enumerate}

\section{CU-31: Buscar y Filtrar Tickets}

\noindent\textbf{Actor:} Administrador Financiero

\noindent\textbf{Descripción:} El administrador financiero busca y filtra tickets según criterios avanzados como fecha, categoría, monto o palabras clave desde la plataforma web.

\noindent\textbf{Precondiciones:}
\begin{itemize}
    \item El usuario debe estar autenticado en la plataforma web
    \item Deben existir tickets registrados en el sistema
\end{itemize}

\noindent\textbf{Postcondiciones:}
\begin{itemize}
    \item Se muestran los tickets que coinciden con los criterios
    \item Los resultados pueden ser exportados o procesados
    \item Se guarda la consulta para reutilización
\end{itemize}

\noindent\textbf{Flujo Principal:}
\begin{enumerate}
    \item El usuario accede a ``Búsqueda Avanzada de Tickets'' en la web
    \item El sistema presenta criterios de búsqueda:
    \begin{itemize}
        \item Rango de fechas específico
        \item Categorías múltiples
        \item Rango de montos (mínimo/máximo)
        \item Palabras clave en descripción
        \item Establecimiento o lugar
        \item Método de registro (OCR/Manual)
    \end{itemize}
    \item El usuario configura los filtros deseados
    \item El usuario ejecuta la búsqueda
    \item El sistema procesa y muestra resultados paginados
    \item El usuario puede:
    \begin{itemize}
        \item Refinar la búsqueda
        \item Exportar resultados
        \item Guardar consulta como favorita
    \end{itemize}
\end{enumerate}

\section{CU-32: Generar Reportes Excel}

\noindent\textbf{Actor:} Administrador Financiero

\noindent\textbf{Descripción:} El administrador financiero genera y descarga reportes financieros en formato Excel con hojas separadas por categorías o períodos desde la plataforma web.

\noindent\textbf{Precondiciones:}
\begin{itemize}
    \item El usuario debe estar autenticado en la plataforma web
    \item Deben existir datos financieros para incluir en el reporte
\end{itemize}

\noindent\textbf{Postcondiciones:}
\begin{itemize}
    \item Se genera un archivo Excel completo y estructurado
    \item El archivo está listo para descarga
    \item Se registra la generación del reporte
\end{itemize}

\noindent\textbf{Flujo Principal:}
\begin{enumerate}
    \item El usuario accede a ``Reportes Excel'' en la plataforma web
    \item El usuario configura el reporte:
    \begin{itemize}
        \item Período de tiempo
        \item Categorías a incluir
        \item Tipo de datos (transacciones, resúmenes, gráficas)
        \item Formato de hojas (por categoría, por mes, consolidado)
    \end{itemize}
    \item El usuario presiona ``Generar Reporte Excel''
    \item El sistema procesa los datos y crea el archivo Excel
    \item El sistema presenta el archivo para descarga
    \item El usuario descarga el archivo Excel generado
\end{enumerate}

\section{CU-33: Compartir Reportes por Correo}

\noindent\textbf{Actor:} Administrador Financiero

\noindent\textbf{Descripción:} El administrador financiero comparte reportes generados directamente por correo electrónico desde la plataforma web, sin necesidad de descargarlos localmente.

\noindent\textbf{Precondiciones:}
\begin{itemize}
    \item El usuario debe estar autenticado en la plataforma web
    \item Debe existir un reporte generado o datos para crear uno
    \item El sistema debe tener configuración de correo habilitada
\end{itemize}

\noindent\textbf{Postcondiciones:}
\begin{itemize}
    \item El reporte se envía por correo electrónico
    \item Se confirma el envío al usuario
    \item Se registra la acción de compartir
\end{itemize}

\noindent\textbf{Flujo Principal:}
\begin{enumerate}
    \item El usuario genera o selecciona un reporte existente
    \item El usuario presiona ``Compartir por Correo''
    \item El sistema presenta formulario de envío:
    \begin{itemize}
        \item Destinatarios (múltiples direcciones)
        \item Asunto personalizable
        \item Mensaje opcional
        \item Formato del archivo adjunto
    \end{itemize}
    \item El usuario completa la información de envío
    \item El sistema valida las direcciones de correo
    \item El sistema envía el correo con el reporte adjunto
    \item Se muestra confirmación de envío exitoso
\end{enumerate}

\section{CU-34: Comparar Gastos entre Meses}

\noindent\textbf{Actor:} Administrador Financiero

\noindent\textbf{Descripción:} El administrador financiero visualiza comparativas gráficas y tabulares de gastos entre diferentes meses, mostrando incrementos o reducciones porcentuales desde la plataforma web.

\noindent\textbf{Precondiciones:}
\begin{itemize}
    \item El usuario debe estar autenticado en la plataforma web
    \item Deben existir datos de al menos dos meses diferentes
\end{itemize}

\noindent\textbf{Postcondiciones:}
\begin{itemize}
    \item Se muestran comparativas detalladas entre períodos
    \item Se identifican tendencias y patrones
    \item Los análisis pueden ser exportados
\end{itemize}

\noindent\textbf{Flujo Principal:}
\begin{enumerate}
    \item El usuario accede a ``Comparación Mensual'' en la web
    \item El usuario selecciona los meses a comparar
    \item El sistema genera comparativas mostrando:
    \begin{itemize}
        \item Gráficas de barras comparativas
        \item Tablas con diferencias porcentuales
        \item Análisis por categorías
        \item Tendencias de crecimiento/reducción
    \end{itemize}
    \item El usuario puede personalizar la visualización
    \item El sistema destaca cambios significativos
    \item El usuario puede exportar la comparativa
\end{enumerate}

\section{CU-35: Consultar Balance vs Egresos}

\noindent\textbf{Actor:} Administrador Financiero

\noindent\textbf{Descripción:} El administrador financiero consulta un balance general que compara ingresos frente a egresos, calculando el resultado neto del período seleccionado desde la plataforma web.

\noindent\textbf{Precondiciones:}
\begin{itemize}
    \item El usuario debe estar autenticado en la plataforma web
    \item Deben existir registros de ingresos y egresos
\end{itemize}

\noindent\textbf{Postcondiciones:}
\begin{itemize}
    \item Se muestra el balance financiero actualizado
    \item Se calcula el resultado neto del período
    \item Se identifican áreas de mejora
\end{itemize}

\noindent\textbf{Flujo Principal:}
\begin{enumerate}
    \item El usuario accede a ``Balance Financiero'' en la web
    \item El usuario selecciona el período de análisis
    \item El sistema calcula y presenta:
    \begin{itemize}
        \item Total de ingresos del período
        \item Total de egresos del período
        \item Balance neto (ingresos - egresos)
        \item Porcentaje de gastos vs ingresos
        \item Gráfica de flujo de efectivo
    \end{itemize}
    \item El sistema destaca:
    \begin{itemize}
        \item Meses con balance positivo/negativo
        \item Tendencias de mejora o deterioro
        \item Recomendaciones basadas en el balance
    \end{itemize}
\end{enumerate}

\section{CU-36: Visualizar Estadísticas Avanzadas}

\noindent\textbf{Actor:} Administrador Financiero

\noindent\textbf{Descripción:} El administrador financiero accede a un panel con estadísticas detalladas del comportamiento financiero, incluyendo tendencias, promedios y porcentajes avanzados desde la plataforma web.

\noindent\textbf{Precondiciones:}
\begin{itemize}
    \item El usuario debe estar autenticado en la plataforma web
    \item Deben existir datos históricos suficientes para análisis
\end{itemize}

\noindent\textbf{Postcondiciones:}
\begin{itemize}
    \item Se presentan estadísticas avanzadas completas
    \item El usuario obtiene insights profundos sobre su comportamiento
    \item Los análisis pueden ser utilizados para planificación
\end{itemize}

\noindent\textbf{Flujo Principal:}
\begin{enumerate}
    \item El usuario accede a ``Estadísticas Avanzadas'' en la web
    \item El sistema presenta dashboard con:
    \begin{itemize}
        \item Análisis de tendencias temporales
        \item Promedios móviles de gastos
        \item Desviaciones estándar por categoría
        \item Análisis de correlaciones
        \item Predicciones basadas en patrones históricos
    \end{itemize}
    \item El usuario puede personalizar el análisis:
    \begin{itemize}
        \item Seleccionar métricas específicas
        \item Ajustar períodos de análisis
        \item Filtrar por categorías
    \end{itemize}
    \item El sistema genera insights automatizados
    \item El usuario puede exportar análisis completos
\end{enumerate}

\section{CU-37: Cargar Ticket Digital}

\noindent\textbf{Actor:} Administrador Financiero

\noindent\textbf{Descripción:} El administrador financiero carga tickets en formato digital (PDF o imagen) desde el explorador web, procesándolos automáticamente con el módulo OCR.

\noindent\textbf{Precondiciones:}
\begin{itemize}
    \item El usuario debe estar autenticado en la plataforma web
    \item Debe tener tickets digitales disponibles para cargar
    \item El módulo OCR debe estar operativo
\end{itemize}

\noindent\textbf{Postcondiciones:}
\begin{itemize}
    \item Los tickets digitales quedan procesados y registrados
    \item Los datos extraídos están disponibles para edición
    \item Se almacenan los archivos originales
\end{itemize}

\noindent\textbf{Flujo Principal:}
\begin{enumerate}
    \item El usuario accede a ``Cargar Ticket Digital'' en la web
    \item El usuario selecciona archivos para cargar:
    \begin{itemize}
        \item Imágenes (JPG, PNG, GIF)
        \item Documentos PDF
        \item Múltiples archivos simultáneamente
    \end{itemize}
    \item El sistema valida los archivos cargados
    \item El sistema procesa cada archivo con OCR
    \item El sistema presenta datos extraídos para confirmación
    \item El usuario revisa y confirma los datos
    \item El sistema registra las transacciones procesadas
\end{enumerate}

\section{CU-38: Configurar Metas desde Web}

\noindent\textbf{Actor:} Administrador Financiero

\noindent\textbf{Descripción:} El administrador financiero define y modifica metas de ahorro directamente desde la plataforma web, sincronizándose con los datos de la aplicación móvil.

\noindent\textbf{Precondiciones:}
\begin{itemize}
    \item El usuario debe estar autenticado en la plataforma web
    \item Debe tener presupuestos configurados
    \item La sincronización con la app móvil debe estar activa
\end{itemize}

\noindent\textbf{Postcondiciones:}
\begin{itemize}
    \item Las metas quedan configuradas y sincronizadas
    \item Los cambios se reflejan en la aplicación móvil
    \item Se establecen alertas y recordatorios automáticos
\end{itemize}

\noindent\textbf{Flujo Principal:}
\begin{enumerate}
    \item El usuario accede a ``Gestión de Metas'' en la plataforma web
    \item El usuario puede:
    \begin{itemize}
        \item Crear nuevas metas de ahorro
        \item Modificar metas existentes
        \item Eliminar metas completadas
        \item Configurar metas complejas con submetas
    \end{itemize}
    \item El usuario define parámetros avanzados:
    \begin{itemize}
        \item Metas escalonadas
        \item Contribuciones automáticas
        \item Alertas personalizadas
        \item Integración con presupuestos
    \end{itemize}
    \item El sistema valida y guarda las configuraciones
    \item El sistema sincroniza con la aplicación móvil
    \item Se confirma la sincronización exitosa
\end{enumerate}