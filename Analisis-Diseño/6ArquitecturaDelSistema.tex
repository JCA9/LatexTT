% !TeX root = analisis-diseno.tex

%=========================================================
\chapter{Arquitectura del sistema}
\label{cap:arquitectura}

Este capítulo presenta la arquitectura técnica de FinanzApp, describiendo la estructura de componentes, patrones arquitectónicos utilizados, tecnologías seleccionadas y decisiones de diseño que soportan los requerimientos funcionales y no funcionales del sistema.

%---------------------------------------------------------
\section{Visión general de la arquitectura}

FinanzApp utiliza una arquitectura de microservicios distribuida que separa las responsabilidades en componentes especializados, permitiendo escalabilidad, mantenibilidad y desarrollo independiente de cada módulo.

\begin{figure}[htpb!]
	\begin{center}
		% TODO: Agregar imagen correspondiente
		\caption{Visión general de la arquitectura de FinanzApp}
		\label{fig:arquitecturaGeneral}
	\end{center}
\end{figure}

% - - - - - - - - - - - - - - - - - - - - - - - - - - - - 
\subsection{Componentes principales}

\begin{description}
	\item[\textbf{Frontend Móvil}] Aplicación nativa desarrollada en React Native para iOS y Android
	\item[\textbf{Frontend Web}] Aplicación web responsive desarrollada en React.js
	\item[\textbf{API Gateway}] Punto de entrada único para todas las peticiones del cliente
	\item[\textbf{Servicio de Autenticación}] Gestión de usuarios, autenticación y autorización
	\item[\textbf{Servicio de Transacciones}] Gestión de gastos, ingresos y operaciones financieras
	\item[\textbf{Servicio OCR}] Procesamiento de imágenes y extracción de texto
	\item[\textbf{Servicio de Categorización}] Clasificación automática inteligente de transacciones
	\item[\textbf{Servicio de Analytics}] Análisis predictivo y generación de insights
	\item[\textbf{Servicio de Notificaciones}] Gestión y envío de notificaciones multiplataforma
	\item[\textbf{Base de Datos Principal}] PostgreSQL para datos transaccionales
	\item[\textbf{Cache Distribuido}] Redis para optimización de rendimiento
	\item[\textbf{File Storage}] AWS S3 para almacenamiento de imágenes
\end{description}

%---------------------------------------------------------
\section{Patrones arquitectónicos}

% - - - - - - - - - - - - - - - - - - - - - - - - - - - - 
\subsection{Arquitectura hexagonal (Ports and Adapters)}

Cada microservicio implementa arquitectura hexagonal para aislar la lógica de negocio de los detalles de implementación externa.

\begin{figure}[htpb!]
	\begin{center}
		% TODO: Agregar imagen correspondiente
		\caption{Arquitectura hexagonal implementada en microservicios}
		\label{fig:arquitecturaHexagonal}
	\end{center}
\end{figure}

\textbf{Beneficios implementados:}
\begin{itemize}
	\item Testabilidad mejorada mediante inyección de dependencias
	\item Flexibilidad para cambiar tecnologías de persistencia
	\item Separación clara entre lógica de negocio e infraestructura
	\item Facilita implementación de diferentes adaptadores (REST, GraphQL, gRPC)
\end{itemize}

% - - - - - - - - - - - - - - - - - - - - - - - - - - - - 
\subsection{CQRS (Command Query Responsibility Segregation)}

Los servicios críticos implementan CQRS para optimizar operaciones de lectura y escritura independientemente.

\textbf{Aplicación en FinanzApp:}
\begin{itemize}
	\item \textbf{Commands}: Registro de transacciones, actualización de presupuestos
	\item \textbf{Queries}: Consulta de analytics, reportes históricos, dashboards
	\item \textbf{Separación de bases de datos}: Write DB optimizada para transacciones, Read DB optimizada para consultas
	\item \textbf{Event Sourcing}: Para auditoria completa de cambios financieros
\end{itemize}

% - - - - - - - - - - - - - - - - - - - - - - - - - - - - 
\subsection{Event-Driven Architecture}

Los componentes se comunican mediante eventos asincrónicos para mejorar el desacoplamiento y la escalabilidad.

\begin{figure}[htpb!]
	\begin{center}
		% TODO: Agregar imagen correspondiente
		\caption{Arquitectura basada en eventos}
		\label{fig:eventDriven}
	\end{center}
\end{figure}

\textbf{Eventos principales:}
\begin{itemize}
	\item \texttt{TransactionCreated}: Nueva transacción registrada
	\item \texttt{CategoryAssigned}: Transacción categorizada automáticamente
	\item \texttt{BudgetExceeded}: Presupuesto superado en una categoría
	\item \texttt{UserGoalAchieved}: Meta financiera alcanzada
	\item \texttt{AnomalousSpendingDetected}: Gasto atípico identificado
\end{itemize}

%---------------------------------------------------------
\section{Stack tecnológico}

% - - - - - - - - - - - - - - - - - - - - - - - - - - - - 
\subsection{Frontend}

\begin{description}
	\item[\textbf{Aplicación Móvil}] 
	\begin{itemize}
		\item React Native 0.72+ para desarrollo multiplataforma
		\item Redux Toolkit para gestión de estado
		\item React Navigation 6 para navegación
		\item React Native Camera para funcionalidad OCR
		\item Async Storage para persistencia local
	\end{itemize}
	
	\item[\textbf{Aplicación Web}]
	\begin{itemize}
		\item React.js 18+ con TypeScript
		\item Next.js 13+ para SSR y optimizaciones
		\item Tailwind CSS para styling responsivo
		\item Chart.js para visualizaciones de datos
		\item PWA capabilities para experiencia nativa
	\end{itemize}
\end{description}

% - - - - - - - - - - - - - - - - - - - - - - - - - - - - 
\subsection{Backend}

\begin{description}
	\item[\textbf{API Gateway}] Kong con plugins para autenticación y rate limiting
	\item[\textbf{Microservicios}] Node.js con Express.js y TypeScript
	\item[\textbf{Base de datos principal}] PostgreSQL 15+ con particionamiento
	\item[\textbf{Cache}] Redis 7+ para sesiones y cache de aplicación
	\item[\textbf{Message Broker}] Apache Kafka para eventos asincrónicos
	\item[\textbf{Container Orchestration}] Docker + Kubernetes
\end{description}

% - - - - - - - - - - - - - - - - - - - - - - - - - - - - 
\subsection{Servicios externos}

\begin{description}
	\item[\textbf{OCR}] Google Cloud Vision API con fallback a Tesseract.js
	\item[\textbf{Machine Learning}] TensorFlow.js para categorización en el cliente
	\item[\textbf{Storage}] AWS S3 para imágenes con CloudFront CDN
	\item[\textbf{Monitoring}] Prometheus + Grafana + AlertManager
	\item[\textbf{Logging}] ELK Stack (Elasticsearch, Logstash, Kibana)
\end{description}

%---------------------------------------------------------
\section{Diseño de la base de datos}

% - - - - - - - - - - - - - - - - - - - - - - - - - - - - 
\subsection{Modelo físico de datos}

\begin{figure}[htpb!]
	\begin{center}
		% TODO: Agregar imagen correspondiente
		\caption{Modelo físico de la base de datos}
		\label{fig:modeloFisicoBD}
	\end{center}
\end{figure}

% - - - - - - - - - - - - - - - - - - - - - - - - - - - - 
\subsection{Estrategias de optimización}

\begin{itemize}
	\item \textbf{Particionamiento}: Tabla de transacciones particionada por fecha
	\item \textbf{Índices compuestos}: Para consultas frecuentes por usuario+fecha+categoría
	\item \textbf{Materialised views}: Para agregaciones complejas de analytics
	\item \textbf{Connection pooling}: PgBouncer para optimizar conexiones
	\item \textbf{Read replicas}: Para distribuir carga de consultas de solo lectura
\end{itemize}

%---------------------------------------------------------
\section{Seguridad}

% - - - - - - - - - - - - - - - - - - - - - - - - - - - - 
\subsection{Autenticación y autorización}

\begin{itemize}
	\item \textbf{JWT}: Tokens con expiración corta (15 min) y refresh tokens
	\item \textbf{OAuth 2.0}: Integración con Google y Apple para registro social
	\item \textbf{2FA}: Autenticación de dos factores opcional con TOTP
	\item \textbf{RBAC}: Control de acceso basado en roles (User, Admin, Analyst)
	\item \textbf{Rate limiting}: Protección contra ataques de fuerza bruta
\end{itemize}

% - - - - - - - - - - - - - - - - - - - - - - - - - - - - 
\subsection{Protección de datos}

\begin{itemize}
	\item \textbf{Cifrado en tránsito}: TLS 1.3 para todas las comunicaciones
	\item \textbf{Cifrado en reposo}: AES-256 para datos sensibles en base de datos
	\item \textbf{PII Encryption}: Información personal cifrada a nivel de aplicación
	\item \textbf{Secrets management}: AWS Secrets Manager para credenciales
	\item \textbf{Data masking}: Ofuscación de datos en ambientes de desarrollo
\end{itemize}

%---------------------------------------------------------
\section{Escalabilidad y rendimiento}

% - - - - - - - - - - - - - - - - - - - - - - - - - - - - 
\subsection{Estrategias de escalabilidad}

\begin{itemize}
	\item \textbf{Horizontal scaling}: Auto-scaling de pods en Kubernetes
	\item \textbf{Database sharding}: Preparación para sharding por región geográfica
	\item \textbf{CDN global}: CloudFlare para contenido estático mundial
	\item \textbf{Microservices isolation}: Fallas aisladas por dominio de negocio
	\item \textbf{Circuit breakers}: Hystrix para prevenir cascading failures
\end{itemize}

% - - - - - - - - - - - - - - - - - - - - - - - - - - - - 
\subsection{Objetivos de rendimiento}

\begin{itemize}
	\item \textbf{Latencia API}: <200ms para 95\% de requests
	\item \textbf{Throughput}: 1000 requests/segundo sostenido
	\item \textbf{Tiempo de procesamiento OCR}: <10 segundos por imagen
	\item \textbf{Disponibilidad}: 99.9\% uptime (SLA objetivo)
	\item \textbf{Recovery Time}: <5 minutos para recuperación de fallas
\end{itemize}
