% !TeX root = analisis-diseno.tex

%=========================================================
\chapter{Análisis del Sistema}
\label{cap:analisis}

Este capítulo presenta el análisis completo del sistema FinanzApp, definiendo sus características principales, actores, requerimientos y restricciones operacionales.

%---------------------------------------------------------
\section{Descripción general}

FinanzApp es una aplicación móvil de gestión de finanzas personales diseñada específicamente para jóvenes adultos de 18 a 30 años. El sistema utiliza tecnología OCR (Reconocimiento Óptico de Caracteres) para automatizar la captura de gastos mediante el escaneo de tickets y recibos, proporcionando categorización automática inteligente y análisis predictivo de patrones de consumo.

La aplicación se basa en una arquitectura de tres capas (presentación, lógica de negocio y datos) con patrón cliente-servidor, ofreciendo una experiencia de usuario intuitiva mientras mantiene la integridad y seguridad de los datos financieros.

Las funcionalidades principales incluyen:
\begin{itemize}
	\item Registro automático de gastos mediante OCR
	\item Categorización inteligente de transacciones
	\item Gestión de presupuestos personalizables
	\item Analytics y visualización de patrones de gasto
	\item Sincronización multiplataforma
	\item Notificaciones proactivas de control financiero
\end{itemize}

%---------------------------------------------------------
\section{Actores}

Los actores identificados en el sistema FinanzApp representan las entidades externas que interactúan con el sistema:

\begin{description}
	\item[\textcolor{userColor}{\textbf{Usuario Joven}}] Persona de 18 a 30 años que utiliza la aplicación para gestionar sus finanzas personales. Es el actor principal del sistema y puede realizar todas las operaciones de registro, consulta y configuración.
	
	\item[\textcolor{userColor}{\textbf{Servicio OCR}}] Sistema externo especializado en procesamiento de imágenes y reconocimiento óptico de caracteres. Procesa las fotografías de tickets y recibos para extraer información estructurada.
	
	\item[\textcolor{userColor}{\textbf{Sistema de Notificaciones}}] Servicio externo que gestiona el envío de notificaciones push, SMS y correo electrónico hacia los usuarios registrados.
	
	\item[\textcolor{userColor}{\textbf{Administrador del Sistema}}] Personal técnico responsable del mantenimiento, monitoreo y configuración de parámetros del sistema.
\end{description}

%---------------------------------------------------------
\section{Requerimientos funcionales}

Los requerimientos funcionales definen las capacidades específicas que debe proporcionar el sistema:

\begin{description}
	\item[\textbf{RF-001}] El sistema debe permitir el registro de usuarios mediante correo electrónico y contraseña
	\item[\textbf{RF-002}] El usuario debe poder autenticarse en el sistema usando sus credenciales
	\item[\textbf{RF-003}] El sistema debe procesar imágenes de tickets mediante OCR para extraer información de gastos
	\item[\textbf{RF-004}] El sistema debe categorizar automáticamente los gastos basándose en el comercio y tipo de productos
	\item[\textbf{RF-005}] El usuario debe poder crear y configurar presupuestos por categoría
	\item[\textbf{RF-006}] El sistema debe generar reportes visuales de gastos y patrones de consumo
	\item[\textbf{RF-007}] El usuario debe poder sincronizar sus datos entre dispositivos
	\item[\textbf{RF-008}] El sistema debe enviar notificaciones cuando se aproxime a límites de presupuesto
	\item[\textbf{RF-009}] El usuario debe poder consultar el historial de transacciones con filtros
	\item[\textbf{RF-010}] El sistema debe permitir la edición manual de transacciones procesadas por OCR
	\item[\textbf{RF-011}] El usuario debe poder gestionar categorías personalizadas
	\item[\textbf{RF-012}] El sistema debe proporcionar analytics predictivos de gastos futuros
\end{description}

%---------------------------------------------------------
\section{Requerimientos no funcionales}

Los requerimientos no funcionales establecen criterios de calidad y restricciones técnicas:

\begin{description}
	\item[\textbf{RNF-001}] \textbf{Rendimiento:} El procesamiento OCR debe completarse en menos de 5 segundos
	\item[\textbf{RNF-002}] \textbf{Disponibilidad:} El sistema debe estar disponible 99.5\% del tiempo
	\item[\textbf{RNF-003}] \textbf{Seguridad:} Los datos financieros deben encriptarse usando AES-256
	\item[\textbf{RNF-004}] \textbf{Usabilidad:} La aplicación debe ser utilizable por usuarios sin experiencia técnica
	\item[\textbf{RNF-005}] \textbf{Escalabilidad:} El sistema debe soportar hasta 100,000 usuarios concurrentes
	\item[\textbf{RNF-006}] \textbf{Compatibilidad:} Debe funcionar en iOS 12+ y Android 8.0+
	\item[\textbf{RNF-007}] \textbf{Precisión OCR:} La extracción de datos debe tener una precisión mínima del 85\%
	\item[\textbf{RNF-008}] \textbf{Tiempo respuesta:} Las consultas de datos deben responder en menos de 2 segundos
	\item[\textbf{RNF-009}] \textbf{Backup:} Los datos deben respaldarse automáticamente cada 24 horas
	\item[\textbf{RNF-010}] \textbf{Offline:} Funcionalidades básicas deben estar disponibles sin conexión
\end{description}

%---------------------------------------------------------
\section{Reglas de negocio}

Las reglas de negocio establecen las políticas y restricciones operacionales:

% !TeX root = analisis-diseno.tex

%=========================================================
% Reglas de negocio para FinanzApp

\begin{description}
	\item[\hypertarget{RN-001}{RN-001}] \textbf{Registro único de usuario:} Cada usuario debe tener un email único en el sistema. No se permiten cuentas duplicadas con el mismo email.
	
	\item[\hypertarget{RN-002}{RN-002}] \textbf{Edad mínima:} Los usuarios deben tener al menos 18 años para registrarse en el sistema, validado mediante fecha de nacimiento.
	
	\item[\hypertarget{RN-003}{RN-003}] \textbf{Límite de procesamiento OCR:} Cada usuario puede procesar máximo 50 tickets por día mediante OCR para optimizar recursos del sistema.
	
	\item[\hypertarget{RN-004}{RN-004}] \textbf{Validación de montos:} Los montos de transacciones deben ser positivos y no pueden exceder \$100,000.00 MXN por transacción individual.
	
	\item[\hypertarget{RN-005}{RN-005}] \textbf{Retención de imágenes:} Las imágenes de tickets se conservan por 90 días después del procesamiento exitoso, luego se eliminan automáticamente por privacidad.
	
	\item[\hypertarget{RN-006}{RN-006}] \textbf{Categorización automática:} El sistema debe categorizar automáticamente las transacciones con un nivel de confianza mínimo del 60\%. Transacciones con menor confianza requieren validación manual.
	
	\item[\hypertarget{RN-007}{RN-007}] \textbf{Presupuestos activos:} Un usuario puede tener máximo 10 presupuestos activos simultáneamente para evitar complejidad excesiva.
	
	\item[\hypertarget{RN-008}{RN-008}] \textbf{Período de presupuesto:} Los presupuestos pueden configurarse para períodos de 1 semana, 1 mes o 3 meses únicamente.
	
	\item[\hypertarget{RN-009}{RN-009}] \textbf{Alertas de presupuesto:} Se envían alertas automáticas al alcanzar 70\%, 90\% y 100\% del presupuesto asignado a cada categoría.
	
	\item[\hypertarget{RN-010}{RN-010}] \textbf{Metas financieras:} Las metas de ahorro deben tener una fecha límite entre 1 mes y 5 años desde su creación.
	
	\item[\hypertarget{RN-011}{RN-011}] \textbf{Sincronización de datos:} Los datos deben sincronizarse entre dispositivos en un máximo de 5 segundos después de una transacción.
	
	\item[\hypertarget{RN-012}{RN-012}] \textbf{Backup automático:} El sistema debe crear respaldos automáticos de los datos del usuario cada 24 horas.
	
	\item[\hypertarget{RN-013}{RN-013}] \textbf{Inactividad de cuenta:} Las cuentas inactivas por más de 365 días se marcan para archivado, con notificación previa de 30 días al usuario.
	
	\item[\hypertarget{RN-014}{RN-014}] \textbf{Exportación de datos:} Los usuarios pueden exportar todos sus datos personales en formato JSON o CSV en cualquier momento.
	
	\item[\hypertarget{RN-015}{RN-015}] \textbf{Eliminación de cuenta:} Al eliminar una cuenta, todos los datos se eliminan permanentemente después de 30 días de período de gracia.
	
	\item[\hypertarget{RN-016}{RN-016}] \textbf{Validación de tickets:} Solo se aceptan imágenes en formato JPEG, PNG o PDF, con tamaño máximo de 10 MB.
	
	\item[\hypertarget{RN-017}{RN-017}] \textbf{Moneda única:} El sistema opera únicamente con pesos mexicanos (MXN). No se soportan múltiples monedas.
	
	\item[\hypertarget{RN-018}{RN-018}] \textbf{Categorías personalizadas:} Los usuarios pueden crear máximo 20 categorías personalizadas además de las categorías predefinidas del sistema.
	
	\item[\hypertarget{RN-019}{RN-019}] \textbf{Histórico mínimo:} Se requieren mínimo 30 días de datos históricos para generar análisis predictivos y recomendaciones personalizadas.
	
	\item[\hypertarget{RN-020}{RN-020}] \textbf{Notificaciones:} Los usuarios pueden recibir máximo 5 notificaciones push por día para evitar fatiga de notificaciones.
	
	\item[\hypertarget{RN-021}{RN-021}] \textbf{Sesión de usuario:} Las sesiones expiran automáticamente después de 30 días de inactividad por seguridad.
	
	\item[\hypertarget{RN-022}{RN-022}] \textbf{Corrección de OCR:} Los usuarios pueden corregir manualmente los datos extraídos por OCR dentro de las primeras 48 horas del procesamiento.
	
	\item[\hypertarget{RN-023}{RN-023}] \textbf{Análisis de tendencias:} El sistema requiere mínimo 90 días de datos para mostrar análisis de tendencias confiables.
	
	\item[\hypertarget{RN-024}{RN-024}] \textbf{Compartir datos:} Los datos financieros personales no pueden compartirse con terceros sin consentimiento explícito del usuario.
	
	\item[\hypertarget{RN-025}{RN-025}] \textbf{Recuperación de contraseña:} Los tokens de recuperación de contraseña expiran en 1 hora y solo pueden usarse una vez.
\end{description}


%---------------------------------------------------------
\section{Lista de casos de uso}

Los casos de uso del sistema FinanzApp se organizan en los siguientes grupos funcionales:

\begin{table}[htpb!]
\centering
\begin{tabular}{|l|l|l|}
\hline
\textbf{ID} & \textbf{Nombre} & \textbf{Prioridad} \\
\hline
CU-001 & Registrar gasto mediante OCR & Alta \\
CU-002 & Configurar presupuesto & Alta \\
CU-003 & Consultar analytics & Media \\
CU-004 & Gestionar categorías & Media \\
CU-005 & Registro gasto OCR simple & Alta \\
CU-006 & Sincronizar datos & Media \\
CU-007 & Autenticar usuario & Alta \\
CU-008 & Editar transacción & Baja \\
CU-009 & Generar reportes & Media \\
CU-010 & Configurar notificaciones & Baja \\
\hline
\end{tabular}
\caption{Lista completa de casos de uso}
\label{tab:casosDeUso}
\end{table}

%---------------------------------------------------------
\section{Descripción breve de casos de uso clave}

\subsection{CU-001: Registrar gasto mediante OCR}
\textbf{Descripción:} El usuario fotografía un ticket, el sistema procesa la imagen mediante OCR, extrae la información y registra automáticamente la transacción con categorización inteligente.

\textbf{Actor principal:} Usuario Joven

\textbf{Flujo básico:} Usuario abre cámara → Fotografía ticket → Sistema procesa OCR → Extrae datos → Categoriza automáticamente → Registra transacción

\subsection{CU-002: Configurar presupuesto}
\textbf{Descripción:} El usuario define límites de gasto por categoría y período de tiempo para controlar sus finanzas personales.

\textbf{Actor principal:} Usuario Joven

\textbf{Flujo básico:} Usuario accede a configuración → Selecciona categoría → Define límite y período → Sistema valida y guarda configuración

\subsection{CU-003: Consultar analytics}
\textbf{Descripción:} El usuario visualiza reportes gráficos de sus patrones de gasto, tendencias y proyecciones futuras.

\textbf{Actor principal:} Usuario Joven

\textbf{Flujo básico:} Usuario accede a analytics → Selecciona período → Sistema genera visualizaciones → Muestra insights y recomendaciones

%---------------------------------------------------------
\section{Suposiciones y restricciones}

\subsection{Suposiciones}
\begin{itemize}
	\item Los usuarios objetivo tienen smartphones con cámara de al menos 8MP
	\item Los tickets y recibos están en español y formato estándar mexicano
	\item Los usuarios tienen acceso regular a internet móvil
	\item Los comercios utilizan sistemas de punto de venta que generan tickets legibles
	\item Los usuarios están dispuestos a adoptar herramientas digitales de gestión financiera
\end{itemize}

\subsection{Restricciones}
\begin{itemize}
	\item Límite de 50 procesamiento OCR por usuario por día en versión gratuita
	\item Almacenamiento máximo de 1GB por usuario para imágenes de tickets
	\item Soporte inicial únicamente para tickets en pesos mexicanos
	\item Disponibilidad limitada a México en la fase inicial
	\item Dependencia de servicios externos para procesamiento OCR
\end{itemize}

%---------------------------------------------------------
\section{Alcance del sistema}

\subsection{Dentro del alcance}
\begin{itemize}
	\item Aplicación móvil nativa para iOS y Android
	\item Procesamiento automático de tickets mediante OCR
	\item Categorización inteligente de gastos
	\item Gestión de presupuestos personalizables
	\item Analytics y reportes visuales
	\item Sincronización de datos en la nube
	\item Sistema de notificaciones proactivas
\end{itemize}

\subsection{Fuera del alcance}
\begin{itemize}
	\item Conexión con bancos o instituciones financieras
	\item Procesamiento de estados de cuenta bancarios
	\item Funcionalidades de inversión o ahorro automatizado
	\item Asesoría financiera personalizada con humanos
	\item Soporte para múltiples monedas en la versión inicial
	\item Aplicación web completa (solo móvil)
\end{itemize}