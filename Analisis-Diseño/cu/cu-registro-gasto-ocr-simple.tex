% !TeX root = ../analisis-diseno.tex

%=========================================================
% Caso de uso: Registrar gasto mediante OCR

\section{CU-001 Registrar gasto mediante OCR}

\subsection{Descripción}
El usuario utiliza la cámara de su dispositivo móvil para fotografiar un ticket o recibo, el sistema procesa la imagen mediante OCR, extrae la información relevante y registra automáticamente la transacción con categorización inteligente.

\subsection{Información del Caso de Uso}
\begin{itemize}
    \item \textbf{Actor Principal:} Usuario Joven (Persona de 18-30 años)
    \item \textbf{Versión:} 1.0
    \item \textbf{Prioridad:} Alta
    \item \textbf{Estado:} Activo
    \item \textbf{Fecha:} Noviembre 2024
\end{itemize}

\subsection{Precondiciones}
\begin{itemize}
    \item El usuario debe estar autenticado en la aplicación
    \item El dispositivo debe tener cámara funcional
    \item El usuario debe tener conexión a internet activa
    \item El ticket o recibo debe estar en formato físico legible
\end{itemize}

\subsection{Postcondiciones}
\begin{itemize}
    \item La transacción queda registrada en la base de datos del usuario
    \item El gasto se categoriza automáticamente
    \item Los presupuestos relevantes se actualizan
    \item Se genera una notificación de confirmación
\end{itemize}

\subsection{Flujo Principal}
\begin{enumerate}
    \item[\UCactor] El usuario accede a la función de registro de gastos desde el dashboard principal
    \item[\UCactor] El usuario selecciona la opción ``Escanear ticket''
    \item[\UCsystem] El sistema activa la cámara del dispositivo y muestra la interfaz de captura
    \item[\UCactor] El usuario posiciona el ticket dentro de las guías de alineación
    \item[\UCactor] El usuario toma la fotografía presionando el botón de captura
    \item[\UCsystem] El sistema muestra un preview de la imagen capturada
    \item[\UCactor] El usuario confirma que la imagen es legible
    \item[\UCsystem] El sistema envía la imagen al servicio OCR para procesamiento
    \item[\UCsystem] El sistema extrae: monto, fecha, establecimiento, y posibles ítems
    \item[\UCsystem] El sistema aplica algoritmos de categorización automática
    \item[\UCsystem] El sistema presenta los datos extraídos al usuario para confirmación
    \item[\UCactor] El usuario revisa la información y realiza correcciones si es necesario
    \item[\UCactor] El usuario confirma el registro de la transacción
    \item[\UCsystem] El sistema registra el gasto y actualiza los presupuestos correspondientes
    \item[\UCsystem] El sistema muestra confirmación de registro exitoso
\end{enumerate}

\subsection{Flujos Alternativos}

\textbf{Flujo A: OCR no puede procesar la imagen correctamente}
\begin{itemize}
    \item[Condición:] En el paso 9, el OCR no logra extraer información clara
    \item[Acción:] El sistema muestra mensaje de error y permite retomar la foto o registrar manualmente
\end{itemize}

\textbf{Flujo B: Imagen de baja calidad}
\begin{itemize}
    \item[Condición:] En el paso 6, la imagen está borrosa o mal enfocada
    \item[Acción:] El sistema sugiere tomar una nueva fotografía con mejores condiciones de luz
\end{itemize}

\textbf{Flujo C: Datos extraídos incorrectos}
\begin{itemize}
    \item[Condición:] En el paso 12, el usuario detecta errores en la información extraída
    \item[Acción:] El usuario corrige los campos erróneos antes de confirmar
\end{itemize}

\subsection{Referencias}
\begin{itemize}
    \item RN-001: Autenticación requerida
    \item RN-005: Validación de montos
    \item RN-008: Categorización automática
    \item RF-001: Captura mediante OCR
\end{itemize}