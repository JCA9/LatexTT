% !TeX root = ../analisis-diseno.tex

%=========================================================
% Caso de uso: Registrar gasto mediante OCR

\begin{UseCase}{CU-001}{Registrar gasto mediante OCR}{
	El usuario utiliza la cámara de su dispositivo móvil para fotografiar un ticket o recibo, el sistema procesa la imagen mediante OCR, extrae la información relevante y registra automáticamente la transacción con categorización inteligente.
}
	\UCactor{\cdtActorRef{Usuario Joven}{Persona que desea registrar un gasto de manera rápida y automática}}
	\UCitem[Versión]{1.0}
	\UCitem[Autor]{Equipo FinanzApp}
	\UCitem[Fuente]{Análisis de requerimientos del usuario}
	\UCitem[Prioridad]{Alta}
	\UCitem[Estatus]{Activo}
	\UCitem[Fecha]{Noviembre 2024}

	% Precondiciones
	\UCpre{
		\UCli{El usuario debe estar autenticado en la aplicación}
		\UCli{El dispositivo debe tener cámara funcional}
		\UCli{El usuario debe tener conexión a internet activa}
		\UCli{El usuario no debe haber excedido el límite diario de 50 procesamiento OCR}
		\UCli{El ticket o recibo debe estar en formato físico legible}
	}

	% Postcondiciones
	\UCpost{
		\UCli{La transacción queda registrada en la base de datos del usuario}
		\UCli{El gasto se categoriza automáticamente}
		\UCli{Los presupuestos relevantes se actualizan}
		\UCli{Se genera una notificación de confirmación}
		\UCli{La imagen del ticket se almacena temporalmente (90 días)}
	}

	% Flujo principal
	\UCstart
		\UCstep[\UCactor] El usuario accede a la función de registro de gastos desde el dashboard principal
		\UCstep[\UCactor] El usuario selecciona la opción ``Escanear ticket'' 
		\UCstep[\UCsystem] El sistema activa la cámara del dispositivo y muestra la interfaz de captura
		\UCstep[\UCactor] El usuario posiciona el ticket dentro de las guías de alineación
		\UCstep[\UCactor] El usuario toma la fotografía presionando el botón de captura
		\UCstep[\UCsystem] El sistema muestra un preview de la imagen capturada
		\UCstep[\UCactor] El usuario confirma que la imagen es legible o decide tomarla nuevamente
		\UCstep[\UCsystem] El sistema envía la imagen al servicio OCR para procesamiento
		\UCstep[\UCsystem] El sistema muestra un indicador de progreso ``Procesando ticket...''
		\UCstep[\UCsystem] El servicio OCR extrae: monto, fecha, establecimiento, y posibles ítems
		\UCstep[\UCsystem] El sistema aplica algoritmos de categorización automática
		\UCstep[\UCsystem] El sistema presenta los datos extraídos al usuario para confirmación
		\UCstep[\UCactor] El usuario revisa la información y realiza correcciones si es necesario
		\UCstep[\UCactor] El usuario confirma el registro de la transacción
		\UCstep[\UCsystem] El sistema guarda la transacción en la base de datos
		\UCstep[\UCsystem] El sistema actualiza los presupuestos y estadísticas relevantes
		\UCstep[\UCsystem] El sistema muestra mensaje de confirmación: \MSGref{MSG-003}{OCR exitoso}
		\UCstep[\UCsystem] El sistema retorna al dashboard con la nueva transacción visible
	\UCend

	% Trayectorias alternativas
	\UCitem[Trayectorias alternativas]{
		\UCaltBegin{A}
			\UCaltitem[Condición]{En el paso 5 la calidad de la imagen es insuficiente}
			\UCaltitem[\UCsystem] El sistema detecta automáticamente baja calidad de imagen
			\UCaltitem[\UCsystem] El sistema muestra sugerencia: ``Imagen poco clara. ¿Deseas tomarla nuevamente?''
			\UCaltitem[\UCactor] El usuario decide tomar una nueva fotografía
			\UCaltitem[\UCsystem] Regresa al paso 4 del flujo principal
		\UCaltEnd

		\UCaltBegin{B}
			\UCaltitem[Condición]{En el paso 10 el OCR no puede procesar la imagen}
			\UCaltitem[\UCsystem] El sistema muestra mensaje de error: \MSGref{MSG-004}{Error en OCR}
			\UCaltitem[\UCsystem] El sistema ofrece opciones: ``Reintentar'' o ``Registro manual''
			\UCaltitem[\UCactor] El usuario selecciona ``Registro manual''
			\UCaltitem[\UCsystem] El sistema abre el formulario de registro manual
			\UCaltitem[\UCactor] El usuario ingresa manualmente: monto, fecha, descripción, categoría
			\UCaltitem[\UCsystem] Salta al paso 15 del flujo principal
		\UCaltEnd

		\UCaltBegin{C}
			\UCaltitem[Condición]{En el paso 13 la categorización automática tiene baja confianza (<60\%)}
			\UCaltitem[\UCsystem] El sistema marca la transacción como ``Requiere validación''
			\UCaltitem[\UCsystem] El sistema muestra opciones de categoría sugeridas
			\UCaltitem[\UCactor] El usuario selecciona la categoría correcta manualmente
			\UCaltitem[\UCsystem] El sistema aprende de la corrección para futuras categorizaciones
			\UCaltitem[\UCsystem] Continúa con el paso 15 del flujo principal
		\UCaltEnd

		\UCaltBegin{D}
			\UCaltitem[Condición]{El usuario ha alcanzado el límite diario de procesamiento OCR}
			\UCaltitem[\UCsystem] El sistema muestra mensaje: \MSGref{MSG-017}{Límite OCR diario}
			\UCaltitem[\UCsystem] El sistema ofrece la opción de registro manual únicamente
			\UCaltitem[\UCactor] El usuario puede proceder con registro manual o cancelar
		\UCaltEnd
	}

	% Referencias
	\UCitem[Referencias]{
		\UCli \BRref{RN-003}{Límite de procesamiento OCR}
		\UCli \BRref{RN-006}{Categorización automática}
		\UCli \BRref{RN-016}{Validación de tickets}
		\UCli \RFref{RF-001}{Digitalización automática de tickets}
		\UCli \RFref{RF-002}{Categorización inteligente de gastos}
	}
\end{UseCase}
