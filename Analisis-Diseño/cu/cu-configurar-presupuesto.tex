% !TeX root = ../analisis-diseno.tex

%=========================================================
% Caso de uso: Configurar presupuesto personal

\begin{UseCase}{CU-002}{Configurar presupuesto personal}{
	El usuario crea un nuevo presupuesto definiendo el monto límite, período de tiempo, categorías incluidas y umbrales de alerta para controlar sus gastos de manera proactiva.
}
	\UCactor{\cdtActorRef{Usuario Joven}{Persona que desea establecer límites de gasto para controlar sus finanzas}}
	\UCitem[Versión]{1.0}
	\UCitem[Autor]{Equipo FinanzApp}
	\UCitem[Fuente]{Análisis de requerimientos de control financiero}
	\UCitem[Prioridad]{Alta}
	\UCitem[Estatus]{Activo}
	\UCitem[Fecha]{Noviembre 2024}

	% Precondiciones
	\UCpre{
		\UCli{El usuario debe estar autenticado en la aplicación}
		\UCli{El usuario debe tener al menos 30 días de historial de transacciones para sugerencias}
		\UCli{El usuario no debe tener más de 10 presupuestos activos simultáneamente}
		\UCli{Debe existir al menos una categoría de gastos disponible}
	}

	% Postcondiciones
	\UCpost{
		\UCli{El presupuesto queda creado y activo en el sistema}
		\UCli{Se configuran las alertas automáticas según los umbrales definidos}
		\UCli{El dashboard se actualiza mostrando el nuevo presupuesto}
		\UCli{Se inicia el monitoreo automático de gastos contra el presupuesto}
		\UCli{Se programa el envío de reportes periódicos}
	}

	% Flujo principal
	\UCstart
		\UCstep[\UCactor] El usuario accede a la sección ``Presupuestos'' desde el menú principal
		\UCstep[\UCsystem] El sistema muestra la lista de presupuestos existentes y botón ``Crear nuevo presupuesto''
		\UCstep[\UCactor] El usuario selecciona ``Crear nuevo presupuesto''
		\UCstep[\UCsystem] El sistema muestra el formulario de configuración de presupuesto
		\UCstep[\UCactor] El usuario ingresa el nombre del presupuesto (ej: ``Entretenimiento Diciembre'')
		\UCstep[\UCactor] El usuario define el monto total del presupuesto
		\UCstep[\UCsystem] El sistema sugiere un monto basado en el historial del usuario si está disponible
		\UCstep[\UCactor] El usuario selecciona el período: semanal, mensual o trimestral
		\UCstep[\UCactor] El usuario define las fechas de inicio y fin del período
		\UCstep[\UCsystem] El sistema muestra las categorías disponibles para incluir en el presupuesto
		\UCstep[\UCactor] El usuario selecciona una o más categorías a incluir (ej: Restaurantes, Entretenimiento)
		\UCstep[\UCsystem] El sistema muestra el gasto promedio histórico en las categorías seleccionadas
		\UCstep[\UCactor] El usuario configura los umbrales de alerta (por defecto: 70\% y 90\%)
		\UCstep[\UCactor] El usuario selecciona los tipos de notificación deseados (push, email, in-app)
		\UCstep[\UCactor] El usuario revisa el resumen de configuración y confirma la creación
		\UCstep[\UCsystem] El sistema valida que el presupuesto no exceda límites del sistema
		\UCstep[\UCsystem] El sistema crea el presupuesto y activa el monitoreo automático
		\UCstep[\UCsystem] El sistema muestra mensaje de confirmación y retorna a la lista de presupuestos
		\UCstep[\UCsystem] El sistema actualiza el dashboard con el nuevo presupuesto visible
	\UCend

	% Trayectorias alternativas
	\UCitem[Trayectorias alternativas]{
		\UCaltBegin{A}
			\UCaltitem[Condición]{En el paso 16 el usuario ya tiene 10 presupuestos activos}
			\UCaltitem[\UCsystem] El sistema muestra mensaje: ``Límite de presupuestos alcanzado. Desactiva un presupuesto existente para crear uno nuevo''
			\UCaltitem[\UCsystem] El sistema muestra lista de presupuestos activos con opción de desactivar
			\UCaltitem[\UCactor] El usuario selecciona un presupuesto para desactivar
			\UCaltitem[\UCsystem] El sistema desactiva el presupuesto seleccionado
			\UCaltitem[\UCsystem] Regresa al paso 16 del flujo principal
		\UCaltEnd

		\UCaltBegin{B}
			\UCaltitem[Condición]{En el paso 6 el sistema no puede sugerir monto por falta de historial}
			\UCaltitem[\UCsystem] El sistema muestra mensaje informativo: ``Sin historial suficiente para sugerencias''
			\UCaltitem[\UCsystem] El sistema proporciona rangos de referencia típicos para la categoría
			\UCaltitem[\UCactor] El usuario ingresa el monto deseado manualmente
			\UCaltitem[\UCsystem] Continúa con el paso 8 del flujo principal
		\UCaltEnd

		\UCaltBegin{C}
			\UCaltitem[Condición]{En el paso 11 no hay categorías creadas por el usuario}
			\UCaltitem[\UCsystem] El sistema muestra solo las categorías predefinidas del sistema
			\UCaltitem[\UCsystem] El sistema ofrece la opción ``Crear nueva categoría personalizada''
			\UCaltitem[\UCactor] El usuario puede seleccionar categorías existentes o crear nuevas
			\UCaltitem[\UCsystem] Si se crean nuevas categorías, se procesan antes de continuar
			\UCaltitem[\UCsystem] Continúa con el paso 12 del flujo principal
		\UCaltEnd

		\UCaltBegin{D}
			\UCaltitem[Condición]{En el paso 9 las fechas seleccionadas son inválidas}
			\UCaltitem[\UCsystem] El sistema valida que la fecha de fin sea posterior a la de inicio
			\UCaltitem[\UCsystem] El sistema valida que la fecha de inicio no sea en el pasado
			\UCaltitem[\UCsystem] El sistema muestra mensaje de error y resalta los campos incorrectos
			\UCaltitem[\UCactor] El usuario corrige las fechas
			\UCaltitem[\UCsystem] Continúa con el paso 10 del flujo principal
		\UCaltEnd
	}

	% Referencias
	\UCitem[Referencias]{
		\UCli \BRref{RN-007}{Presupuestos activos}
		\UCli \BRref{RN-008}{Período de presupuesto}
		\UCli \BRref{RN-009}{Alertas de presupuesto}
		\UCli \RFref{RF-003}{Sistema de presupuestos personalizables}
		\UCli \RFref{RF-004}{Alertas proactivas de gastos}
	}
\end{UseCase}
