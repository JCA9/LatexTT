% !TeX root = analisis-diseno.tex

%=========================================================
\chapter{Modelo de datos}
\label{cap:datos}

Este capítulo presenta el diseño detallado del modelo de datos de FinanzApp, incluyendo el esquema de base de datos, relaciones entre entidades, estrategias de indexación, y consideraciones de rendimiento y escalabilidad para el almacenamiento de información financiera.

%---------------------------------------------------------
\section{Modelo conceptual de datos}

El modelo de datos de FinanzApp está diseñado para soportar eficientemente las operaciones de gestión financiera personal, análisis predictivo y procesamiento OCR, manteniendo la integridad y seguridad de los datos sensibles.

\begin{figure}[htpb!]
	\begin{center}
		% TODO: Agregar imagen correspondiente
		\caption{Modelo conceptual de datos de FinanzApp}
		\label{fig:modeloConceptualDatos}
	\end{center}
\end{figure}

% - - - - - - - - - - - - - - - - - - - - - - - - - - - - 
\subsection{Entidades principales}

\begin{description}
	\item[\textbf{users}] Información de usuarios del sistema con datos de perfil y configuración
	\item[\textbf{transactions}] Registro de todas las transacciones financieras (gastos e ingresos)
	\item[\textbf{categories}] Categorías de clasificación para transacciones, predefinidas y personalizadas
	\item[\textbf{budgets}] Presupuestos configurados por usuarios para control de gastos
	\item[\textbf{financial\_goals}] Metas de ahorro y objetivos financieros de usuarios
	\item[\textbf{receipts}] Información de tickets y recibos procesados por OCR
	\item[\textbf{notifications}] Historial de notificaciones enviadas a usuarios
	\item[\textbf{user\_sessions}] Gestión de sesiones de usuario para seguridad
\end{description}

%---------------------------------------------------------
\section{Modelo lógico de datos}

% - - - - - - - - - - - - - - - - - - - - - - - - - - - - 
\subsection{Esquema de base de datos}

\begin{figure}[htpb!]
	\begin{center}
		% TODO: Agregar imagen correspondiente
		\caption{Esquema lógico de la base de datos}
		\label{fig:esquemaBaseDatos}
	\end{center}
\end{figure}

% - - - - - - - - - - - - - - - - - - - - - - - - - - - - 
\subsection{Definición de tablas principales}

\textbf{Tabla users:}
\begin{verbatim}
CREATE TABLE users (
    user_id UUID PRIMARY KEY DEFAULT gen_random_uuid(),
    email VARCHAR(255) UNIQUE NOT NULL,
    password_hash VARCHAR(255) NOT NULL,
    first_name VARCHAR(100) NOT NULL,
    last_name VARCHAR(100) NOT NULL,
    date_of_birth DATE,
    phone_number VARCHAR(20),
    profile_image_url TEXT,
    preferred_language VARCHAR(5) DEFAULT 'es-MX',
    timezone VARCHAR(50) DEFAULT 'America/Mexico_City',
    monthly_income DECIMAL(10,2),
    currency VARCHAR(3) DEFAULT 'MXN',
    notification_preferences JSONB,
    privacy_settings JSONB,
    created_at TIMESTAMP WITH TIME ZONE DEFAULT NOW(),
    updated_at TIMESTAMP WITH TIME ZONE DEFAULT NOW(),
    last_login_at TIMESTAMP WITH TIME ZONE,
    is_active BOOLEAN DEFAULT true,
    is_email_verified BOOLEAN DEFAULT false
);
\end{verbatim}

\textbf{Tabla transactions:}
\begin{verbatim}
CREATE TABLE transactions (
    transaction_id UUID PRIMARY KEY DEFAULT gen_random_uuid(),
    user_id UUID NOT NULL REFERENCES users(user_id) ON DELETE CASCADE,
    category_id UUID REFERENCES categories(category_id),
    receipt_id UUID REFERENCES receipts(receipt_id),
    amount DECIMAL(10,2) NOT NULL CHECK (amount > 0),
    transaction_type VARCHAR(10) NOT NULL CHECK (transaction_type IN ('expense', 'income')),
    description TEXT NOT NULL,
    merchant_name VARCHAR(200),
    transaction_date DATE NOT NULL,
    location JSONB, -- {lat, lng, address}
    payment_method VARCHAR(50), -- cash, card, transfer, etc.
    is_recurring BOOLEAN DEFAULT false,
    recurrence_pattern JSONB, -- frequency, next_date, etc.
    tags TEXT[], -- user-defined tags
    confidence_score DECIMAL(3,2), -- OCR confidence (0.00-1.00)
    status VARCHAR(20) DEFAULT 'confirmed', -- pending, confirmed, deleted
    created_at TIMESTAMP WITH TIME ZONE DEFAULT NOW(),
    updated_at TIMESTAMP WITH TIME ZONE DEFAULT NOW()
);
\end{verbatim}

\textbf{Tabla categories:}
\begin{verbatim}
CREATE TABLE categories (
    category_id UUID PRIMARY KEY DEFAULT gen_random_uuid(),
    user_id UUID REFERENCES users(user_id) ON DELETE CASCADE, -- NULL for system categories
    parent_category_id UUID REFERENCES categories(category_id),
    name VARCHAR(100) NOT NULL,
    description TEXT,
    icon_name VARCHAR(50),
    color_hex VARCHAR(7), -- #RRGGBB format
    is_system_category BOOLEAN DEFAULT false,
    is_active BOOLEAN DEFAULT true,
    budget_limit DECIMAL(10,2),
    created_at TIMESTAMP WITH TIME ZONE DEFAULT NOW(),
    updated_at TIMESTAMP WITH TIME ZONE DEFAULT NOW(),
    UNIQUE(user_id, name) -- Unique per user, system categories have user_id NULL
);
\end{verbatim}

\textbf{Tabla budgets:}
\begin{verbatim}
CREATE TABLE budgets (
    budget_id UUID PRIMARY KEY DEFAULT gen_random_uuid(),
    user_id UUID NOT NULL REFERENCES users(user_id) ON DELETE CASCADE,
    name VARCHAR(100) NOT NULL,
    description TEXT,
    total_amount DECIMAL(10,2) NOT NULL CHECK (total_amount > 0),
    period_type VARCHAR(10) NOT NULL CHECK (period_type IN ('weekly', 'monthly', 'quarterly')),
    start_date DATE NOT NULL,
    end_date DATE NOT NULL,
    categories UUID[] NOT NULL, -- Array of category_ids
    alert_thresholds DECIMAL(3,2)[] DEFAULT '{0.70, 0.90}', -- Alert at 70% and 90%
    is_active BOOLEAN DEFAULT true,
    created_at TIMESTAMP WITH TIME ZONE DEFAULT NOW(),
    updated_at TIMESTAMP WITH TIME ZONE DEFAULT NOW(),
    CHECK (end_date > start_date)
);
\end{verbatim}

\textbf{Tabla financial\_goals:}
\begin{verbatim}
CREATE TABLE financial_goals (
    goal_id UUID PRIMARY KEY DEFAULT gen_random_uuid(),
    user_id UUID NOT NULL REFERENCES users(user_id) ON DELETE CASCADE,
    name VARCHAR(100) NOT NULL,
    description TEXT,
    target_amount DECIMAL(10,2) NOT NULL CHECK (target_amount > 0),
    current_amount DECIMAL(10,2) DEFAULT 0.00,
    target_date DATE NOT NULL,
    goal_type VARCHAR(20) NOT NULL CHECK (goal_type IN ('savings', 'debt_reduction', 'expense_reduction')),
    priority VARCHAR(10) DEFAULT 'medium' CHECK (priority IN ('low', 'medium', 'high')),
    status VARCHAR(15) DEFAULT 'active' CHECK (status IN ('active', 'paused', 'completed', 'cancelled')),
    auto_contribution DECIMAL(10,2) DEFAULT 0.00,
    reminder_frequency VARCHAR(10) DEFAULT 'weekly',
    created_at TIMESTAMP WITH TIME ZONE DEFAULT NOW(),
    updated_at TIMESTAMP WITH TIME ZONE DEFAULT NOW(),
    CHECK (target_date > CURRENT_DATE)
);
\end{verbatim}

\textbf{Tabla receipts:}
\begin{verbatim}
CREATE TABLE receipts (
    receipt_id UUID PRIMARY KEY DEFAULT gen_random_uuid(),
    user_id UUID NOT NULL REFERENCES users(user_id) ON DELETE CASCADE,
    original_image_url TEXT NOT NULL,
    processed_image_url TEXT,
    ocr_text TEXT,
    extracted_data JSONB, -- {amount, date, merchant, items, etc.}
    ocr_confidence DECIMAL(3,2), -- Overall confidence score
    processing_status VARCHAR(20) DEFAULT 'pending',
    -- pending, processing, completed, failed, manual_review
    error_message TEXT,
    processing_time_ms INTEGER,
    ocr_provider VARCHAR(50), -- google_vision, tesseract, etc.
    file_size_bytes INTEGER,
    image_dimensions JSONB, -- {width, height}
    created_at TIMESTAMP WITH TIME ZONE DEFAULT NOW(),
    processed_at TIMESTAMP WITH TIME ZONE,
    expires_at TIMESTAMP WITH TIME ZONE DEFAULT (NOW() + INTERVAL '90 days')
);
\end{verbatim}

%---------------------------------------------------------
\section{Índices y optimizaciones}

% - - - - - - - - - - - - - - - - - - - - - - - - - - - - 
\subsection{Estrategia de indexación}

\textbf{Índices principales:}
\begin{verbatim}
-- Transacciones por usuario y fecha (consulta más frecuente)
CREATE INDEX idx_transactions_user_date ON transactions(user_id, transaction_date DESC);

-- Transacciones por categoría para análisis
CREATE INDEX idx_transactions_category ON transactions(category_id) WHERE status = 'confirmed';

-- Búsqueda de texto en descripciones
CREATE INDEX idx_transactions_description_gin ON transactions USING gin(to_tsvector('spanish', description));

-- Presupuestos activos por usuario
CREATE INDEX idx_budgets_user_active ON budgets(user_id) WHERE is_active = true;

-- Recibos por estado de procesamiento
CREATE INDEX idx_receipts_status ON receipts(processing_status, created_at);

-- Usuarios activos para consultas de sistema
CREATE INDEX idx_users_active ON users(last_login_at) WHERE is_active = true;
\end{verbatim}

% - - - - - - - - - - - - - - - - - - - - - - - - - - - - 
\subsection{Particionamiento de datos}

\textbf{Particionamiento de transacciones por fecha:}
\begin{verbatim}
-- Tabla principal particionada
CREATE TABLE transactions_partitioned (
    LIKE transactions INCLUDING DEFAULTS INCLUDING CONSTRAINTS
) PARTITION BY RANGE (transaction_date);

-- Particiones mensuales para los últimos 12 meses
CREATE TABLE transactions_2024_01 PARTITION OF transactions_partitioned
    FOR VALUES FROM ('2024-01-01') TO ('2024-02-01');

CREATE TABLE transactions_2024_02 PARTITION OF transactions_partitioned
    FOR VALUES FROM ('2024-02-01') TO ('2024-03-01');
-- ... más particiones
\end{verbatim}

%---------------------------------------------------------
\section{Seguridad de datos}

% - - - - - - - - - - - - - - - - - - - - - - - - - - - - 
\subsection{Cifrado de información sensible}

\textbf{Campos cifrados a nivel de aplicación:}
\begin{itemize}
	\item \textbf{users.email}: Cifrado reversible para consultas
	\item \textbf{users.phone\_number}: Cifrado con salt único por usuario
	\item \textbf{transactions.description}: Cifrado para transacciones >1000 MXN
	\item \textbf{receipts.ocr\_text}: Cifrado completo del texto extraído
	\item \textbf{users.monthly\_income}: Hash con salt para análisis agregados
\end{itemize}

% - - - - - - - - - - - - - - - - - - - - - - - - - - - - 
\subsection{Control de acceso}

\textbf{Row Level Security (RLS):}
\begin{verbatim}
-- Activar RLS en tablas sensibles
ALTER TABLE transactions ENABLE ROW LEVEL SECURITY;
ALTER TABLE budgets ENABLE ROW LEVEL SECURITY;
ALTER TABLE financial_goals ENABLE ROW LEVEL SECURITY;

-- Política: usuarios solo ven sus propios datos
CREATE POLICY user_transactions_policy ON transactions
    FOR ALL TO app_user
    USING (user_id = current_setting('app.current_user_id')::uuid);

-- Política: administradores ven datos agregados
CREATE POLICY admin_transactions_policy ON transactions
    FOR SELECT TO admin_user
    USING (true);
\end{verbatim}

%---------------------------------------------------------
\section{Modelo de datos para analytics}

% - - - - - - - - - - - - - - - - - - - - - - - - - - - - 
\subsection{Tablas de agregación}

\textbf{Vista materializada para analytics diarios:}
\begin{verbatim}
CREATE MATERIALIZED VIEW daily_user_analytics AS
SELECT 
    user_id,
    transaction_date,
    COUNT(*) as transaction_count,
    SUM(CASE WHEN transaction_type = 'expense' THEN amount ELSE 0 END) as daily_expenses,
    SUM(CASE WHEN transaction_type = 'income' THEN amount ELSE 0 END) as daily_income,
    array_agg(DISTINCT category_id) as categories_used,
    AVG(confidence_score) as avg_ocr_confidence
FROM transactions 
WHERE status = 'confirmed'
GROUP BY user_id, transaction_date;

-- Índice para consultas rápidas
CREATE UNIQUE INDEX idx_daily_analytics ON daily_user_analytics(user_id, transaction_date);

-- Refresh automático cada hora
CREATE OR REPLACE FUNCTION refresh_daily_analytics()
RETURNS void AS $$
BEGIN
    REFRESH MATERIALIZED VIEW CONCURRENTLY daily_user_analytics;
END;
$$ LANGUAGE plpgsql;
\end{verbatim}

% - - - - - - - - - - - - - - - - - - - - - - - - - - - - 
\subsection{Métricas de comportamiento}

\textbf{Tabla de eventos de usuario:}
\begin{verbatim}
CREATE TABLE user_events (
    event_id UUID PRIMARY KEY DEFAULT gen_random_uuid(),
    user_id UUID NOT NULL REFERENCES users(user_id),
    event_type VARCHAR(50) NOT NULL, -- login, ocr_scan, budget_created, etc.
    event_data JSONB, -- Additional event-specific data
    session_id UUID,
    ip_address INET,
    user_agent TEXT,
    device_info JSONB, -- {platform, version, model}
    created_at TIMESTAMP WITH TIME ZONE DEFAULT NOW()
);

-- Índice para análisis de comportamiento
CREATE INDEX idx_user_events_type_date ON user_events(event_type, created_at);
CREATE INDEX idx_user_events_user_session ON user_events(user_id, session_id);
\end{verbatim}

%---------------------------------------------------------
\section{Respaldo y recuperación}

% - - - - - - - - - - - - - - - - - - - - - - - - - - - - 
\subsection{Estrategia de backup}

\textbf{Configuración de respaldos automatizados:}
\begin{itemize}
	\item \textbf{Full backup}: Diario a las 2:00 AM con retención de 30 días
	\item \textbf{Incremental backup}: Cada 6 horas con retención de 7 días
	\item \textbf{WAL archiving}: Continuo para point-in-time recovery
	\item \textbf{Cross-region replication}: Replica en región secundaria con lag <5 min
	\item \textbf{Backup verification}: Restauración automática en ambiente de prueba
\end{itemize}

% - - - - - - - - - - - - - - - - - - - - - - - - - - - - 
\subsection{Plan de recuperación ante desastres}

\textbf{Objetivos de recuperación:}
\begin{itemize}
	\item \textbf{RTO (Recovery Time Objective)}: 4 horas máximo
	\item \textbf{RPO (Recovery Point Objective)}: 15 minutos máximo de pérdida de datos
	\item \textbf{Failover automático}: Detección de fallas en <60 segundos
	\item \textbf{Procedimientos documentados}: Runbooks para diferentes escenarios
	\item \textbf{Testing periódico}: Simulacros de DR cada trimestre
\end{itemize}

%---------------------------------------------------------
\section{Monitoreo y mantenimiento}

% - - - - - - - - - - - - - - - - - - - - - - - - - - - - 
\subsection{Métricas de rendimiento}

\textbf{KPIs de base de datos:}
\begin{itemize}
	\item \textbf{Query performance}: <100ms para 95\% de consultas frecuentes
	\item \textbf{Connection pooling}: Utilización <80\% del pool
	\item \textbf{Index usage}: >90\% de consultas utilizan índices apropiados
	\item \textbf{Storage growth}: Monitoreo de crecimiento y proyecciones
	\item \textbf{Replication lag}: <5 segundos en replicas de lectura
\end{itemize}

% - - - - - - - - - - - - - - - - - - - - - - - - - - - - 
\subsection{Mantenimiento automatizado}

\textbf{Tareas programadas:}
\begin{verbatim}
-- Limpieza de recibos expirados
CREATE OR REPLACE FUNCTION cleanup_expired_receipts()
RETURNS void AS $$
BEGIN
    DELETE FROM receipts WHERE expires_at < NOW();
    VACUUM ANALYZE receipts;
END;
$$ LANGUAGE plpgsql;

-- Actualización de estadísticas
CREATE OR REPLACE FUNCTION update_table_statistics()
RETURNS void AS $$
BEGIN
    ANALYZE transactions;
    ANALYZE users;
    ANALYZE categories;
    ANALYZE budgets;
END;
$$ LANGUAGE plpgsql;

-- Programar ejecución con pg_cron
SELECT cron.schedule('cleanup-receipts', '0 2 * * *', 'SELECT cleanup_expired_receipts();');
SELECT cron.schedule('update-stats', '0 1 * * 0', 'SELECT update_table_statistics();');
\end{verbatim}
