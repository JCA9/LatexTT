% !TeX root = analisis-diseno.tex

%=========================================================
\chapter{Modelo de interacción}
\label{cap:interaccion}

Este capítulo describe las interacciones entre los usuarios y el sistema FinanzApp, así como las interacciones entre los diferentes componentes del sistema. Se detallan los casos de uso principales, las interfaces de interacción y los mensajes intercambiados entre los actores del sistema.

%---------------------------------------------------------
\section{Casos de uso detallados}

Los casos de uso representan las funcionalidades principales que ofrece FinanzApp a sus usuarios, describiendo las interacciones específicas entre actores y sistema.

% !TeX root = ../analisis-diseno.tex

%=========================================================
% Caso de uso: Registrar gasto mediante OCR

\begin{UseCase}{CU-001}{Registrar gasto mediante OCR}{
	El usuario utiliza la cámara de su dispositivo móvil para fotografiar un ticket o recibo, el sistema procesa la imagen mediante OCR, extrae la información relevante y registra automáticamente la transacción con categorización inteligente.
}
	\UCactor{\cdtActorRef{Usuario Joven}{Persona que desea registrar un gasto de manera rápida y automática}}
	\UCitem[Versión]{1.0}
	\UCitem[Autor]{Equipo FinanzApp}
	\UCitem[Fuente]{Análisis de requerimientos del usuario}
	\UCitem[Prioridad]{Alta}
	\UCitem[Estatus]{Activo}
	\UCitem[Fecha]{Noviembre 2024}

	% Precondiciones
	\UCpre{
		\UCli{El usuario debe estar autenticado en la aplicación}
		\UCli{El dispositivo debe tener cámara funcional}
		\UCli{El usuario debe tener conexión a internet activa}
		\UCli{El usuario no debe haber excedido el límite diario de 50 procesamiento OCR}
		\UCli{El ticket o recibo debe estar en formato físico legible}
	}

	% Postcondiciones
	\UCpost{
		\UCli{La transacción queda registrada en la base de datos del usuario}
		\UCli{El gasto se categoriza automáticamente}
		\UCli{Los presupuestos relevantes se actualizan}
		\UCli{Se genera una notificación de confirmación}
		\UCli{La imagen del ticket se almacena temporalmente (90 días)}
	}

	% Flujo principal
	\UCstart
		\UCstep[\UCactor] El usuario accede a la función de registro de gastos desde el dashboard principal
		\UCstep[\UCactor] El usuario selecciona la opción ``Escanear ticket'' 
		\UCstep[\UCsystem] El sistema activa la cámara del dispositivo y muestra la interfaz de captura
		\UCstep[\UCactor] El usuario posiciona el ticket dentro de las guías de alineación
		\UCstep[\UCactor] El usuario toma la fotografía presionando el botón de captura
		\UCstep[\UCsystem] El sistema muestra un preview de la imagen capturada
		\UCstep[\UCactor] El usuario confirma que la imagen es legible o decide tomarla nuevamente
		\UCstep[\UCsystem] El sistema envía la imagen al servicio OCR para procesamiento
		\UCstep[\UCsystem] El sistema muestra un indicador de progreso ``Procesando ticket...''
		\UCstep[\UCsystem] El servicio OCR extrae: monto, fecha, establecimiento, y posibles ítems
		\UCstep[\UCsystem] El sistema aplica algoritmos de categorización automática
		\UCstep[\UCsystem] El sistema presenta los datos extraídos al usuario para confirmación
		\UCstep[\UCactor] El usuario revisa la información y realiza correcciones si es necesario
		\UCstep[\UCactor] El usuario confirma el registro de la transacción
		\UCstep[\UCsystem] El sistema guarda la transacción en la base de datos
		\UCstep[\UCsystem] El sistema actualiza los presupuestos y estadísticas relevantes
		\UCstep[\UCsystem] El sistema muestra mensaje de confirmación: \MSGref{MSG-003}{OCR exitoso}
		\UCstep[\UCsystem] El sistema retorna al dashboard con la nueva transacción visible
	\UCend

	% Trayectorias alternativas
	\UCitem[Trayectorias alternativas]{
		\UCaltBegin{A}
			\UCaltitem[Condición]{En el paso 5 la calidad de la imagen es insuficiente}
			\UCaltitem[\UCsystem] El sistema detecta automáticamente baja calidad de imagen
			\UCaltitem[\UCsystem] El sistema muestra sugerencia: ``Imagen poco clara. ¿Deseas tomarla nuevamente?''
			\UCaltitem[\UCactor] El usuario decide tomar una nueva fotografía
			\UCaltitem[\UCsystem] Regresa al paso 4 del flujo principal
		\UCaltEnd

		\UCaltBegin{B}
			\UCaltitem[Condición]{En el paso 10 el OCR no puede procesar la imagen}
			\UCaltitem[\UCsystem] El sistema muestra mensaje de error: \MSGref{MSG-004}{Error en OCR}
			\UCaltitem[\UCsystem] El sistema ofrece opciones: ``Reintentar'' o ``Registro manual''
			\UCaltitem[\UCactor] El usuario selecciona ``Registro manual''
			\UCaltitem[\UCsystem] El sistema abre el formulario de registro manual
			\UCaltitem[\UCactor] El usuario ingresa manualmente: monto, fecha, descripción, categoría
			\UCaltitem[\UCsystem] Salta al paso 15 del flujo principal
		\UCaltEnd

		\UCaltBegin{C}
			\UCaltitem[Condición]{En el paso 13 la categorización automática tiene baja confianza (<60\%)}
			\UCaltitem[\UCsystem] El sistema marca la transacción como ``Requiere validación''
			\UCaltitem[\UCsystem] El sistema muestra opciones de categoría sugeridas
			\UCaltitem[\UCactor] El usuario selecciona la categoría correcta manualmente
			\UCaltitem[\UCsystem] El sistema aprende de la corrección para futuras categorizaciones
			\UCaltitem[\UCsystem] Continúa con el paso 15 del flujo principal
		\UCaltEnd

		\UCaltBegin{D}
			\UCaltitem[Condición]{El usuario ha alcanzado el límite diario de procesamiento OCR}
			\UCaltitem[\UCsystem] El sistema muestra mensaje: \MSGref{MSG-017}{Límite OCR diario}
			\UCaltitem[\UCsystem] El sistema ofrece la opción de registro manual únicamente
			\UCaltitem[\UCactor] El usuario puede proceder con registro manual o cancelar
		\UCaltEnd
	}

	% Referencias
	\UCitem[Referencias]{
		\UCli \BRref{RN-003}{Límite de procesamiento OCR}
		\UCli \BRref{RN-006}{Categorización automática}
		\UCli \BRref{RN-016}{Validación de tickets}
		\UCli \RFref{RF-001}{Digitalización automática de tickets}
		\UCli \RFref{RF-002}{Categorización inteligente de gastos}
	}
\end{UseCase}

% !TeX root = ../analisis-diseno.tex

%=========================================================
% Caso de uso: Configurar presupuesto personal

\begin{UseCase}{CU-002}{Configurar presupuesto personal}{
	El usuario crea un nuevo presupuesto definiendo el monto límite, período de tiempo, categorías incluidas y umbrales de alerta para controlar sus gastos de manera proactiva.
}
	\UCactor{\cdtActorRef{Usuario Joven}{Persona que desea establecer límites de gasto para controlar sus finanzas}}
	\UCitem[Versión]{1.0}
	\UCitem[Autor]{Equipo FinanzApp}
	\UCitem[Fuente]{Análisis de requerimientos de control financiero}
	\UCitem[Prioridad]{Alta}
	\UCitem[Estatus]{Activo}
	\UCitem[Fecha]{Noviembre 2024}

	% Precondiciones
	\UCpre{
		\UCli{El usuario debe estar autenticado en la aplicación}
		\UCli{El usuario debe tener al menos 30 días de historial de transacciones para sugerencias}
		\UCli{El usuario no debe tener más de 10 presupuestos activos simultáneamente}
		\UCli{Debe existir al menos una categoría de gastos disponible}
	}

	% Postcondiciones
	\UCpost{
		\UCli{El presupuesto queda creado y activo en el sistema}
		\UCli{Se configuran las alertas automáticas según los umbrales definidos}
		\UCli{El dashboard se actualiza mostrando el nuevo presupuesto}
		\UCli{Se inicia el monitoreo automático de gastos contra el presupuesto}
		\UCli{Se programa el envío de reportes periódicos}
	}

	% Flujo principal
	\UCstart
		\UCstep[\UCactor] El usuario accede a la sección ``Presupuestos'' desde el menú principal
		\UCstep[\UCsystem] El sistema muestra la lista de presupuestos existentes y botón ``Crear nuevo presupuesto''
		\UCstep[\UCactor] El usuario selecciona ``Crear nuevo presupuesto''
		\UCstep[\UCsystem] El sistema muestra el formulario de configuración de presupuesto
		\UCstep[\UCactor] El usuario ingresa el nombre del presupuesto (ej: ``Entretenimiento Diciembre'')
		\UCstep[\UCactor] El usuario define el monto total del presupuesto
		\UCstep[\UCsystem] El sistema sugiere un monto basado en el historial del usuario si está disponible
		\UCstep[\UCactor] El usuario selecciona el período: semanal, mensual o trimestral
		\UCstep[\UCactor] El usuario define las fechas de inicio y fin del período
		\UCstep[\UCsystem] El sistema muestra las categorías disponibles para incluir en el presupuesto
		\UCstep[\UCactor] El usuario selecciona una o más categorías a incluir (ej: Restaurantes, Entretenimiento)
		\UCstep[\UCsystem] El sistema muestra el gasto promedio histórico en las categorías seleccionadas
		\UCstep[\UCactor] El usuario configura los umbrales de alerta (por defecto: 70\% y 90\%)
		\UCstep[\UCactor] El usuario selecciona los tipos de notificación deseados (push, email, in-app)
		\UCstep[\UCactor] El usuario revisa el resumen de configuración y confirma la creación
		\UCstep[\UCsystem] El sistema valida que el presupuesto no exceda límites del sistema
		\UCstep[\UCsystem] El sistema crea el presupuesto y activa el monitoreo automático
		\UCstep[\UCsystem] El sistema muestra mensaje de confirmación y retorna a la lista de presupuestos
		\UCstep[\UCsystem] El sistema actualiza el dashboard con el nuevo presupuesto visible
	\UCend

	% Trayectorias alternativas
	\UCitem[Trayectorias alternativas]{
		\UCaltBegin{A}
			\UCaltitem[Condición]{En el paso 16 el usuario ya tiene 10 presupuestos activos}
			\UCaltitem[\UCsystem] El sistema muestra mensaje: ``Límite de presupuestos alcanzado. Desactiva un presupuesto existente para crear uno nuevo''
			\UCaltitem[\UCsystem] El sistema muestra lista de presupuestos activos con opción de desactivar
			\UCaltitem[\UCactor] El usuario selecciona un presupuesto para desactivar
			\UCaltitem[\UCsystem] El sistema desactiva el presupuesto seleccionado
			\UCaltitem[\UCsystem] Regresa al paso 16 del flujo principal
		\UCaltEnd

		\UCaltBegin{B}
			\UCaltitem[Condición]{En el paso 6 el sistema no puede sugerir monto por falta de historial}
			\UCaltitem[\UCsystem] El sistema muestra mensaje informativo: ``Sin historial suficiente para sugerencias''
			\UCaltitem[\UCsystem] El sistema proporciona rangos de referencia típicos para la categoría
			\UCaltitem[\UCactor] El usuario ingresa el monto deseado manualmente
			\UCaltitem[\UCsystem] Continúa con el paso 8 del flujo principal
		\UCaltEnd

		\UCaltBegin{C}
			\UCaltitem[Condición]{En el paso 11 no hay categorías creadas por el usuario}
			\UCaltitem[\UCsystem] El sistema muestra solo las categorías predefinidas del sistema
			\UCaltitem[\UCsystem] El sistema ofrece la opción ``Crear nueva categoría personalizada''
			\UCaltitem[\UCactor] El usuario puede seleccionar categorías existentes o crear nuevas
			\UCaltitem[\UCsystem] Si se crean nuevas categorías, se procesan antes de continuar
			\UCaltitem[\UCsystem] Continúa con el paso 12 del flujo principal
		\UCaltEnd

		\UCaltBegin{D}
			\UCaltitem[Condición]{En el paso 9 las fechas seleccionadas son inválidas}
			\UCaltitem[\UCsystem] El sistema valida que la fecha de fin sea posterior a la de inicio
			\UCaltitem[\UCsystem] El sistema valida que la fecha de inicio no sea en el pasado
			\UCaltitem[\UCsystem] El sistema muestra mensaje de error y resalta los campos incorrectos
			\UCaltitem[\UCactor] El usuario corrige las fechas
			\UCaltitem[\UCsystem] Continúa con el paso 10 del flujo principal
		\UCaltEnd
	}

	% Referencias
	\UCitem[Referencias]{
		\UCli \BRref{RN-007}{Presupuestos activos}
		\UCli \BRref{RN-008}{Período de presupuesto}
		\UCli \BRref{RN-009}{Alertas de presupuesto}
		\UCli \RFref{RF-003}{Sistema de presupuestos personalizables}
		\UCli \RFref{RF-004}{Alertas proactivas de gastos}
	}
\end{UseCase}

% !TeX root = ../analisis-diseno.tex

%=========================================================
% Caso de uso: Consultar analytics y predicciones

\begin{UseCase}{CU-003}{Consultar analytics y predicciones}{
	El usuario accede a reportes detallados de sus patrones de gasto, análisis predictivos personalizados, y recomendaciones financieras basadas en su comportamiento histórico y metas establecidas.
}
	\UCactor{\cdtActorRef{Usuario Joven}{Persona que desea comprender sus patrones financieros y recibir insights predictivos}}
	\UCitem[Versión]{1.0}
	\UCitem[Autor]{Equipo FinanzApp}
	\UCitem[Fuente]{Análisis de requerimientos de inteligencia financiera}
	\UCitem[Prioridad]{Media}
	\UCitem[Estatus]{Activo}
	\UCitem[Fecha]{Noviembre 2024}

	% Precondiciones
	\UCpre{
		\UCli{El usuario debe estar autenticado en la aplicación}
		\UCli{El usuario debe tener al menos 90 días de historial de transacciones para análisis de tendencias}
		\UCli{El usuario debe tener al menos 30 días de datos para predicciones básicas}
		\UCli{El sistema debe haber procesado los datos más recientes del usuario}
	}

	% Postcondiciones
	\UCpost{
		\UCli{Se muestran los analytics actualizados del usuario}
		\UCli{Se generan predicciones personalizadas basadas en patrones identificados}
		\UCli{Se registra la consulta para personalización futura de reportes}
		\UCli{Se actualizan las recomendaciones contextuales del usuario}
		\UCli{Se programa la próxima actualización de analytics}
	}

	% Flujo principal
	\UCstart
		\UCstep[\UCactor] El usuario accede a la sección ``Analytics'' desde el menú principal
		\UCstep[\UCsystem] El sistema verifica la disponibilidad de datos suficientes para análisis
		\UCstep[\UCsystem] El sistema muestra el dashboard de analytics con resumen general
		\UCstep[\UCsystem] El sistema presenta gráficos de tendencias de gastos por categoría (últimos 3 meses)
		\UCstep[\UCsystem] El sistema muestra comparativa mensual de gastos vs ingresos
		\UCstep[\UCactor] El usuario puede seleccionar diferentes períodos de análisis (mes, trimestre, año)
		\UCstep[\UCsystem] El sistema actualiza las visualizaciones según el período seleccionado
		\UCstep[\UCsystem] El sistema muestra patrones identificados: días de mayor gasto, categorías dominantes
		\UCstep[\UCsystem] El sistema presenta análisis predictivo: ``Es probable que gastes \$X en Y categoría esta semana''
		\UCstep[\UCactor] El usuario puede explorar predicciones específicas tocando en las tarjetas de insights
		\UCstep[\UCsystem] El sistema muestra detalles de la predicción con nivel de confianza y factores
		\UCstep[\UCsystem] El sistema presenta recomendaciones personalizadas basadas en los análisis
		\UCstep[\UCactor] El usuario puede marcar recomendaciones como útiles o descartarlas
		\UCstep[\UCsystem] El sistema ajusta futuras recomendaciones basado en el feedback del usuario
		\UCstep[\UCactor] El usuario puede exportar reportes específicos o compartir insights
		\UCstep[\UCsystem] El sistema genera el reporte solicitado y lo envía al email del usuario
		\UCstep[\UCsystem] El sistema registra las preferencias de visualización del usuario para futuras consultas
	\UCend

	% Trayectorias alternativas
	\UCitem[Trayectorias alternativas]{
		\UCaltBegin{A}
			\UCaltitem[Condición]{En el paso 2 no hay suficientes datos para análisis completo}
			\UCaltitem[\UCsystem] El sistema muestra mensaje informativo sobre la limitación de datos
			\UCaltitem[\UCsystem] El sistema presenta análisis básicos disponibles con los datos existentes
			\UCaltitem[\UCsystem] El sistema sugiere: ``Continúa registrando gastos para obtener análisis más precisos''
			\UCaltitem[\UCsystem] El sistema muestra timeline estimado para análisis completos
		\UCaltEnd

		\UCaltBegin{B}
			\UCaltitem[Condición]{En el paso 9 no se pueden generar predicciones confiables}
			\UCaltitem[\UCsystem] El sistema muestra mensaje: ``Predicciones en desarrollo. Necesitamos más datos de tu comportamiento''
			\UCaltitem[\UCsystem] El sistema presenta insights básicos disponibles: promedios, totales, comparaciones
			\UCaltitem[\UCsystem] El sistema muestra consejos generales de educación financiera
			\UCaltitem[\UCsystem] Continúa con el paso 12 del flujo principal (recomendaciones generales)
		\UCaltEnd

		\UCaltBegin{C}
			\UCaltitem[Condición]{El usuario solicita exportar datos pero no hay conexión a internet}
			\UCaltitem[\UCsystem] El sistema detecta la falta de conectividad
			\UCaltitem[\UCsystem] El sistema muestra mensaje: \MSGref{MSG-012}{Error de conexión}
			\UCaltitem[\UCsystem] El sistema ofrece ``Programar envío cuando se reconecte''
			\UCaltitem[\UCactor] El usuario puede aceptar el programa de envío diferido
			\UCaltitem[\UCsystem] El sistema agenda el envío y notifica cuando se complete
		\UCaltEnd

		\UCaltBegin{D}
			\UCaltitem[Condición]{El sistema detecta patrones de gasto preocupantes}
			\UCaltitem[\UCsystem] El sistema genera alerta contextual sobre comportamiento financiero riesgoso
			\UCaltitem[\UCsystem] El sistema presenta recomendaciones específicas para mejorar la situación
			\UCaltitem[\UCsystem] El sistema ofrece conectar con recursos educationales adicionales
			\UCaltitem[\UCactor] El usuario puede aceptar recomendaciones o marcarlas como no relevantes
			\UCaltitem[\UCsystem] Continúa con el flujo principal incorporando las preferencias del usuario
		\UCaltEnd
	}

	% Referencias
	\UCitem[Referencias]{
		\UCli \BRref{RN-019}{Histórico mínimo}
		\UCli \BRref{RN-023}{Análisis de tendencias}
		\UCli \RFref{RF-005}{Análisis predictivo de gastos}
		\UCli \RFref{RF-006}{Reportes y visualizaciones avanzadas}
		\UCli \RFref{RF-007}{Recomendaciones financieras personalizadas}
	}
\end{UseCase}

% !TeX root = ../analisis-diseno.tex

%=========================================================
% Caso de uso: Gestionar categorías personalizadas

\begin{UseCase}{CU-004}{Gestionar categorías personalizadas}{
	El usuario crea, modifica y organiza categorías personalizadas para clasificar sus transacciones de acuerdo a sus necesidades específicas, mejorando la precisión del análisis financiero personal.
}
	\UCactor{\cdtActorRef{Usuario Joven}{Persona que desea personalizar la clasificación de sus gastos según sus hábitos específicos}}
	\UCitem[Versión]{1.0}
	\UCitem[Autor]{Equipo FinanzApp}
	\UCitem[Fuente]{Análisis de requerimientos de personalización}
	\UCitem[Prioridad]{Media}
	\UCitem[Estatus]{Activo}
	\UCitem[Fecha]{Noviembre 2024}

	% Precondiciones
	\UCpre{
		\UCli{El usuario debe estar autenticado en la aplicación}
		\UCli{El usuario no debe tener más de 20 categorías personalizadas creadas}
		\UCli{Deben existir categorías predefinidas del sistema como base}
	}

	% Postcondiciones
	\UCpost{
		\UCli{Las categorías modificadas quedan actualizadas en el sistema}
		\UCli{Las transacciones futuras pueden utilizar las nuevas categorías}
		\UCli{El algoritmo de categorización automática aprende de las nuevas categorías}
		\UCli{Los reportes y analytics incorporan las categorías personalizadas}
	}

	% Flujo principal
	\UCstart
		\UCstep[\UCactor] El usuario accede a ``Configuración'' > ``Gestionar categorías''
		\UCstep[\UCsystem] El sistema muestra la lista de categorías disponibles: predefinidas y personalizadas
		\UCstep[\UCsystem] El sistema diferencia visualmente las categorías del sistema vs personalizadas
		\UCstep[\UCactor] El usuario selecciona ``Crear nueva categoría''
		\UCstep[\UCsystem] El sistema muestra el formulario de creación de categoría
		\UCstep[\UCactor] El usuario ingresa el nombre de la nueva categoría (ej: ``Mascotas'')
		\UCstep[\UCactor] El usuario añade una descripción opcional de la categoría
		\UCstep[\UCactor] El usuario selecciona un icono de la galería disponible
		\UCstep[\UCactor] El usuario elige un color representativo de una paleta predefinida
		\UCstep[\UCsystem] El sistema muestra preview de cómo se verá la categoría en la interfaz
		\UCstep[\UCactor] El usuario puede definir una categoría padre si desea crear subcategorías
		\UCstep[\UCactor] El usuario puede establecer un presupuesto por defecto para la categoría
		\UCstep[\UCactor] El usuario confirma la creación de la categoría
		\UCstep[\UCsystem] El sistema valida que el nombre no esté duplicado
		\UCstep[\UCsystem] El sistema crea la categoría y la añade a la lista personalizada
		\UCstep[\UCsystem] El sistema actualiza el motor de categorización para reconocer la nueva categoría
		\UCstep[\UCsystem] El sistema muestra mensaje de confirmación y retorna a la lista de categorías
	\UCend

	% Trayectorias alternativas
	\UCitem[Trayectorias alternativas]{
		\UCaltBegin{A}
			\UCaltitem[Condición]{En el paso 14 el nombre de la categoría ya existe}
			\UCaltitem[\UCsystem] El sistema muestra mensaje de error: ``Ya existe una categoría con este nombre''
			\UCaltitem[\UCsystem] El sistema sugiere nombres alternativos similares
			\UCaltitem[\UCactor] El usuario modifica el nombre o selecciona una sugerencia
			\UCaltitem[\UCsystem] Continúa con el paso 15 del flujo principal
		\UCaltEnd

		\UCaltBegin{B}
			\UCaltitem[Condición]{El usuario selecciona ``Editar'' en una categoría existente}
			\UCaltitem[\UCsystem] El sistema verifica si la categoría es editable (no es categoría del sistema)
			\UCaltitem[\UCsystem] El sistema abre el formulario con los datos actuales pre-cargados
			\UCaltitem[\UCactor] El usuario modifica los campos deseados
			\UCaltitem[\UCactor] El usuario confirma los cambios
			\UCaltitem[\UCsystem] El sistema actualiza la categoría y todas las transacciones asociadas
			\UCaltitem[\UCsystem] El sistema muestra confirmación de actualización exitosa
		\UCaltEnd

		\UCaltBegin{C}
			\UCaltitem[Condición]{El usuario intenta eliminar una categoría con transacciones asociadas}
			\UCaltitem[\UCsystem] El sistema detecta transacciones vinculadas a la categoría
			\UCaltitem[\UCsystem] El sistema muestra mensaje: ``Esta categoría tiene X transacciones asociadas''
			\UCaltitem[\UCsystem] El sistema ofrece opciones: ``Reasignar transacciones'' o ``Cancelar eliminación''
			\UCaltitem[\UCactor] El usuario selecciona una nueva categoría para reasignar las transacciones
			\UCaltitem[\UCsystem] El sistema migra las transacciones y elimina la categoría original
		\UCaltEnd

		\UCaltBegin{D}
			\UCaltitem[Condición]{El usuario ha alcanzado el límite de 20 categorías personalizadas}
			\UCaltitem[\UCsystem] El sistema muestra mensaje de límite alcanzado
			\UCaltitem[\UCsystem] El sistema sugiere: ``Elimina categorías no utilizadas o combina categorías similares''
			\UCaltitem[\UCsystem] El sistema muestra estadísticas de uso de cada categoría personalizada
			\UCaltitem[\UCactor] El usuario puede eliminar categorías poco utilizadas
			\UCaltitem[\UCsystem] Una vez liberado espacio, permite crear nueva categoría
		\UCaltEnd

		\UCaltBegin{E}
			\UCaltitem[Condición]{El usuario selecciona ``Importar categorías'' de otro usuario}
			\UCaltitem[\UCsystem] El sistema muestra categorías públicas populares de la comunidad
			\UCaltitem[\UCsystem] El sistema permite filtrar por tipo de usuario (estudiante, profesional, etc.)}
			\UCaltitem[\UCactor] El usuario selecciona categorías de interés para importar
			\UCaltitem[\UCsystem] El sistema añade las categorías seleccionadas como personalizadas del usuario
			\UCaltitem[\UCsystem] El sistema actualiza el motor de categorización con las nuevas categorías
		\UCaltEnd
	}

	% Referencias
	\UCitem[Referencias]{
		\UCli \BRref{RN-018}{Categorías personalizadas}
		\UCli \RFref{RF-008}{Gestión de categorías personalizadas}
		\UCli \RFref{RF-009}{Importación de categorías de comunidad}
	}
\end{UseCase}

% !TeX root = ../analisis-diseno.tex

%=========================================================
% Caso de uso: Sincronizar datos entre dispositivos

\begin{UseCase}{CU-005}{Sincronizar datos entre dispositivos}{
	El sistema mantiene automáticamente sincronizados los datos financieros del usuario entre la aplicación móvil y la plataforma web, garantizando consistencia y acceso actualizado desde cualquier dispositivo.
}
	\UCactor{\cdtActorRef{Usuario Joven}{Persona que utiliza FinanzApp desde múltiples dispositivos y requiere datos actualizados}}
	\UCsysactor{\cdtActorRef{Sistema de Sincronización}{Componente automático que gestiona la consistency de datos}}
	\UCitem[Versión]{1.0}
	\UCitem[Autor]{Equipo FinanzApp}
	\UCitem[Fuente]{Análisis de requerimientos multiplataforma}
	\UCitem[Prioridad]{Alta}
	\UCitem[Estatus]{Activo}
	\UCitem[Fecha]{Noviembre 2024}

	% Precondiciones
	\UCpre{
		\UCli{El usuario debe estar autenticado en al menos un dispositivo}
		\UCli{Al menos uno de los dispositivos debe tener conexión a internet activa}
		\UCli{Los dispositivos deben tener la aplicación actualizada a versión compatible}
		\UCli{El usuario debe tener una sesión válida no expirada}
	}

	% Postcondiciones
	\UCpost{
		\UCli{Todos los dispositivos activos muestran los mismos datos actualizados}
		\UCli{Los conflictos de sincronización se resuelven automáticamente}
		\UCli{Se mantiene un log de sincronización para auditoría}
		\UCli{Los dispositivos offline almacenan cambios para sincronizar al reconectarse}
	}

	% Flujo principal
	\UCstart
		\UCstep[\UCactor] El usuario realiza una modificación en cualquier dispositivo (ej: registra un gasto en móvil)
		\UCstep[\UCsystem] El sistema local guarda el cambio con timestamp y marca como ``pendiente de sincronización''
		\UCstep[\UCsystem] El sistema detecta conexión a internet disponible
		\UCstep[\UCsystem] El sistema de sincronización inicia el proceso de upload de cambios pendientes
		\UCstep[\UCsystem] El servidor recibe los cambios y valida la integridad de los datos
		\UCstep[\UCsystem] El servidor procesa los cambios y actualiza la base de datos central
		\UCstep[\UCsystem] El servidor genera eventos de notificación para otros dispositivos del usuario
		\UCstep[\UCsystem] Los dispositivos conectados reciben notificación de cambios disponibles
		\UCstep[\UCsystem] Cada dispositivo descarga únicamente los cambios incrementales
		\UCstep[\UCsystem] Los dispositivos actualizan su cache local y interfaz de usuario
		\UCstep[\UCsystem] El sistema confirma sincronización exitosa a todos los dispositivos
		\UCstep[\UCactor] El usuario ve los cambios reflejados en todos sus dispositivos activos
		\UCstep[\UCsystem] El sistema marca la sincronización como completada y actualiza timestamps
	\UCend

	% Trayectorias alternativas
	\UCitem[Trayectorias alternativas]{
		\UCaltBegin{A}
			\UCaltitem[Condición]{En el paso 3 no hay conexión a internet disponible}
			\UCaltitem[\UCsystem] El sistema almacena los cambios localmente con flag ``offline''
			\UCaltitem[\UCsystem] El sistema muestra indicador de ``Trabajando offline'' al usuario
			\UCaltitem[\UCsystem] El sistema activa listener para detectar reconexión automáticamente
			\UCaltitem[\UCsystem] Al detectar conectividad, retoma desde el paso 4 del flujo principal
			\UCaltitem[\UCsystem] El sistema procesa todos los cambios offline acumulados
		\UCaltEnd

		\UCaltBegin{B}
			\UCaltitem[Condición]{En el paso 5 se detecta conflicto de datos (edición concurrente)}
			\UCaltitem[\UCsystem] El sistema identifica el conflicto comparando timestamps y versiones
			\UCaltitem[\UCsystem] El sistema aplica estrategia ``Last-Write-Wins'' para resolver automáticamente
			\UCaltitem[\UCsystem] El sistema notifica al usuario sobre la resolución del conflicto
			\UCaltitem[\UCsystem] El sistema ofrece opción de deshacer la resolución automática si es incorrecta
			\UCaltitem[\UCactor] El usuario puede aceptar la resolución o realizar corrección manual
			\UCaltitem[\UCsystem] Continúa con el paso 6 del flujo principal
		\UCaltEnd

		\UCaltBegin{C}
			\UCaltitem[Condición]{La sincronización falla por error del servidor}
			\UCaltitem[\UCsystem] El sistema detecta el error y activa mecanismo de reintentos
			\UCaltitem[\UCsystem] El sistema espera backoff exponencial (1s, 2s, 4s, 8s) entre reintentos
			\UCaltitem[\UCsystem] El sistema intenta hasta 3 veces antes de marcar como fallida
			\UCaltitem[\UCsystem] Si todos los reintentos fallan, muestra mensaje: \MSGref{MSG-012}{Error de conexión}
			\UCaltitem[\UCsystem] El sistema programa reintento automático en 15 minutos
			\UCaltitem[\UCactor] El usuario puede forzar reintento manual o continuar trabajando offline
		\UCaltEnd

		\UCaltBegin{D}
			\UCaltitem[Condición]{El usuario cambia de dispositivo durante una sincronización}
			\UCaltitem[\UCsystem] El sistema detecta login desde nuevo dispositivo
			\UCaltitem[\UCsystem] El sistema prioriza la descarga completa de datos para el nuevo dispositivo
			\UCaltitem[\UCsystem] El sistema muestra indicador de ``Sincronizando datos...'' durante la descarga inicial
			\UCaltitem[\UCsystem] Una vez completada la descarga inicial, se integra al flujo normal de sincronización
			\UCaltitem[\UCsystem] El sistema notifica: \MSGref{MSG-008}{Sincronización completa}
		\UCaltEnd

		\UCaltBegin{E}
			\UCaltitem[Condición]{Se detecta actividad sospechosa de sincronización}
			\UCaltitem[\UCsystem] El sistema detecta patrones anómalos (múltiples dispositivos, ubicaciones extrañas)}
			\UCaltitem[\UCsystem] El sistema suspende temporalmente la sincronización automática
			\UCaltitem[\UCsystem] El sistema envía alerta de seguridad al email del usuario
			\UCaltitem[\UCsystem] El sistema requiere re-autenticación para continuar sincronizando
			\UCaltitem[\UCactor] El usuario debe confirmar la actividad legítima y re-autenticarse
			\UCaltitem[\UCsystem] Tras confirmación, se reanuda la sincronización normal
		\UCaltEnd
	}

	% Referencias
	\UCitem[Referencias]{
		\UCli \BRref{RN-011}{Sincronización de datos}
		\UCli \BRref{RN-012}{Backup automático}
		\UCli \RFref{RF-010}{Sincronización en tiempo real}
		\UCli \RFref{RF-011}{Resolución automática de conflictos}
		\UCli \RFref{RF-012}{Funcionalidad offline}
	}
\end{UseCase}


%---------------------------------------------------------
\section{Interfaces de usuario}

Las interfaces de usuario definen cómo los usuarios interactúan con las diferentes funcionalidades de FinanzApp a través de las plataformas móvil y web.

% Interfaces de usuario simplificadas
\section{IU-001 Dashboard Principal}
\textbf{Propósito:} Pantalla principal que muestra resumen de gastos y acceso rápido a funciones principales.

\textbf{Elementos principales:}
\begin{itemize}
    \item Balance actual del mes
    \item Gráfico de gastos por categoría
    \item Botón ``Registrar Gasto''
    \item Últimas transacciones
    \item Notificaciones de presupuesto
\end{itemize}

\section{IU-002 Interfaz de Cámara OCR}
\textbf{Propósito:} Interfaz para capturar tickets mediante la cámara del dispositivo.

\textbf{Elementos principales:}
\begin{itemize}
    \item Vista previa de la cámara
    \item Guías de alineación para el ticket
    \item Botón de captura
    \item Flash automático/manual
    \item Botón de confirmación/rehacer
\end{itemize}

\section{IU-003 Configuración de Presupuestos}
\textbf{Propósito:} Permite al usuario establecer y modificar presupuestos por categoría.

\textbf{Elementos principales:}
\begin{itemize}
    \item Lista de categorías con presupuestos actuales
    \item Deslizadores para ajustar montos
    \item Indicadores de progreso de gasto
    \item Botones para activar/desactivar alertas
    \item Histórico de cambios de presupuesto
\end{itemize}

\section{IU-004 Analytics de Gastos}
\textbf{Propósito:} Presenta análisis visuales y predicciones sobre los patrones de gasto del usuario.

\textbf{Elementos principales:}
\begin{itemize}
    \item Gráficos de tendencias mensuales
    \item Comparativas por período
    \item Predicciones de gastos futuros
    \item Filtros por categoría y fecha
    \item Exportación de reportes
\end{itemize}

\section{IU-005 Perfil de Usuario}
\textbf{Propósito:} Configuración de datos personales y preferencias de la aplicación.

\textbf{Elementos principales:}
\begin{itemize}
    \item Información personal del usuario
    \item Configuración de notificaciones
    \item Preferencias de moneda y región
    \item Configuración de seguridad
    \item Sincronización entre dispositivos
\end{itemize}

%---------------------------------------------------------
\section{Mensajes del sistema}

Los mensajes del sistema proporcionan retroalimentación al usuario sobre el estado de las operaciones, errores, confirmaciones y notificaciones importantes.

% !TeX root = ../analisis-diseno.tex

%=========================================================
% Mensajes del sistema para FinanzApp

\begin{mensajes}
   \mensaje{MSG-001}{Bienvenida al usuario}{¡Hola! Bienvenido a FinanzApp. Estamos aqui para ayudarte a tomar control de tus finanzas de manera fácil y automática.}
   \mensaje{MSG-002}{OCR en progreso}{Procesando tu ticket... Esto puede tomar unos segundos. ¡No cierres la aplicación!}
   \mensaje{MSG-003}{OCR exitoso}{¡Perfecto! Hemos extraído la información de tu ticket. Verifica que todo esté correcto antes de guardar.}
   \mensaje{MSG-004}{Error en OCR}{No pudimos procesar la imagen claramente. Intenta tomar otra foto con mejor iluminación o ingresa los datos manualmente.}
   \mensaje{MSG-005}{Presupuesto 70\%}{⚠️ Has gastado el 70\% de tu presupuesto en [Categoría]. Te quedan \$[Monto] para el resto del mes.}
   \mensaje{MSG-006}{Presupuesto excedido}{🚨 Has superado tu presupuesto de [Categoría] por \$[Monto]. ¿Quieres ajustar tu presupuesto o revisar tus gastos?}
   \mensaje{MSG-007}{Meta alcanzada}{🎉 ¡Felicidades! Has alcanzado tu meta de ahorro: [Nombre de la meta]. ¿Quieres establecer una nueva meta?}
   \mensaje{MSG-008}{Sincronización completa}{✅ Tus datos se han sincronizado correctamente en todos tus dispositivos.}
   \mensaje{MSG-009}{Gasto atípico detectado}{🤔 Detectamos un gasto inusual de \$[Monto] en [Categoría]. ¿Es correcto o necesitas reclasificarlo?}
   \mensaje{MSG-010}{Recordatorio de presupuesto}{💡 Llevas 3 días sin registrar gastos. ¿Has tenido gastos que no hemos capturado?}
   \mensaje{MSG-011}{Consejo de ahorro}{💰 Tip: Has gastado 30\% más en [Categoría] este mes. Considera usar [Sugerencia específica] para ahorrar.}
   \mensaje{MSG-012}{Error de conexión}{Sin conexión a internet. Trabajando en modo offline. Los cambios se sincronizarán cuando te reconectes.}
   \mensaje{MSG-013}{Actualización disponible}{Nueva versión disponible con mejoras en OCR y nuevas funciones. ¿Quieres actualizar ahora?}
   \mensaje{MSG-014}{Backup completado}{✅ Respaldo de datos completado exitosamente. Tus datos financieros están seguros.}
   \mensaje{MSG-015}{Sesión expirada}{Tu sesión ha expirado por seguridad. Por favor inicia sesión nuevamente para continuar.}
   \mensaje{MSG-016}{Categorización automática}{🤖 Hemos categorizado automáticamente tu gasto como [Categoría]. ¿Es correcto?}
   \mensaje{MSG-017}{Límite OCR diario}{Has alcanzado el límite de 50 tickets procesados hoy. Intenta mañana o ingresa datos manualmente.}
   \mensaje{MSG-018}{Predicción de gasto}{📊 Según tus patrones, es probable que gastes \$[Monto] adicionales en [Categoría] esta semana.}
   \mensaje{MSG-019}{Invitación a calificar}{¿Te está gustando FinanzApp? Nos encantaría conocer tu opinión. ¿Podrías calificarnos en la App Store?}
   \mensaje{MSG-020}{Exportación exitosa}{📄 Tus datos han sido exportados exitosamente. El archivo se ha enviado a tu email registrado.}
\end{mensajes}


%---------------------------------------------------------
\section{Flujos de interacción principales}

% - - - - - - - - - - - - - - - - - - - - - - - - - - - - 
\subsection{Flujo de onboarding del usuario}

\begin{figure}[htpb!]
	\begin{center}
		% TODO: Agregar imagen correspondiente
		\caption{Flujo de onboarding del usuario}
		\label{fig:flujoOnboarding}
	\end{center}
\end{figure}

El proceso de onboarding guía al nuevo usuario a través de la configuración inicial de su perfil, establecimiento de categorías de gastos, definición de presupuestos iniciales y comprensión de las funcionalidades principales de la aplicación.

\textbf{Pasos del flujo:}
\begin{enumerate}
	\item Registro con email y confirmación
	\item Configuración de perfil básico (nombre, edad, objetivos financieros)
	\item Tutorial interactivo de funcionalidades principales
	\item Primer escaneo de ticket (demo)
	\item Configuración de presupuesto inicial
	\item Personalización de categorías
	\item Configuración de notificaciones
	\item Acceso al dashboard principal
\end{enumerate}

% - - - - - - - - - - - - - - - - - - - - - - - - - - - - 
\subsection{Flujo de digitalización completa}

\begin{figure}[htpb!]
	\begin{center}
		% TODO: Agregar imagen correspondiente
		\caption{Flujo de digitalización completa de gasto}
		\label{fig:flujoDigitalizacion}
	\end{center}
\end{figure}

Este flujo muestra la experiencia completa del usuario desde la decisión de registrar un gasto hasta la confirmación de la transacción categorizada y almacenada en su perfil financiero.

\textbf{Pasos del flujo:}
\begin{enumerate}
	\item Usuario accede a la función de cámara
	\item Fotografía el ticket o recibo
	\item Sistema procesa imagen con OCR
	\item Extracción de datos (monto, fecha, establecimiento)
	\item Categorización automática inteligente
	\item Presentación de resultados al usuario
	\item Confirmación o corrección por parte del usuario
	\item Almacenamiento en base de datos
	\item Actualización de presupuestos y estadísticas
	\item Notificación de éxito
\end{enumerate}

% - - - - - - - - - - - - - - - - - - - - - - - - - - - - 
\subsection{Flujo de análisis predictivo}

\begin{figure}[htpb!]
	\begin{center}
		% TODO: Agregar imagen correspondiente
		\caption{Flujo de análisis predictivo y alertas}
		\label{fig:flujoAnalisisPredictivo}
	\end{center}
\end{figure}

Este flujo describe cómo el sistema analiza automáticamente los patrones de gasto del usuario para generar insights, predicciones y alertas proactivas que ayuden en la toma de decisiones financieras.

%---------------------------------------------------------
\section{Patrones de interacción}

% - - - - - - - - - - - - - - - - - - - - - - - - - - - - 
\subsection{Interacción táctil optimizada}

La aplicación móvil está diseñada para maximizar la usabilidad en dispositivos táctiles:

\begin{itemize}
	\item \textbf{Gestos principales}: Tap, swipe, pinch-to-zoom, pull-to-refresh
	\item \textbf{Navegación}: Bottom tab bar para secciones principales
	\item \textbf{Acciones rápidas}: Float action button para registro de gastos
	\item \textbf{Feedback háptico}: Confirmaciones y notificaciones importantes
	\item \textbf{Accesibilidad}: Soporte para TalkBack y tamaños de texto dinámicos
\end{itemize}

% - - - - - - - - - - - - - - - - - - - - - - - - - - - - 
\subsection{Interacción web responsive}

La plataforma web se adapta a diferentes tamaños de pantalla y métodos de interacción:

\begin{itemize}
	\item \textbf{Breakpoints}: Mobile (<768px), tablet (768-1024px), desktop (>1024px)
	\item \textbf{Navegación adaptativa}: Hamburger menu en móvil, sidebar en desktop
	\item \textbf{Dashboard configurable}: Widgets reorganizables por drag-and-drop
	\item \textbf{Teclado shortcuts}: Acceso rápido a funciones principales
	\item \textbf{Tooltips contextuales}: Ayuda integrada sin interrumpir el flujo
\end{itemize}

% - - - - - - - - - - - - - - - - - - - - - - - - - - - - 
\subsection{Interacción por voz (futura implementación)}

Consideraciones para futuras implementaciones de interacción por voz:

\begin{itemize}
	\item \textbf{Comandos básicos}: ``Registrar gasto de \$50 en comida''
	\item \textbf{Consultas}: ``¿Cuánto he gastado esta semana?''
	\item \textbf{Configuración}: ``Crear presupuesto de \$2000 para entretenimiento''
	\item \textbf{Privacidad}: Procesamiento local cuando sea posible
	\item \textbf{Confirmación visual}: Siempre mostrar resultado de comando de voz
\end{itemize}

%---------------------------------------------------------
\section{Especificaciones de usabilidad}

% - - - - - - - - - - - - - - - - - - - - - - - - - - - - 
\subsection{Métricas de usabilidad objetivo}

\begin{itemize}
	\item \textbf{Tiempo de onboarding}: Máximo 3 minutos para configuración inicial
	\item \textbf{Tiempo de registro de gasto}: Máximo 15 segundos desde cámara hasta confirmación
	\item \textbf{Tasa de error en OCR}: Máximo 15\% de tickets requieren corrección manual
	\item \textbf{Satisfacción del usuario}: Score NPS mínimo de 7/10
	\item \textbf{Retención de usuarios}: 70\% activos después de 30 días
	\item \textbf{Tiempo de carga}: Máximo 2 segundos para cualquier pantalla
\end{itemize}

% - - - - - - - - - - - - - - - - - - - - - - - - - - - - 
\subsection{Principios de diseño UX}

\begin{itemize}
	\item \textbf{Simplicidad}: Interfaces limpias con máximo 3 niveles de navegación
	\item \textbf{Consistencia}: Patrones de interacción uniformes en toda la aplicación
	\item \textbf{Retroalimentación}: Respuesta inmediata a todas las acciones del usuario
	\item \textbf{Tolerancia a errores}: Deshacer acciones críticas, confirmaciones para eliminaciones
	\item \textbf{Personalización}: Adaptación a preferencias y patrones de uso individual
	\item \textbf{Gamificación sutil}: Logros y progreso sin interrumpir la funcionalidad principal
\end{itemize}

%---------------------------------------------------------
\section{Integración con sistemas externos}

% - - - - - - - - - - - - - - - - - - - - - - - - - - - - 
\subsection{APIs bancarias}

\begin{figure}[htpb!]
	\begin{center}
		% TODO: Agregar imagen correspondiente
		\caption{Integración con APIs bancarias}
		\label{fig:integracionBancaria}
	\end{center}
\end{figure}

La integración con APIs bancarias permite la importación automática de transacciones y la validación de gastos registrados manualmente.

\textbf{Funcionalidades de integración:}
\begin{itemize}
	\item Importación automática de transacciones con tarjeta
	\item Validación de gastos registrados manualmente
	\item Detección de transacciones duplicadas
	\item Reconciliación de saldos
	\item Notificaciones de discrepancias
\end{itemize}

% - - - - - - - - - - - - - - - - - - - - - - - - - - - - 
\subsection{Servicios de terceros}

\begin{itemize}
	\item \textbf{Servicio OCR}: Google Cloud Vision API para procesamiento de imágenes
	\item \textbf{Notificaciones push}: Firebase Cloud Messaging para notificaciones móviles
	\item \textbf{Analytics}: Google Analytics para métricas de uso y comportamiento
	\item \textbf{Crashlytics}: Firebase Crashlytics para monitoreo de errores
	\item \textbf{Storage}: AWS S3 para almacenamiento de imágenes de tickets
	\item \textbf{CDN}: CloudFlare para distribución global de contenido estático
\end{itemize}
