% !TeX root = analisis-diseno.tex

%=========================================================
\chapter{Diseño de interfaz de usuario}
\label{cap:interfaces}

Este capítulo presenta el diseño de la interfaz de usuario de FinanzApp, incluyendo los principios de diseño UX/UI aplicados, wireframes, mockups de alta fidelidad, y especificaciones de interacción para las plataformas móvil y web.

%---------------------------------------------------------
\section{Principios de diseño}

El diseño de FinanzApp se basa en principios de usabilidad y experiencia de usuario específicamente adaptados para jóvenes adultos y el contexto de finanzas personales.

% - - - - - - - - - - - - - - - - - - - - - - - - - - - - 
\subsection{Principios fundamentales}

\begin{description}
	\item[\textbf{Simplicidad intuitiva}] Interfaces limpias que minimizan la carga cognitiva y permiten completar tareas financieras sin fricción.
	
	\item[\textbf{Feedback inmediato}] Respuesta visual instantánea a todas las acciones del usuario, especialmente importante en el procesamiento OCR.
	
	\item[\textbf{Gamificación sutil}] Elementos de gamificación que motivan el uso consistente sin ser intrusivos o infantiles.
	
	\item[\textbf{Personalización adaptativa}] La interfaz se adapta a los patrones de uso y preferencias individuales del usuario.
	
	\item[\textbf{Accesibilidad universal}] Diseño inclusivo que cumple con WCAG 2.1 AA para usuarios con diferentes capacidades.
	
	\item[\textbf{Confianza y seguridad}] Elementos visuales que transmiten seguridad y confiabilidad en el manejo de datos financieros.
\end{description}

% - - - - - - - - - - - - - - - - - - - - - - - - - - - - 
\subsection{Guía de estilo visual}

\begin{figure}[htpb!]
	\begin{center}
		% TODO: Agregar imagen correspondiente
		\caption{Guía de estilo visual de FinanzApp}
		\label{fig:guiaEstilo}
	\end{center}
\end{figure}

\textbf{Paleta de colores:}
\begin{itemize}
	\item \textbf{Primario}: \#2563EB (Azul confianza) - Botones principales, enlaces
	\item \textbf{Secundario}: \#059669 (Verde prosperidad) - Ingresos, metas alcanzadas
	\item \textbf{Acento}: \#DC2626 (Rojo alerta) - Gastos, alertas, presupuestos excedidos
	\item \textbf{Neutro}: \#6B7280 (Gris moderno) - Texto secundario, bordes
	\item \textbf{Fondo}: \#F9FAFB (Blanco hueso) - Fondos, cards
\end{itemize}

\textbf{Tipografía:}
\begin{itemize}
	\item \textbf{Primaria}: Inter (Sans-serif moderna, alta legibilidad)
	\item \textbf{Monoespaciada}: JetBrains Mono (Para montos y códigos)
	\item \textbf{Jerarquía}: H1(32px), H2(24px), H3(20px), Body(16px), Caption(14px)
\end{itemize}

%---------------------------------------------------------
\section{Arquitectura de información}

% - - - - - - - - - - - - - - - - - - - - - - - - - - - - 
\subsection{Estructura de navegación móvil}

\begin{figure}[htpb!]
	\begin{center}
		% TODO: Agregar imagen correspondiente
		\caption{Estructura de navegación de la aplicación móvil}
		\label{fig:navegacionMovil}
	\end{center}
\end{figure}

La navegación móvil utiliza un patrón de tabs inferiores para acceso rápido a las funciones principales:

\begin{itemize}
	\item \textbf{Inicio}: Dashboard con resumen financiero y acciones rápidas
	\item \textbf{Gastos}: Historial de transacciones y funcionalidad de escaneo
	\item \textbf{Analytics}: Gráficos, insights y análisis predictivo  
	\item \textbf{Presupuestos}: Configuración y monitoreo de presupuestos
	\item \textbf{Perfil}: Configuración de usuario y preferencias
\end{itemize}

% - - - - - - - - - - - - - - - - - - - - - - - - - - - - 
\subsection{Estructura de navegación web}

\begin{figure}[htpb!]
	\begin{center}
		% TODO: Agregar imagen correspondiente
		\caption{Estructura de navegación de la plataforma web}
		\label{fig:navegacionWeb}
	\end{center}
\end{figure}

La plataforma web utiliza un sidebar colapsible que se adapta al tamaño de pantalla:

\begin{itemize}
	\item \textbf{Dashboard}: Vista general con widgets configurables
	\item \textbf{Transacciones}: Gestión completa de gastos e ingresos
	\item \textbf{Análisis}: Reportes detallados y visualizaciones avanzadas
	\item \textbf{Presupuestos y Metas}: Herramientas de planificación financiera
	\item \textbf{Configuración}: Ajustes avanzados y administración de cuenta
\end{itemize}

%---------------------------------------------------------
\section{Diseño de pantallas principales}

% - - - - - - - - - - - - - - - - - - - - - - - - - - - - 
\subsection{Dashboard principal móvil}

\begin{figure}[htpb!]
	\begin{center}
		% TODO: Agregar imagen correspondiente
		\caption{Mockup del dashboard principal móvil}
		\label{fig:dashboardMovil}
	\end{center}
\end{figure}

El dashboard móvil presenta información financiera clave de forma concisa:

\textbf{Componentes principales:}
\begin{itemize}
	\item Header con saludo personalizado y notificaciones
	\item Card de balance actual con tendencia visual
	\item Resumen de gastos del mes con gráfico circular
	\item Lista de transacciones recientes (últimas 5)
	\item Floating Action Button para registro rápido de gastos
	\item Quick actions para funciones frecuentes
\end{itemize}

% - - - - - - - - - - - - - - - - - - - - - - - - - - - - 
\subsection{Interfaz de cámara OCR}

\begin{figure}[htpb!]
	\begin{center}
		% TODO: Agregar imagen correspondiente
		\caption{Mockup de la interfaz de cámara OCR}
		\label{fig:camaraOCR}
	\end{center}
\end{figure}

La interfaz de captura está optimizada para facilitar el escaneo de tickets:

\textbf{Elementos de la interfaz:}
\begin{itemize}
	\item Viewfinder con guías de alineación para tickets
	\item Botón de captura con feedback háptico
	\item Toggle para flash automático según condiciones de luz
	\item Preview de imagen capturada con opción de rehacer
	\item Indicador de progreso durante procesamiento OCR
	\item Resultados extraídos con opción de corrección manual
\end{itemize}

% - - - - - - - - - - - - - - - - - - - - - - - - - - - - 
\subsection{Dashboard web con widgets}

\begin{figure}[htpb!]
	\begin{center}
		% TODO: Agregar imagen correspondiente
		\caption{Mockup del dashboard web con widgets configurables}
		\label{fig:dashboardWeb}
	\end{center}
\end{figure}

El dashboard web aprovecha el espacio adicional para mostrar más información:

\textbf{Widgets disponibles:}
\begin{itemize}
	\item Resumen financiero mensual con comparativa
	\item Gráfico de tendencias de gastos por categoría
	\item Progreso de presupuestos con alertas visuales
	\item Metas financieras con timeline de progreso
	\item Transacciones recientes con filtros rápidos
	\item Quick insights y recomendaciones personalizadas
\end{itemize}

%---------------------------------------------------------
\section{Patrones de interacción}

% - - - - - - - - - - - - - - - - - - - - - - - - - - - - 
\subsection{Gestos táctiles móviles}

\begin{figure}[htpb!]
	\begin{center}
		% TODO: Agregar imagen correspondiente
		\caption{Patrones de gestos táctiles implementados}
		\label{fig:gestosTactiles}
	\end{center}
\end{figure}

\textbf{Gestos implementados:}
\begin{itemize}
	\item \textbf{Swipe right}: Categorizar transacción como gasto frecuente
	\item \textbf{Swipe left}: Eliminar o archivar transacción
	\item \textbf{Pull to refresh}: Actualizar datos y sincronizar
	\item \textbf{Long press}: Mostrar menú contextual con acciones secundarias
	\item \textbf{Pinch to zoom}: Ampliar gráficos para análisis detallado
	\item \textbf{Double tap}: Marcar transacción como favorita o recurrente
\end{itemize}

% - - - - - - - - - - - - - - - - - - - - - - - - - - - - 
\subsection{Estados de interacción}

\textbf{Estados visuales definidos:}
\begin{itemize}
	\item \textbf{Default}: Estado normal de elementos interactivos
	\item \textbf{Hover}: Retroalimentación en web para elementos clickeables
	\item \textbf{Active/Pressed}: Feedback táctil inmediato al tocar
	\item \textbf{Focused}: Estado de navegación por teclado (accesibilidad)
	\item \textbf{Disabled}: Elementos no disponibles con indicación visual clara
	\item \textbf{Loading}: Estados de carga con spinners y skeleton screens
	\item \textbf{Error}: Indicadores de error con guías de recuperación
\end{itemize}

%---------------------------------------------------------
\section{Responsive design}

% - - - - - - - - - - - - - - - - - - - - - - - - - - - - 
\subsection{Breakpoints y adaptaciones}

\begin{figure}[htpb!]
	\begin{center}
		% TODO: Agregar imagen correspondiente
		\caption{Adaptaciones responsive para diferentes dispositivos}
		\label{fig:responsiveBreakpoints}
	\end{center}
\end{figure}

\textbf{Breakpoints definidos:}
\begin{itemize}
	\item \textbf{Mobile}: 320px - 767px (Navegación por tabs, stacks verticales)
	\item \textbf{Tablet}: 768px - 1023px (Navegación híbrida, 2 columnas)
	\item \textbf{Desktop}: 1024px+ (Sidebar fijo, múltiples columnas)
	\item \textbf{Large Desktop}: 1440px+ (Dashboard expandido, widgets adicionales)
\end{itemize}

% - - - - - - - - - - - - - - - - - - - - - - - - - - - - 
\subsection{Adaptaciones específicas}

\textbf{Mobile-first adaptaciones:}
\begin{itemize}
	\item Navegación colapsible con hamburger menu
	\item Cards apiladas verticalmente con scroll
	\item Formularios de una columna con inputs optimizados para touch
	\item Gráficos simplificados con gestos de zoom
	\item FAB (Floating Action Button) para acción principal
\end{itemize}

\textbf{Desktop enhancements:}
\begin{itemize}
	\item Sidebar persistente con navegación contextual
	\item Dashboard de múltiples columnas con drag-and-drop
	\item Tooltips informativos sin interferir con el flujo
	\item Keyboard shortcuts para power users
	\item Multi-panel views para comparación de datos
\end{itemize}

%---------------------------------------------------------
\section{Accesibilidad y usabilidade}

% - - - - - - - - - - - - - - - - - - - - - - - - - - - - 
\subsection{Cumplimiento WCAG 2.1}

\textbf{Nivel AA compliance:}
\begin{itemize}
	\item \textbf{Perceptible}: Contraste mínimo 4.5:1, texto escalable hasta 200\%
	\item \textbf{Operable}: Navegación por teclado completa, tiempo ajustable
	\item \textbf{Comprensible}: Lenguaje claro, patrones consistentes
	\item \textbf{Robusto}: Compatible con lectores de pantalla, HTML semántico
\end{itemize}

% - - - - - - - - - - - - - - - - - - - - - - - - - - - - 
\subsection{Consideraciones especiales}

\textbf{Finanzas y accesibilidad:}
\begin{itemize}
	\item Representación alternativa de gráficos (tablas, texto descriptivo)
	\item Alertas de presupuesto en múltiples modalidades (visual, auditiva, háptica)
	\item Confirmaciones adicionales para acciones financieras críticas
	\item Soporte para tecnologías asistivas (VoiceOver, TalkBack)
	\item Modo alto contraste para usuarios con problemas visuales
	\item Tamaños de texto adaptables para diferentes necesidades
\end{itemize}

%---------------------------------------------------------
\section{Prototipado y validación}

% - - - - - - - - - - - - - - - - - - - - - - - - - - - - 
\subsection{Proceso de diseño iterativo}

\begin{figure}[htpb!]
	\begin{center}
		% TODO: Agregar imagen correspondiente
		\caption{Proceso de diseño iterativo con validación de usuarios}
		\label{fig:procesoIterativo}
	\end{center}
\end{figure}

\textbf{Fases del proceso:}
\begin{enumerate}
	\item \textbf{Research}: Entrevistas con usuarios target, análisis de competencia
	\item \textbf{Wireframing}: Sketches y wireframes de baja fidelidad
	\item \textbf{Prototyping}: Prototipos interactivos en Figma
	\item \textbf{User Testing}: Pruebas de usabilidad con usuarios reales
	\item \textbf{Iteration}: Refinamiento basado en feedback
	\item \textbf{Implementation}: Desarrollo con design system
	\item \textbf{Validation}: A/B testing y métricas de uso
\end{enumerate}

% - - - - - - - - - - - - - - - - - - - - - - - - - - - - 
\subsection{Métricas de validación UX}

\textbf{KPIs de experiencia de usuario:}
\begin{itemize}
	\item \textbf{Task completion rate}: >90\% para flujos principales
	\item \textbf{Time on task}: <15 segundos para registro de gasto
	\item \textbf{Error rate}: <5\% en formularios críticos
	\item \textbf{User satisfaction}: SUS score >80/100
	\item \textbf{Retention}: >70\% usuarios activos a 30 días
	\item \textbf{Feature adoption}: >60\% usuarios usan OCR regularmente
\end{itemize}
