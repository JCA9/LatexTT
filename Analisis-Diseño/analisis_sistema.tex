\chapter{Análisis del Sistema}

\section{Descripción General}

El sistema de gestión financiera personal es una aplicación móvil diseñada para ayudar a jóvenes adultos a desarrollar hábitos financieros saludables mediante la automatización del registro de gastos y la aplicación de metodologías de presupuestación probadas.

\section{Objetivos del Sistema}

\subsection{Objetivo General}
Desarrollar una aplicación móvil que facilite la gestión financiera personal a través del registro automatizado de gastos mediante OCR y la implementación del método de presupuestación por sobres de Dave Ramsey.

\subsection{Objetivos Específicos}
\begin{itemize}
    \item Automatizar el registro de gastos mediante captura y procesamiento OCR de tickets
    \item Implementar un sistema de categorización inteligente de gastos
    \item Proporcionar herramientas de planificación y seguimiento de presupuestos
    \item Ofrecer funcionalidades de gestión y planificación de pagos de deudas
    \item Generar reportes y estadísticas de comportamiento financiero
\end{itemize}

\section{Actores del Sistema}

\subsection{Administrador Financiero}
\textbf{Descripción:} Usuario principal del sistema, propietario de la cuenta financiera.

\textbf{Responsabilidades:}
\begin{itemize}
    \item Registrar ingresos y egresos
    \item Configurar presupuestos por categorías
    \item Gestionar información de deudas
    \item Consultar reportes y estadísticas
    \item Configurar alertas y notificaciones
\end{itemize}

\section{Requisitos Funcionales}

Los requerimientos funcionales describen las funciones específicas que el sistema debe realizar para satisfacer las necesidades del usuario. Se han identificado 40 requerimientos funcionales organizados por funcionalidad y plataforma.

\subsection{Requerimientos Compartidos (App Móvil y Plataforma Web)}

\subsubsection{Gestión de Usuarios}
\textbf{RF-01: Registro de usuario} - El sistema debe permitir el registro de nuevos usuarios mediante correo electrónico y contraseña, validando los datos ingresados antes de crear la cuenta.

\textbf{RF-02: Inicio de sesión} - El sistema debe permitir que los usuarios registrados inicien sesión de forma segura tanto en la aplicación móvil como en la plataforma web.

\textbf{RF-03: Recuperación de contraseña} - El sistema debe permitir al usuario recuperar su contraseña mediante un enlace de restablecimiento enviado a su correo electrónico registrado.

\textbf{RF-04: Edición de perfil} - El usuario podrá modificar su información personal, como nombre, correo electrónico o fotografía de perfil, desde la aplicación o la web.

\textbf{RF-05: Eliminación de cuenta} - El sistema debe permitir al usuario eliminar permanentemente su cuenta y los datos asociados, solicitando confirmación previa antes de proceder.

\subsubsection{Gestión de Transacciones}
\textbf{RF-06: Registro de ingresos} - El usuario podrá registrar manualmente sus ingresos indicando el monto, la fecha y una descripción opcional, tanto desde la aplicación móvil como desde la plataforma web.

\textbf{RF-07: Registro de egresos} - El usuario podrá registrar egresos de forma manual o automática, especificando monto, fecha, categoría y descripción.

\textbf{RF-08: Registro de egresos manualmente} - El sistema permitirá al usuario ingresar egresos de manera manual cuando no disponga de un ticket digital o imagen para procesar automáticamente.

\textbf{RF-09: Registro de ticket manualmente} - El usuario podrá registrar un ticket introduciendo manualmente los datos del comprobante, como fecha, monto y establecimiento.

\subsubsection{Gestión de Presupuestos}
\textbf{RF-10: Configuración de presupuesto} - El sistema debe permitir al usuario definir presupuestos por categoría o periodo, estableciendo límites de gasto que serán monitoreados automáticamente.

\subsubsection{Consultas y Visualización}
\textbf{RF-11: Visualización de historial de gastos} - El usuario podrá consultar el historial de gastos registrados en ambos entornos, filtrando por fecha, categoría o tipo de transacción.

\textbf{RF-12: Visualización de gráficas o estadísticas} - El sistema debe generar gráficas y estadísticas interactivas que representen el comportamiento financiero del usuario de manera clara y comprensible.

\subsubsection{Reportes}
\textbf{RF-13: Generación de reportes en PDF} - La plataforma web y la aplicación móvil deberán permitir la generación de reportes financieros en formato PDF, con opción de descarga o visualización en pantalla.

\subsubsection{Metas de Ahorro}
\textbf{RF-14: Configuración de metas de ahorro} - El usuario podrá establecer metas de ahorro personalizadas, definir plazos y montos objetivo, y consultar su progreso mediante indicadores visuales.

\subsection{Requerimientos Exclusivos de la Aplicación Móvil}

\subsubsection{Personalización de Interfaz}
\textbf{RF-15: Cambio de tema de la aplicación} - El sistema debe permitir al usuario cambiar el tema visual de la aplicación (modo claro u oscuro) para mejorar la experiencia de uso.

\subsubsection{Procesamiento OCR}
\textbf{RF-16: Captura de foto del ticket} - La aplicación móvil debe permitir al usuario tomar una fotografía del ticket de compra utilizando la cámara del dispositivo para su posterior procesamiento.

\textbf{RF-17: Captura múltiple de tickets} - El sistema debe permitir tomar varias fotografías del mismo ticket en caso de que la información no sea visible en una sola imagen.

\textbf{RF-18: Confirmación de datos escaneados} - El sistema debe mostrar los datos extraídos por el OCR para que el usuario los revise y confirme antes de su almacenamiento definitivo.

\textbf{RF-19: Etiquetado personalizado de tickets} - El usuario podrá agregar etiquetas personalizadas a los tickets para clasificar los gastos según sus propios criterios o proyectos personales.

\textbf{RF-20: Corrección de datos escaneados} - El sistema debe permitir al usuario editar manualmente los datos que fueron mal reconocidos durante el proceso de OCR.

\subsubsection{Notificaciones y Alertas}
\textbf{RF-21: Visualización de historial de notificaciones} - El usuario podrá consultar un historial de notificaciones dentro de la aplicación móvil, incluyendo alertas de presupuesto, deudas y movimientos recientes.

\subsubsection{Análisis y Estadísticas}
\textbf{RF-22: Visualización de estadísticas por categoría} - El sistema debe permitir al usuario visualizar las estadísticas de gasto organizadas por categoría y periodo, mostrando porcentajes o montos totales.

\subsubsection{Gestión de Categorías}
\textbf{RF-23: Modificación de categorías de gasto} - El usuario podrá crear, editar o eliminar categorías de gasto personalizadas para una organización más detallada de sus finanzas.

\subsubsection{Gestión de Deudas}
\textbf{RF-24: Establecimiento de alertas de deudas} - El sistema debe permitir configurar alertas automáticas para recordar al usuario fechas de vencimiento o pago de sus deudas registradas.

\textbf{RF-25: Configuración de deudas} - El usuario podrá registrar nuevas deudas, especificando monto, acreedor, fecha de vencimiento y frecuencia de pago.

\textbf{RF-26: Registro de deudas} - El sistema debe almacenar la información de las deudas registradas por el usuario y permitir su actualización o eliminación posterior.

\textbf{RF-27: Planificación de pago de deudas} - El sistema debe generar un plan de pago de deudas que muestre fechas, montos parciales y estado de cada obligación financiera.

\subsubsection{Presupuestos Personalizados}
\textbf{RF-28: Planificación de presupuestos personales} - El sistema debe permitir al usuario definir presupuestos personalizados por categoría o tipo de gasto para un mejor control financiero.

\subsubsection{Gestión de Tickets}
\textbf{RF-29: Consulta de tickets} - El usuario podrá consultar los tickets previamente registrados, filtrando por fecha, categoría o monto.

\textbf{RF-30: Visualización de gastos} - El sistema debe mostrar al usuario un resumen de sus gastos organizados por día, semana y mes, junto con las categorías más frecuentes.

\textbf{RF-31: Eliminación de tickets} - El sistema debe permitir al usuario eliminar tickets registrados de forma manual o automática, previa confirmación para evitar pérdidas accidentales de información.

\subsection{Requerimientos Exclusivos de la Plataforma Web}

\subsubsection{Consejos Personalizados}
\textbf{RF-32: Consulta de consejos personalizados} - La plataforma web debe mostrar consejos financieros personalizados basados en los hábitos de consumo del usuario, apoyándose en el análisis de sus gastos registrados.

\subsubsection{Búsqueda Avanzada}
\textbf{RF-33: Búsqueda y filtrado de tickets} - El sistema debe permitir al usuario buscar y filtrar tickets según criterios como fecha, categoría, monto o palabras clave dentro de las descripciones.

\subsubsection{Reportes Especializados}
\textbf{RF-34: Generación de reportes en Excel} - El sistema debe permitir generar y descargar reportes financieros en formato Excel, con hojas separadas por categorías o periodos.

\textbf{RF-35: Compartir reportes por correo electrónico} - El usuario podrá compartir los reportes generados directamente por correo electrónico desde la plataforma web, sin necesidad de descargarlos localmente.

\subsubsection{Análisis Comparativo}
\textbf{RF-36: Comparación de gastos entre meses} - El sistema debe generar comparativas gráficas y tabulares de los gastos registrados en distintos meses, mostrando incrementos o reducciones porcentuales.

\textbf{RF-37: Consulta de balance vs egresos} - El sistema debe mostrar un balance general que compare los ingresos frente a los egresos, calculando el resultado neto del periodo seleccionado.

\subsubsection{Estadísticas Avanzadas}
\textbf{RF-38: Visualización de estadísticas avanzadas} - La plataforma debe ofrecer un panel con estadísticas detalladas del comportamiento financiero del usuario, incluyendo tendencias, promedios y porcentajes de gasto por categoría.

\subsubsection{Carga de Documentos}
\textbf{RF-39: Carga de ticket digital} - El sistema debe permitir la carga de tickets en formato digital (PDF o imagen) desde el explorador web, procesándolos automáticamente con el módulo OCR.

\subsubsection{Gestión Web de Metas}
\textbf{RF-40: Configuración de metas de ahorro desde la web} - El usuario podrá definir y modificar metas de ahorro directamente desde la plataforma web, sincronizándose con los datos de la aplicación móvil.

\section{Requisitos No Funcionales}

Los requisitos no funcionales definen las características y restricciones del sistema que no están directamente relacionadas con las funcionalidades específicas, pero que son esenciales para garantizar su calidad, rendimiento, seguridad y usabilidad.

\subsection{RNF-01: Seguridad}
Las contraseñas y datos personales deben almacenarse cifrados utilizando un algoritmo seguro (bcrypt) y no se guardará texto plano. Toda la comunicación entre cliente y servidor debe utilizar protocolos seguros (HTTPS/TLS).

\subsection{RNF-02: Mantenibilidad}
El código debe estar modularizado, documentado y versionado con control GitHub para facilitar actualizaciones. Cada módulo podrá ser actualizado sin afectar la funcionalidad global del sistema.

\subsection{RNF-03: Portabilidad}
La aplicación móvil debe funcionar correctamente en dispositivos Android 10 o superior, manteniendo su funcionalidad y diseño sin errores en diferentes tipos de pantallas y resoluciones.

\subsection{RNF-04: Compatibilidad}
La plataforma web debe ser compatible con navegadores modernos como Chrome (v135), Edge (v135) y Firefox (v140), garantizando funcionalidad completa en cada uno de ellos.

\subsection{RNF-05: Usabilidad}
La interfaz de usuario de la aplicación móvil será desarrollada conforme a las Flutter Usability Guidelines y Material Design, asegurando botones, menús y elementos interactivos claros, consistentes y accesibles. La versión web seguirá las React Accessibility Guidelines, priorizando navegación intuitiva y tiempos de respuesta óptimos. El sistema deberá permitir su uso sin capacitación previa.

\subsection{RNF-06: Rendimiento}
\begin{itemize}
    \item El sistema debe procesar imágenes OCR en menos de 5 segundos
    \item Los tiempos de respuesta para consultas no deben exceder 2 segundos
    \item La aplicación debe cargar completamente en menos de 3 segundos
    \item El sistema debe soportar al menos 1000 usuarios concurrentes
\end{itemize}

\subsection{RNF-07: Disponibilidad}
El sistema debe mantener una disponibilidad del 99.5% durante horarios de uso típicos (6:00 AM - 11:00 PM), con períodos de mantenimiento programados fuera de estos horarios.

\subsection{RNF-08: Escalabilidad}
El sistema debe ser capaz de escalar horizontalmente para manejar incrementos en la carga de usuarios, soportando un crecimiento del 200% en la base de usuarios sin degradación significativa del rendimiento.

\subsection{RNF-09: Integridad de Datos}
\begin{itemize}
    \item Todas las transacciones financieras deben ser registradas con marca de tiempo inmutable
    \item El sistema debe mantener respaldos automáticos diarios de todos los datos
    \item Debe implementarse un sistema de auditoría que registre todas las modificaciones
\end{itemize}

\subsection{RNF-10: Privacidad}
\begin{itemize}
    \item Los datos personales y financieros deben cumplir con regulaciones de protección de datos
    \item El sistema debe permitir la eliminación completa de datos del usuario bajo solicitud
    \item Las imágenes de tickets deben ser procesadas localmente cuando sea posible
\end{itemize}