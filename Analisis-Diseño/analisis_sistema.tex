\chapter{Análisis del Sistema}

\section{Descripción General}

El sistema de gestión financiera personal es una aplicación móvil diseñada para ayudar a jóvenes adultos a desarrollar hábitos financieros saludables mediante la automatización del registro de gastos y la aplicación de metodologías de presupuestación probadas.

\section{Objetivos del Sistema}

\subsection{Objetivo General}
Desarrollar una aplicación móvil que facilite la gestión financiera personal a través del registro automatizado de gastos mediante OCR y la implementación del método de presupuestación por sobres de Dave Ramsey.

\subsection{Objetivos Específicos}
\begin{itemize}
    \item Automatizar el registro de gastos mediante captura y procesamiento OCR de tickets
    \item Implementar un sistema de categorización inteligente de gastos
    \item Proporcionar herramientas de planificación y seguimiento de presupuestos
    \item Ofrecer funcionalidades de gestión y planificación de pagos de deudas
    \item Generar reportes y estadísticas de comportamiento financiero
\end{itemize}

\section{Actores del Sistema}

Los actores del sistema representan entidades externas que interactúan directamente con el sistema de gestión financiera personal. Para este prototipo se ha identificado un actor principal que abarca el alcance funcional definido para la aplicación.

\subsection{Administrador Financiero}

\subsubsection{Definición}
Persona física que utiliza el sistema para gestionar sus finanzas personales de manera individual. Corresponde al usuario final propietario de la cuenta y los datos financieros almacenados en el sistema.

\subsubsection{Características Demográficas}
\begin{itemize}
    \item \textbf{Edad:} Jóvenes adultos de 18 años en adelante
    \item \textbf{Nivel tecnológico:} Usuario intermedio de dispositivos móviles y navegadores web
    \item \textbf{Contexto de uso:} Personas que buscan desarrollar hábitos financieros saludables y mejorar su responsabilidad económica
\end{itemize}

\subsubsection{Objetivos y Motivaciones}
\begin{itemize}
    \item Automatizar el registro de gastos para reducir el esfuerzo manual
    \item Obtener visibilidad clara de sus patrones de consumo
    \item Implementar estrategias de presupuestación efectivas
    \item Desarrollar disciplina financiera a través de herramientas digitales
    \item Planificar y controlar el pago de deudas personales
\end{itemize}

\subsubsection{Responsabilidades en el Sistema}
\begin{itemize}
    \item \textbf{Gestión de cuenta:} Registrarse, autenticarse y mantener actualizada su información personal
    \item \textbf{Registro de transacciones:} Capturar tickets mediante fotografías, registrar ingresos y egresos manuales
    \item \textbf{Configuración de presupuestos:} Definir límites de gasto por categorías y períodos específicos
    \item \textbf{Gestión de deudas:} Registrar obligaciones financieras y configurar recordatorios de pago
    \item \textbf{Análisis de datos:} Consultar reportes, estadísticas y gráficas de comportamiento financiero
    \item \textbf{Personalización:} Configurar categorías, alertas, notificaciones y preferencias de la aplicación
\end{itemize}

\subsubsection{Interacciones con el Sistema}
\begin{itemize}
    \item \textbf{Interfaz móvil:} Uso principal a través de la aplicación Android para registro cotidiano de gastos
    \item \textbf{Interfaz web:} Acceso complementario para análisis detallado, generación de reportes y configuraciones avanzadas
    \item \textbf{Frecuencia de uso:} Interacción diaria para registro de gastos, consulta semanal de reportes y configuración mensual de presupuestos
\end{itemize}

\subsubsection{Justificación del Alcance}
Para este prototipo académico se ha definido un único actor principal dado que:
\begin{itemize}
    \item El sistema se enfoca en finanzas personales individuales, no corporativas o familiares
    \item La complejidad de múltiples roles excedería el alcance del proyecto
    \item No se requiere administración técnica del sistema en el entorno de prototipo controlado
\end{itemize}

\section{Requisitos Funcionales}

Los requerimientos funcionales describen las funciones específicas que el sistema debe realizar para satisfacer las necesidades del usuario. Se han identificado 40 requerimientos funcionales organizados por funcionalidad y plataforma.

\subsection{Requerimientos Compartidos (App Móvil y Plataforma Web)}

\subsubsection{Gestión de Usuarios}
\textbf{RF-01: Registro de usuario} - El sistema debe permitir el registro de nuevos usuarios mediante correo electrónico y contraseña, validando los datos ingresados antes de crear la cuenta.

\textbf{RF-02: Inicio de sesión} - El sistema debe permitir que los usuarios registrados inicien sesión de forma segura tanto en la aplicación móvil como en la plataforma web.

\textbf{RF-03: Recuperación de contraseña} - El sistema debe permitir al usuario recuperar su contraseña mediante un enlace de restablecimiento enviado a su correo electrónico registrado.

\textbf{RF-04: Edición de perfil} - El usuario podrá modificar su información personal, como nombre, correo electrónico o fotografía de perfil, desde la aplicación o la web.

\textbf{RF-05: Eliminación de cuenta} - El sistema debe permitir al usuario eliminar permanentemente su cuenta y los datos asociados, solicitando confirmación previa antes de proceder.

\subsubsection{Gestión de Transacciones}
\textbf{RF-06: Registro de ingresos} - El usuario podrá registrar manualmente sus ingresos indicando el monto, la fecha y una descripción opcional, tanto desde la aplicación móvil como desde la plataforma web.

\textbf{RF-07: Registro de egresos} - El usuario podrá registrar egresos de forma manual o automática, especificando monto, fecha, categoría y descripción.

\textbf{RF-08: Registro de egresos manualmente} - El sistema permitirá al usuario ingresar egresos de manera manual cuando no disponga de un ticket digital o imagen para procesar automáticamente.

\textbf{RF-09: Registro de ticket manualmente} - El usuario podrá registrar un ticket introduciendo manualmente los datos del comprobante, como fecha, monto y establecimiento.

\subsubsection{Gestión de Presupuestos}
\textbf{RF-10: Configuración de presupuesto} - El sistema debe permitir al usuario definir presupuestos por categoría o periodo, estableciendo límites de gasto que serán monitoreados automáticamente.

\subsubsection{Consultas y Visualización}
\textbf{RF-11: Visualización de historial de gastos} - El usuario podrá consultar el historial de gastos registrados en ambos entornos, filtrando por fecha, categoría o tipo de transacción.

\textbf{RF-12: Visualización de gráficas o estadísticas} - El sistema debe generar gráficas y estadísticas interactivas que representen el comportamiento financiero del usuario de manera clara y comprensible.

\subsubsection{Reportes}
\textbf{RF-13: Generación de reportes en PDF} - La plataforma web y la aplicación móvil deberán permitir la generación de reportes financieros en formato PDF, con opción de descarga o visualización en pantalla.

\subsubsection{Metas de Ahorro}
\textbf{RF-14: Configuración de metas de ahorro} - El usuario podrá establecer metas de ahorro personalizadas, definir plazos y montos objetivo, y consultar su progreso mediante indicadores visuales.

\subsection{Requerimientos Exclusivos de la Aplicación Móvil}

\subsubsection{Personalización de Interfaz}
\textbf{RF-15: Cambio de tema de la aplicación} - El sistema debe permitir al usuario cambiar el tema visual de la aplicación (modo claro u oscuro) para mejorar la experiencia de uso.

\subsubsection{Procesamiento OCR}
\textbf{RF-16: Captura de foto del ticket} - La aplicación móvil debe permitir al usuario tomar una fotografía del ticket de compra utilizando la cámara del dispositivo para su posterior procesamiento.

\textbf{RF-17: Captura múltiple de tickets} - El sistema debe permitir tomar varias fotografías del mismo ticket en caso de que la información no sea visible en una sola imagen.

\textbf{RF-18: Confirmación de datos escaneados} - El sistema debe mostrar los datos extraídos por el OCR para que el usuario los revise y confirme antes de su almacenamiento definitivo.

\textbf{RF-19: Etiquetado personalizado de tickets} - El usuario podrá agregar etiquetas personalizadas a los tickets para clasificar los gastos según sus propios criterios o proyectos personales.

\textbf{RF-20: Corrección de datos escaneados} - El sistema debe permitir al usuario editar manualmente los datos que fueron mal reconocidos durante el proceso de OCR.

\subsubsection{Notificaciones y Alertas}
\textbf{RF-21: Visualización de historial de notificaciones} - El usuario podrá consultar un historial de notificaciones dentro de la aplicación móvil, incluyendo alertas de presupuesto, deudas y movimientos recientes.

\subsubsection{Análisis y Estadísticas}
\textbf{RF-22: Visualización de estadísticas por categoría} - El sistema debe permitir al usuario visualizar las estadísticas de gasto organizadas por categoría y periodo, mostrando porcentajes o montos totales.

\subsubsection{Gestión de Categorías}
\textbf{RF-23: Modificación de categorías de gasto} - El usuario podrá crear, editar o eliminar categorías de gasto personalizadas para una organización más detallada de sus finanzas.

\subsubsection{Gestión de Deudas}
\textbf{RF-24: Establecimiento de alertas de deudas} - El sistema debe permitir configurar alertas automáticas para recordar al usuario fechas de vencimiento o pago de sus deudas registradas.

\textbf{RF-25: Configuración de deudas} - El usuario podrá registrar nuevas deudas, especificando monto, acreedor, fecha de vencimiento y frecuencia de pago.

\textbf{RF-26: Registro de deudas} - El sistema debe almacenar la información de las deudas registradas por el usuario y permitir su actualización o eliminación posterior.

\textbf{RF-27: Planificación de pago de deudas} - El sistema debe generar un plan de pago de deudas que muestre fechas, montos parciales y estado de cada obligación financiera.

\subsubsection{Presupuestos Personalizados}
\textbf{RF-28: Planificación de presupuestos personales} - El sistema debe permitir al usuario definir presupuestos personalizados por categoría o tipo de gasto para un mejor control financiero.

\subsubsection{Gestión de Tickets}
\textbf{RF-29: Consulta de tickets} - El usuario podrá consultar los tickets previamente registrados, filtrando por fecha, categoría o monto.

\textbf{RF-30: Visualización de gastos} - El sistema debe mostrar al usuario un resumen de sus gastos organizados por día, semana y mes, junto con las categorías más frecuentes.

\textbf{RF-31: Eliminación de tickets} - El sistema debe permitir al usuario eliminar tickets registrados de forma manual o automática, previa confirmación para evitar pérdidas accidentales de información.

\subsection{Requerimientos Exclusivos de la Plataforma Web}

\subsubsection{Consejos Personalizados}
\textbf{RF-32: Consulta de consejos personalizados} - La plataforma web debe mostrar consejos financieros personalizados basados en los hábitos de consumo del usuario, apoyándose en el análisis de sus gastos registrados.

\subsubsection{Búsqueda Avanzada}
\textbf{RF-33: Búsqueda y filtrado de tickets} - El sistema debe permitir al usuario buscar y filtrar tickets según criterios como fecha, categoría, monto o palabras clave dentro de las descripciones.

\subsubsection{Reportes Especializados}
\textbf{RF-34: Generación de reportes en Excel} - El sistema debe permitir generar y descargar reportes financieros en formato Excel, con hojas separadas por categorías o periodos.

\textbf{RF-35: Compartir reportes por correo electrónico} - El usuario podrá compartir los reportes generados directamente por correo electrónico desde la plataforma web, sin necesidad de descargarlos localmente.

\subsubsection{Análisis Comparativo}
\textbf{RF-36: Comparación de gastos entre meses} - El sistema debe generar comparativas gráficas y tabulares de los gastos registrados en distintos meses, mostrando incrementos o reducciones porcentuales.

\textbf{RF-37: Consulta de balance vs egresos} - El sistema debe mostrar un balance general que compare los ingresos frente a los egresos, calculando el resultado neto del periodo seleccionado.

\subsubsection{Estadísticas Avanzadas}
\textbf{RF-38: Visualización de estadísticas avanzadas} - La plataforma debe ofrecer un panel con estadísticas detalladas del comportamiento financiero del usuario, incluyendo tendencias, promedios y porcentajes de gasto por categoría.

\subsubsection{Carga de Documentos}
\textbf{RF-39: Carga de ticket digital} - El sistema debe permitir la carga de tickets en formato digital (PDF o imagen) desde el explorador web, procesándolos automáticamente con el módulo OCR.

\subsubsection{Gestión Web de Metas}
\textbf{RF-40: Configuración de metas de ahorro desde la web} - El usuario podrá definir y modificar metas de ahorro directamente desde la plataforma web, sincronizándose con los datos de la aplicación móvil.

\section{Requisitos No Funcionales}

Los requisitos no funcionales definen las características y restricciones del sistema que no están directamente relacionadas con las funcionalidades específicas, pero que son esenciales para garantizar su calidad, rendimiento, seguridad y usabilidad.

\subsection{RNF-01: Seguridad}
Las contraseñas y datos personales deben almacenarse cifrados utilizando un algoritmo de hash y no se guardará texto plano en la base de datos MySQL. Para el prototipo se utilizarán datos ficticios durante las demostraciones y el sistema funcionará en un entorno controlado de servidor local.

\subsection{RNF-02: Mantenibilidad}
El código debe estar organizado en módulos separados para facilitar el mantenimiento por parte del equipo de 2 desarrolladores. Se implementará una arquitectura simple que separe la lógica de la aplicación móvil y plataforma web, el backend, y la base de datos.

\subsection{RNF-03: Portabilidad}
La aplicación móvil debe funcionar correctamente en dispositivos Android 10 o superior con al menos 3GB de RAM, adaptándose automáticamente a diferentes tamaños de pantalla entre 5" y 6.7".

\subsection{RNF-04: Compatibilidad}
La plataforma web debe ser compatible con navegadores modernos como Chrome (v120+), Firefox (v115+) y Edge (v120+), funcionando correctamente en resoluciones desde 768px (tablet) hasta 1920px (desktop).

\subsection{RNF-05: Usabilidad}
La interfaz de usuario de la aplicación móvil será desarrollada conforme a las Flutter Usability Guidelines y Material Design, asegurando botones, menús y elementos interactivos claros, consistentes y accesibles. La versión web seguirá las React Accessibility Guidelines, priorizando navegación intuitiva y tiempos de respuesta óptimos. El sistema deberá permitir su uso sin capacitación previa.

\section{Análisis de Riesgos}

El análisis de riesgos permite identificar, evaluar y gestionar los eventos que podrían afectar negativamente el desarrollo y éxito del proyecto. Esta sección presenta una evaluación sistemática de los riesgos del proyecto, incluyendo su probabilidad de ocurrencia, impacto potencial y estrategias de mitigación preventivas.

\subsection{Metodología de Evaluación}

La evaluación de riesgos se realiza mediante una matriz de probabilidad/impacto que clasifica cada riesgo según:

\begin{itemize}
    \item \textbf{Probabilidad:} Muy probable (81-100\%), Probable (61-80\%), Posible (41-60\%), Poco probable (21-40\%), Casi imposible (1-20\%)
    \item \textbf{Impacto:} Severo, Mayor, Significativo, Menor, Insignificante
    \item \textbf{Estrategia:} Evitar (E), Mitigar (M), Monitorear (B), Acción Recomendada (A)
\end{itemize}

\subsection{Matriz de Evaluación de Riesgos}

\begin{table}[H]
\centering
\small
\renewcommand{\arraystretch}{1.2}
\begin{tabular}{|l|c|c|c|c|c|}
\hline
\textbf{PROBABILIDAD} & \multicolumn{5}{c|}{\textbf{IMPACTO}} \\
\hline
& \textbf{Insignificante} & \textbf{Menor} & \textbf{Significativo} & \textbf{Mayor} & \textbf{Severo} \\
\hline
\textbf{Muy probable (81-100\%)} & A & A & M & E & E \\
\hline
\textbf{Probable (61-80\%)} & M & A & M & E & E \\
\hline
\textbf{Posible (41-60\%)} & B & B & M & E & A \\
\hline
\textbf{Poco probable (21-40\%)} & B & B & M & M & A \\
\hline
\textbf{Casi imposible (1-20\%)} & B & B & M & E & A \\
\hline
\end{tabular}
\caption{Matriz de Evaluación Probabilidad/Impacto}
\label{tab:matriz-evaluacion-riesgos}
\end{table}

\textbf{Leyenda:} E=Evitar, M=Mitigar, B=Monitorear, A=Acción Recomendada

\subsection{Identificación y Evaluación de Riesgos}

\begin{table}[H]
\centering
\scriptsize
\renewcommand{\arraystretch}{1.1}
\begin{tabular}{|p{0.6cm}|p{2.8cm}|p{1cm}|p{1cm}|p{1cm}|p{0.6cm}|p{3.5cm}|p{1.8cm}|}
\hline
\textbf{ID} & \textbf{Descripción del Riesgo} & \textbf{Cat.} & \textbf{Prob.} & \textbf{Imp.} & \textbf{Est.} & \textbf{Plan de Mitigación Preventiva} & \textbf{Indicador} \\
\hline

\multicolumn{8}{|c|}{\textbf{RIESGOS TÉCNICOS}} \\
\hline

R01 & Pérdida del código o archivos del proyecto & Técnico & Posible & Mayor & E & Respaldos automáticos en GitHub diarios, branches protegidas, respaldos locales semanales. & >24h sin commit, fallos Git \\
\hline

R02 & Errores críticos en OCR con Tesseract & Técnico & Probable & Significativo & M & Validación confianza >80\%, dataset pruebas, preprocesamiento imágenes mejorado. & Precisión <70\%, quejas usuarios \\
\hline

R03 & Fallas integración Flutter-Python-MySQL & Técnico & Posible & Mayor & M & APIs REST documentadas, pruebas integración automatizadas, mocks desarrollo. & Errores conexión >10\%, timeouts \\
\hline

R04 & Corrupción/pérdida base datos & Técnico & Poco probable & Severo & E & Respaldos BD diarios, scripts recuperación, transacciones críticas. & Errores integridad, consultas falla >5\% \\
\hline

\multicolumn{8}{|c|}{\textbf{RIESGOS ORGANIZACIONALES}} \\
\hline

R05 & Cambios requerimientos en desarrollo & Organ. & Probable & Significativo & M & Documentar requerimientos, control cambios, priorizar funcionalidades core. & Cambios >2 por semana \\
\hline

R06 & Fallas comunicación equipo & Organ. & Posible & Mayor & M & Reuniones semanales, herramientas seguimiento, canales comunicación claros. & Reuniones perdidas, tareas >3d sin actualizar \\
\hline

\multicolumn{8}{|c|}{\textbf{RIESGOS PERSONALES}} \\
\hline

R07 & Abandono integrante equipo & Personal & Poco probable & Severo & E & Documentar código, compartir conocimiento, respaldos actualizados, roles intercambiables. & Ausencias frecuentes, baja participación \\
\hline

R08 & Retrasos carga académica excesiva & Personal & Probable & Significativo & M & Cronograma con períodos exámenes, priorizar tareas críticas, buffer time. & Retraso milestones >1 semana \\
\hline

\multicolumn{8}{|c|}{\textbf{RIESGOS DE SEGURIDAD}} \\
\hline

R09 & Exposición datos sensibles usuarios & Seguridad & Posible & Severo & E & Cifrado SHA-1 contraseñas, validar inputs SQL, HTTPS, datos ficticios demos. & Vulnerabilidades código, datos texto plano \\
\hline

\multicolumn{8}{|c|}{\textbf{RIESGOS EXTERNOS}} \\
\hline

R10 & Fallas hardware/energía eléctrica & Externo & Posible & Menor & B & Laptops con batería, autoguardado frecuente, trabajo offline. & Interrupciones energía >2/semana \\
\hline

R11 & Problemas conectividad demostraciones & Externo & Probable & Significativo & M & Demos offline, datos locales prueba, videos respaldo, hotspot móvil. & Problemas red ubicación demo \\
\hline

\end{tabular}
\caption{Matriz Completa de Análisis de Riesgos}
\label{tab:analisis-riesgos-completo}
\end{table}

\subsection{Resumen de Estrategias de Mitigación}

\subsubsection{Riesgos Críticos (Evitar)}
\begin{itemize}
    \item \textbf{R01, R04, R07, R09:} Requieren implementación inmediata de controles preventivos
    \item \textbf{Monitoreo:} Revisión diaria de indicadores de alerta
    \item \textbf{Responsable:} Ambos integrantes del equipo
\end{itemize}

\subsubsection{Riesgos Altos (Mitigar)}
\begin{itemize}
    \item \textbf{R02, R03, R05, R06, R08, R11:} Implementar controles preventivos y planes de contingencia
    \item \textbf{Monitoreo:} Revisión semanal en reuniones de equipo
    \item \textbf{Responsable:} Distribuir según especialización técnica
\end{itemize}

\subsubsection{Riesgos Moderados (Acción Recomendada/Monitorear)}
\begin{itemize}
    \item \textbf{R10:} Seguimiento mensual, acción si se materializa
    \item \textbf{Monitoreo:} Revisión durante milestones del proyecto
\end{itemize}

\section{Matriz de Cumplimiento de Reglas de Negocio}

Las reglas de negocio definen las políticas y restricciones operacionales que el sistema debe implementar. La siguiente matriz presenta las reglas críticas del sistema junto con los escenarios de cumplimiento e incumplimiento para garantizar el correcto funcionamiento del sistema.

\begin{table}[H]
\centering
\tiny
\renewcommand{\arraystretch}{1.1}
\begin{tabular}{|p{1.2cm}|p{3.5cm}|p{4cm}|p{4cm}|}
\hline
\textbf{Regla} & \textbf{Descripción} & \textbf{Caso de Cumplimiento} & \textbf{Caso de Incumplimiento} \\
\hline
\hline

\multicolumn{4}{|c|}{\textbf{GESTIÓN DE USUARIOS Y SEGURIDAD}} \\
\hline

RN01 & Registro único de usuario: Cada usuario debe tener una cuenta única asociada a un correo electrónico válido & 
\textbf{Escenario:} Usuario registra con email ``usuario@ejemplo.com''. \newline
\textbf{Resultado:} Cuenta creada exitosamente, email almacenado como identificador único. &
\textbf{Escenario:} Segundo usuario intenta registrarse con ``usuario@ejemplo.com''. \newline
\textbf{Resultado:} Sistema rechaza registro mostrando ``El correo ya está registrado''. \\
\hline

RN03 & Tiempo de sesión: Las sesiones expiran automáticamente después de 5 minutos de inactividad & 
\textbf{Escenario:} Usuario permanece inactivo 5 minutos. \newline
\textbf{Resultado:} Sistema cierra sesión automáticamente y redirige al login. &
\textbf{Escenario:} Usuario inactivo 6 minutos intenta realizar acción. \newline
\textbf{Resultado:} Sistema muestra ``Sesión expirada'' y solicita autenticación. \\
\hline

\hline
\multicolumn{4}{|c|}{\textbf{PROCESAMIENTO OCR Y TICKETS}} \\
\hline

RN05 & Límite de tamaño de imagen: Las imágenes no deben superar 10 MB & 
\textbf{Escenario:} Usuario carga imagen de 8 MB. \newline
\textbf{Resultado:} Imagen procesada correctamente por OCR. &
\textbf{Escenario:} Usuario intenta cargar imagen de 12 MB. \newline
\textbf{Resultado:} Sistema rechaza con mensaje ``Imagen excede 10 MB''. \\
\hline

RN07 & Idioma OCR: Solo procesa tickets en español & 
\textbf{Escenario:} Ticket contiene texto ``Total: \$150.00 Fecha: 15/11/2025''. \newline
\textbf{Resultado:} OCR extrae monto y fecha correctamente. &
\textbf{Escenario:} Ticket en inglés ``Total: \$150.00 Date: 11/15/2025''. \newline
\textbf{Resultado:} OCR falla, solicita registro manual. \\
\hline

RN10 & Validación manual obligatoria: Cuando OCR presenta baja confianza, requiere validación & 
\textbf{Escenario:} OCR extrae datos con 95\% confianza. \newline
\textbf{Resultado:} Datos guardados automáticamente sin intervención. &
\textbf{Escenario:} OCR extrae con 60\% confianza. \newline
\textbf{Resultado:} Sistema solicita validación manual antes de guardar. \\
\hline

\hline
\multicolumn{4}{|c|}{\textbf{CATEGORIZACIÓN Y PRESUPUESTOS}} \\
\hline

RN11 & Límite categorías: Máximo 10 categorías personalizadas por usuario & 
\textbf{Escenario:} Usuario crea 8va categoría personalizada ``Entretenimiento''. \newline
\textbf{Resultado:} Categoría creada y disponible para asignación. &
\textbf{Escenario:} Usuario intenta crear 11va categoría personalizada. \newline
\textbf{Resultado:} Sistema bloquea con ``Límite de 10 categorías alcanzado''. \\
\hline

RN13 & Asignación única: Cada gasto se asigna a una sola categoría & 
\textbf{Escenario:} Gasto de \$500 asignado a categoría ``Alimentos''. \newline
\textbf{Resultado:} Gasto registrado completamente en esa categoría. &
\textbf{Escenario:} Usuario intenta dividir \$500 entre ``Alimentos'' y ``Entretenimiento''. \newline
\textbf{Resultado:} Sistema requiere selección de una sola categoría. \\
\hline

RN23 & Alertas presupuesto: Alerta al superar 80\% del presupuesto & 
\textbf{Escenario:} Presupuesto \$1000, gasto acumulado \$850. \newline
\textbf{Resultado:} Sistema envía alerta ``85\% del presupuesto utilizado''. &
\textbf{Escenario:} Presupuesto \$1000, gasto \$750. \newline
\textbf{Resultado:} No se genera alerta (bajo el umbral del 80\%). \\
\hline

\hline
\multicolumn{4}{|c|}{\textbf{GESTIÓN DE DEUDAS Y ALERTAS}} \\
\hline

RN19 & Límite deudas: Máximo 20 deudas activas simultáneamente & 
\textbf{Escenario:} Usuario registra su 15va deuda activa. \newline
\textbf{Resultado:} Deuda almacenada y visible en panel de deudas. &
\textbf{Escenario:} Usuario intenta registrar 21va deuda activa. \newline
\textbf{Resultado:} Sistema bloquea con ``Límite de 20 deudas alcanzado''. \\
\hline

RN21 & Alertas vencimiento: Recordatorios automáticos de pagos & 
\textbf{Escenario:} Deuda con vencimiento 20/11/2025, hoy 18/11/2025. \newline
\textbf{Resultado:} Sistema envía notificación ``Pago vence en 2 días''. &
\textbf{Escenario:} Configuración de alertas deshabilitada por usuario. \newline
\textbf{Resultado:} No se envían recordatorios aunque la deuda esté próxima a vencer. \\
\hline

\hline
\multicolumn{4}{|c|}{\textbf{INTEGRIDAD Y SEGURIDAD DE DATOS}} \\
\hline

RN27 & Eliminación irreversible: Confirmación explícita requerida & 
\textbf{Escenario:} Usuario confirma eliminación de transacción en diálogo. \newline
\textbf{Resultado:} Transacción eliminada permanentemente del sistema. &
\textbf{Escenario:} Usuario cancela diálogo de confirmación. \newline
\textbf{Resultado:} Transacción permanece intacta en el sistema. \\
\hline

RN29 & Validación montos: Valores numéricos positivos con máximo 2 decimales & 
\textbf{Escenario:} Usuario ingresa monto ``\$125.50''. \newline
\textbf{Resultado:} Monto validado y almacenado correctamente. &
\textbf{Escenario:} Usuario ingresa ``\$-50.00'' o ``\$100.255''. \newline
\textbf{Resultado:} Sistema rechaza con ``Monto debe ser positivo con máximo 2 decimales''. \\
\hline

\hline
\multicolumn{4}{|c|}{\textbf{REPORTES Y ANÁLISIS}} \\
\hline

RN25 & Rango fechas reportes: Solo fechas con transacciones registradas & 
\textbf{Escenario:} Usuario solicita reporte de noviembre 2025 con 15 transacciones. \newline
\textbf{Resultado:} Reporte generado con todos los datos del mes. &
\textbf{Escenario:} Usuario solicita reporte de enero 2025 sin transacciones. \newline
\textbf{Resultado:} Sistema muestra ``No hay datos para el período seleccionado''. \\
\hline

RN26 & Formatos exportación: PDF para visualización, Excel para análisis & 
\textbf{Escenario:} Usuario selecciona ``Exportar a PDF''. \newline
\textbf{Resultado:} Reporte generado en formato PDF con gráficos y tablas. &
\textbf{Escenario:} Sistema no puede generar formato por error técnico. \newline
\textbf{Resultado:} Mensaje ``Error al generar reporte, intente nuevamente''. \\
\hline

\end{tabular}
\caption{Matriz de Cumplimiento de Reglas de Negocio}
\label{tab:matriz-cumplimiento-rn}
\end{table}