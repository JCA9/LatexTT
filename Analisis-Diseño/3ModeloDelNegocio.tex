% !TeX root = analisis-diseno.tex

%=========================================================
\chapter{Modelo del negocio}
\label{cap:negocio}

Este capítulo presenta el modelo del negocio de FinanzApp, definiendo las reglas de negocio que rigen el funcionamiento del sistema, los procesos principales de la gestión financiera personal, y las entidades de negocio involucradas. Se establecen las bases conceptuales que guiarán el diseño e implementación del sistema.

%---------------------------------------------------------
\section{Reglas de negocio}

Las reglas de negocio definen las políticas, restricciones y lógica que debe cumplir FinanzApp para garantizar su correcto funcionamiento y alineación con los objetivos del proyecto.

% !TeX root = analisis-diseno.tex

%=========================================================
% Reglas de negocio para FinanzApp

\begin{description}
	\item[\hypertarget{RN-001}{RN-001}] \textbf{Registro único de usuario:} Cada usuario debe tener un email único en el sistema. No se permiten cuentas duplicadas con el mismo email.
	
	\item[\hypertarget{RN-002}{RN-002}] \textbf{Edad mínima:} Los usuarios deben tener al menos 18 años para registrarse en el sistema, validado mediante fecha de nacimiento.
	
	\item[\hypertarget{RN-003}{RN-003}] \textbf{Límite de procesamiento OCR:} Cada usuario puede procesar máximo 50 tickets por día mediante OCR para optimizar recursos del sistema.
	
	\item[\hypertarget{RN-004}{RN-004}] \textbf{Validación de montos:} Los montos de transacciones deben ser positivos y no pueden exceder \$100,000.00 MXN por transacción individual.
	
	\item[\hypertarget{RN-005}{RN-005}] \textbf{Retención de imágenes:} Las imágenes de tickets se conservan por 90 días después del procesamiento exitoso, luego se eliminan automáticamente por privacidad.
	
	\item[\hypertarget{RN-006}{RN-006}] \textbf{Categorización automática:} El sistema debe categorizar automáticamente las transacciones con un nivel de confianza mínimo del 60\%. Transacciones con menor confianza requieren validación manual.
	
	\item[\hypertarget{RN-007}{RN-007}] \textbf{Presupuestos activos:} Un usuario puede tener máximo 10 presupuestos activos simultáneamente para evitar complejidad excesiva.
	
	\item[\hypertarget{RN-008}{RN-008}] \textbf{Período de presupuesto:} Los presupuestos pueden configurarse para períodos de 1 semana, 1 mes o 3 meses únicamente.
	
	\item[\hypertarget{RN-009}{RN-009}] \textbf{Alertas de presupuesto:} Se envían alertas automáticas al alcanzar 70\%, 90\% y 100\% del presupuesto asignado a cada categoría.
	
	\item[\hypertarget{RN-010}{RN-010}] \textbf{Metas financieras:} Las metas de ahorro deben tener una fecha límite entre 1 mes y 5 años desde su creación.
	
	\item[\hypertarget{RN-011}{RN-011}] \textbf{Sincronización de datos:} Los datos deben sincronizarse entre dispositivos en un máximo de 5 segundos después de una transacción.
	
	\item[\hypertarget{RN-012}{RN-012}] \textbf{Backup automático:} El sistema debe crear respaldos automáticos de los datos del usuario cada 24 horas.
	
	\item[\hypertarget{RN-013}{RN-013}] \textbf{Inactividad de cuenta:} Las cuentas inactivas por más de 365 días se marcan para archivado, con notificación previa de 30 días al usuario.
	
	\item[\hypertarget{RN-014}{RN-014}] \textbf{Exportación de datos:} Los usuarios pueden exportar todos sus datos personales en formato JSON o CSV en cualquier momento.
	
	\item[\hypertarget{RN-015}{RN-015}] \textbf{Eliminación de cuenta:} Al eliminar una cuenta, todos los datos se eliminan permanentemente después de 30 días de período de gracia.
	
	\item[\hypertarget{RN-016}{RN-016}] \textbf{Validación de tickets:} Solo se aceptan imágenes en formato JPEG, PNG o PDF, con tamaño máximo de 10 MB.
	
	\item[\hypertarget{RN-017}{RN-017}] \textbf{Moneda única:} El sistema opera únicamente con pesos mexicanos (MXN). No se soportan múltiples monedas.
	
	\item[\hypertarget{RN-018}{RN-018}] \textbf{Categorías personalizadas:} Los usuarios pueden crear máximo 20 categorías personalizadas además de las categorías predefinidas del sistema.
	
	\item[\hypertarget{RN-019}{RN-019}] \textbf{Histórico mínimo:} Se requieren mínimo 30 días de datos históricos para generar análisis predictivos y recomendaciones personalizadas.
	
	\item[\hypertarget{RN-020}{RN-020}] \textbf{Notificaciones:} Los usuarios pueden recibir máximo 5 notificaciones push por día para evitar fatiga de notificaciones.
	
	\item[\hypertarget{RN-021}{RN-021}] \textbf{Sesión de usuario:} Las sesiones expiran automáticamente después de 30 días de inactividad por seguridad.
	
	\item[\hypertarget{RN-022}{RN-022}] \textbf{Corrección de OCR:} Los usuarios pueden corregir manualmente los datos extraídos por OCR dentro de las primeras 48 horas del procesamiento.
	
	\item[\hypertarget{RN-023}{RN-023}] \textbf{Análisis de tendencias:} El sistema requiere mínimo 90 días de datos para mostrar análisis de tendencias confiables.
	
	\item[\hypertarget{RN-024}{RN-024}] \textbf{Compartir datos:} Los datos financieros personales no pueden compartirse con terceros sin consentimiento explícito del usuario.
	
	\item[\hypertarget{RN-025}{RN-025}] \textbf{Recuperación de contraseña:} Los tokens de recuperación de contraseña expiran en 1 hora y solo pueden usarse una vez.
\end{description}


%---------------------------------------------------------
\section{Procesos de negocio}

Los procesos de negocio describen las actividades principales que los usuarios realizan dentro del contexto de gestión financiera personal, y cómo FinanzApp los soporta y automatiza.

% - - - - - - - - - - - - - - - - - - - - - - - - - - - - 
\subsection{Proceso de registro y digitalización de gastos}

\begin{figure}[htpb!]
	\begin{center}
		% TODO: Agregar imagen correspondiente
		\caption{Proceso de registro y digitalización de gastos}
		\label{fig:procesoRegistro}
	\end{center}
\end{figure}

Este proceso representa la actividad principal de FinanzApp: la digitalización automática de gastos mediante OCR. El usuario fotografía un ticket o recibo, el sistema procesa la imagen, extrae la información relevante, la categoriza automáticamente y la registra en el perfil financiero del usuario.

\textbf{Actores involucrados:} \textcolor{userColor}{\textbf{Usuario Joven}}, \textcolor{userColor}{\textbf{Servicio OCR}}

\textbf{Precondiciones:} Usuario autenticado, cámara del dispositivo funcional, conexión a internet activa.

\textbf{Postcondiciones:} Gasto registrado y categorizado en la base de datos del usuario.

% - - - - - - - - - - - - - - - - - - - - - - - - - - - - 
\subsection{Proceso de análisis y generación de insights financieros}

\begin{figure}[htpb!]
	\begin{center}
		% TODO: Agregar imagen correspondiente
		\caption{Proceso de análisis y generación de insights financieros}
		\label{fig:procesoAnalisis}
	\end{center}
\end{figure}

Este proceso describe cómo el sistema analiza los patrones de gasto del usuario para generar insights, predicciones y recomendaciones personalizadas. Incluye la identificación de tendencias, detección de gastos atípicos y generación de alertas proactivas.

\textbf{Actores involucrados:} Sistema de análisis predictivo (interno)

\textbf{Precondiciones:} Datos históricos suficientes (mínimo 30 días), algoritmos de ML entrenados.

\textbf{Postcondiciones:} Insights generados y disponibles en el dashboard del usuario.

% - - - - - - - - - - - - - - - - - - - - - - - - - - - - 
\subsection{Proceso de educación financiera contextual}

\begin{figure}[htpb!]
	\begin{center}
		% TODO: Agregar imagen correspondiente
		\caption{Proceso de educación financiera contextual}
		\label{fig:procesoEducacion}
	\end{center}
\end{figure}

Este proceso detalla cómo FinanzApp proporciona educación financiera just-in-time, integrando consejos y recomendaciones en el contexto específico de las acciones del usuario.

\textbf{Actores involucrados:} \textcolor{userColor}{\textbf{Usuario Joven}}, Motor de recomendaciones (interno)

\textbf{Precondiciones:} Perfil del usuario establecido, contenido educativo disponible.

\textbf{Postcondiciones:} Conocimiento financiero del usuario incrementado, mejores decisiones financieras.

%---------------------------------------------------------
\section{Entidades de negocio}

Las entidades de negocio representan los conceptos principales del dominio financiero personal que maneja FinanzApp.

% - - - - - - - - - - - - - - - - - - - - - - - - - - - - 
\subsection{Usuario}

Representa a la persona que utiliza FinanzApp para gestionar sus finanzas personales.

\textbf{Atributos principales:}
\begin{itemize}
	\item Información personal: nombre, edad, email, teléfono
	\item Preferencias: categorías personalizadas, metas financieras, frecuencia de notificaciones
	\item Perfil financiero: ingresos promedio, patrón de gastos, nivel de educación financiera
	\item Configuración: idioma, moneda, zona horaria
\end{itemize}

% - - - - - - - - - - - - - - - - - - - - - - - - - - - - 
\subsection{Transacción}

Representa cualquier movimiento financiero (gasto o ingreso) registrado en el sistema.

\textbf{Atributos principales:}
\begin{itemize}
	\item Información básica: monto, fecha, descripción, tipo (gasto/ingreso)
	\item Categorización: categoría principal, subcategoría, tags personalizados
	\item Origen: método de registro (OCR, manual, automático), establecimiento
	\item Metadatos: ubicación, imagen del ticket, nivel de confianza OCR
\end{itemize}

% - - - - - - - - - - - - - - - - - - - - - - - - - - - - 
\subsection{Categoría}

Representa las clasificaciones utilizadas para organizar las transacciones financieras.

\textbf{Atributos principales:}
\begin{itemize}
	\item Identificación: nombre, descripción, icono, color
	\item Jerarquía: categoría padre, subcategorías
	\item Configuración: presupuesto asignado, alertas, metas específicas
	\item Estadísticas: promedio mensual, tendencia, comparación con usuarios similares
\end{itemize}

% - - - - - - - - - - - - - - - - - - - - - - - - - - - - 
\subsection{Presupuesto}

Representa los límites de gasto establecidos por el usuario para diferentes categorías o períodos.

\textbf{Atributos principales:}
\begin{itemize}
	\item Definición: monto límite, período (semanal/mensual), categorías incluidas
	\item Estado: progreso actual, porcentaje utilizado, días restantes
	\item Alertas: umbrales de notificación, tipos de alerta configurados
	\item Histórico: cumplimiento en períodos anteriores, tendencias
\end{itemize}

% - - - - - - - - - - - - - - - - - - - - - - - - - - - - 
\subsection{Meta Financiera}

Representa objetivos de ahorro o reducción de gastos establecidos por el usuario.

\textbf{Atributos principales:}
\begin{itemize}
	\item Objetivo: descripción, monto target, fecha límite
	\item Progreso: monto actual, porcentaje completado, tiempo restante
	\item Estrategia: plan de ahorro sugerido, acciones recomendadas
	\item Motivación: recordatorios, recompensas, hitos intermedios
\end{itemize}

% - - - - - - - - - - - - - - - - - - - - - - - - - - - - 
\subsection{Ticket/Recibo}

Representa la digitalización de un comprobante de compra físico.

\textbf{Atributos principales:}
\begin{itemize}
	\item Imagen: archivo original, imagen procesada, metadatos de captura
	\item Extracción OCR: texto extraído, nivel de confianza, campos identificados
	\item Validación: estado de verificación, correcciones manuales
	\item Procesamiento: timestamp, tiempo de procesamiento, versión del algoritmo OCR
\end{itemize}

%---------------------------------------------------------
\section{Estados de las entidades}

% !TeX root = analisis-diseno.tex

%=========================================================
% Estados de las entidades principales

% - - - - - - - - - - - - - - - - - - - - - - - - - - - - 
\subsection{Estados del Usuario}

\begin{figure}[htpb!]
	\begin{center}
		% TODO: Agregar imagen correspondiente
		\caption{Diagrama de estados del Usuario}
		\label{fig:estadosUsuario}
	\end{center}
\end{figure}

\begin{description}
	\item[\textbf{Registro pendiente}] Usuario ha iniciado el proceso de registro pero no ha confirmado su email.
	\item[\textbf{Activo}] Usuario con cuenta confirmada y acceso completo al sistema.
	\item[\textbf{Onboarding}] Usuario completando la configuración inicial de perfil y preferencias.
	\item[\textbf{Suspendido}] Cuenta temporalmente deshabilitada por violación de términos de uso.
	\item[\textbf{Inactivo}] Usuario que no ha utilizado la aplicación por más de 90 días.
	\item[\textbf{Eliminado}] Cuenta marcada para eliminación, en período de gracia de 30 días.
\end{description}

% - - - - - - - - - - - - - - - - - - - - - - - - - - - - 
\subsection{Estados de la Transacción}

\begin{figure}[htpb!]
	\begin{center}
		% TODO: Agregar imagen correspondiente
		\caption{Diagrama de estados de la Transacción}
		\label{fig:estadosTransaccion}
	\end{center}
\end{figure}

\begin{description}
	\item[\textbf{Pendiente}] Transacción creada pero no procesada completamente.
	\item[\textbf{Procesando OCR}] Imagen de ticket siendo procesada por el servicio OCR.
	\item[\textbf{OCR Completado}] Datos extraídos exitosamente del ticket.
	\item[\textbf{Categorizando}] Sistema de ML determinando la categoría apropiada.
	\item[\textbf{Validación requerida}] Categorización con baja confianza, requiere revisión manual.
	\item[\textbf{Confirmada}] Transacción validada y lista para análisis.
	\item[\textbf{Modificada}] Usuario ha editado información de la transacción.
	\item[\textbf{Eliminada}] Transacción marcada como eliminada (soft delete).
\end{description}

% - - - - - - - - - - - - - - - - - - - - - - - - - - - - 
\subsection{Estados del Ticket/Recibo}

\begin{figure}[htpb!]
	\begin{center}
		% TODO: Agregar imagen correspondiente
		\caption{Diagrama de estados del Ticket}
		\label{fig:estadosTicket}
	\end{center}
\end{figure}

\begin{description}
	\item[\textbf{Subido}] Imagen del ticket cargada en el sistema.
	\item[\textbf{En cola}] Ticket esperando procesamiento OCR.
	\item[\textbf{Procesando}] OCR extrayendo información del ticket.
	\item[\textbf{Procesado}] Información extraída exitosamente.
	\item[\textbf{Error OCR}] Fallo en el procesamiento, requiere intervención.
	\item[\textbf{Validado}] Información confirmada por el usuario.
	\item[\textbf{Archivado}] Ticket procesado y almacenado para referencia histórica.
	\item[\textbf{Eliminado}] Imagen eliminada después del período de retención.
\end{description}

% - - - - - - - - - - - - - - - - - - - - - - - - - - - - 
\subsection{Estados del Presupuesto}

\begin{figure}[htpb!]
	\begin{center}
		% TODO: Agregar imagen correspondiente
		\caption{Diagrama de estados del Presupuesto}
		\label{fig:estadosPresupuesto}
	\end{center}
\end{figure}

\begin{description}
	\item[\textbf{Borrador}] Presupuesto creado pero no activado.
	\item[\textbf{Activo}] Presupuesto en funcionamiento, monitoreando gastos.
	\item[\textbf{Advertencia}] 70\% del presupuesto utilizado, alertas enviadas.
	\item[\textbf{Crítico}] 90\% del presupuesto utilizado, alertas urgentes.
	\item[\textbf{Excedido}] Presupuesto superado, notificaciones de exceso.
	\item[\textbf{Completado}] Período del presupuesto finalizado exitosamente.
	\item[\textbf{Cancelado}] Presupuesto desactivado antes de su vencimiento.
	\item[\textbf{Pausado}] Presupuesto temporalmente deshabilitado.
\end{description}

% - - - - - - - - - - - - - - - - - - - - - - - - - - - - 
\subsection{Estados de la Meta Financiera}

\begin{figure}[htpb!]
	\begin{center}
		% TODO: Agregar imagen correspondiente
		\caption{Diagrama de estados de la Meta Financiera}
		\label{fig:estadosMeta}
	\end{center}
\end{figure}

\begin{description}
	\item[\textbf{Planificada}] Meta creada con objetivo y fecha límite definidos.
	\item[\textbf{En progreso}] Meta activa con progreso siendo monitoreado.
	\item[\textbf{En riesgo}] Progreso insuficiente para alcanzar la meta en tiempo.
	\item[\textbf{Pausada}] Meta temporalmente suspendida por decisión del usuario.
	\item[\textbf{Alcanzada}] Objetivo de la meta cumplido exitosamente.
	\item[\textbf{Vencida}] Fecha límite superada sin alcanzar el objetivo.
	\item[\textbf{Cancelada}] Meta abandonada por decisión del usuario.
	\item[\textbf{Extendida}] Fecha límite modificada para continuar el progreso.
\end{description}


%---------------------------------------------------------
\section{Modelo conceptual}

\begin{figure}[htpb!]
	\begin{center}
		% TODO: Agregar imagen correspondiente
		\caption{Modelo conceptual de FinanzApp}
		\label{fig:modeloConceptual}
	\end{center}
\end{figure}

El modelo conceptual muestra las relaciones entre las entidades principales del sistema. Un Usuario puede tener múltiples Transacciones, las cuales se organizan en Categorías. Los Presupuestos se asocian a categorías específicas, mientras que las Metas Financieras son objetivos generales del usuario. Los Tickets representan la fuente de información para las transacciones registradas mediante OCR.

%---------------------------------------------------------
\section{Políticas de negocio}

% - - - - - - - - - - - - - - - - - - - - - - - - - - - - 
\subsection{Políticas de privacidad y seguridad}

\begin{itemize}
	\item Los datos financieros se cifran con AES-256 tanto en tránsito como en reposo
	\item Las imágenes de tickets se eliminan automáticamente después de 90 días del procesamiento exitoso
	\item Los datos analíticos se anonimizan antes de utilizarse para mejoras del sistema
	\item Los usuarios pueden exportar todos sus datos en cualquier momento
	\item La información personal no se comparte con terceros sin consentimiento explícito
\end{itemize}

% - - - - - - - - - - - - - - - - - - - - - - - - - - - - 
\subsection{Políticas de uso del servicio OCR}

\begin{itemize}
	\item Máximo 50 procesamiento OCR por usuario por día
	\item Imágenes deben ser menores a 10 MB
	\item Formatos soportados: JPEG, PNG, PDF
	\item Tiempo máximo de procesamiento: 30 segundos
	\item Reintento automático hasta 3 veces en caso de falla
\end{itemize}

% - - - - - - - - - - - - - - - - - - - - - - - - - - - - 
\subsection{Políticas de categorización automática}

\begin{itemize}
	\item Precisión mínima requerida del 80\% para categorización automática
	\item Aprendizaje continuo basado en correcciones del usuario
	\item Revisión manual requerida para transacciones mayores a \$5,000 MXN
	\item Categoría ``Sin clasificar'' para casos de baja confianza (<60\%)
	\item Actualización de algoritmos cada 30 días con nuevos datos de entrenamiento
\end{itemize}
