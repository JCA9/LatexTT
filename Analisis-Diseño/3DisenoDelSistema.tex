% !TeX root = analisis-diseno.tex

%=========================================================
\chapter{Diseño del Sistema}
\label{cap:diseno}

Este capítulo presenta el diseño técnico completo del sistema FinanzApp, incluyendo la arquitectura de software, diseños estático y dinámico que guiarán la implementación del sistema.

%---------------------------------------------------------
\section{Arquitectura}

FinanzApp implementa una arquitectura de tres capas con patrón cliente-servidor, separando claramente las responsabilidades de presentación, lógica de negocio y acceso a datos.

\subsection{Arquitectura de tres capas}

\begin{figure}[htpb!]
	\begin{center}
		% TODO: Agregar diagrama de arquitectura de 3 capas
		\caption{Arquitectura de tres capas de FinanzApp}
		\label{fig:arquitecturaTresCapas}
	\end{center}
\end{figure}

\begin{description}
	\item[\textbf{Capa de Presentación}] 
	\begin{itemize}
		\item Aplicación móvil nativa (React Native)
		\item Interfaces de usuario responsivas
		\item Gestión de eventos y validaciones del lado cliente
		\item Componentes reutilizables de UI
	\end{itemize}
	
	\item[\textbf{Capa de Lógica de Negocio}] 
	\begin{itemize}
		\item API REST con Node.js y Express
		\item Servicios de procesamiento OCR
		\item Motor de categorización inteligente
		\item Lógica de validación y reglas de negocio
		\item Servicios de analytics y reportes
	\end{itemize}
	
	\item[\textbf{Capa de Datos}] 
	\begin{itemize}
		\item Base de datos PostgreSQL
		\item Cache distribuido con Redis
		\item Almacenamiento de archivos (AWS S3)
		\item Servicios de respaldo y sincronización
	\end{itemize}
\end{description}

\subsection{Patrón Cliente-Servidor}

La comunicación entre las capas sigue el patrón cliente-servidor:

\begin{itemize}
	\item \textbf{Cliente:} Aplicación móvil que presenta la interfaz y gestiona la interacción del usuario
	\item \textbf{Servidor:} Backend que procesa las peticiones, ejecuta la lógica de negocio y gestiona los datos
	\item \textbf{Protocolo:} HTTPS con API REST para comunicación segura
	\item \textbf{Formato:} JSON para intercambio de datos estructurados
\end{itemize}

%---------------------------------------------------------
\section{Diseño estático}

El diseño estático define la estructura de clases, entidades y relaciones de datos del sistema.

\subsection{Diagrama de clases}

\begin{figure}[htpb!]
	\begin{center}
		% TODO: Agregar diagrama de clases
		\caption{Diagrama de clases principal de FinanzApp}
		\label{fig:diagramaClases}
	\end{center}
\end{figure}

Las clases principales del sistema incluyen:

\begin{description}
	\item[\textbf{Usuario}] Gestiona la información y autenticación de usuarios
	\item[\textbf{Transaccion}] Representa gastos e ingresos registrados
	\item[\textbf{Categoria}] Define las categorías de clasificación de transacciones
	\item[\textbf{Presupuesto}] Gestiona límites de gasto por categoría
	\item[\textbf{TicketOCR}] Procesa imágenes de tickets mediante OCR
	\item[\textbf{Analytics}] Genera reportes y análisis de patrones
	\item[\textbf{Notificacion}] Gestiona alertas y mensajes al usuario
\end{description}

\subsection{Modelo entidad-relación}

\begin{figure}[htpb!]
	\begin{center}
		% TODO: Agregar diagrama entidad-relación
		\caption{Modelo entidad-relación de FinanzApp}
		\label{fig:modeloER}
	\end{center}
\end{figure}

Las entidades principales y sus relaciones:

\begin{itemize}
	\item Un \textbf{Usuario} puede tener múltiples \textbf{Transacciones}
	\item Una \textbf{Transacción} pertenece a una \textbf{Categoría}
	\item Un \textbf{Usuario} puede definir múltiples \textbf{Presupuestos}
	\item Un \textbf{Presupuesto} se asocia a una \textbf{Categoría}
	\item Una \textbf{Transacción} puede originarse de un \textbf{TicketOCR}
	\item Un \textbf{Usuario} puede recibir múltiples \textbf{Notificaciones}
\end{itemize}

\subsection{Modelo relacional}

\begin{figure}[htpb!]
	\begin{center}
		% TODO: Agregar esquema de base de datos
		\caption{Esquema relacional de la base de datos}
		\label{fig:esquemaRelacional}
	\end{center}
\end{figure}

Tablas principales del esquema relacional:

\begin{description}
	\item[\textbf{usuarios}] (id, email, password\_hash, nombre, fecha\_registro, activo)
	\item[\textbf{categorias}] (id, nombre, descripcion, color, icono, usuario\_id)
	\item[\textbf{transacciones}] (id, usuario\_id, categoria\_id, monto, descripcion, fecha, tipo, ticket\_id)
	\item[\textbf{presupuestos}] (id, usuario\_id, categoria\_id, limite, periodo, fecha\_inicio, activo)
	\item[\textbf{tickets\_ocr}] (id, usuario\_id, imagen\_url, texto\_extraido, precision, fecha\_procesamiento)
	\item[\textbf{notificaciones}] (id, usuario\_id, tipo, titulo, mensaje, leida, fecha\_envio)
\end{description}

%---------------------------------------------------------
\section{Diseño dinámico}

El diseño dinámico describe el comportamiento del sistema a través de diagramas de secuencia y actividad.

\subsection{Diagrama de secuencia 1: Registro}

\begin{figure}[htpb!]
	\begin{center}
		% TODO: Agregar diagrama de secuencia de registro
		\caption{Diagrama de secuencia - Registro de usuario}
		\label{fig:secuenciaRegistro}
	\end{center}
\end{figure}

El proceso de registro involucra:
\begin{enumerate}
	\item Usuario ingresa datos de registro en la aplicación
	\item Aplicación valida formato de email y fortaleza de contraseña
	\item Aplicación envía petición de registro al servidor
	\item Servidor valida que email no exista en base de datos
	\item Servidor encripta contraseña y crea registro de usuario
	\item Servidor retorna token de autenticación
	\item Aplicación almacena token y redirige a pantalla principal
\end{enumerate}

\subsection{Diagrama de secuencia 2: Procesamiento OCR}

\begin{figure}[htpb!]
	\begin{center}
		% TODO: Agregar diagrama de secuencia OCR
		\caption{Diagrama de secuencia - Procesamiento OCR}
		\label{fig:secuenciaOCR}
	\end{center}
\end{figure}

El flujo de procesamiento OCR incluye:
\begin{enumerate}
	\item Usuario captura imagen de ticket con la cámara
	\item Aplicación comprime y optimiza imagen
	\item Aplicación envía imagen al servidor
	\item Servidor invoca servicio OCR externo
	\item Servicio OCR procesa imagen y extrae texto
	\item Servidor parsea texto para identificar monto, fecha y comercio
	\item Servidor invoca servicio de categorización automática
	\item Servidor crea transacción preliminar
	\item Servidor retorna datos extraídos a la aplicación
	\item Aplicación muestra preview para confirmación del usuario
\end{enumerate}

\subsection{Diagrama de secuencia 3: Categorización automática}

\begin{figure}[htpb!]
	\begin{center}
		% TODO: Agregar diagrama de secuencia de categorización
		\caption{Diagrama de secuencia - Categorización automática}
		\label{fig:secuenciaCategorizacion}
	\end{center}
\end{figure}

La categorización automática funciona mediante:
\begin{enumerate}
	\item Sistema recibe datos de transacción (comercio, descripción, monto)
	\item Motor de categorización consulta base de conocimiento
	\item Sistema analiza patrones históricos del usuario
	\item Algoritmo de ML asigna probabilidades a categorías posibles
	\item Sistema selecciona categoría con mayor probabilidad
	\item Si confianza es baja, marca para revisión manual
	\item Sistema asigna categoría y actualiza modelo de aprendizaje
\end{enumerate}

\subsection{Diagrama de actividad del flujo principal}

\begin{figure}[htpb!]
	\begin{center}
		% TODO: Agregar diagrama de actividad
		\caption{Diagrama de actividad - Flujo principal de registro de gasto}
		\label{fig:actividadPrincipal}
	\end{center}
\end{figure}

El flujo principal de la aplicación comprende:

\begin{enumerate}
	\item \textbf{Inicio:} Usuario autenticado accede a la aplicación
	\item \textbf{Decisión:} Usuario elige registrar nuevo gasto
	\item \textbf{Captura:} Usuario fotografía ticket o recibo
	\item \textbf{Procesamiento:} Sistema ejecuta OCR automáticamente
	\item \textbf{Extracción:} Sistema parsea y estructura datos
	\item \textbf{Categorización:} Motor inteligente asigna categoría
	\item \textbf{Confirmación:} Usuario revisa y confirma datos
	\item \textbf{Registro:} Sistema guarda transacción en base de datos
	\item \textbf{Actualización:} Sistema actualiza presupuestos y analytics
	\item \textbf{Notificación:} Si aplica, sistema envía alertas relevantes
	\item \textbf{Finalización:} Usuario ve confirmación y puede continuar
\end{enumerate}

Flujos alternativos incluyen corrección manual de datos OCR, creación de nuevas categorías, y manejo de errores de conectividad.