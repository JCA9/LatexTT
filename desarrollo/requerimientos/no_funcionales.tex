
\subsection{Requerimientos No Funcionales}
Los requisitos no funcionales definen las características y restricciones del sistema que no están directamente relacionadas con las funcionalidades específicas,
pero que son esenciales para garantizar su calidad, rendimiento, seguridad y usabilidad. A continuación, presentamos los principales requisitos no funcionales
identificados para nuestro sistema. \cite{RequisitosNoFuncionales}

\begin{table}[H]
\centering
\renewcommand{\arraystretch}{1.2}
\begin{tabularx}{\textwidth}{l l X}
\toprule
\textbf{Código} & \textbf{Nombre} & \textbf{Descripción} \\
\midrule

RNF01 & Seguridad &
Las contraseñas y datos personales deben almacenarse cifrados utilizando un algoritmo seguro (bcrypt) y no se guardará texto plano. \\

RNF02 & Mantenibilidad &
El código debe estar modularizado, documentado y versionado con control GitHub para facilitar actualizaciones. Cada módulo podrá ser actualizado sin afectar la funcionalidad global. \\

RNF03 & Portabilidad &
La aplicación móvil debe funcionar correctamente en dispositivos Android 10 o superior, manteniendo su funcionalidad y diseño sin errores en diferentes tipos de pantallas. \\

RNF04 & Compatibilidad &
La plataforma web debe ser compatible con navegadores modernos como Chrome (v135), Edge (v135) y Firefox (v140). \\

RNF05 & Usabilidad &
La interfaz de usuario de la aplicación móvil será desarrollada conforme a las Flutter Usability Guidelines y Material Design \cite{flutter2024}, asegurando botones, menús y elementos interactivos claros, consistentes y accesibles. La versión web seguirá las React Accessibility Guidelines \cite{react2024} \cite{infinum2024}, priorizando navegación intuitiva y tiempos de respuesta óptimos. El sistema deberá permitir su uso sin capacitación previa. \\

\bottomrule
\end{tabularx}
\caption{Requisitos No Funcionales (RNF01–RNF05)}
\end{table}
