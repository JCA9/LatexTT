\section{Análisis de Riesgos}

\begin{table}[H]
\centering
\small
\label{tab:riesgos}
\begin{tabular}{|p{1.2cm}|p{4cm}|p{3cm}|p{3cm}|p{4cm}|}
\hline
\textbf{ID} & \textbf{Riesgo} & \textbf{Categoría} & \textbf{Probabilidad} & \textbf{Gestión del riesgo} \\
\hline
R01 & Pérdida del código o archivos del proyecto & Técnico & Alta & Realizar un respaldo completo en GitHub y en la nube cada que se realice un cambio, sea minimo o no. \\
\hline
R02 & Un integrante deja el equipo & Personal & Baja & Replantear objetivos y priorizar módulos principales. Postergar funciones secundarias. \\
\hline
R03 & Fallas en la base de datos & Técnico & Moderada & Realizar respaldos semanalmente. \\
\hline
R04 & Problemas de conexión o caídas en Azure & Externo & Moderada & Tener respaldo local para pruebas. \\
\hline
R05 & Errores en el reconocimiento de texto del OCR & Técnico & Alta & Entrenar modelos como Regex con distintos tipos de tickets y validar resultados manualmente. \\
\hline
R06 & Cambios de requerimientos del proyecto & Organizacional & Alta & Documentar requerimientos y registrar cambios del proyecto. \\
\hline
R07 & Retrasos por carga académica o personal & Personal & Moderada & Reajustar cronograma y redistribuir tareas según prioridades. \\
\hline
R08 & Fallas en la comunicación del equipo & Organizacional & Moderada & Establecer reuniones semanales. \\
\hline
R09 & Exposición o pérdida de datos de usuarios & Seguridad & Alta & Cifrar datos sensibles como la contraseña. \\
\hline
R10 & Falla del equipo o energía eléctrica & Externo & Moderada & Usar laptops con batería, respaldos y almacenamiento en la nube. \\
\hline
\end{tabular}
\caption{Matriz riesgos identificados y su gestión}
\end{table}