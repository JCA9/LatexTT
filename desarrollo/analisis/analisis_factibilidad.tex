\section{Análisis de factibilidad}
El análisis de factibilidad del proyecto se realiza considerando tres dimensiones: técnica, económica y operativa.
\subsection{Factibilidad técnica}

El proyecto es técnicamente factible, ya que se implementa con herramientas y frameworks compatibles entre sí, ampliamente documentados y accesibles para desarrollo local.\\
Las pruebas del sistema se ejecutan en un entorno controlado, sin necesidad de servidores externos, lo cual permite validar el comportamiento completo del prototipo: registro de usuarios, escaneo de tickets, categorización automática y generación de reportes.
\subsection{Factibilidad económica}

Dado que el proyecto no contempla su comercialización ni costos de operación en producción, no genera gastos significativos.
Todo el desarrollo se realiza con versiones gratuitas o de código abierto.\\
El costo principal se asocia únicamente al tiempo de desarrollo, al uso de equipos personales y uso de aplicaciones de diseño como Balsamiq, por lo que su inversión económica es mínima.

\subsection{Factibilidad operativa}

El sistema es operativamente factible, ya que su despliegue local permite una demostración completa de las funcionalidades, sin depender de servicios externos.\\
La plataforma web puede ejecutarse desde el entorno de desarrollo y la aplicación móvil se distribuye mediante su archivo APK.
Esto facilita su presentación, evaluación y futura extensión hacia un entorno en producción, si se desea continuar el proyecto en fases posteriores.

En conclusión, el proyecto cumple con los criterios de viabilidad técnica y académica, siendo completamente funcional dentro de su entorno de desarrollo local y sustentado con herramientas tecnológicas sostenibles y accesibles.