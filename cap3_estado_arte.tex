\chapter{Estado del Arte}

El estado del arte de este proyecto se centra en analizar diversas aplicaciones, artículos y trabajos previos que abordan la gestión de finanzas personales mediante 
tecnologías digitales, particularmente aquellas que integran reconocimiento óptico de caracteres (OCR), inteligencia artificial (IA) o mecanismos automatizados de control 
de gastos. El objetivo es identificar las ventajas y limitaciones de las soluciones existentes para fundamentar la propuesta de una aplicación móvil y web que automatice el 
registro de gastos a través del escaneo de tickets de compra.

\section{Aplicación Expensify}
Expensify es una de las aplicaciones más consolidadas en el ámbito del control de gastos y reportes financieros. Su principal característica es el uso 
de \textbf{Reconocimiento Óptico de Caracteres (OCR)} para escanear recibos y extraer información relevante como montos, fechas y categorías. Los usuarios pueden 
generar reportes automáticos y exportarlos en formatos compatibles con software contable como QuickBooks o Xero.  
Entre sus ventajas destacan la integración con múltiples plataformas, la generación automática de informes y la posibilidad de sincronizar cuentas corporativas. 
No obstante, presenta desventajas como la necesidad de validar manualmente las categorías de gasto, errores ocasionales en el reconocimiento del texto y una interfaz más 
enfocada al entorno empresarial que al usuario común.  
Tras probar la aplicación, se observa que su interfaz resulta funcional, pero algo recargada, y que el proceso de configuración inicial puede ser lento. Es útil para 
profesionales o pequeñas empresas, pero excesiva para usuarios individuales que buscan simplicidad.

\section{Aplicación Veryfi}
Veryfi combina el OCR con \textbf{inteligencia artificial (IA)} para procesar automáticamente recibos y facturas. Su sistema categoriza los gastos y genera estadísticas 
detalladas en tiempo real, sin intervención manual. Está orientada tanto a usuarios individuales como a pequeñas empresas, destacando por su velocidad y precisión.  
La aplicación utiliza aprendizaje automático para mejorar continuamente el reconocimiento de texto y la clasificación de transacciones. Sin embargo, su versión gratuita 
es limitada y el costo de la suscripción puede ser alto para quienes solo buscan funcionalidades básicas.  
Al probarla, se percibe una gran precisión en el reconocimiento de datos, aunque su diseño está más orientado al entorno empresarial. Es una de las soluciones más avanzadas, 
pero con una curva de aprendizaje considerable.

\section{Aplicación Mint}
Mint, desarrollada por Intuit, es una de las aplicaciones de finanzas personales más populares a nivel global. Permite vincular cuentas bancarias, tarjetas y préstamos, 
ofreciendo una visión general del estado financiero del usuario. Aunque no cuenta con OCR, su sistema de categorización automática identifica patrones de gasto de forma inteligente.  
Entre sus ventajas se encuentran la integración bancaria, la creación de presupuestos personalizados y las alertas de gastos. No obstante, su desventaja principal es la 
falta de un módulo de escaneo de tickets y su dependencia de bancos con soporte API.  
En pruebas, destaca por su diseño intuitivo y reportes detallados, aunque su enfoque está más en la planificación que en el registro automatizado.

\section{Aplicación Shoeboxed}
Shoeboxed permite escanear recibos mediante OCR y convertirlos en reportes de gastos. Es útil para declaraciones de impuestos y control contable. Su principal ventaja es 
la digitalización de documentos con valor fiscal, pero la interfaz puede resultar poco amigable y la clasificación de gastos no siempre es precisa.  
La aplicación requiere tiempo de aprendizaje y ocasionalmente presenta errores de interpretación, especialmente con tickets en español. Pese a ello, ofrece un buen nivel 
de organización documental y respaldo digital.

\section{Aplicación Fyle}
Fyle integra OCR e IA para la gestión inteligente de gastos, generando reportes y recomendaciones automáticas. Se orienta principalmente al ámbito corporativo, 
pero puede adaptarse al uso personal. Ofrece integración con software contable y reportes automatizados.  
Las ventajas incluyen su precisión y automatización completa, mientras que sus desventajas radican en la necesidad de suscripción y la limitada personalización 
para usuarios comunes.  
En la práctica, Fyle logra procesar recibos con alta precisión, pero su modelo de negocio orientado a empresas restringe su adopción masiva.
