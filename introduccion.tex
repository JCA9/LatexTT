\chapter*{Introducción}

En el presente proyecto proponemos el desarrollo de un sistema para la gestión de finanzas personales, 
diseñado para asistir a los usuarios en el control, registro y análisis de sus ingresos y gastos de 
manera sencilla, automatizada y accesible. La solución se compone de una aplicación móvil y una 
plataforma web, que en conjunto permiten capturar la información de los tickets de compra, clasificar los 
gastos en categorías y generar reportes financieros dinámicos que facilitan la toma de decisiones 
económicas informadas.

El sistema está orientado a personas que buscan mejorar la administración de sus recursos personales o 
familiares, así como a pequeños negocios que requieren una herramienta básica para llevar un control de 
sus movimientos financieros sin necesidad de conocimientos contables avanzados. A través de un proceso 
de automatización y simplificación de tareas, se busca reducir la intervención manual y los errores 
comunes en el registro de datos financieros.

Para la extracción automática de información, el sistema emplea técnicas de Reconocimiento Óptico de 
Caracteres (OCR), las cuales permiten leer y procesar los datos contenidos en tickets físicos o digitales. Posteriormente, la información se clasifica según reglas heurísticas basadas en palabras clave y patrones de texto. Los datos procesados se almacenan en una base de datos relacional, desde donde pueden ser consultados, filtrados y analizados por el usuario en tiempo real.

En cuanto a su arquitectura, el sistema adopta un modelo cliente-servidor en tres capas, que separa la 
interfaz de usuario, la lógica de negocio y la gestión de datos, garantizando así escalabilidad y 
facilidad de mantenimiento. La plataforma web se desarrolla con Next.js, aprovechando sus capacidades de 
renderizado eficiente y gestión modular de componentes, mientras que la aplicación móvil se implementa en 
Flutter, lo que permite una experiencia nativa multiplataforma. La persistencia de datos se gestiona 
mediante MySQL, un sistema gestor robusto y de código abierto.

El proyecto tiene un carácter académico y de prototipo funcional, sin fines comerciales, y se implementa 
utilizando herramientas gratuitas y de código abierto. Su ejecución en entornos locales garantiza 
independencia tecnológica y bajo impacto económico. En términos generales, esta propuesta busca fomentar
el uso responsable de la tecnología para el fortalecimiento de la educación financiera, ofreciendo una 
alternativa práctica, sostenible y adaptable a las necesidades de los usuarios.